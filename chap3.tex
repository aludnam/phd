%!TEX root = thesis.tex
\chapter{Line Scan - Structured Illumination Microscopy\label{ch:LSSIM}}

In this chapter we discuss a combination of structured illumination microscopy with line scanning. The work has been done in collaboration with Institute of photonic technology (IPHT) Jena in Germany. The work has been recently published in Optics Express \cite{Mandula2012b}:

\href{http://www.opticsinfobase.org/oe/abstract.cfm?uri=oe-20-22-24167}{\textcolor{black}{O. Mandula, M. Kielhorn, K. Wicker, G. Krampert, I. Kleppe, and R. Heintzmann, "Line scan - structured illumination microscopy super-resolution imaging in thick fluorescent samples," Optics Express {\bf20}, 24167 (2012).}}

%==========================================
%==========================================

\section{Structured illumination microscopy}

Structured illumination microscopy (SIM) is a fluorescence microscopy technique providing images of biological samples with resolution surpassing the classical diffraction limit \cite{Heintzmann1999, Gustafsson2000}. SIM requires modification of the illumination part of a wide-field fluorescent microscope such that the sample is illuminated with a spatially varying intensity pattern. The most common pattern used in SIM consists of dense stripes with a sinusoidal profile. This pattern is typically generated by laser light passing through an optical grating (we denote this as the SIM grating) and focused with an objective into a sample. The SIM grating is placed in the plane conjugate to the sample plane and therefore translation and rotation of the optical grating result in translation and rotation of the illumination pattern. High-resolution information is extracted by processing images with different translation (henceforth referred to as phase) and rotation of the illumination pattern \cite{Heintzmann1999, Gustafsson2000}.

Despite the capability of producing optically sectioned images \cite{Schermelleh2008}, SIM becomes increasingly challenging when applied to thick ($>20\unit{\mu m}$) densely labelled fluorescent samples. The wide-field-like (WF) illumination of the SIM system generates a high intensity of out-of- focus fluorescent light. This homogeneous background is added to the spatially modulated fluorescent emission and reduces the pattern modulation in the recorded images \autoref{fig:Fig1}\bbb. The additive background also increases the noise level, which further deteriorates the quality of the pattern. As a result, the reconstructed images are corrupted by strong noise artifacts (\autoref{fig:Fig2}\ccc,\ddd\ and \autoref{fig:Fig3}\bbb).

SIM with sparse illumination patterns \cite{benedetti2000method} in order to reduce the out-of-focus light has been demonstrated recently \cite{York2012}. The method was used in conjunction with assigning detected light (pixel reassignment) to the most likely position of the emitter \cite{Cox1982,Cox1982b}. The sparse patterns were generated by a digital micromirror device (DMD). In this article, we propose generation of the sparse illumination patterns by combining the structured illumination with line scanning (LS) microscopy.

In the LS microscopy, the excitation light is focused into a thin line and swept across a fluorescent sample. The out-of-focus light is discarded either by a physical confocal slit (line confocal microscope) or computationally post acquisition  \cite{benedetti2000method,Neil1997}. The axial response of a line confocal system is discussed in \cite{Poher2008}.

The combination of the LS and SIM methods, which is the focus of this manuscript, merges the ability of a line scanning system to physically suppress out-of-focus light with the resolution enhancement of structured illumination. LS-SIM therefore enables high-resolution imaging in thick fluorescent samples. This idea has been demonstrated on simulated data \cite{Kim2009} and in this article we show LS-SIM reconstructed images of thick fluorescent sample.

%==========================================
%==========================================

\section{Experimental methods}

%==========================================

\subsection{Setup}
\begin{figure}[!hbt]
	\centering
	\newcommand{\wf}{1\textwidth}
	\includegraphics[width=\wf]{\home Documents/Publications/LineScanPaper/OPEX_finalsubmission/Figure1}
	\caption{Illustration of the LS-SIM setup. The black arrows indicate the movement of the line scan. Upper left inset: (a) one frame of the LS-SIM raw data $I_{n,o,p}$ with indicated ON (illuminated) and OFF (not illuminated) regions (see further explanation in the main text). A close-up of the red box region is shown as (b) a wide-field SIM image $I^{WF-SIM}$ and (c) a line-scan SIM $I^{LS-SIM}_{o,p}$ image. Scale bar (a-c) $2\unit{\mu m}$. Bottom right inset: Illustration of the LS-SIM pattern formation. Real space (top row) corresponds to a SIM grating plane. Fourier space (bottom row) represents the distribution of the intensity in the back focal plane (BFP) of the objective (h, i, j). The aperture of the BFP is indicated as a red circle. The multiplication ($\times$) of the images in real space corresponds to the convolution operation ($\bigotimes$) in Fourier space. The intensity distribution in the sample plane is shown in (g). The Fourier transform of (g) is shown in (k) with the border of the optical transfer function indicated as a green circle and the position of the Fourier transformed intensity orders of the grating are indicated with red arrows.} 
	\label{fig:Fig1}
\end{figure}
%
We used a pre-commercial prototype of the ZEISS ELYRA-S system ($63\times$ /$1.4$ N.A. objective) with a line-scanning module (ZEISS LSM DuoScan SL). A schematic illustration of the setup is shown in \autoref{fig:Fig1}. A cylindrical lens was used to focus the laser light ($488\unit{nm}$) into a thin scanning line. The line was focused onto a SIM grating, with bars perpendicular to the scanning line. This produced a sinusoidal intensity modulation along the scanning line in the sample plane (\autoref{fig:Fig1}\gggg). The phase of the fine sinusoidal SIM pattern was controlled by translation of the SIM grating while the orientation was changed with an image rotator (integral part of the ZEISS ELYRA-S system) positioned between the SIM grating and the objective. This setup therefore avoids the physical rotation of the cylindrical lens and the SIM grating.

%==========================================
\clearpage
\subsection{Data acquisition}
During each camera (Andor iXon DU-885K) acquisition frame of $120 \unit{ms}$, the line was swept across the sample. The scanning galvo-mirror was synchronised with an AOM-based modulation of the excitation laser light such that a sparse periodic pattern consisting of thin bright (ON) lines separated by wide dark (OFF) areas was generated in the sample (\autoref{fig:Fig1}\aaa). The ratio of the ON area to the total area (ON + OFF), the mark/area ratio (MAR) \cite{Pawley2006}, was set to approximately 1/26 with 32 partially overlapping scan positions of the line. The fine sinusoidal (SIM) modulation was superimposed on each bright (ON) line creating a pattern resembling beads on a string (\autoref{fig:Fig1}\aaa). The synchronisation was controlled with an Arduino microcontroller (MC-NOVE, MultiComp). An illustration of the trigger signals in the microscope is shown in supplementary figure (\autoref{app:LSSIM electronics}).

For each orientation ($o$) and phase ($p$) of the SIM grating we acquired $N = 32$ images $I_{n,o,p}$ with different positions ($n = 1..N$) of the bright (ON) lines (\autoref{fig:Fig1}\aaa). The ON lines were shifted in each image $I_{n,o,p}$, such that when summed over the N images, they fill the dark areas between the lines. The sum of acquired images therefore corresponds to a standard wide-field SIM (WF-SIM) raw image with slightly reduced modulation of the SIM pattern:
%
\begin{equation}
	I^{WF-SIM}_{o,p}=\sum_{n=1}^NI_{n,o,p}.
\end{equation}

A conventional WF image can be computed by summing all WF-SIM frames $I^{WF}\approx\sum_{o,p}I^{WF-SIM}_{o,p}$. However, the instrument can also be switched between the line scanning illumination and the wide-field mode. We therefore acquired genuine WF-SIM data for comparison (\autoref{fig:Fig1}\bbb) with approximately the same total number of photons per pixel over all necessary images.

We treat the raw data $I_{n,o,p}$ in two distinct steps. In the first step, we take advantage of the line scan to produce background-reduced, optically sectioned images $I^{LS-SIM}$ (\autoref{eq:LSSIM}) with improved SIM pattern visibility as compared to conventional WF-SIM. In the consecutive step, we extract the high-resolution information from the structured illumination using conventional WF-SIM treatment of the data.

%==========================================

\subsection{Data evaluation}

A background reduced image with a SIM pattern of superior quality and high modulation (\autoref{fig:Fig1}\ccc) can be computed from the raw data $I_{n,o,p}$ using \cite{benedetti2000method,Heintzmann2006}
%
\begin{equation}
	I^{LS-SIM}_{o,p}=\max_n(I_{n,o,p})+\min_n(I_{n,o,p})-2\underset{n}{\unit{mean}}(I_{n,o,p}),
	\label{eq:LSSIM}
\end{equation}
%
which is known to yield good results for sparse illumination patterns \cite{Heintzmann2006} (low MAR). \Autoref{eq:LSSIM} is one of many ways to obtain optical sectioning from the raw data \cite{Schermelleh2008,benedetti2000method}, and this equation has been chosen for simplicity and robustness. Images $I^{LS-SIM}_{o,p}$ with different phase translations ($p$) and orientations ($o$) of the SIM pattern are passed to a SIM reconstruction algorithm \cite{Wicker2010a} to produce a final reconstructed image shown in \autoref{fig:Fig2} and \autoref{fig:Fig3}\aaa.

We used five phases ($p = 1..5$) and three rotations ($o = 1..3$) of the SIM pattern ($5 \times 3 = 15$  $I^{LS-SIM}$ images). Each $	I^{LS-SIM}$ image requires $n=1..32$ individual scan images $I_{n,o,p}$. Therefore we captured $5\times3\times32 = 480$ images for one reconstructed plane. The total acquisition time was approximately $75\unit{s}$ with $120\unit{ms}$ acquisition time for each frame $I_{n,o,p}$.

Corresponding line-confocal images (without SIM) can be computed by summing over orientations ($o$) and phases ($p$) of the SIM pattern using the preprocessed sectioned data $I^{LS}=\sum_{o,p}I^{LS-SIM}_{o,p}$. The LS image is shown in \autoref{fig:Fig2}\eee,\fff.

%==========================================
%==========================================

\section{Results}

\begin{figure}[!hbt]
	\centering
	\newcommand{\wf}{1\textwidth}
	\includegraphics[width=\wf]{\home Data/lineScan/120116/HomeMade_g561_ROI4_LS/Results/images/comparisonROI123lineprofile_labels}
	\caption{ LS-SIM reconstructed image of a Calliphora salivary gland (i). WF image (bottom-left corner of (i)) shown for comparison. Scale bar $2 \unit{\mu m}$. Selected regions (red, green) reveal details of the actin structures: (a, b) WF image, (c, d) WF-SIM reconstruction, (e, f) LS image, (g, h) LS-SIM reconstruction. Intensity profiles along red line in (e) and blue line in (g) is plotted in (j) in corresponding colours. Scale bar (a-h) $1 \unit{\mu m}$.} 
	\label{fig:Fig2}
\end{figure}

A LS-SIM reconstruction of a Calliphora salivary gland stained with Alexa488-Phalloidin is shown in \autoref{fig:Fig2}\iii. The sample was $\sim30 \unit{\mu m}$ thick and the section was taken $\sim5 \unit{\mu m}$ below the surface. The bottom left corner of \autoref{fig:Fig2}\iii\  shows a wide-field (WF) image for comparison. The out-of-focus light is dramatically reduced in the LS-SIM reconstructed images, revealing complex structural detail within the sample. The fine actin structure remains completely unresolved in the WF image \autoref{fig:Fig2}\aaa,\bbb. Reconstructed conventional WF-SIM data reveal some structural detail, but the image is severely corrupted by noise artifacts (\autoref{fig:Fig2}\ccc,\ddd\ and \autoref{fig:Fig3}\bbb). The WF-SIM data were taken in the wide-field mode of the microscope prior to the LS-SIM data acquisition with the same number of phases and orientations of the SIM pattern. The acquisition time was $120 \unit{ms/frame}$ and the laser power was adjusted to achieve an approximately similar number of photons/pixels as for the whole series of LS-SIM data. The appropriate adjustment of the laser power was determined from the previous measurement using a different region of the same sample.

\begin{figure}[!hbt]
	\centering
	\newcommand{\wf}{1\textwidth}
	\includegraphics[width=\wf]{\home Data/lineScan/120116/HomeMade_g561_ROI4_LS/Results/images/comparisonROI3_WFSIMandLSSIM_arrows_letters}
	\caption{Comparison of (a) LS-SIM and (b) a conventional WF-SIM reconstruction of the blue-framed region from \autoref{fig:Fig2}. LS-SIM image is less affected with noise artifacts, which results in cleaner image. Arrows are pointing to the structures revealed in LS-SIM image. Scale bar $2\unit{\mu m}$.} 
	\label{fig:Fig3}
\end{figure}

The line scan image $I^{LS}$, corresponds to an image from a line confocal microscope. It provides optical sectioning with strongly reduced out-of-focus light (\autoref{fig:Fig2}\eee,\fff). However, when compared with a wide field image, resolution in the lateral direction has not been improved. In LS-SIM the optical sectioning capability of the line scanning and the resolution improvement of the structured illumination combine to show very fine details of the specimen's inner structure (see \autoref{fig:Fig2}\hhh\  and arrows in \autoref{fig:Fig3}\aaa), unresolved in WF (\autoref{fig:Fig2}\bbb) and barely visible in WF-SIM (\autoref{fig:Fig2}\ddd\  and arrows in \autoref{fig:Fig3}\bbb) and LS mode (\autoref{fig:Fig2}\fff). Due to the higher quality of the SIM pattern in the $I^{LS-SIM}_{o,p}$ images (\autoref{fig:Fig1}\ccc), the reconstructed images are less affected by noise artifacts compared with the conventional WF-SIM reconstruction. Some of the very dim structures in LS-SIM are completely concealed in the WF-SIM image (magenta arrow in \autoref{fig:Fig3} � best visible directly on the screen).

\begin{figure}[!hbt]
	\centering
	\newcommand{\wf}{1\textwidth}
	\includegraphics[width=\wf]{\home Documents/Publications/LineScanPaper/OPEX_finalsubmission/Figure4}
	\caption{Fourier transforms (a) of a single raw scan frame $I_{n,o,p}$ and (b) of a single $I^{LS-SIM}_{n,o,p}$ image computed from \autoref{eq:LSSIM}. Red arrows point at five peaks of the illumination pattern (see \autoref{fig:Fig1}\kkk). The second diffraction peaks are located at 70\% of the cut-off frequency (green circle). The division of the diffraction lines into several points in (a) stems from the illumination of a multitude of lines per exposure.} 
	\label{fig:Fig4}
\end{figure}

Intensity profiles measured along a cross-section of a fine, vertically oriented structure revealed in the LS (blue line in \autoref{fig:Fig2}\eee) and the LS-SIM image (red line in \autoref{fig:Fig2}\gggg) are shown in \autoref{fig:Fig2}\jjj. The cross-section was measured as an average of a stripe five pixels ($\sim200\unit{nm}$) in width to provide smoother curves and is plotted with circular marks of corresponding colour in \autoref{fig:Fig2}\jjj. Gaussian fits are plotted as smooth curves. The estimated full width in half maximum (FWHM) of the LS-SIM profile was 1.6 times smaller than the one extracted from the LS image. This ratio is a rough estimate for the resolution improvement achieved with structured illumination. In our setup, the second diffraction orders were located at about 70\% of the BFP aperture radius (\autoref{fig:Fig4}), which corresponds to the expansion of the cut-off frequency border (i.e. the OTF support) by a factor of 1.7 after image reconstruction \autoref{fig:Fig5}. This is consistent with the above stated narrowing of the line by a factor of 1.6.

\begin{figure}[!hbt]
	\centering
	\newcommand{\wf}{1\textwidth}
	\includegraphics[width=\wf]{\home Documents/Publications/LineScanPaper/OPEX_finalsubmission/Figure5}
	\caption{Fourier transform of the LS image (a) with a green circle marking the cut-off frequency region. The Fourier transform of the reconstructed LS-SIM image (b) shows the extension of the transferred frequencies. The extended cut-off border is shown as a red circle. The arrows point at suspicious peaks in the spectrum possibly giving rise to the artifacts in the reconstructed image.} 
	\label{fig:Fig5}
\end{figure}

%==========================================
%==========================================
\clearpage
\section{Discussion}

The sectioning capability and reconstruction artifact reduction demonstrated in this article are significant advantages of LS-SIM over conventional structured illumination (WF-SIM). Slit scanning systems are known to be well capable of suppressing out-of-focus light \cite{Poher2008} and can be operated at quite high speed \cite{Botcherby2009}.
Even though, when imaging a fluorescent sheet, the asymptotic intensity decay of slit scan systems is typically inferior to that of pinhole based systems ($1/u$ vs. $1/u^2$ for axial distance $u$), axial resolutions below a micrometer are achievable as seen in the images presented in \cite{Botcherby2009}.

The sectioning performance depends on the size of the illuminated volume - in our case on the width of the ON lines - and on the mark/area ratio (sparsity) of the line illumination pattern (\autoref{fig:Fig1}\aaa). The pattern must be sufficiently sparse to avoid cross talk between adjacent ON regions, and the ON lines should be sufficiently thin to reduce the illuminated volume in each frame. However, sparser patterns with thinner ON lines require more scans, increasing the total acquisition time. There is therefore a trade-off between the sparseness of the illumination pattern and the acquisition time. In LS-SIM we have to capture 32 times more frames than in standard WF-SIM. This requires a highly stable setup as movement of the sample during acquisition (75 seconds for a single slice) may impair the reconstruction.

Noise artifact reduction is associated with an improvement of the signal-to-noise ratio in the line-scanned data. Assuming Poisson noise in the CCD camera images, the variance of the noise is proportional to the signal intensity. Only a small in-focus fraction of the sample volume is illuminated at any given time during the line scan. The fluorophores in the out-of- focus regions of the specimen are excited less, which reduces their contribution to the pixels imaging the fluorescence signal collected from the line-illuminated in-focus region. This physical reduction of the out-of-focus background decreases the total intensity and hence the noise variance in the detected data, rendering the SIM reconstruction more accurate. A further refinement of optical sectioning is achieved by application of \autoref{eq:LSSIM}, which has the effect of essentially subtracting the out-of-focus background estimated from the OFF (not illuminated) regions in the raw data (\autoref{fig:Fig1}\aaa).

Despite the image quality improvement, LS-SIM images (like SIM images) are not always free of artifacts. Arrows in \autoref{fig:Fig5}\bbb\ point to suspicious peaks in the reconstructed Fourier transform of the LS-SIM image. These regions are a likely source of the reconstruction artifacts and might be caused by non-uniform bleaching of the sample during the line scanning process as suggested by peaks in the Fourier transform of the LS image (arrows in \autoref{fig:Fig4}\aaa) or by fluctuations in laser intensity. We tried to reduce the bleaching artifacts by shifting the ON lines (\autoref{fig:Fig1}\aaa) in consecutive scans not in a natural consecutive order, but in an order maximising the distance between each two consecutive ON regions.

The illustration of the LS-SIM pattern generation process in \autoref{fig:Fig1}\ddd-\kkk\ highlights specific features and limitations of the LS-SIM method. The diffraction peaks of the SIM pattern, shown \autoref{fig:Fig1}\hhh, are smeared into line structures (\autoref{fig:Fig1}\jjj) due to the line focusing of the illumination (\autoref{fig:Fig1}\eee,\iii). While the first and zero diffraction orders of the SIM grating lie well within the back focal plane (BFP) aperture, as shown in \autoref{fig:Fig1}\hhh\  (BFP aperture shown as red circle), their line-illumination versions are spread out along an orthogonally oriented line and then partially blocked (\autoref{fig:Fig1}\jjj). The strength of the diffracted orders is therefore reduced, which in turn reduces the modulation of the illumination SIM pattern in the sample plane. In other words, the line scanning improves the modulation of the detected pattern (because of optical sectioning) but can reduce the modulation of the actual illumination pattern. This effect becomes stronger for finer SIM gratings, where the diffracted orders are near the border of the BFP aperture. Finer gratings are required for further resolution improvement, but the advantages of LS-SIM become less prominent when the line-shaped diffracted orders (\autoref{fig:Fig1}\jjj) are nearly at the edge of the BFP aperture.

Note that the whole LS-SIM data can be treated by a SIM reconstruction algorithm assuming a series of 2D illumination patterns \cite{Heintzmann2003}. This might provide better sectioning and superior performance for noisy data. However, this requires an excessive amount of computation and therefore we opted for two-step treatment of the data.

The LS and WF images have been up-sampled to the size of the reconstructed images. This has been achieved by padding the corresponding Fourier transform by zeros. To reduce hexagonal ``knitting artifacts'', a circular window smoothly decaying outside the support of the original optical transfer function was applied in Fourier space prior to padding (see, for example, \autoref{fig:Fig5}\aaa).

In a recent publication \cite{York2012} the sample was illuminated with a multitude of point focuses. In this case, the contrast for high illumination frequencies follows the transfer curve of a wide field system and thus almost no contrast is present for the very high frequencies. In contrast, the line-scan SIM illumination used in our work conserves the high-frequency contrast very well up to only shortly below the cut-off frequency. Therefore an enhanced super-resolution capability can be expected in our case. The image-processing scheme in our work is also significantly different from the photon reassignment \cite{Cox1982b} used in \cite{York2012}.

%==========================================
%==========================================

\section{Conclusion}

LS-SIM provides optically sectioned images of thick fluorescent samples with a significant resolution improvement (1.6 times as measured by a line scan) in the lateral plane. The LS-SIM reconstructed images suffer significantly less from reconstruction artifacts than conventional structured illumination. LS-SIM reveals fine details of a biological specimen's inner structure with higher resolution than line-confocal microscopy and with image quality superior to conventional structure illumination.
