%!TEX root = thesis.tex
\chapter{Introduction\label{ch:Introduction}}

%==========================================
%==========================================

\section{Optical microscope} % no articles in the headings

A microscope is an instrument allowing us to see objects, which are too small for a naked eye. An optical microscope (often referred to as a ``light microscope'') uses light in the visible spectral range (wavelength $\approx400-700\unit{nm}$), which makes it particularly suitable for biological exploration. Visible light is minimally invasive for sensitive biological samples and allows observation of living specimens. Visible light is also minimally absorbed by water, which prevents heating of the sample.

 The most common optical microscope is a ``far-field'' microscope, where the light has to propagate over a distance significantly longer than its wavelength. The specimen is observed with transmitted, reflected or fluorescent light. Fluorescence microscopy is discussed further in \autoref{sec:Fluorescence microscopy}. The focus of this thesis is on far-field fluorescence optical microscopy.

%==========================================
%==========================================

\section{Brief historical overview}

Optical microscopy has been around for over 400 years. Since the very early versions of Zacharias Janssen's or Galileo's compound microscopes from the beginning of 17\ths{} century, optical microscopy has undergone a long and steady process of development. Despite the speculation as to who was the actual inventor of the optical microscope, it was Anton van Leeuwenhoek who largely popularised the use of the microscope as an instrument for observing the minute details of the specimen. Leeuwenhoek also introduced his simple instrument into biological research during the 17\ths{} century.

An important milestone was the pioneering work of Ernst Abbe \cite{Abbe1873} in the second half of the 19\ths{} century. Abbe set the theoretical resolution limit for the optical microscope and mastered the design of objective lenses highly corrected for optical aberrations.

With advances in technology in the 20\ths{} century, the manufactures have produced lenses reaching the theoretical limits of the optical microscope performance. The 1953 Nobel prize in physics was awarded to Frits Zernike for discovery of the phase contrast \cite{Zernike1942}. This method allows observation of transparent specimens, and had major impact on biological research such as in vivo study of cell cycle. 

The emergence of new microscopy methods surpassing the classical resolution limit (super-resolution microscopy) at the end of the 20\ths{} century and at the beginning of the 21\st{} century has given another boost to optical microscopy research. The resolution of the super-resolution optical microscopes has reached the order of ten nanometres and some researchers have proposed the term ``optical nanoscopy'' to be used \cite{Egner2007, Hell2007, Hell2009}. However, super resolution micro/nano-scopy remains a challenging task, especially when applied to living biological specimens. While most of the super-resolution techniques require a highly specialised and expensive hardware, some of the techniques, such as localisation microscopy (discussed in \autoref{ch:NMF}) can be performed with a conventional fluorescent microscope. 

The number of scientific publications in recent years shows that even after four centuries of development the optical microscopy remains a vibrant and exciting scientific domain.

%==========================================
%==========================================

\section{Point spread function}

An important characteristic of a microscope is the so-called ``point spread function'' (PSF). The PSF represents an image of a point source. The image $i(x)$ of a specimen produced by an optical microscope can be described as a convolution between the object (specimen) $o(x)$ and the point spread function $q(x)$:
%
\begin{equation}
	i(x)=\int q(x-x')o(x')dx'.
	\label{eq:conv}
\end{equation}  

The PSF therefore defines how much the image of the specimen is ``blurred'' during the imaging process. The integration in \autoref{eq:conv} is over the whole space of acquired data (typically 2D or 3D). Note, that \autoref{eq:conv} applies to the situation with spatially invariant PSF. It also assumes that PSF is fully determined by the optical system. The influence of the specimen on the shape of the PSF is neglected. In a real experiment, PSF can be locally distorted by the aberrations introduced by the specimen or by the imperfections in the optical setup.

\begin{figure}[!bht]
	\centering
	\newcommand{\wf}{.48\textwidth}
	\subfloat[Linear]{
	\includegraphics[width=\wf]{figures/psfillustration/PSFAiryGauss}}
	\subfloat[Logarithmic]{
	\includegraphics[width=\wf]{figures/psfillustration/PSFAiryGauss_semilogy}}
	\caption{Intensity profile of a PSF for 1.2 NA objective using $\lambda_{em}=625$ nm emission light. The blue line represents profile of the Airy pattern, red dashed line shows the Gaussian approximation. Green vertical lines mark the first minima of the Airy pattern at $\delta=318$ nm. $\delta$ corresponds to the radius of the Airy disk. (a) Linear, (b) logarithmic plot of the intensity highlighting the secondary maxima in the Airy pattern and the differences of the Gaussian approximation at the periphery of the function.}
	\label{fig:PSF}
\end{figure}
%
Neglecting the effect of polarisation (scalar theory), the two-dimensional PSF of an optical microscope, known as the ``Airy pattern'', is described by \cite{Born1999}
%
\begin{equation}
	q(\rho)=\frac{1}{Z}\left(\frac{J_1(\alpha \rho)}{\alpha \rho}\right),
\end{equation}
%
where $Z=\int q(\rho) d\rho$ is the normalising constant, $J_1$ is the Bessel function of the first kind of order one and $\rho$ is the distance from the centre of the image. The parameter $\alpha$ depends on the emission wavelength $\lambda_{em}$ and the numerical aperture of the objective $\unit{NA}$:
%
\begin{equation}
	\alpha=2\pi\frac{\unit{NA}}{\lambda_{em}}.
\end{equation}
% 
The $\unit{NA}$ is defined as
\begin{equation}
	\unit{NA}=n\sin(\theta),
	\label{eq:NA}
\end{equation}
%
where $n$ is the refractive index of the immersion medium and $2\theta$ is the angle of the light cone entering the objective. Note that the refractive index is a fucntion of wevelength $n\sim n(\lambda)$. Dispersion in the sample and in the optics can lead to the PSF corrupted with spherical aberration when a range of wavelengths is used \cite{SCALETTAR1996}.

An intensity profile of an unaberrated PSF is shown in \autoref{fig:PSF}. The Airy pattern is also compared to the popular approximation of the PSF with a Gaussian function \cite{Zhang2007} defined by a standard deviation
%
\begin{equation}
	\sigma=\frac{\sqrt{2}}{2\pi}\frac{\lambda_{em}}{\unit{NA}}.
\end{equation}

%==========================================
%==========================================

\section{Resolution limit \label{sec:Resolution limit}}

\begin{figure}[!bht]
	\centering
	\newcommand{\wf}{.3\textwidth}	
	\newcommand{\ndir}{figures/psfillustration/}
	\begin{tabular}{ccc}
		\subfloat[$d=\delta/2$]{\includegraphics[width=\wf]{\ndir Airy05_AiryDisk}}
		& \subfloat[$d=\delta$]{\includegraphics[width=\wf]{\ndir Airy1_AiryDisk}}
		& \subfloat[$d=2\delta$]{\includegraphics[width=\wf]{\ndir Airy2_AiryDisk}}
		\tabularnewline
		\subfloat[$d=\delta/2$]{\includegraphics[width=\wf]{\ndir AiryPix05_AiryDisk}}
		& \subfloat[$d=\delta$]{\includegraphics[width=\wf]{\ndir AiryPix1_AiryDisk}}
		& \subfloat[$d=2\delta$]{\includegraphics[width=\wf]{\ndir AiryPix2_AiryDisk}}
		\tabularnewline
		\subfloat[$d=\delta/2$]{\includegraphics[width=\wf]{\ndir Airy05profile}}
		& \subfloat[$d=\delta$]{\includegraphics[width=\wf]{\ndir Airy1profile}}
		& \subfloat[$d=2\delta$]{\includegraphics[width=\wf]{\ndir Airy2profile}}
		\tabularnewline
	\end{tabular}
	\caption{Two PSFs (1.2 NA objective, $\lambda=625$ nm) separated by distance $d$. Location of the sources is indicated with red dots. Airy disk is indicated with green dashed circles. The radius of the Airy disk corresponds to $\delta=318$ nm. (a) Continuous representation of the PSFs. (b) Pixelated version with pixel-size 80 nm. (c) Blue lines show the intensity profiles along the line intersecting the sources' locations. The profiles of the individual PSFs are shown as red and magenta dashed lines.}
	\label{fig:Rayleigh}
\end{figure}
%
The ability of an optical microscope to show spatial details in the specimen structure is fundamentally limited by diffraction \cite{Born1999}. The radius $\delta$ of the Airy disk (see \autoref{fig:PSF}) is often considered as the ``classical resolution limit''. It is given by \cite{Born1999}
%
\begin{equation}
 	\delta=0.61\frac{\lambda_{em}}{\unit{NA}}, 
	\label{eq:Airy}
\end{equation}
%
where $\lambda_{em}$ is the wavelength of the emitted light and $\unit{NA}$ is the numerical aperture of the objective \autoref{eq:NA}. The resolution limit \autoref{eq:Airy} comes from the \emph{empirical} observation: two sources separated by a distance greater than $\delta$ can be ``resolved'' as two individual objects (see \autoref{fig:Rayleigh}\ccc). If the separation is smaller than $\delta$, the point spread functions overlap significantly and the sources become ``unresolved'' (see \autoref{fig:Rayleigh}\aaa). The resolution limit defined by \autoref{eq:Airy}, sometimes called  the ``Raleigh resolution limit'',  is often taken as the benchmark for different resolution techniques. Note that the ``Abbe resolution limit'' $\delta=0.5\lambda_{em}/\unit{NA}$ related to the passband of spatial frequencies is also used.

The resolution limit \autoref{eq:Airy} relates to the noise-free situation with continuous representation of the PSF (the top line of \autoref{fig:Rayleigh}). However, the pixelation and noise associated with the photon-detection process can deteriorate the resolution significantly. The influence of noise, pixelation or brightness intermittency to the resolution is discussed in \autoref{ch:Theoretical-limits-of the LM}. 

%==========================================
%==========================================

\section{Fluorescence microscopy\label{sec:Fluorescence microscopy}}

The introduction of fluorescence microscopy in the 20\ths\ century has revolutionised the use of optical microscopy in biological and medical science. Fluorescence is generated by fluorescent molecules or nanostructures (commonly called fluorophores) during the relaxation of their electronic structure to the ground state.  The fluorophores are driven into higher energetic state with an excitation light of a specific wavelength. The fluorescent light is typically shifted towards the longer wavelengths (Stokes shift),  which allows an efficient filtering of the excitation light from the fluorescence signal. Fluorescence microscopy detects the fluorescence from the sample itself (auto-fluorescence) or more commonly from the fluorophores attached to the specimen \cite{PawleyHandbook2006}.

Fluorescence labels can be attached to the structures of interest with high specificity and provide a strong intensity contrast in the microscopic image of the specimen. Fluorescent proteins (FPs) allow for direct expression of the fluorescent marker by the organism itself \cite{Fernandez-Suarez2008}. FPs have further redefined the use of fluorescence microscopy in cell biology as a nearly non-invasive and highly specific technique. 

The first fluorescent protein used in microscopy (green fluorescent protein - GFP) has been isolated from the jellyfish {\it Aequorea victoria} \cite{Tsien1998}. Nowadays, modified versions of GFP and a rich variety of fluorescent dyes cover the emission across the whole visible spectral range \cite{Fernandez-Suarez2008}. Photo-activable and photo-switchable fluorescent proteins and dyes have also been discovered, and are used in specific applications and microscopy methods \cite{Bock2007,Hirvonen2008, Rego2011a}. The 2008 Nobel Prize in Chemistry was awarded to Martin Chalfie, Osamu Shimomura, and Roger Y. Tsien for their discovery and development of the GFP.
 
%==========================================
%==========================================

\section{Quantum Dots\label{sec:Quantum-dots}}

Quantum dots (QDs), recently used in biological research, are promising fluorescent labels.  QDs are inorganic crystals composed of 100-100,000 atoms of substances such as cadmium selenide (CdSe), with diameter $\sim2-10\unit{nm}$ \cite{Alivisatos1996}. QDs are often coated with a zinc sulphide (ZnS) shell conjugated with an antibody molecule. The diameter of the coated QDs is $\sim10-30\unit{nm}$. The emission wavelength can be tuned by the diameter of the QD core. QDs tend to have a broad excitation spectrum and a narrow emission spectrum when compared to the standard fluorescent dyes/proteins. This facilitates the multicolour imaging with QDs. QDs are also exceptionally photo-stable. It has been reported that QDs illuminated with continuous $\sim 50\unit{mW}$ laser do not bleach even after $14\unit{h}$, whereas the standard fluorescent dye ``fluorescein'' completely bleaches in less than 20 mins \cite{Jaiswal2004}. Moreover, QDs are an order of magnitude brighter than the traditional fluorophores \cite{Resch-Genger2008,Walling2009}. 

Commercially available QDs have a polymer coating with a covalently attached linker, or are conjugated with an antibody molecule. This allows them to be specifically attached to the structure of interest in the specimen the same way as the standard fluorescent labels. However, coated QDs are relatively big ($\sim10-30\unit{nm}$) compared to the fluorescent dyes (fluorescenin size is $\sim1\unit{nm}$) or fluorescent proteins (GFP size is $\sim5\unit{nm}$). Therefore QDs cannot diffuse through the cell membrane, which complicates the labelling of the structures in the interior of the cell. 

An interesting property of the QDs is that they exhibit ``fluorescence blinking'' (fluorescence intermittency) under continuous excitation. QDs switch between the ON episodes of a rapid absorption-fluorescence cycling and the OFF episodes, where no light is emitted despite the continuous excitation. Both ON-time ($\tau_{ON}$) and OFF-time ($\tau_{OFF}$) probability densities follow an inverse power law $P(\tau_{ON/OFF})\propto1/\tau_{ON/OFF}^{m}$ \cite{Kuno2001, Stefani2009}. A comparison of QDs and the standard fluorescent dyes is described in the Resch-Genger et al. review article \cite{Resch-Genger2008}.

%==========================================
%==========================================

\section{Super-resolution\label{sec:super-resolution}}

The resolution limit \autoref{eq:Airy} for a far-field optical microscope has been challenged in the past two decades. Several research teams have reported sub-diffraction resolution in the fluorescent samples. In fact, the resolution limit is no longer dictated by diffraction but by the signal-to-noise ratio. 

There is a rich variety of super-resolution strategies in fluorescence microscopy. However, all these methods are based on driving the nearby fluorophores into emitting (ON) and non-emitting (OFF) states. This allows distinguishing the individual emitters separated by a sub-resolution distance.

``Selective'' activation can be achieved either by spatially structured excitation (structured illumination) or by stochastic activation of the individual fluorophores. In the stochastic activation approach, a small random subset of fluorophores is activated for each acquisition frame. For the conventional methods, the activated subset must be sufficiently small to ensure that the majority of the activated emitters are separated by distances larger than the diffraction limit. In this case each acquired frame consists of several well-separated (non-overlapping) PSFs. The individual fluorophores can be localised by, for example, fitting each PSF with a Gaussian function. Given enough detected photons, the localisation precision can be significantly higher than the resolution limit. The activation-acquisition cycle is typically repeated for several thousands acquisition frames. Super-resolution fluorescent images are produced by visualisation all the estimated fluorophores locations. 

Methods based on this simple strategy are called by a general term ``localisation microscopy'' (LM). Various names for LM have been proposed: ``Photo-Activation Localisation Microscopy'' (PALM) \cite{Betzig2006}, ``fluorescence PALM'' (fPALM) \cite{Hess2006} or ``Stochastic Optical Reconstruction Microscopy'' (STORM) \cite{Rust2006}.  

``Super-resolution Optical Fluctuation Imaging'' (SOFI) \cite{Dertinger2009} is a LM related technique. SOFI is based on higher-order statistical analysis of temporal intensity fluctuations caused by blinking behaviour of the fluorophores.

\cut{In \autoref{ch:NMF}, we discuss a new localisation microscopy method, which can deal with highly overlapping blinking sources. The individual sources can be ``separated'' computationally post acquisition and localised in the same manner as in the conventional LM techniques. The method allows (but is not limited to) using QDs as exceptionally bright and photo-stable fluorescence labels.}

The structured illumination based methods modulate the fluorescence behaviour of the molecules within the diffraction-limited area. The nearby molecules are driven to either ON or OFF states, which facilitates their discrimination.  These methods include ``Saturated Structured Illumination microscopy'' (SSIM) \cite{Gustafsson2000,Heintzmann2002} and ``STimulated Emission Depletion microscopy'' (STED) \cite{Hell1994}. 

\cut{The last chapter of this thesis (\autoref{ch:LSSIM}) introduces a combination of ``(linear) structured illumination microscopy'' with line scanning approach. This combination allows high resolution imaging in relatively thick fluorescent samples, which are challenging for most super-resolution techniques. The ``linear'' stands for the fact that the response of the fluorophores remains proportional to the excitation light. In this case the resolution remains limited, but can be increased by up to a factor of two. }


%==========================================
%==========================================
\clearpage
\section{Overview of the thesis}

In \autoref{ch:NMF} we discuss a new approach to localisation microscopy. Application of the machine learning technique non-negative matrix factorisation (NMF) enables us to computationally separate images of individual blinking fluorophores despite their high mutual overlap in the recorded data. We show that with this approach we can use quantum dots (QDs) as extremely bright and stable fluorescent labels for super-resolution microscopy. \cut{QDs are extremely bright and stable fluorophores, and the single QDs can be recorded using a standard CCD camera even without electron-multiplying (EM) enhancement. Data acquisition is therefore simple and does not require any specialised hardware. It can be performed on any wide-field epi-fluorescence microscope equipped with a camera, which makes this super-resolution method very accessible to a wide scientific community. }

\Autoref{ch:Theoretical-limits-of the LM} discusses the resolution criterion for noisy and pixelated data in terms of so-called fundamental resolution measure (FREM). We show that intermittency of the sources' brightness (blinking) can be beneficial and provide higher resolution when compared to sources with static intensity. 

In \autoref{ch:LSSIM} we present structured illumination microscopy (SIM) combined with line scanning (LS). Our goal is to introduce the SIM into more realistic and biologically relevant settings. Line scanning reduces the out-of-focus fluorescence background, which improves the quality of the illumination pattern. The method enables resolution improvement in relatively thick and densely labelled fluorescent samples. The reconstructed images reveal high details of the specimen's inner structure, and suffer less from the artefacts when compared to the conventional SIM methods.  

\Autoref{ch:Summary} gives a short overview of possible extensions of the current work and contains the final summary of the thesis. 

% This is copied from the first year report: 
%Recently, several super-resolution fluorescence microscopy methods (providing resolution beyond the classical resolution limit) have been developed. Microscopic images of fluorescently labelled samples with sub-resolution details have been obtained by using wide range of ideas and strategies. STED (Stimulated Emission Depletion) microscopy \citep{Westphal2003} reduces the individual excitation focal spots through non-linear interaction between two laser pulses of particular spatio-temporal profiles. This excitation spot is swept across the sample (scanning microscopy) and fluorescence signal from sub-resolution volume is registered at each position. Reducing the size of the PSF must be compensated with smaller displacement of the excitation spot and requires then higher acquisition time. SIM (Structured illumination microscopy) \citep{Gustafsson2000,Heintzmann2002} can be regarded as highly parallel form of scanning microscopy. The sample is illuminated with spatially periodic patterns which enables to encode high spatial Fourier frequencies into the detected image. Several `scans' of the pattern across the sample enable us to extract this information and the region of the observed Fourier frequencies can be enlarged. This in turn leads to super resolution in the real space. $\mathrm{I^{n}M}$ \citep{Gustafsson1999} and 4pi microscopy \citep{HellStelzer1992} introduce interference of excitation and (or) emission light and improve resolution mainly along the axial direction. A different approach is used in single-molecule localisation microscopy (LM) \citep{Betzig1995}. This technique takes advantage of the fact that a single isolated point source can be localised with high precision. A Gaussian approximation (mean $\mu$, variance $s^{2}$) of the point spread function (PSF - an ideal image of a point source) can be fit to the recorded image of a point source and its position ($\mathbf{\hat{\mu}}$) can then be estimated with precision $\mathbf{\hat{\sigma}}_{xy}\approx s/\sqrt{N}$ where $N$ is the number of detected photons \citep{Hess2006}. It is possible to detect more than $10^{4}$ photons from a single fluorophore and thus the achievable localisation precision can be better than $10\,\mathrm{nm}$ \citep{Churchman2005}. The most successful in terms of number of biological publications has been techniques based on (f)PALM/STORM principles: \textbf{PALM} (\emph{Photo-Activated Localisation Microscopy}) \citep{Betzig2006}, \textbf{fPALM} (\emph{Fluorescence PALM}) \citep{Hess2006} and \textbf{STORM} (\emph{Stochastic Optical Reconstruction Microscopy}) \citep{Rust2006_STORM} are three different names for almost identical technique (differences being the type of fluorophore used and fitting algorithms). (f)PAM/STORM methods use the photo-activation property of several fluorescent dyes (or pairs of dyes, STORM) or photo-activable fluorescence proteins ((f)PALM). Under normal conditions these molecules are in a non-fluorescent (`OFF') state. However, after an exposure to the light with $\lambda_{\mathrm{act}}$ wavelength they can be activated into a fluorescent (`ON') state. In this state they can be excited by light with a different wavelength $\lambda_{\mathrm{exc}}$ to emit fluorescent light with a peak at $\lambda_{\mathrm{em}}$ (typically $\lambda_{\mathrm{act}}<\lambda_{\mathrm{exc}}<\lambda_{\mathrm{em}}$). Usually, the sample is continuously exposed to the excitation light $\lambda_{\mathrm{exc}}$ and the activation light $\lambda_{\mathrm{act}}$ is applied in discrete pulses. After each activating pulse $\lambda_{\mathrm{act}}$ a small random subset of fluorophores is activated. Due to the continuous excitation $\lambda_{\mathrm{exc}}$ it emits fluorescence light. If the activated subset is small enough, the individual fluorophores randomly scattered across the sample are well separated (their PSFs do not overlap) and can be localised by fitting the individual PSFs into the pattern. After a certain time of emitting the fluorophore turns back into the dark (OFF) state again. This can be either reversible (`switching', STORM) or irreversible (`bleaching', (f)PALM). An extension of the STORM to three dimensions has been shown in \citep{Huang2008}. A cylindrical lens was inserted in the imaging system introducing astigmatism. Out-of-focus PSF becomes asymmetric (elongated along one lateral direction). The z-dimension is thus encoded in the asymmetry of the PSF when shifting along the axial (z) direction. The resolution improvement down to 20~nm (corresponding to $\lambda_{\mathrm{em}}/30$) was demonstrated \citep{Betzig2006,Rust2006_STORM}. However, a long acquisition time (2-12 hours in the original publications \citep{Betzig2006}) required to collect sufficient density of the fluorophores is the main drawback of these methods even though the required acquisition time has been shortened by several orders of magnitude recently (in 2008 PALM images with $\sim60\,\mathrm{nm}$ resolution ($\lambda_{\mathrm{em}}/8$) with a frame rate 25~s was reported in studies of slow cellular motion dynamics \citep{Shroff2008LiveFPALM}). A fast algorithm for estimation of the position of a large number of individual molecules ($10^{5}$ combined fits and Cramer-Rao lower bound on parameter precision calculation per second) has been presented \citep{Smith2010}. The algorithm is implemented on graphical processing unit and can speed up the post-acquisition analysis of the images by a factor of 10-100 when compared to a modern central processing unit. The algorithm has been shown to reach Cramer-Rao lower bounds on a parameter estimation. Another fast and robust algorithm for data processing consisting of noise reduction, detection of likely fluorophore position, high precision localisation and subsequent visualisation of found fluorophores position has been proposed in \citep{Wolter2010}. fPALM/STORM based techniques can only deal with non-overlapping sources. Separation of individual emitters with overlapping PSFs can significantly speed the acquisition procedure (scaling inversely with the number of molecules localised per bright spot \citep{Small2009}). In 2005 there has been published a method exploiting the fluorescence intermittency (`blinking') of quantum dots under continuous excitation \citep{Lidke2005}. A time series of the blinking quantum dots was recorded and analysed using \emph{Independent Component Analysis} (ICA). The fastICA algorithm \citep{Hyvarinen2000} has been used. As the blinking of the individual quantum dots is independent with respect to each other, individual emitters even with overlapping PSFs can be separated. Localisation of two quantum dots separated down to $23$~nm (corresponding to $\lambda_{\mathrm{em}}/30$) has been reported \citep{Lidke2005}. Further exploration of the technique for more than two sources and for different configuration of the experiment can be found in \citep{Lidke2007}. A Bayesian approach to the blinking of the individual fluorophores has been presented in poster in 2008 \citep{Harrington2008}. A localisation of several quantum dots within the diffracted limited volume has been shown. A method for measurement of sub-resolution distances using the quantum dots has been published in \citep{Lagerholm2006}. However, discrete ON-OFF blinking is required (only one source being ON and others OFF) as opposed to \citep{Lidke2005,Harrington2008} where only fluctuation of the individual sources is required. \textbf{SOFI} (\emph{Super-resolution Optical Fluctuation Imaging}), which was published in 2009 \citep{Dertinger2009} relies on higher-order statistical analysis of temporal intensity fluctuations (caused by blinking behaviour of the quantum dots) recorded in a sequence of images. It has been shown that the \emph{n}th order cumulant function (over time dimension) of the recorded sequence is composed of individual emitters with \emph{n}th power of their intensity images (PSFs). This yields the resolution enhancement by a factor $\sqrt{n}$ (for Gaussian approximation of PSF). Resolution improvement down to 60~nm ($\lambda_{\mathrm{em}}/10$) has been reported. In contrast to ICA based methods no prior knowledge about the number of sources is needed. Although there is no fundamental limit for resolution enhancement there are practical ones. Because the PSF is raised to the \emph{n}th power so is the molecular brightness (intensity) of each emitter. Thus the emitter that has intensity two times higher will appear $2^{n}$ times brighter in the \emph{n}th-order SOFI image. Additionally the brightness of each emitter in the SOFI image will be altered by its specific blinking behaviour. An emitter that doesn't fluctuate over the recorded time will not appear in the SOFI image at all.
%

