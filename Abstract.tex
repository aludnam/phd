\documentclass[11pt]{amsart}
\usepackage{geometry}                % See geometry.pdf to learn the layout options. There are lots.
\geometry{letterpaper}                   % ... or a4paper or a5paper or ... 
%\geometry{landscape}                % Activate for for rotated page geometry
%\usepackage[parfill]{parskip}    % Activate to begin paragraphs with an empty line rather than an indent
\usepackage{graphicx}
\usepackage{amssymb}
\usepackage{epstopdf}
\DeclareGraphicsRule{.tif}{png}{.png}{`convert #1 `dirname #1`/`basename #1 .tif`.png}

\title{Abstract}

\author{}
\date{\today}                                           % Activate to display a given date or no date

\begin{document}
\maketitle
Title sugesstions (there should be mentined that it is for fluorescence microcopy and that the aim is to get sub-diffraction/super resolution): 
\begin{enumerate}
\item
Super-resolution fluorescence microscopy
\item
Super-resolution techniques in fluorescence microscopy
\item
Methods for optical microscopy with sub-diffraction resolution.
\item 
...?
\end{enumerate}
\vspace{1cm}

Fluorescence microcopy is an important tool for biological research. In this thesis we discuss two methods for super-resolution imaging of fluorescent samples. 

The first method represents an extension of localisation microcopy. We used non-negative matrix factorisation (NMF) to model a noisy dataset of highly overlapping fluorophores with intermittent intensities. We can recover images of individual sources from the optimised model, despite their high mutual overlap in the original dataset. This allows us to consider blinking quantum dots as bright and stable fluorophores for localisation microscopy. Moreover, NMF allows recovering individually different sources. Such situation can arrive when the sources are located in different focal planes, for example. We discuss the practical aspects of application of NMF to real dataset and show super-resolution images of biological samples labelled with quantum dots.

The second optical microcopy method we discuss in this theis is a member of the growing family of structured illumination techniques. Our main goal is to apply the structured illumination to thick fluorescent samples generating large out-of-focus backgfound. The out-of-focus fluorescence background deteriorates the illumination pattern and the reconstructed images suffer from influence of noise. We present a combination of structured illumination microscopy with line scanning. This technique reduces the out-of-focus fluorescence background, which improves the modulation and the quality of the illumination pattern and therefore facilitates the reconstruction. We present super-resolution, optically sectioned images of a thick fluorescent sample, revealing details of the specimen�s inner structure. 

We also discuss a theoretical resolution limit in noisy and pixellated data. We corrected published expression of so called fundamental resolution measure (FREM) a derive FREM for two fluorophores with intermittent intensity. We show that the fluorescent blinking, such as can be observed for quantum dots, can increase the resolution defined in terms of FREM.

\end{document}  