%!TEX root =  thesis.tex
\chapter{Non-negative matrix factorisation for localisation microscopy}
%==========================================
%==========================================
\clearpage
{\singlespace
\section*{Structure}
\begin{enumerate}
	\item
	Intro LM, overlapping sources and NMF. Main competitors 3B \cite{Cox2011}, CSSTORM \cite{Zhu2012} and SOFI \cite{Dertinger2009}.	
	\item
	Main points for using NMF in LM: 
	\begin{itemize}
		\item
		`Suitable' model for QDs. 
		\item
		Takes reappearance of the sources into account (like 3B, unlike CSSTORM).
		\item
		Can deal with high density of overlapping sources. Does it do better? QDs are too dense to deal with CSSTORM (max $12\unit{\mu m^{-2}}$).
		\item 
		Can deal with different PSF - no other method can do this!	
		\item
		Show link of Lee's NMF algorithm to Richardson Lucy deconvolution (2nd year review) (done). 
	\end{itemize}�
	\item
	NMF
	\begin{itemize}
		\item 
		Number of sources estimation - second year report. 
		\item
		Local optima.
		\item
		Iterative partial restarts � address local optima + weak indirect enhancing sparsity (discuss Hoyer's sparse NMF \cite{Hoyer2004} \texttt{S433}). 
		Number of sources $K$ can be overestimated but $K$ estimation needs  a classifier which determines whether the source is good or not. 
		Show linear classifier performance. 	
	\end{itemize}	
	\item
	In-focus PSF - dealing with high densities $>10\unit{um^{-2}}$.
	\begin{itemize}
		\item
		Comparison of simulated parameters in: \\ \texttt{DTC/LMsimulationNotes/LMsimulationNotes.tex}
		\item
		Simulated data: Estimated density, localisation error, average precision (\texttt{S422}). 
		Comparison to 3B and CSSTORM software (\texttt{S421, S427, S428}).
		\item 
		Real data - in focus tubulin stained QDs (\texttt{S364}). Show how 3B and CSSTORM perform on these.
		\item
		Show Hoyer's sparse NMF for in focus PSF with the sparsity set to right level.
	\end{itemize}
	\item
	Different individual PSFs - no other method can do this!
	\begin{itemize}
		\item
		Real data - out of focus QDs. (\texttt{S392})
		\item 
		Astigmatic data - for 3D localisation \cite{Huang2008}.
	\end{itemize}
\end{enumerate}

\clearpage

%==========================================
%==========================================

\section*{Working plan}
\begin{enumerate}
	\item
	Introduction from the the previous paper drafts.
	\item
	Discussion of Hoyer's sparse NMF.  Show on simulated in focus PSF - all have the same sparsity. (Simulated data with in-focus PSF randomly scattered \texttt{S433}, with different PSFs \texttt{S401}. Real data with Hoyer in \texttt{S402}). (done) Berry an Appendix A from the first year review. 
	\item
	Discussion of ICA. Show ICA failure (done)
	\item
	Software issues.
	\begin{itemize}
		\item
		Make 3B software work! ugh. (done) - simulations with `telegraph process'. (done)
		\item
		Turn CSSTORM output to locations. Adjust scanning window (or simulations) such that it fits... (done)
		\item
		Multiple sources: detection with ``non maximum suppression?'' OR  ``linear classifier?''. Splitting OR localising with PALM software? 
		
		Show AP for individual frames of CSSTORM and for the mean projection. (done)
	\end{itemize}
	\item
	Simulations - compare to CSSTORM and 3B. 
	\begin{itemize}
		\item
		Randomly scattered PSFs with different densities (range $0-50 \unit{um^{-2}}$). Evaluate AP, estimated density, localisation error. Show limits of the method. (Density, intensity, blinking behaviour, number of frames.) (done)
		\item
		Double cross - different linear densities. 
		\item
		Subsampling the the simulations in time and see the performance: more and less bright/less and brighter - what is better?  (!)
	\end{itemize}
	\item
	Real data - tubulin data. 3B and CSSTORM on these data.
	\item
	Get astigmatic data?
\end{enumerate}
\clearpage
}

%==========================================
%==========================================

\section{Introduction}

\subsection{Localisation microscopy}

Localisation microscopy (LM) is a conceptually simple and accessible technique for super-resolution imaging of fluorescent samples. LM takes as input a stack of images containing a number of fluorescent sources with time-varying intensity and identify the locations and point-spread functions (PSFs) of these sources. If the sources are attached to structures of interest (e.g.\ in biological samples), then this provides useful information about the target structures. By exploiting multiple images of time-varying sources, LM can achieve a resolution beyond the classical diffraction limit of $\sim \lambda/2$ \cite{Abbe1873}, where $\lambda$ is the wavelength of the light source. Provided enough photons are collected, the localisation of an individual source can be an order of magnitude below $\lambda/2$ \cite{Ober2004}, meaning that sources whose point spread functions overlap heavily can be resolved. 
%See \autoref{fig:Iterative restarts}, \autoref{fig:Real-data-QDrandom} and \autoref{fig:Real-data-patch-B24} for examples.

LM techniques are based on sources with transition between bright (ON) and dark (OFF) states. Fluorescent proteins or organic dyes are used as fluorophores in the standard techniques (fPALM \cite{Hess2006}, STORM \cite{Rust2006}). The density of the ON sources in each captured frame can be controlled by photo-switching and must be optimised experimentally. High density of ON fluorophores results in overlapping sources and complicates localisation (overlapping sources are usually discarded), whereas low density leads to a excessive total acquisition time. Several thousands frames are typically required for an image reconstruction.

%==========================================

\subsection{Quantum dots for localisation microscopy}

In last years there has been interest in using quantum dots (QDs) as sources for localisation microscopy. QDs are an order of magnitude brighter compared to the organic dyes or fluorescent proteins used in conventional LM \cite{Resch-Genger2008}. Under continuous excitation QDs exhibit a stochastic blinking between ON and OFF states. Excellent photo-stability, low cyto-toxicity and distinctive spectral properties make QDs very attractive for biological research. However, the stochastic blinking of QDs is impractical for standard LM techniques because the rate of switching, and hence the density of ON sources, is difficult to control. Thus QD-labeled data typically consist of highly overlapping sources which cannot be localised with standard techniques.

%==========================================

\subsection{Overlapping sources � related work}

Several techniques dealing with overlapping sources have been proposed recently. Most of these methods model the LM data from known image of a single source, so called point spread function (PSF). Most often a single PSF is assumed to be shared by all sources in the dataset.

There are two main groups of the algorithms addressing the overlapping sources in the LM data. 

The first group operates separately on each frame of the LM dataset: a method proposed in \cite{Huang2011} tries to fit multiple PSFs into each frame of the dataset while \cite{Zhu2012} make use of compressed sampling to recover the sparse vector representing the fluorophores distribution.  DAOSTORM algorithm \cite{Holden2011} applies iterative fitting and subtracting procedure in each frame. These methods ignores the reappearance of the sources and can generally deal with moderately overlapping sources with densities $<10\unit{sources/\mu m^{2}}$ \cite{Huang2011,Holden2011,Zhu2012}.

The second group of the algorithms models LM dataset as a collection of blinking sources. They can improve the localisation for higher densities of the overlapping sources by taking the the reappearance of fluorophores into account. However, these algorithms are in general computationally more expensive. Modelling the whole dataset from known PSF with maximum aposterioiry (MAP) fitting has been proposed in \cite{Harrington2008}. 

Bayesian analysis of the intermittent sources (Bayesian Blinking and Bleaching (3B) analysis) has been proposed in \cite{Cox2011}. The blinking behaviour of the fluorophores is modelled as a hidden markov model with three distinct states: emitting, not-emitting and bleached. Each source is described by position, size of the PSF and intensity. MAP estimates of the positions obtained from different sampling of state sequences are used as an estimated locations of the fluorophores. While the 3B analysis adjust the width of the PSF (gaussian approximation of the PSF \cite{Zhang2007}), it cannot deal with PSFs of different shapes. This situation can arise, for example, in three dimensional samples, where the overlapping sources can be located in different focal planes (see \autoref{fig:Simulted-PSF-different-focal-depths}). Moreover, the 3B analysis assumes a mono-exponential decay of the fluorescence for the individual sources. QDs have a complex blinking behaviour with power-law distribution in the histogram of on and off times \cite{Shimizu2001}. This can complicate the 3B analysis of the QD data. 

Yet another approach to the LM data with overlapping sources problem has been proposed in a method called SOFI (Superresolution Optical Fluctuation Imaging) \cite{Dertinger2010b}. Instead of separating the individual sources, SOFI analyses the higher order statistic of the intensity fluctuation. The intensity values in the SOFI image, however, reflect the fluctuation behaviour, rather than the strength of the emitters. Sources which do not blink will not appear in the SOFI image. This issue has been addressed recently by bSOFI (balanced SOFI) \cite{Geissbuehler2012}.
% The balanced SOFI paper to read!!

\begin{figure}[!htb]
	\newcommand{\widthfig}{.95\textwidth}
	\newcommand{\barspace}{-.5cm}
	\centering
	\condcomment{\boolean{includefigs}}{ 
	\begin{tabular}{l}
		\includegraphics[width=\widthfig]{\qd S393/images/psf_outOfFocus}\vspace{\barspace}\tabularnewline		
		\includegraphics[width=\widthfig]{\qd S393/images/psf_outOfFocus_intBars}
	\end{tabular}
	}	
	\caption{Simulated PSF in different depths of focus. The number in each figure indicates the distance in $\unit{\mu m}$ from the in-focus plane. The maximum intensity relative to the in-focus PSF is indicated in the bars below and corresponds to about 10\% at $1\unit{\mu m}$ and 3\% at $1.5\unit{\mu m}$.}
	\label{fig:Simulted-PSF-different-focal-depths}
\end{figure}

%==========================================
%==========================================

\clearpage
\section{Non-negative matrix factorisation for QD data}
We propose non-negative matrix factorisation (NMF) \cite{Lee1999} as a natural model for LM data with QD labeled samples. NMF can separate the individual highly overlapping sources with individual different shapes. This situation that can arise in a real microscopic data when aberrations are present or when the sources are not in one focal plane (\autoref{fig:Simulted-PSF-different-focal-depths}). Moreover we use NMF algorithm which takes the Poisson noise in the recorded images into account. This allows us to recover the individual sources from noisy QD data.

%==========================================

\subsection{NMF introduction}
Non-negative matrix factorisation solves the approximative factorisation of an $N\times T$ data matrix $\bm{D}$ with non-negative entries:
%
\begin{equation}
	\bm{D}\approx\bm{WH},
	\label{eq:NMF approx}
\end{equation}
%
where $\bm{W}$ and $\bm{H}$ are $N\times K$ and $K\times T$  matrices ($K<N,T$), respectively. The factorisation is constraint to $\bm{W}$ and $\bm{H}$ with non-negative entries. 

Initial factorisation algorithms (so called positive matrix factorisation) \cite{Paatero1994} were published in 1994. However, it was in 1999 when NMF attracted attention of researchers after publication of the \emph{Nature} article by Daniel Lee and Sebastian Seung \cite{Lee1999}. NMF was presented as an efficient and powerful method for approximation of non-negative data (in their case a database of facial images) by linear combination of non-negative localised basis vectors (images of the nose, mouth, ears, eyes, etc.) An individual face from the data-set can be recovered as a non-subtractive composition of individual basis vectors. Lee and Seung proposed simple multiplicative updates \cite{Lee2001} for $\bm{W}$ and $\bm{H}$. 

%==========================================

\subsection{NMF as a natural model for QD data\label{sec: NMF}}
Non-negative matrix factorisation (NMF) \cite{Lee1999,Lee2001} is a natural model for QD data. NMF decomposes a movie of the blinking QDs into spatial and temporal parts, i.e.,\ time independent emission profiles of the individual sources and fluctuating intensities of each source, respectively. NMF imposes non-negativity constraints on both the spatial and the temporal components which are natural constraints for the source profiles and intensities of blinking QDs.

Consider a $N\times T$ data matrix $\bm{D}$, where $N$ is the number of pixels in each frame, and $T$ is the number of time frames. The columns $\bm{D}$ are the individual frames of the movie reshaped into $N\times 1$ vector by concatenating the columns the image. All entries in $\bm{D}$ are non-negative, i.e.,\ $d_{xt}\geq 0$. Under the NMF model, the expectation value of the $\bm{D}$ is assumed to be decomposed into a $N\times K$ spatial component matrix $\bm{W}$ (images of the $K$ individual sources) and the $K\times T$ temporal component matrix $\bm{H}$ (the intensities of the sources):
%
\begin{equation}
	\mathbb{E}\left[\bm{D}\right]=\bm{WH};\;w_{xk},\, h_{kt}\geq0
	\label{eq:NMF model}
\end{equation}
%
or in element-wise form
%
\begin{equation}
	\mathbb{E}\left[d_{xk}\right]=\sum_{k=1}^{K}w_{xk}h_{kt};\;w_{xk},\, h_{kt}\geq0
	\label{eq:NMF model element-wise}
\end{equation}

The predominant noise model in microscopy imaging is Poisson noise \cite{PawleyHandbook2006}. Therefore the log-likelihood function can be expressed as
%
\begin{equation}
	\log p(\bm{D}|\bm{W},\bm{H})=\sum_{xt}\left(d_{xt}\log\sum_{k=1}^{K}w_{xk}h_{kt}-\sum_{k=1}^{K}w_{xk}h_{kt}\right)+C_{1},
	\label{eq:NMF model likelihood}
\end{equation}
%
where $C_{1}$ is independent of $\bm{W}$and $\bm{H}$. 

The Lee and Seung NMF algorithm \cite{Lee2001} minimises the divergence between the data and the NMF model
%
\begin{equation}
	\mbox{KL}(\bm{D}\parallel\bm{WH})=-\sum_{xt}\left(d_{xt}\log\sum_{k=1}^{K}w_{xk}h_{kt}-\sum_{k=1}^{K}w_{xk}h_{kt}\right)+C_{2}
	\label{eq:KL divergence}
\end{equation}
%
where $C_{2}$ is independent of $\bm{W}$and $\bm{H}$. Comparison with the log-likelihood \autoref{eq:NMF model likelihood} shows that the minimum of the divergence with positivity constrains on $\bm{W}$ and $\bm{H}$ is equivalent to the maximum of the log-likelihood. 

There is a scaling indeterminacy between $\bm{W}$ and $\bm{H}$ in the NMF model. We fix this by setting the $L_1$ norm of each column of $\bm{W}$ to 1. The background fluorescence in the images is modelled as a `flat' component $\bm{w}_{K} = \bfone/N$ with corresponding intensity $\bm{h}_{K}$. The spatial part $\bm{w}_{K}$ is not updated during the algorithm. 

The NMF model is fitted to data iteratively using multiplicative updates for $\bm{H}$ and $\bm{W}$ sequentially \cite{Lee2001}. Note that the divergence (\autoref{eq:KL divergence}) is convex wrt $\bm{H}$ and $\bm{W}$ individually, but not in both variables together \cite{Lee2001}, leading to local optima. Sufficient conditions for uniqueness of solutions to the NMF problem has been studied in \cite{Donoho2004}. 

%==========================================
%==========================================

\clearpage
\section{NMF � related work\label{sec:Related}}

Various alternative minimisation strategies have been explored in an effort to speed up the convergence properties of the Lee \& Seung updates. A comprehensive discussion on the variety of these algorithms can be found in \cite{Berry2007}. 

Additional constraints can be imposed on $\bm{W}$ and $\bm{H}$ matrices. For example, imposing a defined ``sparsity'' on either columns of $W$ or rows of $H$ has been proposed in \cite{Hoyer2004} and is discussed below. Enforcing the temporal smoothness of $H$ in the analysis of EEG recordings has been published in \cite{Chen2005}. Multiplicative updates for various constraints have been proposed in \cite{Chen2005,Pauca2006}.

A generative version of NMF (Gamma-Poisson model) is shown in \cite{Canny2004} as a probabilistic model for a document corpus.

%==========================================

\subsection{Hoyer's sparse NMF \label{sub:Hoyer}}
NMF with explicit sparsity constraints has been developed by Hoyer  \cite{Hoyer2004}. Specific constraints on ``sparsity'', defined in  \cite{Hoyer2004}, of the columns of $\bm{W}$ can be imposed during the optimisation. 

All the columns $\bm{w}_{k}$s of estimated matrix $\bm{W}$ have an identical ``sparseness'' which might be restrictive when out-of-focus PSFs are present. For example, the in focus PSF in \autoref{fig:Simulted-PSF-different-focal-depths} has Hoyer's sparsity $s=0.83$ while the PSF from $1 \unit{\mu m}$ out-of-focus plane has $s=0.4$ and the PSF from $1.8 \unit{\mu m}$ out-of-focus plane  has $s=0.1$.
%
\begin{figure}[!htb] %copied from S433_report.tex
	\newcommand{\sizefig}{.9}
	\centering
	\subfloat[$10 \unit{sources/\mu m^{2}}$]{	
	\includegraphics[scale=\sizefig]{\qd S429/figures/resN_simiter1_1to14}}\\
%	\subfloat[$20 \unit{sources/\mu m^{2}}$]{	
%	\includegraphics[scale=\sizefig]{\qd S430/figures/resN_simiter1_1to14}}\\
	\subfloat[$30 \unit{sources/\mu m^{2}}$]{	
	\includegraphics[scale=\sizefig]{\qd S431/figures/resN_simiter1_1to14}}\\
%	\subfloat[$40 \unit{sources/\mu m^{2}}$]{	
%	\includegraphics[scale=\sizefig]{\qd S432/figures/resN_simiter1_1to14}}\\
	\subfloat[$50 \unit{sources/\mu m^{2}}$]{	
	\includegraphics[scale=\sizefig]{\qd S433/figures/resN_simiter1_1to14}}\\	
	\caption{No sparsity constraint on $\bm{W}$ estimated with Hoyer's algorithm from simulated data of randomly scattered sources with different density. The density is shown under each subfigure. Shown first 14 estimated components.}
	\label{fig: Hoyer no sparsity constraint}
\end{figure}

\begin{figure}[htbp!] %copied from S433_report.tex
	\newcommand{\sizefig}{.9}
	\centering
	\subfloat[$10 \unit{sources/\mu m^{2}}$]{	
	\includegraphics[scale=\sizefig]{\qd S429/figures/res07_simiter1_1to14}}\\
%	\subfloat[$20 \unit{sources/\mu m^{2}}$]{	
%	\includegraphics[scale=\sizefig]{\qd S430/figures/res07_simiter1_1to14}}\\
	\subfloat[$30 \unit{sources/\mu m^{2}}$]{	
	\includegraphics[scale=\sizefig]{\qd S431/figures/res07_simiter1_1to14}}\\
%	\subfloat[$40 \unit{sources/\mu m^{2}}$]{	
%	\includegraphics[scale=\sizefig]{\qd S432/figures/res07_simiter1_1to14}}\\
	\subfloat[$50 \unit{sources/\mu m^{2}}$]{	
	\includegraphics[scale=\sizefig]{\qd S433/figures/res07_simiter1_1to14}}\\	
	\caption{Sparsity constraints $s=0.7$ on $\bm{W}$ estimated with Hoyer's algorithm from simulated data of randomly scattered sources with different density. The density is shown under each subfigure. Shown first 14 estimated components.}
	\label{fig: Hoyer sparsity 0.7}
\end{figure}

However, simulations suggest that even for data with all the sources in focus, and therefore with identical sparsity, the Hoyer's sparse NMF does not perform well for higher densities of the overlapping sources. \autoref{fig: Hoyer no sparsity constraint} shows estimated $\bm{W}$ with Hoyer's algorithm without sparsity constraints. for simulated data of randomly scattered sources with densities $10-50\,\unit{\mu m^{-2}}$. Note that most of the $\bm{w_{k}}$ for higher densities contain multiple sources \autoref{fig: Hoyer no sparsity constraint}b,c. Imposing the sparsity constraints $s=0.7$ on the columns of $\bm{W}$, estimated from the true PSF, gives better estimated sources for densities$<30\unit{\mu m^{-2}}$  \autoref{fig: Hoyer sparsity 0.7}a, however, for dense data the method fails to recover the individual sources and gives unsatisfactory results \autoref{fig: Hoyer sparsity 0.7}c.

Data were simulated for uniform random distribution of sources with blinking intensity uniformly distributed on the interval $[0, 5000]$ photons. The background was $100$ photons/pixel and data were corrupted with Poisson noise. Prior to the evaluation with the Hoyer's algorithm, the true background value was subtracted from the data (clipping negative pixels to zero). The number of component $K$ was set to correct (true) value. The algorithm was run for 1000 iterations. Running the algorithm for longer (2000, 5000) iterations did not improve the estimated results. 

%==========================================

\subsection{Independent component analysis}
\begin{figure}[!htb] % this figure is copied from ~/DTC/paper/NMFLM.tex
	\condcomment{\boolean{includefigs}}{
	\newcommand{\sizefig}{.4}
	\centering
	\subfloat[NMF (noise)]{
	\begin{tabular}{c}
		\includegraphics[scale=\sizefig]{\qd S310/images/nmf_noise_loc_c1} \tabularnewline
		\includegraphics[scale=\sizefig]{\qd S310/images/nmf_noise_loc_c2}\tabularnewline
	\end{tabular}}
	\subfloat[ICA (noise)]{
	\begin{tabular}{c}
		\includegraphics[scale=\sizefig]{\qd S310/images/ica_noise_loc_c1} \tabularnewline
		\includegraphics[scale=\sizefig]{\qd S310/images/ica_noise_loc_c2}\tabularnewline
	\end{tabular}}	
	\subfloat[ICA (noise free)]{
	\begin{tabular}{c}
		\includegraphics[scale=\sizefig]{\qd S310/images/ica_noNoise_loc_c1} \tabularnewline
		\includegraphics[scale=\sizefig]{\qd S310/images/ica_noNoise_loc_c2}\tabularnewline
	\end{tabular}}}
	\caption{Comparison of the components separated with NMF (a) and ICA (b) for	simulated noisy data of two blinking QDs separated by $d=50\unit{nm}$. ICA for noise free data (c). The true and the estimated positions are shown as blue circle and red crosses, respectively. The radius of the green circle is the resolution limit $\delta$. Blue pixels contain negative values.}
	\label{fig:Comparison of NMF and ICA}
\end{figure}

Independent component analysis (ICA) algorithm \cite{Hyvarinen2000} has been used for separating the overlapping QDs \cite{Lidke2005}. ICA allows each sources to have a different individual PSF, however, ICA model allows negative entries in the individual PSFs and does not account for noise in the measured data which can make recovery of the individual sources difficult in realistic noise levels \autoref{fig:Comparison of NMF and ICA}. 

\autoref{fig:Comparison of NMF and ICA} is a results from a simulated $10^{3}$ frames containing two sources with blinking intensity uniformly distributed on the interval $[0, 1500]$ photons and with background $100$ photons/pixel. The true background level was subtracted (clipping the negative values) prior to the ICA evaluation.

%==========================================

\subsection{Gamma - Poisson model \label{sub: GaP}}
The gamma-Poisson (GaP) model has been proposed \cite{Canny2004} as a probabilistic model for documents. It represents a generative model for NMF \autoref{eq:NMF model}. The entries $h_{kt}$ of the intensity matrix $\bm{H}$ in \autoref{eq:NMF model element-wise} are regarded as latent variables generated from a Gamma distribution with parameters $\alpha_{k}, \beta_{k}$ and the data are modelled as a Poisson variable with mean $\bm{WH}$. Variables $\theta = \{\bm{w_{k}},\alpha_{k}, \beta_{k}\}; k = 1..K$ are then parameters of the GaP model.

%==========================================

\subsection{Richardson-Lucy deconvolution}
There is a link between NMF and classical Richadson-Lucy deconvolution. 
%This is from DecovolutionNotes.lyx file.
An observed `blurred' (diffraction limited) image $\bm{i}$ ($N\times1$ vector) can expressed as a (discretised) convolution 
%
\begin{equation} 
	i_{x}=\sum_{j=1}^{N}o_{j}w_{x-j}, 
\end{equation} 
%
where $\bm{o}$ ($N\times1$) is the original (unblurred) object which represents locations and intensities of fluorescent sources. $\bm{w}$ ($N\times1$) is an image of point spread function (PSF) centred in the middle of the image. Richardson \cite{Richardson1972} and Lucy \cite{Lucy1974} published an iterative deconvolution technique for astronomical images with known PSF. They used Bayes theorem as a `hint' for an iterative update of $\bm{o}$. This update is usually referred to as Richardson-Lucy (RL) deconvolution algorithm and is identical to the Lee-Seung NMF update with generalised KL-divergence objective function \cite{Lee2001}. 

Holmes \cite{Holmes:92} derived the RL updates based on maximum likelihood estimation of the model with Poisson noise using the expectation-maximisation algorithm. He also proposed an update for $\bm{w}$ so that the method can be used as a blind deconvolution algorithm (PSF is not known). This is sometimes referred to as a `blind RL algorithm'. The updates for $\bm{o}$ and $\bm{w}$ are technically identical to the Lee and Seung NMF updates (KL divergence as an objective function). 

However, (blind) RL deconvolution solves a different problem then NMF. RL deconvolution estimates one PSF ($\bm{w}$) which is shared by all sources. The deconvolution is performed for each frame separately, independent on the rest of the dataset. NMF models the whole dataset as a collection of individual (and in general different) PSFs (columns of $\bm{W}$) each changing intensity over time (rows of $\bm{H}$). While one source which appears in $n$ different frames is treated as $n$ different individual sources by RL, NMF can identify it as a single source. 

Modified updates imposing radial symmetry constraints on PSF were also proposed. There exist several modified updates derived using EM algorithm which impose some constraints on $\bm{o}$ or $\bm{w}$. \cite{Joshi:93} gives updates where Good's roughness measure ($\int\frac{\left|\nabla f(x)\right|^{2}}{f(x)}dx$) on the original image $\bm{o}$ is used as a regularisation term. This biases the solution towards the `smooth' images and avoids speckle artefacts in the reconstructions that are sometimes experienced in deconvolution methods. 

\cite{Fish:95} use RL `blind' algorithm (updates on both $\bm{o}$ and $\bm{w}$) but after some number of iterations they fit some approximation of the PSF to the estimated $\bm{w}$ and uses this fit as a new $\bm{w}$. They claim that in noisy images this `semi-blind' deconvolution can perform better than the one with known PSF. The comparison of the regularised RL versions and some other deconvolution techniques has been shown in \cite{Kempen1997BA,Verveer1999}. RL usually performs well for noisy images.

%==========================================
%==========================================

\clearpage
\section{NMF for realistic microscopy datasets \label{sec:NMF-for-real}}

NMF model fitting for blinking quantum dots is challenging when applied to a dataset of around $1,000$ large images (each containing $\sim 10^{5}$ pixels and $\sim10^{3}-10^{4}$ QDs). Beside an excessively large computational time, the local minima in NMF fitting complicate the optimisation \cite{Kim2008}. We address this partly by dividing the data into overlapping patches, so that NMF is applied to each of patch individually, and then these results are ``stitched'' back together. We also develop further methods to reduce local optima problems in the fitting procedure. 

A pipe-line for NMF evaluation of a realistic microscopy datasets is described in this section.


%==========================================

\subsection{Pre-processing \label{sec:preproc}}

Raw data are calibrated such that the image intensity corresponds to photon counts. Each image is divided into patches of size $m\times m$ pixels with $o$ pixels overlap. We usually use $m = 25$ and $o = 5$. The overlap has been chosen as the estimated extent of a single in-focus point spread function. Each patch $p$ is reshaped into a vector by concatenation of columns, so that a patch in all $T$ frames creates a $m^{2}\times T$ data matrix $\bm{D}_{p}$.

To detect patches with low signal, the maximum intensity pixel in the time average of each patch $m_{p}=\max_{i}\left\langle \bm{D}_{p}(i,t)\right\rangle _{t}$ is compared to the maximum intensity pixel of the average of the whole data $m=\max_{i}\left\langle \bm{D}(i,t)\right\rangle _{t}$. The patches with $m_{p}/m< \theta_{m}$ are considered to contain only weak signal and are not considered for further evaluation. In our dataset se set $\theta_{m}=0.25.$

%==========================================

\subsection{Estimation of number of sources $K$\label{sub:Estimation-of-number-of-sources}}

The NMF model \autoref{eq:NMF model element-wise} requires prior knowledge about the number of sources $K$ to be separated. This is a difficult task for noisy datasets. In preliminary work we explored this on simulated data with models fitted for a range of $K$ values. 

\begin{figure}[!h]
	\centering
	\newcommand{\sizee}{.25}		
	\newcommand{\sizebb}{.6}
	\newcommand{\ima}{$2\delta$} 
	\newcommand{\imb}{$1.5\delta$}
	\newcommand{\imc}{$\delta$}
	\newcommand{\pca}{, PCA}
	\newcommand{\data}{, data}
	\newcommand{\lbd}{, lower bound}
	\newcommand{\mxc}{, max correlation}
	
	\subfloat[\ima \data]{
	\includegraphics[scale=\sizebb]{\qd S300/images/reslocalized_nc11_sc20}}
	\subfloat[\imb \data]{
	\includegraphics[scale=\sizebb]{\qd S300/images/reslocalized_nc11_sc15}}
	\subfloat[\imc \data]{
	\includegraphics[scale=\sizebb]{\qd S300/images/reslocalized_nc11_sc10}}\\

	\subfloat[\ima \pca]{
	\includegraphics[scale=\sizee]{\qd S301/images/pca_nc20}}
	\subfloat[\imb \pca]{
	\includegraphics[scale=\sizee]{\qd S301/images/pca_nc15}}
	\subfloat[\imc \pca]{
	\includegraphics[scale=\sizee]{\qd S301/images/pca_nc10}}\\
%	
%	\subfloat[\ima \llk]{
%	\includegraphics[scale=\sizee]{\qd S303/images/LogLikPoisson_nc20}}
%	\subfloat[\imb \llk]{
%	\includegraphics[scale=\sizee]{\qd S303/images/LogLikPoisson_nc15}}
%	\subfloat[\imc \llk]{
%	\includegraphics[scale=\sizee]{\qd S303/images/LogLikPoisson_nc10}}\\
	
	\subfloat[\ima \lbd]{
	\includegraphics[scale=\sizee]{\qd S303/images/LowerBound_nc20}}
	\subfloat[\imb \lbd]{
	\includegraphics[scale=\sizee]{\qd S303/images/LowerBound_nc15}}
	\subfloat[\imc \lbd]{
	\includegraphics[scale=\sizee]{\qd S303/images/LowerBound_nc10}}
	
	\subfloat[\ima \mxc]{
	\includegraphics[scale=\sizee]{\qd S303/images/MaxCorrInResid_nc20_NMF}}
	\subfloat[\imb \mxc]{
	\includegraphics[scale=\sizee]{\qd S303/images/MaxCorrInResid_nc15_NMF}}
	\subfloat[\imc \mxc]{
	\includegraphics[scale=\sizee]{\qd S303/images/MaxCorrInResid_nc10_NMF}}\\
	
	\caption{(a-c) Simulated data. (d-l) $K$ estimation for $10$ sources contained in a circular area with radius \ima \ (left column), \imb \ (middle column) and \imc \ (right column).}
	 \label{fig:K estimation}
\end{figure}

\autoref{fig:K estimation}(a-c) represent three different simulated datasets with sources randomly scattered within an area of radius $2\delta$, $1.5\delta$ and $\delta$, respectively. $\delta$ was equal to diameter of an Airy disk, shown as a green circle in \autoref{fig:K estimation}(a-c) ($\delta=0.61\frac{\lambda_{em}}{NA}$). Each image shows the mean of all frames of the dataset and corresponds to a wide-field image with blue circles indicating the true positions of the sources.

A simple model comparison can obtained by using the Bayesian Information Criterion \cite{Bishop2006} where we find $\hat{\theta}$ that maximises $p(\bm{D}|K, \theta)$, and then add a penalty term for the $NK$ parameters contained in $\bm{W}$, so that models with larger $K$ would be more heavily penalised. BIC, however, did not provide a satisfactory results. We therefore estimated the number of sources $K$ using:

\begin{enumerate}
	\item 
	\emph{Principal Component Analysis (PCA).}
	
	A crude estimation of $K$ can be obtained from a position of a ``kink'' in the plot of sorted principal values \autoref{fig:K estimation}d-f. However, the ``kink'' is not obvious in the noisy data or data with high density of blinking sources \autoref{fig:K estimation}f.
	\item
	\emph{A variational lower bound.} 
	
	A variational approximation of the GaP model \autoref{sub: GaP} provide lower bound $\mathcal{L}$ on the likelihood $p(\bm{D}|K, \theta)$ by approximately integrating out the latent variables $\bm{h}_{k}$ \cite{Buntine2006}. To obtain the marginal likelihood $p(\bm{D}|K)$ it would be necessary to also integrate out $\theta$, but this is computationally challenging. We show that in fact the lower bound already underestimates the value of $K$, so that $p(\bm{D}|K)$ would likely peak at even lower values of $K$ and thus systematically underestimate the number of sources \autoref{fig:K estimation}g-i.
	
	\item 
	\emph{Analysis of correlations in residuals.}
	
	An alternative approach for estimating $K$ is to analyses the residuals ``data-model''.  $N\times T$ residual matrix $\bm{S}$:
	
\begin{equation}
	s_{nt}=\frac{d_{nt}-\sum_{k=1}^{K}w_{nk}h_{kt}}{\sqrt{\sum_{k=1}^{K}w_{nk}h_{kt}}}.
\end{equation}

The factor $1/\sqrt{\sum_{k=1}^{K}w_{nk}h_{kt}}$ is applied in order to standardise the residuals (zero mean and unit variance) of Poisson distributed data. We can then compute the $N\times N$ correlation matrix 

\begin{equation}
	\bm{C}_{S}=\bm{SS}^{T},
\end{equation}

and the $N\times N$ matrix of the correlation coefficients $\bm{R}_{S}$ with entries 

\begin{equation}
	r_{ij}=\frac{c_{ij}}{\sqrt{c_{ii}c_{jj}}}.
	\label{eq:Correlation in residuals}
\end{equation}

Underestimation of the number of sources ($K<K_{\mbox{true}}$) will lead to correlations between some pixels as the model will try to explain multiple sources with one component. For $K\geq K_{\mbox{true}}$ the correlations are expected to drop to a base level and the residuals become uncorrelated. We can pick the value of $K$ for which the residual correlations decrease to a certain level and where further increase of $K$ does not give any further improvement \autoref{fig:K estimation}j-l. This method showed the best performance on simulated datasets \autoref{fig:K estimation hist}.
\end{enumerate}

\begin{figure}[!hbt]
	\newcommand{\sizef}{.4}		
	\centering
	\subfloat[Maximum correlations in residuals]{
	\includegraphics[scale=\sizef]{\qd S301/images/KestHist_maxC2}}
	\subfloat[Variational lower bound]{
	\includegraphics[scale=\sizef]{\qd S300/images/KestHist_lb}}	
	\caption{Histograms of the $K$ estimations with analysis of correlations in residuals (a) and variational lower bound (b). Simulated data with three different sources density: 10 sources within a disk of $\delta$ (blue), $1.5\delta$ (green) and $2\delta$ (red).}
	\label{fig:K estimation hist}
\end{figure}

None of the methods mentioned above provided satisfactory results for higher densities of the source. However, we developed a iterative procedure of the NMF algorithm which can deal with moderate overestimation of $K$ (see \autoref{sub: Iterative restarts}). The credible number of sources is recovered after fitting. Therefore we used a crude over-estimation of $K$ with PCA, because this can be computed directly from data $\bm{D}$ prior to the evaluation of the algorithm: 
%
\begin{enumerate}
	\item
	We compute the sorted principal coefficients $\lambda_{j}$ of $\bm{D}$ ($\lambda_{1}>\lambda_{2}>...$). 
	\item
	$K$ is (over)estimated as the number of components which satisfy $\lambda_{j}/\lambda_{1}>t_{\lambda}$, where $t_{\lambda}$ is a threshold. 
\end{enumerate}
%
User should be able to test the source estimation procedure on a patch where the number of sources can be guessed (e.g.\ an area with sparse sources) to get a notion about the threshold. The threshold should be set such that is overestimates the true number of sources.

%==========================================

\subsection{Tackling local optima in NMF fitting with iterative restarts \label{sub: Iterative restarts}}

Although the Lee and Seung algorithm is convex with respect to $\bm{W}$ and $\bm{H}$ separately, it is non-convex in both simultaneously \cite{Lee2001}. Multiple restarts can be used to address the problem of local optima, but we have not found good solutions with this approach on the QD data. Instead, we exploit some prior knowledge about the problem, namely that PSFs are likely to have a fairly compact structure, see e.g. \autoref{fig:Simulted-PSF-different-focal-depths}. As the estimated sources $\bm{w_{k}}$ are normalised to have an $L_1$-norm of 1 (i.e.,\ $\sum_{j}w_{jk}=1$), we use the inverse $L_2$-norm to rank the columns $\bm{w_{k}}$'s of the matrix $\bm{W}$. Note that Hoyer's sparsity measure $s=\frac{\sqrt{n}-L_1/L_2}{\sqrt{n}-1}$ \cite{Hoyer2004} is a $L_{1}/L_{2}$ measure normalised to the $[0..1]$ interval \cite{Kim2008}. 

This leads to an iterative algorithm (see \autoref{alg:restarts}) where on iteration $j+1$ the first $j$ sorted sources $\{ \bm{w} \}_{1}^{j}$ (and corresponding $\{ \bm{h} \}_{1}^{j}$) are used as initial values for the first $j$ columns of $\bm{W}$ and the corresponding rows of $\bm{H}$. The remaining components are re-initialised from a uniform random distribution. Initial values of $\bm{W}$ and $\bm{H}$ for the $j+1$th iteration are therefore composed of the $j$ `sparsest' components of the previous iteration and $K-j$ randomly initialised components. The procedure runs until $j=K$.

\begin{algorithm}
	\caption{Iterative restarts of the NMF}	
	\label{alg:restarts}
	\begin{enumerate}
		\item Set $\bm{W}_{init}$ and $\bm{H}_{init}$ as random positive matrices.
		\item Iterate for $i=1:K$ ($K$ is the number of (over) estimated sources.)
		\begin{enumerate}
			\item Run NMF with $\bm{W}_{init}$ and $\bm{H}_{init}$ as initial values.
			\item Sort columns of $\bm{W}$ according to $L_2$ norm and permute rows of $\bm{H}$ correspondingly.
			\item Replace first $j$ columns of $\bm{W}_{init}$ with first $j$ columns of sorted $\bm{W}$.
			\item Replace last $j+1:K$ columns of $\bm{W}_{init}$ with positive random vectors.
			\item Replace first $j$ rows of $\bm{H}_{init}$ with first $j$ rows of sorted $\bm{H}$.
			\item Replace last $j+1:K$ rows of $\bm{H}_{init}$ with positive random vectors.
		\end{enumerate}
	\end{enumerate}    
\end{algorithm}

The iterative procedure is illustrated on simulated data of slanted line with eight attached PSFs in \autoref{fig:Iterative restarts}. The parameters of the simulations are discussed in  \autoref{sec:results} with illustration of a typical frame of the dataset \autoref{fig:Data-true-estimations}a and true sources \autoref{fig:Data-true-estimations}b.

The algorithm was run for $K=15$, with the last component reserved for background. The results of the first run (random initialisation) are shown in \autoref{fig:Iterative restarts}a. The individual PSFs (see \autoref{fig:Data-true-estimations}b) are spread across all $\bm{w}$s, and many of them contain a mixture of multiple PSFs. This is a typical solution corresponding to a local minimum of the objective function \autoref{eq:KL divergence}. As the iterative procedure progresses, realistic sources are gradually recovered, see \autoref{fig:Iterative restarts}b. After nine iterations, the first eight $\bm{w}$'s show credible PSFs, while the rest represent only noise, see \autoref{fig:Iterative restarts}c. Further iterations do not have a significant effect on the already estimated PSFs, see \autoref{fig:Iterative restarts}d. 

\begin{figure}[!htb]
	\newcommand{\widthfig}{.95\textwidth}
	\newcommand{\barspace}{-.7cm}
	\condcomment{\boolean{includefigs}}{ 
	\centering
		
		\subfloat[run 1 ]{
		\begin{tabular}{l}
			\includegraphics[width=\widthfig]{\qd S382/images/w_restart0_1toEnd}\vspace{\barspace}\tabularnewline
			\includegraphics[width=\widthfig]{\qd S382/images/w_restart0_1toEnd_intBars}
			\tabularnewline
		\end{tabular}}
		
		\subfloat[run 4 ]{
		\begin{tabular}{l}
			\includegraphics[width=\widthfig]{\qd S382/images/w_restart3_1toEnd}\vspace{\barspace}\tabularnewline
			\includegraphics[width=\widthfig]{\qd S382/images/w_restart3_1toEnd_intBars}
			\tabularnewline
		\end{tabular}}
		
		\subfloat[run 8 ]{
		\begin{tabular}{l}
			\includegraphics[width=\widthfig]{\qd S382/images/w_restart7_1toEnd} \vspace{\barspace}\tabularnewline
			\includegraphics[width=\widthfig]{\qd S382/images/w_restart7_1toEnd_intBars}
			\tabularnewline
		\end{tabular}}
		
		\subfloat[run 14 ]{
		\begin{tabular}{l}
			\includegraphics[width=\widthfig]{\qd S382/images/w_restart13_1toEnd} \vspace{\barspace}\tabularnewline
			\includegraphics[width=\widthfig]{\qd S382/images/w_restart13_1toEnd_intBars}\tabularnewline
		\end{tabular}}		
	}
	\caption{Illustration of the iterative restart procedure. Estimated sources after (a) 1, (b) 4, (c) 9 and (c) 14 runs of the algorithm. Bars below the figures show the maximum of the $\bm{w}$.}
	\label{fig:Iterative restarts}
\end{figure}

The iterative restart procedure leads to better local minima of the NMF optimisation problem. Moreover the $L_{2}$ norm sorting of the recovered $\bm{w}$s after each iteration (step 2a in \autoref{alg:restarts}) indirectly enhances sparsity of the $\bm{W}$'s columns. The sparse $\bm{w}$ with small $L_{2}$ norm are likely to be reused in subsequent restart (step 2c in \autoref{alg:restarts}), whereas the $\bm{w}$ with large $L_{2}$ norm are likely to be replaced by a random vector. This ``soft'' sparsity enhancement allows for higher flexibility of the recovered $\bm{w}$ and allows recovery of the sources with different individual sparsities such as the sources from different focal depths as shown in \autoref{fig:Iterative restarts}d. This is and advantage compared to a ``hard'' sparsity enhancement used in the Hoyer's algorithm (\autoref{sub:Hoyer}).

%==========================================

\subsection{Classification of the estimated sources\label{sub:Classification-of-sources}} \fix

The estimated sources can greatly vary in quality. While some $\bm{w}$'s are credible representation of a PSF, there are often $\bm{w}$'s which contain multiple PSFs. Because of the overestimation of the number of sources $K$ (\autoref{sub:Estimation-of-number-of-sources}), there are also $\bm{w}$'s present which correspond to background noise. Sources located close to the patch border and therefore partially missing should also be identified; these sources will likely appear in the adjacent patch entirely, because the overlap of the patches is set to approximately the extent of the (in focus) PSF (\autoref{sec:preproc}).

\begin{figure}[!htb]
	\newcommand{\fw}{.98\textwidth}
	\newcommand{\barspace}{-.55cm}
	\centering
	\begin{tabular}{l}			
		\includegraphics[width=\fw]{\qd S455/images/resw_1to16_col}\vspace{\barspace}\tabularnewline
		\includegraphics[width=\fw]{\qd S455/images/resw_1to16_col_barInt}\tabularnewline
		\includegraphics[width=\fw]{\qd S455/images/resw_17to32_col}\vspace{\barspace}\tabularnewline
		\includegraphics[width=\fw]{\qd S455/images/resw_17to32_col_barInt}\tabularnewline
		\includegraphics[width=\fw]{\qd S455/images/resw_33to48_col}\vspace{\barspace}\tabularnewline
		\includegraphics[width=\fw]{\qd S455/images/resw_33to48_col_barInt}\tabularnewline
		\includegraphics[width=\fw]{\qd S455/images/resw_49to64_col}\vspace{\barspace}\tabularnewline
		\includegraphics[width=\fw]{\qd S455/images/resw_49to64_col_barInt}\tabularnewline
		\includegraphics[width=\fw]{\qd S455/images/resw_65to80_col}\vspace{\barspace}\tabularnewline
		\includegraphics[width=\fw]{\qd S455/images/resw_65to80_col_barInt}\tabularnewline
	\end{tabular}	
	\caption{Selection of the credible $\bm{w}$s. The green and blue boxes indicates the estimated ``credible'' sources (with only one major global maximum). The sources with blue frame have the maximum closer then 2 pixels to the border. The sources with red frame have local maxima of comparable strength. Bars under the figures show the maximum of the estimated $\bm{w}_{k}$.}
	\label{fig:good w}	
\end{figure}

In the situation were all the sources are expected to be in-focus and therefore have a fairly compact PSF with one global maximum (\autoref{fig:Simulted-PSF-different-focal-depths} left) we can use a simple procedure for identifying reasonable $\bm{w}$s:
%
\begin{enumerate}
	\item
	Each estimated source $\bm{w}$ is convolved with in-focus point spread function (PSF) (generated from the parameters of the experimental setup). This is to smooth the noise in the results and to enhance the structures on a scale of PSF. 
	\item
	Number of local maxima bigger then $50\%$ of the global maximum is counted. 
\end{enumerate}

Only the sources with one local maximum of the convolution of PSF and $\bm{w}$ are considered for further evaluation. The distance of the maximum from the edge can indicate a partially missing source. 

The process is illustrated in \autoref{fig:good w} on $\bm{w}$ estimated from the simulated dataset of randomly scattered sources with density $50\unit{\mu m^{-2}}$ (\autoref{sub:Simul random}). The $\bm{w}$s considered as a ``credible'' are indicated with blue or green frame. The blue frame shows the sources with maximum closer then two pixels from the border. The red frame shows the $\bm{w}$ with two local minima of similar strength (at least $50\%$ of the strength of the stronger maximum).

This approach would, however, fail when used on data with out-of-focus PSF, because the images of the out-of-focus PSF do not have one compact global maximum (\autoref{fig:Simulted-PSF-different-focal-depths} right).

\emph{Discuss here the linear classifier?} \fix

%We use a linear classifier to automate the quality assessment of the estimated $\bm{w}_{k}$. We compute a set of features based on intensity values of $\bm{w}_{k}$, thresholded background image $\bm{b}_{k}=\bm{w}_{k}<t_b$, thresholded foreground image $\bm{f}_{k}=\bm{w}_{k}>t_f$ and an image of $\bm{w}_{k}$ smoothed with estimated in-focus PSF. The features are listed in \autoref{tab:Features}a and five classes for $\bm{w}_{k}$ are listed in \autoref{tab:Classes}b.
%
%
%\begin{table}[!h]
%	\begin{centering}		
%		\footnotesize{\subfloat[Features]{
%		\begin{tabular}{|c|c|}
%			\hline 
%			\textbf{\#} & \textbf{note}\tabularnewline
%			\hline
%			\hline 
%			\textbf{1} & $l^{2}$ norm ($\sqrt{\sum_{j}w_{jk}^{2}}$)\tabularnewline
%			\hline 
%			\textbf{2} & Smoothness of $\bm{w}_{k}$ (discrete version of $\int\left|\frac{\partial}{\partial x}w_{k}(x)\right|dx$).\tabularnewline
%			\hline 
%			\textbf{3} & Smoothness of $\bm{b}_{k}$ (discrete version of $\int\left|\frac{\partial}{\partial x}b_{k}(x)\right|dx$).\tabularnewline
%			\hline 
%			\textbf{4} & Maximum of the cross-correlation between $\bm{w}_{k}$ and PSF.\tabularnewline
%			\hline 
%			\textbf{5} & Difference between PSF smoothed image and original $\bm{w}_{k}$.\tabularnewline
%			\hline 
%			\textbf{6} & Sum of the foreground $\bm{f}_{k}$. $\sum_{j}a_{jk}$ $ $\tabularnewline
%			\hline 
%			\textbf{7} & Distance of the global maximum from the nearest edge.\tabularnewline
%			\hline
%		\end{tabular}}				
%		\textbf{\hspace{.2cm}}\subfloat[Classes]{
%		\begin{tabular}{|c|c|}
%			\hline 
%			\textbf{\#} & \textbf{note}\tabularnewline
%			\hline
%			\hline 
%			\textbf{1} & One credible PSF\tabularnewline
%			\hline 
%			\textbf{5} & One PSF partly missing \tabularnewline
%			\hline 
%			\textbf{2} & Two credible PSFs\tabularnewline
%			\hline 
%			\textbf{3} & Three credible PSFs\tabularnewline
%			\hline 
%			\textbf{4} & Multiple credible PSFs\tabularnewline
%			\hline 
%			\textbf{0} & Noise \tabularnewline
%			\hline
%		\end{tabular}}}
%	\end{centering}	
%	\caption{(a) Features and (b) classes for classification of $\bm{w}_{k}$.}\label{tab:Features}\label{tab:Classes}
%\end{table}

%==========================================

\subsection{Localisation and stitching\label{sub:Localisation-and-stitching}}
The individual estimated sources classified as a credible representation of the PSF can be now localised. Conventional LM techniques applies the maximum likelihood fitting of an in-focus PSF gaussian approximation to the estimated images \cite{Hess2006}. The localisation precision is estimated from the number of photons emitted by the sources in the frame from which the source was localised. NMF estimated intensity $\bm{H}$ matrix gives us much greater flexibility as we have an access to the entire intensity profile of the source. We can therefore estimate number of all photons emitted by the source during the measurement. The sources close to the edge can be problematic to localise. If the source represents an in-focus PSF, then it should appear entirely in the adjacent patch and can be localised there. When dealing with images with mostly in focus PSFs we can simply discard the sources classified as ``partly missing'' (\autoref{sub:Classification-of-sources}). More problematic are the out-of-focus PSFs with extent larger then the overlap area. These sources have to be first stitched together before further processing.

%==========================================

\subsection{Visualisation of the results\label{sub:visualisation}}
To visualise the results of the NMF procedure we can use two different methods: 
\begin{enumerate}
	\item
	The conventional way for LM data visualisation (STORM, PALM) is to sum gaussian functions placed in the estimated locations. The variance $\sigma^{2}$ of each gaussian can be set to reflect the ``uncertainty'' of the estimated position. This is usually set to be proportional to number of photons emitted by the source. $\sigma$ is usually much smaller then the resolution limit and the result represent a super-resolution image of the data. 
	
	In terms of NMF procedure, this method replaces the credible estimated $\bm{w}$s with ideal, sub-resolution PSFs. The intensity values for each source can be estimated from the intensity time profiles of each source (rows of $\bm{H}$). 	
	\item
	We can use the estimated sources $\bm{w}$ directly without replacing them with their ``ideal'' PSFs. By taking the power $p>1$ of the estimated sources $\bm{w}^{p}$ we achieve a ``shrinking'' of the individual $\bm{w}$ while keeping some characteristics of each source's shape (elongation along a certain direction, for example). If we normalise the $L_{1}$ norm of $\bm{w}^{p}$ to 1, we can reconstruct a ``super-resolution'' image by summing all $\bm{w_{k}}^{p}$, weighted by the corresponding mean intensity $\unit{mean}(\bm{h_{k}})$.
\end{enumerate}

%==========================================
%==========================================

\clearpage
\section{Evaluation of the results\label{sec:evaluation}}

The performance of the algorithm applied on a simulated dataset can be quantitatively measured, because the true locations of the sources are known. We used several measures to compare the algorithm performance on simulated datasets consisting of randomly scattered in-focus PSFs. 

\begin{figure}[!h]
	\centering
	\includegraphics[width=1.0\textwidth]{figures/TPFNFPillustration/TPFNFPs}
	\caption{True positives TP, false positives FP and false negatives FN illustration. Red dot represent the true location with a circle of radius $r$. Green cross is an estimated position.}
	\label{fig:TPFPFN}
\end{figure}

The individual estimated sources $\bm{w_{k}}$ were localised by ML fitting of a gaussian approximation of a PSF \cite{Zhang2007}. We used a greedy algorithm to assign the estimated locations ($E$) to their nearest true positions ($T$). Only one estimated position was assigned to each true position. If the distance between and estimated and the true position was smaller than a threshold $r$, then the source was consider as a true positive (TP). A true position with no estimated source within a disk of radius $r$ was false negative (FN), whereas an estimated position further then $r$ from any true position was considered as false positive (FP) \autoref{fig:TPFPFN}. $M$ estimated sources in the proximity of one true source are counted as $1$TP and $(M-1)$FP (\autoref{fig:TPFPFNcombi}, left). One estimated source in proximity of $M$ true sources gives $1$TP and $(M-1)$FN (\autoref{fig:TPFPFNcombi}, right).

We set the threshold $r=\sigma/2$, where $\sigma=\frac{\sqrt{2}}{2\pi}\frac{\lambda_{em}}{NA}$ is the standard deviation of the in-focus PSF gaussian approximation \cite{Zhang2007}. For the parameters used in our simulations (see \autoref{tab:Simulations parameters}) the threshold corresponds to $r=0.7\unit{pixels}$ ($56\unit{nm}$). 

\begin{figure}[!h]
	\centering
	\includegraphics[width=1.0\textwidth]{figures/TPFNFPillustration/TPFNFPs_comb}
	\caption{There is only one estimated position assigned to each true position. Two estimated sources in the proximity of one true source are counted as $1$TP and $1$FP (left). One estimated source in proximity of two true sources gives $1$TP and $1$FN (right)}
	\label{fig:TPFPFNcombi}
\end{figure}

%==========================================

\subsection{Localisation error}
Localisation error was estimated as an average distance between true locations and all estimated locations classified as TP. 

%==========================================

\subsection{Estimated density}
The estimated density of the sources was counted as number of all TP divided by the area of the image.

%==========================================

\subsection{Average precision\label{sub:AP}}
The average precision (AP) \cite{Salton1986,Everingham2009} was used to summarise both localisation precision and ability to recover the individual sources.

The estimated positions $e_{k}$ are ranked according to the square of the source's mean brightness $b_{k}$, because the localisation precision should scale as $1/sqrt(N)$, where $N$ is a number of photons emitted by a source. This is retrieved from the matrix $\bm{H}$ as a mean along the rows.

The interval $[l_{min},l_{max}]$ between the dimmest $l_{min}$ and the brightest $l_{max}$ source is divided into 101 equal intervals (confidence levels) $\bm{l}=[0:.01:1]*(l_{max}-l_{min})+l_{min}$. For each confidence level $l_i$ only the sources with $b_{k}$ above $l_i$ are considered. True positives ($\unit{TP}_{i}$), false negatives ($\unit{FN}_{i}$) and false positives ($\unit{FP}_{i}$) are computed for each confidence level $l_{i}$.

Precision $P$ and recall $R$ are computed from $\unit{TP}(l_{i})$, $\unit{FP}(l_{i})$ and $\unit{FN}(l_{i})$ for each confidence level $l_i$:
%
\begin{align} \label{eq:TP,FN} 
	P(l_i)& = \frac{\unit{TP}(l_i)}{\unit{TP}(l_i)+\unit{FP}(l_i)}\\
	R(l_i)& = \frac{\unit{TP}(l_i)}{\unit{TP}(l_i)+\unit{FN}(l_i)}
\end{align}
%
An example of precision $P(l_{i})$ and recall $R(l_{i})$ curves for different confidence levels is shown in \autoref{fig:PRcurve}a. 
%
\begin{figure}[!h]
	\newcommand{\widthfig}{.5\textwidth}
	\centering
	\subfloat[]{
	\includegraphics[width=\widthfig]{\qd S407/images/PRconfidence}}
	\subfloat[]{
	\includegraphics[width=\widthfig]{\qd S407/images/PR_Pinterpol}}
	\caption{(a) Example of the precision $P(l_{i})$ (blue) and recall $R(l_{i})$ (green) curve. (b) The precision/recall curve $P(R)$ (blue) with interpolated precision $P_{interp}(\tilde{R}$) (red).}
	\label{fig:PRcurve}
\end{figure}


The precision/recall (PR) curve $P(R)$ (\autoref{fig:PRcurve}a) is interpolated for $11$ equally spaced recall levels $\tilde{R}_{i}\in[0:.1:1]$ by taking the maximum precision for which the corresponding recall exceeds $\tilde{R}_{i}$ (\autoref{fig:PRcurve}b):
%
\begin{equation}
	P_{interp}(\tilde{R})=\max_{R;R\geq \tilde{R}}P(R)
\end{equation}
%
The precision/recall (PR) curve is interpolated in order to reduce the impact of ``wiggles'' in the PR curve (see \autoref{fig:PRcurve}b). Note that to obtain a high AP, the method must have precision at all levels of recall penalising methods that can accurately estimate only few very bright sources. 
% ``wiggles'' in the precision/recall curve, caused by small variations in the ranking of examples. 

Average precision (AP) is a mean of interpolated precision.
%
\begin{equation}
	AP=\frac{1}{11}\sum_{\tilde{R}}{P_{interp}(\tilde{R})}
\end{equation}

%==========================================
%==========================================

\clearpage
\section{Simulated data \label{sec:simulations}}
To test the performance of the algorithm in different experimental settings we created several simulated datasets. These datasets were used for comparing the NMF with CSSTORM \cite{Zhu2012} and the 3B \cite{Cox2011} methods. 

%Main competitors 3B \cite{Cox2011}, CSSTORM \cite{Zhu2012} and SOFI \cite{Dertinger2009}.	

The parameters of the simulations such as size of the PSF and the brightness of the fluorophores are chosen to correspond to a real QD data and are summarised in \autoref{tab:Simulations parameters}. 
%
\begin{table}[!h]
	\centering
	\begin{tabular}{|c|c|l|}
		\hline 
		Parameter 		& Note  						& Value\tabularnewline
		\hline \hline 
		$\lambda_{em}$ 	& wavelength of the emission light 	& 625 nm\tabularnewline
		\hline 
		NA 				& numerical aperture 			& 1.3\tabularnewline
		\hline 
		RI 				& refraction index 				& 1.5\tabularnewline
		\hline 
		pixel-size 			& size of a pixel in image plane	 	& 80 nm\tabularnewline
		\hline 
		$n_{t}$ 			& number of frames  				& $50-1000$\tabularnewline
		\hline 
		$\unit{mean}(n_{phot})$ & mean number of photons / source / frame 	& 2500\tabularnewline
		\hline 
		$\max(n_{phot})$ 	& max number of photons / source / frame 	& 5000\tabularnewline
		\hline 
		$b$ 				& background/pixel 				& 70\tabularnewline
		\hline
	\end{tabular}
	\caption{Main parameters for data simulations.}
	\label{tab:Simulations parameters}
\end{table}

The blinking behaviour of the QDs was simulated as:
%
\begin{enumerate}
	\item
	Uniform random number between $0$ and $\max(n_{phot})$.
	\item
	Telegraph process with switching rate $\gamma$: the time axes was oversampled $100\times$, telegraph process was generated between $0$ and $\max(n_{phot})$ with $\gamma/100$ and the result was $100\times$ under-sampled. This way some intermediate values of the intensity are generated in the blinking. 
\end{enumerate}

The blinking behaviour is discussed more in details in \autoref{sub:blinking behaviour}.


 
%==========================================

\subsection{Randomly scattered sources\label{sub:Simul random}}
Several frames from a simulated dataset of randomly scattered sources with defined densities are shown in  \autoref{fig:simulated data random}.

\begin{figure}[!htb]	
	\newcommand{\widthfig}{1\textwidth}
	\centering	
	\subfloat[density $10\unit{\mu m^{-2}}$ ($14$ sources)]{
	\includegraphics[width=\widthfig]{\qd S455/images/dpixc_1to8_dens10}}
	
	\subfloat[density $30\unit{\mu m^{-2}}$ ($43$ sources)]{
	\includegraphics[width=\widthfig]{\qd S455/images/dpixc_1to8_dens30}}
	
	\subfloat[density $50\unit{\mu m^{-2}}$, ($72$ sources)]{
	\includegraphics[width=\widthfig]{\qd S455/images/dpixc_1to8_dens50}}		
	\caption{First eight frames (out of $10^{3}$) of simulated data. Randomly scattered sources with density $10-50\unit{\mu m^{-2}}$. The area of the frame is $1.2\times1.2\unit{\mu m}$ ($15\times15$ pixels)}
	\label{fig:simulated data random}
\end{figure} 

%==========================================

\subsection{Artificial structure: hash symbol\label{sub:Simul hash}}

%==========================================

\subsection{Slanted bar - out of focus PSF\label{sub:Simul bar}}

%==========================================
%==========================================

\clearpage
\section{Results \label{sec:results}}
We used simulated data for exploring the behaviour of the NMF in different experimental regimes. AP and estmated density (described in \autoref{sec:evaluation}) were used as quantitative quality assesments of the algorithm performance. We also used the simulated dataset for comparing NMF witih other techniques dealing with overlapping sources. 

We applied the NMF on real datasets. A randomly scattered out-of-focus PSF were used to demonstrated the ability of NMF to recover different individual sources. Tubulin structures labeled with QDs were used as a biologically relevant sample. 

%==========================================

\subsection{Simulated data: effect of the source density and blinking behaviour \label{sub:blinking behaviour}}
The performance of the algorithm for different source densities and three different types of the blinking behaviour was tested on simulated data. The parameters of the simulations are in \autoref{tab:Simulations parameters}. 

\begin{figure}[!h]
	\centering
	\newcommand{\sizef}{.35}
	\subfloat[uniform]{
	\includegraphics[scale=\sizef]{\qd S455/images/blinkmatS422}}
	\subfloat[telegraph ($\gamma$)]{
	\includegraphics[scale=\sizef]{\qd S455/images/blinkmatS445}}\\
	\subfloat[telegraph oversampled ($\gamma/100$)]{
	\includegraphics[scale=\sizef]{\qd S455/images/blinkmatS450}}
	\subfloat[telegraph oversampled ($\gamma$)]{
	\includegraphics[scale=\sizef]{\qd S455/images/blinkmatS455}}	
	\caption{Examples of blinking behaviour of one source (50 points out of 1000).}
	\label{fig:blinking}
\end{figure}

Four different blinking behaviours with fixed number of emitted photons ($\unit{mean}(n_{phot})$) were considered: 
%
\begin{enumerate}
	\item
	Uniform random distribution between $0$ and $\max(n_{phot})$ (\autoref{fig:blinking}a). 
	\item
	A telegraph process with a rate $\gamma$ (\autoref{fig:blinking}b). The intensity is switching between $0$ and $\max(n_{phot})$
	\item
	The time axes was oversampled $100\times$ and a telegraph process with a rate $\gamma$ was generated. Finally the blinking was under-sampled $100\times$. This corresponds to $100\times$ faster blinking with severely under-sampled measurement, which averages the intermittent behaviour (\autoref{fig:blinking}c).
	\item
	The time axes was oversampled $100\times$ and a telegraph process with switching rate $\gamma/100$ was generated. After $100\times$ under-sampling (\autoref{fig:blinking}d) this represents more realistic blinking behaviour then (\autoref{fig:blinking}b) with the same blinking rate. The switching between two states is no longer synchronised with the sampling and which gives rise to intermediate intensity values. 
\end{enumerate}
	
\begin{figure}[!h]
	\centering
	\includegraphics[scale=.6]{\qd S455/images/blinkmat_varS422S445S450S455}
	\caption{Variance of the intensity time profiles for four different blinking behaviours shown in \autoref{fig:blinking}.}
	\label{fig:blinking var}
\end{figure}

The variance of the four different intensity profiles is shown in \autoref{fig:blinking var}.

We used \autoref{alg:restarts} to evaluate simulated data of randomly scattered sources for a range of densities $5-50\unit{\mu m^{-2}}$. The parameters of the simulations are in \autoref{tab:Simulations parameters} and an illustration of the datasets for is shown  in \autoref{fig:simulated data random}. 

The number of sources $K$ was set to $K=K_{true}+10$, where $K_{true}$ is the true number of sources used for simulation. 

\begin{figure}[!h]
	\centering
	\includegraphics[scale=.6]{\qd S455/images/AP_letters_S422S445S450S455}
	\caption{Average Precision. Comparison of the different blinking behaviours from the \autoref{fig:blinking}.}
	\label{fig:AP blinking}
\end{figure}

The average precision (\autoref{sub:AP}) for the four blinking behaviours from \autoref{fig:blinking} is shown in \autoref{fig:AP blinking}. The severely under-sampled blinking \autoref{fig:blinking}c represents a difficult dataset with little variation of the sources intensity. The poor performance of the NMF on these data is therefore not surprising. 

%==========================================

\subsection{Simulated data: effect of the number of frames}
%Simulated data of randomly scattered sources (\autoref{sec:simulations}) were used to evaluate effect of the number of frames in dataset. We used a dataset with $3000$ frames and used first $[50,\,100,\,200,\,500,\, 1000]$ frames for evaluation. The achieved average precision (AP) for three different densities of the sources are shown in \autoref{fig:AP on frames}. 
%%
%\begin{figure}[!h]
%	\newcommand{\sizef}{.7}
%	\centering
%	\includegraphics[scale=\sizef]{\qd S486/figures/APS486S476S481}
%	\caption{Average Precision}
%	\label{fig:AP on frames}
%\end{figure}
%
%For low densities ($10\unit{\mu m^{-2}}$) the results reach the plateau AP about $90\%$ for 100 frames (blue curve in \autoref{fig:AP on frames}). Adding more frames does not increase AP.

What is the optimal way of acquiring data when we have a limited total acquisition time? Is it better to use longer acquisition time per frame to acquire smaller number of frames with better signal-to-noise ratio in each frame? Or rather use as fast acquisition as possible acquiring many frames? 

\begin{figure}[!htb]	
	\newcommand{\widthfig}{1\textwidth}
	\centering	
	\subfloat[original data]{
	\includegraphics[width=\widthfig]{\qd S537/figures/dpixc_1to8_subf1}}
	
	\subfloat[subsampled $2\times$]{
	\includegraphics[width=\widthfig]{\qd S537/figures/dpixc_1to8_subf2}}
	
	\subfloat[subsampled $10\times$]{
	\includegraphics[width=\widthfig]{\qd S537/figures/dpixc_1to8_subf10}}
	
%	\subfloat[subsampled $20\times$]{
%	\includegraphics[width=\widthfig]{\qd S537/figures/dpixc_1to8_subf20}}		
	\caption{First eight frames of subsampled simulated data. }
	\label{fig:subsampled data}
\end{figure} 

To address these question we tested the algorithm on simulated data of randomly scattered sources (\autoref{sec:simulations}). The parameters of the simulation were taken from \autoref{tab:Simulations parameters} but the maximum of the sources' intensity ($\max(n_{phot})$ was set to 300. This represent weak sources of fast acquisition time. The telegraph process \autoref{fig:blinking}c was used for simulated intensity profiles. 

Several frames of the dataset are shown in \autoref{fig:subsampled data}a. From the dataset of $1000$ frames we generated four more datasets by subsampling the data $q=2,\,5,\,10$ and $20$ times by summing the $q$ subsequent frames. If  we neglect the read-out noise of the camera, this correspond to data taken with $q$ times longer acquisition time per frame. The subsampled data consist of $1000/q$ frames. Several frames of the subsampled data for $q=2$ and $10$ are shown in \autoref{fig:subsampled data}b,c, respectively. 

\begin{figure}[!h]
	\centering
	\newcommand{\wf}{.9\textwidth}
	\subfloat[Average precisoin]{
	\includegraphics[width=\wf]{\qd S537/figures/APS537S527S532}}\\
	\caption{Average precision as a function of number of subsampled frames.}
	\label{fig:AP subsampled}
\end{figure}

AP computed for NMF evaluated data for three different source densities is shown in \autoref{fig:AP subsampled}. 

The subsampling of the data gives higher singna-to-noise rate in each frame, however the sources appear to be more in the ``ON'' state because of their ``OFF'' periods get averaged out. 

\emph{Maybe rather try this experiment with uniform random blinking. Because here the blinking rate is set to 0.5 (every other frame switch) and therefore any subsampling makes it worse.}

%==========================================
\clearpage
\subsection{Simulated data: comparison with other methods} %Main competitors 3B \cite{Cox2011}, CSSTORM \cite{Zhu2012} and SOFI \cite{Dertinger2009}.	
%
\begin{figure}[!h]
	\centering
	\newcommand{\sizef}{.42}
	\subfloat[density]{
	\includegraphics[scale=\sizef]{\qd S455/images/densitytrueestNMFCSSTORMCSSTORMind}}
	\subfloat[AP]{
	\includegraphics[scale=\sizef]{\qd S455/images/AP_NMFCSSTORMCSSTORMind}}
	
	\caption{Comparison of the NMF and CSSTORM evaluation of the randomly scattered PSFs. The blue curve corresponds to NMF, green curve to projected CSSTORM image and green curve to values estimated from individual frames of the CSSTORM results. (a) Density (b) Average precision}
	\label{fig:comparison AP,dens}
\end{figure}

We used simulated data of randomly scattered overlapping sources (\autoref{sub:Simul random}) to compare the performance of the NMF with CSSTORM \cite{Zhu2012}, and 3B analysis \cite{Cox2011}, where the code is freely available. 

Four data-stets with densities $10,\,20,\,30$ and $40\unit{sources/\mu m^{2}}$ were used. An empty margin of three pixels was left along every edge of the frames to ensure all there are no partially missing PSFs. The sum projection of the dataset is shown in \autoref{fig:wf demo}. Several individual data frames are shown in \autoref{fig:simulated data random} and shows highly overlapping sources in each frame. For CSSTORM and 3B evaluation the true background value ($100 \unit{photons/pixel}$) was subtracted before evaluation, clipping the negative values.

The  results of the NMF algorithm are shown as blue curves in \autoref{fig:comparison AP,dens}. The visualisation of the results (\autoref{sub:visualisation}) for datasets with four different densities is shown in \autoref{fig:NMF demo}. 

CSSTORM processes each input frame individually,independent on the rest of the dataset. The output for each frame is a ``high resolution'' frame showing the possible positions of the sources on a sub-pixel grid (8 times oversampled). As described in \cite{Zhu2012} we estimated a position of each source as an centre of mass of the small clusters formed on a sub-pixel grid. The AP (as a mean of AP from individual frames ) and the estimated density (as a maximum of estimated densities from individual frames) are denoted as CSSTORMind and are shown as red curves in \autoref{fig:comparison AP,dens}.

We also processed a sum of all output frames which summarises all the estimated sources. The summed image was filtered with gaussian kernel ($\sigma=1\unit{pixel}$) and all the local maxima above a threshold were identified. The threshold was set to $5\%$ of the global maximum. The positions of the local maxima was considered as estimated positions and used for computation of AP and estimated density (\autoref{sec:evaluation}). Results are plotted as green curves in  \autoref{fig:comparison AP,dens}. The gaussian filtered sum of all frames with true and estimated sources is shown in \autoref{fig:CSSTORM demo}.

The results in \autoref{fig:comparison AP,dens} suggest that CSSTORM cannot recover enough sources in individual frames (red curves). The density is severely underestimated which leads to many FN and therefore low recall values (\autoref{eq:TP,FN}) which penalises AP (\autoref{fig:PRcurve}). However, CSSTORM recovers some subset of the sources in each frame and the the sum projection show dramatically improved AP and density estimation. It should be noted, however, that the estimated density and AP from the sum projection is highly dependent on the threshold for considering local maxima (see above).

We also used 3B analysis for comparison. The prior parameters were adjusted to correspond the the true values (\texttt{blur.mu=0.37261, blur.sigma=0.1}). The true number of sources was used as an initial number of spots (\texttt{intensity.rel\_sigma=}$K_{true}$).

The output coordinates of the 3B analysis were placed on $100\times$ oversampled grid and convolved with a gaussian ($\sigma=10$). The resulting image is shown in \autoref{fig:3B demo}. 

The results of NMF, CSSTORM and 3B for different densities of randomly scattered sources are visualised in \autoref{fig:NMF demo}-\ref{fig:3B demo}.

\begin{figure}[!h]
	\centering
	\newcommand{\sizef}{1}
	\subfloat[density $10\unit{\mu m^{-2}}$, $14$ sources]{
	\includegraphics[scale=\sizef]{\qd /S500/figures/wf_dens10_simiter1}}
	\subfloat[density $20\unit{\mu m^{-2}}$, $29$ sources]{
	\includegraphics[scale=\sizef]{\qd S500/figures/wf_dens20_simiter1}}\\
	\subfloat[density $30\unit{\mu m^{-2}}$, $43$ sources]{
	\includegraphics[scale=\sizef]{\qd S500/figures/wf_dens30_simiter1}}
	\subfloat[density $40\unit{\mu m^{-2}}$, $58$ sources]{
	\includegraphics[scale=\sizef]{\qd S500/figures/wf_dens40_simiter1}}
	
	\caption{Sum projection of four datasets with different densities of the randomly scattered sources. Red circles show the true sources. The radius of the circles indicates the true-positive threshold distance $r=0.7\unit{pixels}$ ($56\unit{nm}$). For further information see \autoref{sec:evaluation}. Size of the image is $1.7\times1.7\unit{\mu m}$ ($21\times21$ pixels, $80 \unit{nm/pixel)}$}
	\label{fig:wf demo}
\end{figure}

\begin{figure}[!h]
	\centering
	\newcommand{\sizef}{1}
	\subfloat[density $10\unit{\mu m^{-2}}$]{
	\includegraphics[scale=\sizef]{\qd /S500/figures/demo_dens10_simiter1}}
	\subfloat[density $20\unit{\mu m^{-2}}$]{
	\includegraphics[scale=\sizef]{\qd S500/figures/demo_dens20_simiter1}}\\
	\subfloat[density $30\unit{\mu m^{-2}}$]{
	\includegraphics[scale=\sizef]{\qd S500/figures/demo_dens30_simiter1}}
	\subfloat[density $40\unit{\mu m^{-2}}$]{
	\includegraphics[scale=\sizef]{\qd S500/figures/demo_dens40_simiter1}}
	
	\caption{Demonstration of the NMF results. The results are visualised as a weighted sum of $\bm{w}^{p}$ as described in \autoref{sub:visualisation}. Red circles show the true sources. The radius of the circles indicates the true-positive threshold distance $r=0.7\unit{pixels}$ ($56\unit{nm}$). For further information see \autoref{sec:evaluation}. Green crosses show the local maxima in the image. Size of the image is $1.7\times1.7\unit{\mu m}$ ($84\times84$ pixels - $4\times$ oversampled original).}
	\label{fig:NMF demo}
\end{figure}

\begin{figure}[!h]
	\centering
	\newcommand{\sizef}{1}
	\subfloat[density $10\unit{\mu m^{-2}}$]{
	\includegraphics[scale=\sizef]{\qd S492/figures/demo_dens10_simiter1}}
	\subfloat[density $20\unit{\mu m^{-2}}$]{
	\includegraphics[scale=\sizef]{\qd S492/figures/demo_dens20_simiter1}}\\
	\subfloat[density $30\unit{\mu m^{-2}}$]{
	\includegraphics[scale=\sizef]{\qd S492/figures/demo_dens30_simiter1}}
	\subfloat[density $40\unit{\mu m^{-2}}$]{
	\includegraphics[scale=\sizef]{\qd S492/figures/demo_dens40_simiter1}}
	
	\caption{Demonstration of the CSSTORM results. Gaussian filtered sum projection  of all output frames is shown. Red circles show the true sources. The radius of the circles indicates the true-positive threshold distance $r=0.7\unit{pixels}$ ($56\unit{nm}$). For further information see \autoref{sec:evaluation}. Green crosses show the local maxima in the image. Size of the image is $1.7\times1.7\unit{\mu m}$ ($168\times168$ pixels - $8\times$ oversampled original).}
	\label{fig:CSSTORM demo}
\end{figure}

\begin{figure}[!h]
	\centering
	\newcommand{\sizef}{1}
	\subfloat[density $10\unit{\mu m^{-2}}$]{
	\includegraphics[scale=\sizef]{\qd S494/figures/demo_dens10_simiter1}}
	\subfloat[density $20\unit{\mu m^{-2}}$]{
	\includegraphics[scale=\sizef]{\qd S501/figures/demo_dens20_simiter1}}\\
	\subfloat[density $30\unit{\mu m^{-2}}$]{
	\includegraphics[scale=\sizef]{\qd S495/figures/demo_dens30_simiter1}}
	\subfloat[density $50\unit{\mu m^{-2}}$]{
	\includegraphics[scale=\sizef]{\qd S516/figures/demo_dens40_simiter1}}
	
	\caption{Demonstration of the 3B results. Red circles show the true sources. The radius of the circles indicates the true-positive threshold distance $r=0.7\unit{pixels}$ ($56\unit{nm}$). For further information see \autoref{sec:evaluation}. Green crosses show the local maxima in the 3B image. Size of the image is $1.7\times1.7\unit{\mu m}$ ($2100\times2100$ pixels - $100\times$ oversampled original).}
	\label{fig:3B demo}
\end{figure}

\fix
\emph{compare some simulations with 3B - under progress...}

%==========================================

\clearpage
\subsection{Simulated data: out of focus PSFs}
%
\begin{figure}[!htb]	
	\newcommand{\widthfig}{0.95\textwidth}
	\newcommand{\barspace}{-.5cm}
	\condcomment{\boolean{includefigs}}{ 
	\centering	
	\subfloat[Data]{	
	\begin{tabular}{l}
		\includegraphics[width=\widthfig]{\qd S382/images/dpixc_1to8}		
	\end{tabular}}\\	
	\subfloat[True sources]{
	\begin{tabular}{l}
		\noindent		
		\includegraphics[width=\widthfig]{\qd S382/images/wtrue_l2sort}\vspace{\barspace}\tabularnewline
		\includegraphics[width=\widthfig]{\qd S382/images/wtrue_l2sort_intBars}\tabularnewline
	\end{tabular}}\\
	\subfloat[True sources corrupted with noise]{
	\begin{tabular}{l}
		\includegraphics[width=\widthfig]{\qd S382/images/wtrue_l2sort_noise}\vspace{\barspace}\tabularnewline
		\includegraphics[width=\widthfig]{\qd S382/images/wtrue_l2sort_noise_intBars}\tabularnewline
	\end{tabular}}\\	
	\subfloat[Estimated sources]{
	\begin{tabular}{l}
		\includegraphics[width=\widthfig]{\qd S382/images/resw_1to8}\vspace{\barspace}\tabularnewline
		\includegraphics[width=\widthfig]{\qd S382/images/resw_1to8_intBars}\tabularnewline
	\end{tabular}}}
	\caption{Simulated data of eight sources. (a) Eight frames (out of 500) of the simulated data set. (b) The true sources. (c) Noisy version of the true sources with their maximum intensity. (d) The first 8 estimated sources (see \autoref{fig:Iterative restarts}d for all $\bm{w}$'s.) Bars under the figures show the maximum of the estimated $\bm{w}_{k}$. }
	\label{fig:Data-true-estimations}
\end{figure} 

%\emph{Aren't the bars under the $\bm{w}_{k}$ confusing? They are representing the maximum of $\bm{w}_{k}$ not their maxumim intensity (estimated from $\bm{h}_{k}$} \fix)

We applied NMF to simulated data of eight blinking QDs attached to a bar slanting in depth, see \autoref{fig:Data-true-estimations}a. The individual QDs were separated by 200 nm and the axial difference between the tips of the bar was $1600$ nm. Other parameters of the simulation are: emission wavelength 625 nm, numerical aperture 1.3, refractive index 1.5, edge size of a pixel in the image plane 100 nm, $T=500$, mean number of photons per source 1500, background photons/pixel 70, uniform distribution of blinking. 

The true sources (individual PSFs) are shown in \autoref{fig:Data-true-estimations}b and their noisy version (obtained from the frame with the maximum intensity of each source) is shown in \autoref{fig:Data-true-estimations}c.

%==========================================

\subsection{Real data: out of focus PSFs}
The results of the same procedure applied on a movie of real out-of-focus QDs are shown in \autoref{fig:Real-data-QDrandom}a. Credible out-of-focus PSFs from different focal depths (cf. \autoref{fig:Simulted-PSF-different-focal-depths}) have been recovered (the first two rows in \autoref{fig:Real-data-QDrandom}b). The $\bm{W}_{k}$s in the last row of \autoref{fig:Real-data-QDrandom}b are mostly noise contribution. The mean brightness of these sources (estimated form $\bm{H}$) is about 10\% of the brightest $\bm{W}_{k}$   \emph{say more or cut this??} \fix 
%
\begin{figure}[!htb]
	\newcommand{\sizeresw}{.85}
	\newcommand{\barspace}{-.6cm}
	\condcomment{\boolean{includefigs}}{ 
	\centering
	\begin{tabular}{l}
		\subfloat[Data]{
		\includegraphics[scale=\sizeresw]{\qd S392/images/dpixc_randind}}
	\end{tabular}		
	\subfloat[Estimated sources]{
	\begin{tabular}{l}			
		\includegraphics[scale=\sizeresw]{\qd S392/images/resw_1to11}\vspace{\barspace}\tabularnewline
		\includegraphics[scale=\sizeresw]{\qd S392/images/resw_1to11_intBars}\tabularnewline
		\includegraphics[scale=\sizeresw]{\qd S392/images/resw_12to22}\vspace{\barspace}\tabularnewline
		\includegraphics[scale=\sizeresw]{\qd S392/images/resw_12to22_intBars}\tabularnewline
		\includegraphics[scale=\sizeresw]{\qd S392/images/resw_23to33}\vspace{\barspace}\tabularnewline
		\includegraphics[scale=\sizeresw]{\qd S392/images/resw_23to33_intBars}\tabularnewline			
	\end{tabular}}	
	}
	\caption{Real data � randomly scattered QDs. (a) Eleven randomly selected frames (out of $1,000$) of the overlapping PSFs produced by blinking QDs. (b) Estimated sources $\bm{w}_{k}$ sorted according to their estimated mean brightness. Bars below each figure show the maximum of the $\bm{w}_{k}$.}
	\label{fig:Real-data-QDrandom}	
\end{figure}

%==========================================

\subsection{Real data: QD stained tubulin fibres}
We have applied the pipeline to a stack of $T=1000$, $128\times128$ images of tubulin fibres labelled with QDs (\autoref{fig:Patches}). The parameters of the experiment were: excitation wavelength $405 \unit{nm}$, emission wavelength $625 \unit{nm}$, exposure time $50 \unit{ms/frame}$, numerical aperture $1.4$, refractive index $1.52$, edge size of a pixel in image plane $79 \unit{nm}$, quantum dots (the emission wavelength $\lambda=625 \unit{nm}$).The dataset was divided into $25 \times 25$ patches (see \autoref{sec:preproc}), and only patches with sufficiently strong signal (thick boxes in \autoref{fig:Patches}) are considered for further evaluation (see \autoref{sec:preproc}). The number of sources within each patch is over-estimated via principal components analysis (\autoref{sub:Estimation-of-number-of-sources}). Results of the procedure applied to the patch \texttt{24} from \autoref{fig:Patches}, where all the sources are approximately in the same focal plane, are shown in \autoref{fig:Real-data-patch-B24}b. \emph{say more} \fix

\begin{figure}[!htb]
	\centering
	\condcomment{\boolean{includefigs}}{ 
	\includegraphics[scale=.9]{\qd S364/results/dataChunks}}
	
	\caption{Division of the dataset into smaller patches. The time average of the data is shown as a grey valued image. Boxes with thick lines will be considered for NMF evaluation. Boxes with thin lines are considered to be empty. The identifier of the patch and the (over)estimated number of sources ($K$) are shown in each box.}
	\label{fig:Patches}
\end{figure}

\begin{figure}[!htb]
	\newcommand{\widthf}{.95\textwidth}
	\newcommand{\barspace}{-.6cm}
	\condcomment{\boolean{includefigs}}{ 
	\centering
	\subfloat[Data]{			
	\begin{tabular}{l}
		\includegraphics[width=\widthf]{\qd S364/results/dpixc_randind}\tabularnewline
	\end{tabular}}	
					
	\subfloat[Estimated sources]{				
	\begin{tabular}{l}
		\includegraphics[width=\widthf]{\qd S364/results/resw_B24_1to14}\vspace{\barspace}\tabularnewline
		\includegraphics[width=\widthf]{\qd S364/results/resw_B24_1to14_intBars}\tabularnewline
		\includegraphics[width=\widthf]{\qd S364/results/resw_B24_15to28}\vspace{\barspace}\tabularnewline
		\includegraphics[width=\widthf]{\qd S364/results/resw_B24_15to28_intBars}\tabularnewline
		\includegraphics[width=\widthf]{\qd S364/results/resw_B24_29to42}\vspace{\barspace}\tabularnewline
		\includegraphics[width=\widthf]{\qd S364/results/resw_B24_29to42_intBars}\tabularnewline
		\includegraphics[width=\widthf]{\qd S364/results/resw_B24_43to56}\vspace{\barspace}\tabularnewline
		\includegraphics[width=\widthf]{\qd S364/results/resw_B24_43to56_intBars}\tabularnewline			
	\end{tabular}}
	}
	\caption{Real data � patch \texttt{B24} from \autoref{fig:Patches}. (a) 14 randomly selected frames (out of $10^{3}$) of the tubulin structure stained with QDs. (b) Estimated sources $\bm{w}_{k}$ sorted according to their $l^{2}$ norm (shown all $K=56$ sources). Bars below each figure show the maximum of the $\bm{w}_{k}$.}
	\label{fig:Real-data-patch-B24}
	
\end{figure}

\begin{figure}[!htb]
	\newcommand{\wf}{.9\textwidth}
	\centering
	\includegraphics[width=\wf]{\qd S364/results/all_mean_zoh}
	\caption{Wide - filed data.}
\end{figure}

\begin{figure}[!htb]
	\newcommand{\wf}{.9\textwidth}
	\centering
	\includegraphics[width=\wf]{\qd S364/results/all_gauss3_int}
	\caption{QD - labelled tubulin. Gaussian filtered.}
\end{figure}

\begin{figure}[!htb]
	\newcommand{\wf}{.9\textwidth}
	\centering
	\includegraphics[width=\wf]{\qd S364/results/all_powFromLabel}
	\caption{QD - labelled tubulin. PSF squeezed.}
\end{figure}


%We trained the classifier on $10^{3}$ labelled $\bm{w}_{k}$, computed by NMF from a real dataset (\autoref{fig:Patches}). Confusion matrix of the ten-fold cross validation is shown in \autoref{tab:Confusion-matrix}. From all $\bm{w}_{k}$s classified as good sources (class 1) 89\% were correct, while the rest 11\% being spread into classes for two sources (6\%), half missing source (3\%), noise (2\%) and multiple sources (1\%).
%
%\begin{table}[!h]
%	\subfloat[Counts]{
%	\begin{tabular}{|c||c|c|c|c|c|c|c|}
%		\hline 
%		\textbf{Class} & \textbf{0} & \textbf{1} & \textbf{2} & \textbf{3} & \textbf{4} & \textbf{5} & \tabularnewline
%		\hline
%		\hline 
%		\textbf{0} & \textcolor{red}{130} & 4 & 6 & 1 & 18 & 26 & \tabularnewline
%		\textbf{1} & 6 & \textcolor{red}{335} & 21 & 0 & 3 & 13 & \tabularnewline
%		\textbf{2} & 15 & 43 & \textcolor{red}{89} & 0 & 9 & 16 & \tabularnewline
%		\textbf{3} & 3 & 0 & 6 & \textcolor{red}{3} & 9 & 2 & \tabularnewline
%		\textbf{4} & 29 & 7 & 23 & 1 & \textcolor{red}{32} & 1 & \tabularnewline
%		\textbf{5} & 12 & 12 & 8 & 0 & 0 & \textcolor{red}{187} & \tabularnewline
%		\hline
%	\end{tabular}}
%	\hspace{.2cm}
%	\subfloat[Percentage (sum over rows gives 100\%).]{
%	\begin{tabular}{|c||c|c|c|c|c|c|c|}
%		\hline 
%		\textbf{Class} & \textbf{0} & \textbf{1} & \textbf{2} & \textbf{3} & \textbf{4} & \textbf{5} & \tabularnewline
%		\hline
%		\hline 
%		\textbf{0} & \textcolor{red}{70} & 2 & 3 & 1 & 10 & 14 & \tabularnewline
%		\textbf{1} & 2 & \textcolor{red}{89} & 6 & 0 & 1 & 3 & \tabularnewline
%		\textbf{2} & 9 & 25 & \textcolor{red}{52} & 0 & 5 & 9 & \tabularnewline
%		\textbf{3} & 13 & 0 & 26 & \textcolor{red}{13} & 39 & 9 & \tabularnewline
%		\textbf{4} & 31 & 8 & 25 & 1 & \textcolor{red}{34} & 1 & \tabularnewline
%		\textbf{5} & 5 & 5 & 4 & 0 & 0 & \textcolor{red}{85} & \tabularnewline
%		\hline 
%	\end{tabular}}
%	\caption{Confusion matrix for 10-fold cross validation. Correctly classified $\bm{w}_{k}$ are on diagonal highlighted in red.} \label{tab:Confusion-matrix}
%\end{table}

\begin{table}[!h]	
	\subfloat[Real data]{
	\begin{tabular}{|c|c|c|}
		\hline 
		Parameter & Note  & Value\tabularnewline
		\hline
		\hline 
		$\lambda_{ex}$ & excitation light & 405 nm\tabularnewline
		\hline 
		$\lambda_{em}$ & emission light & 625 nm\tabularnewline
		\hline 
		$t_{exp}$ & exposure time  & 50 ms\tabularnewline
		\hline 
		NA & numerical aperture & 1.4\tabularnewline
		\hline 
		RI & refraction index & 1.52\tabularnewline
		\hline 
		pixel-size & size of a pixel in image plane & 79 nm\tabularnewline
		\hline 
		QD & quantum dots  & QD625\tabularnewline
		\hline 
		$n_{t}$ & number of frames  & $10^{3}$\tabularnewline
		\hline
	\end{tabular}}
	\caption{Parameters of the experiment.}\label{tab:Parameters of the (a) simulations (b) real data}
\end{table}

