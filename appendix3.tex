%!TEX root = thesis.tex
\chapter{LS-SIM Electronics\label{app:LSSIM electronics}}
%
\begin{figure}[!hbt]
	\centering
	\newcommand{\wf}{.8\textwidth}
%	\subfloat[Schematic of the trigger signals]{
%	\includegraphics[width=\wf]{\home Documents/Publications/LineScanPaper/figures/setupElectoricsCorrected/Media1a}}\\
%	\subfloat[Program listing]{
%	\includegraphics[width=\wf]{\home Documents/Publications/LineScanPaper/figures/setupFigure/listing}}		
	\includegraphics[width=\wf]{\home Documents/Publications/LineScanPaper/figures/setupElectoricsCorrected/Media1a}
	\caption{Schematic of the trigger signals in the microscope. The scan-mirror is the master and triggers the camera when it starts moving in one edge of the field. Upon receiving {\sf EXTERNAL\_TRIGGER} the camera starts integrating in less than a microsecond and sets {\sf FIRE} to high. This triggers an interrupt in the micro-controller and it generates {\sf LIGHT\_ENABLE} to modulate the laser. The initial $p_i$ delays determine the position of the scanning lines.
The values in the timing diagram (bottom right) were measured with a logic analyser (Logic16, Salea). These values correspond to an optimised timing of the LS-SIM system, such that the sum of the 32 line positions imprinted on the sample produced a homogeneous wide-field illumination.} 
\end{figure}

