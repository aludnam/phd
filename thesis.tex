%%%%%%%%%%%%%%%%%%%%%%%%
%% Sample use of the infthesis class to prepare a thesis. This can be used as 
%% a template to produce your own thesis.
%%
%% The title, abstract and so on are taken from Martin Reddy's csthesis class
%% documentation.
%%
%% MEF, October 2002
%%%%%%%%%%%%%%%%%%%%%%%%

%%%%
%% Load the class. Put any options that you want here (see the documentation
%% for the list of options). The following are samples for each type of
%% thesis:
%%
%% Note: you can also specify any of the following options:
%%  logo: put a University of Edinburgh logo onto the title page
%%  frontabs: put the abstract onto the title page
%%  deptreport: produce a title page that fits into a Computer Science departmental cover [not sure if this actually works]
%%  singlespacing, fullspacing, doublespacing: choose line spacing
%%  oneside, twoside: specify a one-sided or two-sided thesis
%%  10pt, 11pt, 12pt: choose a font size
%%  centrechapter, leftchapter, rightchapter: alignment of chapter headings
%%  sansheadings, normalheadings: headings and captions in sans-serif  (default) or in the same font as the rest of the thesis
%%  [no]listsintoc: put list of figures/tables in table of contents (default:  not)
%%  romanprepages, plainprepages: number the preliminary pages with Roman  numerals (default) or consecutively with the rest of the thesis
%%  parskip: don't indent paragraphs, put a blank line between instead
%%  abbrevs: define a list of useful abbreviations (see documentation)
%%  draft: produce a single-spaced, double-sided thesis with narrow margins
%%
%% For a PhD thesis -- you must also specify a research institute:
%\documentclass[phd,ianc,twoside,logo]{infthesis}
\documentclass[phd,ianc,oneside,logo,leftchapter,12pt,doublespacing]{infthesis}
%\documentclass[phd,ianc,oneside,logo,leftchapter,12pt,fullspacing]{infthesis}

%% Put any \usepackage commands you want to use right here; the following is an example:
\usepackage{natbib}
\usepackage{graphicx}
\usepackage{subfig}
\usepackage{algorithm}
\usepackage{ifthen} 
\usepackage{bm}
\usepackage{amsmath, amsthm, amssymb}
\usepackage{color}
\usepackage[colorlinks=true, pdfstartview=FitV, linkcolor=blue, citecolor=black, urlcolor=blue,pagebackref,breaklinks=true]{hyperref}
\usepackage{datetime}
\usepackage{lineno}
\usepackage{enumerate}
\usepackage{rotating}
\usepackage{pdfpages}

\hyphenation{a-na-ly-sis}
\hyphenation{CS-STORM}
\hyphenation{Cra-mer}
\hyphenation{fluo-res-cence}
\hyphenation{fluo-res-cent}
\hyphenation{fluo-ro-pho-res}
\hyphenation{di-men-si-on-al}
\hyphenation{de-conv-lucy}

%\linenumbers

%This makes brakcets around equations numbers in \autoref
\makeatletter
\def\tagform@#1{\maketag@@@{\ignorespaces#1\unskip\@@italiccorr}}
\let\orgtheequation\theequation
\def\theequation{(\orgtheequation)}
\makeatother

% \Autoref is for the beginning of the sentence
\let\orgautoref\autoref
\providecommand{\Autoref}[1]
{%
\def\equationautorefname{Equation}%
\def\figureautorefname{Figure}%
\def\subfigureautorefname{Figure}%
\def\chapterautorefname{Chapter}%
\def\sectionautorefname{Section}%
\def\subsectionautorefname{Section}%
\def\tableautorefname{Table}%
\orgautoref{#1}%
}
% \autoref is used inside the sentence to produce Fig., and Eq. for figures, subfigures, and equations
\renewcommand{\autoref}[1]
{%
\def\equationautorefname{Eq.}%
\def\figureautorefname{Fig.}%
\def\subfigureautorefname{Fig.}%
\def\algorithmautorefname{Algorithm}%
\def\chapterautorefname{Chapter}%
\def\sectionautorefname{Sect.}%
\def\subsectionautorefname{Sect.}%
\def\tableautorefname{Tab.}%
%\def\appendixautorefname{App.}%
%\def\appendixsectionautorefname{App.}%
\orgautoref{#1}%
}

%this is to make captions with small bold labels
%	\renewcommand{\captionfont}{\small}
\renewcommand{\captionlabelfont}{\small\bfseries}

 %   General parameters, for ALL pages (copied from http://mintaka.sdsu.edu/GF/bibliog/latex/floats.html):
 \renewcommand{\topfraction}{0.9}	% max fraction of floats at top
 \renewcommand{\bottomfraction}{0.8}	% max fraction of floats at bottom
 
 
% fancy backreferencing
\renewcommand*{\backref}[1]{}
\renewcommand*{\backrefalt}[4]{%
    \ifcase #1 (Not cited.)%
    \or        (Cited on page~#2.)%
    \else      (Cited on pages~#2.)%
    \fi}


\newboolean{includefigs} 
\setboolean{includefigs}{true} % set to {false} to make output without figures

\newcommand{\fix}{\marginpar{FIX}}
\newcommand{\new}{\marginpar{NEW}}
\newcommand{\fixme}[1]{\textcolor{red}{*} \marginpar{FIX:\\ {\scriptsize \textcolor{red}{*}\emph{#1}}}}
\newcommand{\home}{/Users/ondrejmandula/}
\newcommand{\qd}{\home project/data/qdots/}
\newcommand{\bfone}{\mathbf{1}}
\newcommand{\cut}[1]{}
\newcommand{\unit}[1]{\ensuremath{\, \mathrm{#1}}}
\newcommand{\condcomment}[2]{\ifthenelse{#1}{#2}{}}
\newcommand{\um}{\unit{\mu m}}
\newcommand{\inmf}{\hyperref[alg:restarts]{\textcolor{black}{iNMF}}}
\newcommand{\aaa}{\textcolor{blue}{a}}
\newcommand{\bbb}{\textcolor{blue}{b}}
\newcommand{\ccc}{\textcolor{blue}{c}}
\newcommand{\ddd}{\textcolor{blue}{d}}
\newcommand{\eee}{\textcolor{blue}{e}}
\newcommand{\fff}{\textcolor{blue}{f}}
\newcommand{\gggg}{\textcolor{blue}{g}}
\newcommand{\hhh}{\textcolor{blue}{h}}
\newcommand{\iii}{\textcolor{blue}{i}}
\newcommand{\jjj}{\textcolor{blue}{j}}
\newcommand{\kkk}{\textcolor{blue}{k}}
\newcommand{\CR}{Cram\'er \textendash{} Rao\ }
\newcommand{\var}{\mathrm{var}}
\newcommand{\Po}{\mathrm{Po}}
\newcommand{\E}{\mathbb{E}}
\newcommand{\LL}{\mathcal{L}}
\newcommand{\superscript}[1]{\ensuremath{^{\textrm{#1}}}}
\newcommand{\ths}[0]{\superscript{th}}
\newcommand{\st}[0]{\superscript{st}}
\newcommand{\nd}[0]{\superscript{nd}}
\newcommand{\rd}[0]{\superscript{rd}}

\setlength\parindent{0pt} % Remove indenting of paragraphs.
%% Information about the title, etc.
\title{Super-resolution methods for fluorescence microscopy}
% Super-resolution techniques in optical microscopy
% Super-resolution optical microcopy techniques
% Localisation microsopy using quantum dots. 

\author{Ond\v rej Mandula}

\submityear{2012}
% \graduationdate{February 1786}

\abstract{
\singlespace
\newcommand{\vs}{3}

\vspace{10 mm}
Fluorescence microscopy is an important tool for biological research. However, the resolution of a standard fluorescence microscope is limited by diffraction, which makes it difficult to observe small details of a specimen's structure. We have developed two fluorescence microscopy methods that achieve resolution beyond the classical diffraction limit.

\vspace{\vs mm}
The first method represents an extension of localisation microscopy. We used non-negative matrix factorisation (NMF) to model a noisy dataset of highly overlapping fluorophores with intermittent intensities. We can recover images of individual sources from the optimised model, despite their high mutual overlap in the original dataset. This allows us to consider blinking quantum dots as bright and stable fluorophores for localisation microscopy. Moreover, NMF allows recovery of sources each having a unique shape. Such a situation can arise, for example, when the sources are located in different focal planes, and NMF can potentially be used for three dimensional super-resolution imaging. We discuss the practical aspects of applying NMF to real datasets, and show super-resolution images of biological samples labelled with quantum dots. It should be noted that this technique can be performed on any wide-field epifluorescence microscope equipped with a camera, which makes this super-resolution method very accessible to a wide scientific community.

\vspace{\vs mm}
The second optical microscopy method we discuss in this thesis is a member of the growing family of structured illumination techniques. Our main goal is to apply structured illumination to thick fluorescent samples generating a large out-of-focus background. The out-of-focus fluorescence background degrades the illumination pattern, and the reconstructed images suffer from the influence of noise. We present a combination of structured illumination microscopy and line scanning. This technique reduces the out-of-focus fluorescence background, which improves the quality of the illumination pattern and therefore facilitates reconstruction. We present super-resolution, optically sectioned images of a thick fluorescent sample, revealing details of the specimen's inner structure. 

\vspace{\vs mm}
In addition, in this thesis we also discuss a theoretical resolution limit for noisy and pixelated data. We correct a previously published expression for the so-called fundamental resolution measure (FREM) and derive FREM for two fluorophores with intermittent intensity. We show that the intensity intermittency of the sources (observed for quantum dots, for example) can increase the ``resolution'' defined in terms of FREM.
}

%% Now we start with the actual document.
\begin{document}

%% First, the preliminary pages
\begin{preliminary}

%================================
 % Comment this line out when finished
%\noindent Version: \today \ \currenttime \fix\\
%Up to date version on:\\ 
%\url{https://www.dropbox.com/s/5dtktv8cygan971/thesis.pdf}

%================================

\maketitle

\begin{acknowledgements}
I would like to thank to my supervisors Chris Williams and Rainer Heintzmann for their guidance. I would like to give special thanks to Ivana \v Sumanovac for preparation of 3T3 fibroblast cells and Otto Baumann and Eva Simb\" urger for providing the Calliphora sample. My thanks go to Ingo Kleppe and Gerhard Krampert for providing us with ZEISS ELYRA-S system with a line-scanning module. I would also like to acknowledge Aur\' elie Jost, Helen Ramsden, Martin Kielhorn, Jakub Nedbal, Kai Wicker and Hugh Pastoll for help with revision of this manuscript. 
\end{acknowledgements}

\standarddeclaration
% \dedication{To my mummy.}
{

\singlespace
\include{glossary}
%\setcounter{tocdepth}{1}
\tableofcontents
}
% \listoffigures
% \listoftables
\end{preliminary}

%!TEX root = thesis.tex
\chapter{Introduction\label{ch:Introduction}}

%==========================================
%==========================================

\section{Optical microscope} % no articles in the headings

A microscope is an instrument allowing us to see objects, which are too small for a naked eye. An optical microscope (often referred to as a ``light microscope'') uses light in the visible spectral range (wavelength $\approx400-700\unit{nm}$), which makes it particularly suitable for biological exploration. Visible light is minimally invasive for sensitive biological samples and allows observation of living specimens. Visible light is also minimally absorbed by water, which prevents heating of the sample.

 The most common optical microscope is a ``far-field'' microscope, where the light has to propagate over a distance significantly longer than its wavelength. The specimen is observed with transmitted, reflected or fluorescent light. Fluorescence microscopy is discussed further in \autoref{sec:Fluorescence microscopy}. The focus of this thesis is on far-field fluorescence optical microscopy.

%==========================================
%==========================================

\section{Brief historical overview}

Optical microscopy has been around for over 400 years. Since the very early versions of Zacharias Janssen's or Galileo's compound microscopes from the beginning of 17\ths{} century, optical microscopy has undergone a long and steady process of development. Despite the speculation as to who was the actual inventor of the optical microscope, it was Anton van Leeuwenhoek who largely popularised the use of the microscope as an instrument for observing the minute details of the specimen. Leeuwenhoek also introduced his simple instrument into biological research during the 17\ths{} century.

An important milestone was the pioneering work of Ernst Abbe \cite{Abbe1873} in the second half of the 19\ths{} century. Abbe set the theoretical resolution limit for the optical microscope and mastered the design of objective lenses highly corrected for optical aberrations.

With advances in technology in the 20\ths{} century, the manufactures have produced lenses reaching the theoretical limits of the optical microscope performance. The 1953 Nobel prize in physics was awarded to Frits Zernike for discovery of the phase contrast \cite{Zernike1942}. This method allows observation of transparent specimens, and had major impact on biological research such as in vivo study of cell cycle. 

The emergence of new microscopy methods surpassing the classical resolution limit (super-resolution microscopy) at the end of the 20\ths{} century and at the beginning of the 21\st{} century has given another boost to optical microscopy research. The resolution of the super-resolution optical microscopes has reached the order of ten nanometres and some researchers have proposed the term ``optical nanoscopy'' to be used \cite{Egner2007, Hell2007, Hell2009}. However, super resolution micro/nano-scopy remains a challenging task, especially when applied to living biological specimens. While most of the super-resolution techniques require a highly specialised and expensive hardware, some of the techniques, such as localisation microscopy (discussed in \autoref{ch:NMF}) can be performed with a conventional fluorescent microscope. 

The number of scientific publications in recent years shows that even after four centuries of development the optical microscopy remains a vibrant and exciting scientific domain.

%==========================================
%==========================================

\section{Point spread function}

An important characteristic of a microscope is the so-called ``point spread function'' (PSF). The PSF represents an image of a point source. The image $i(x)$ of a specimen produced by an optical microscope can be described as a convolution between the object (specimen) $o(x)$ and the point spread function $q(x)$:
%
\begin{equation}
	i(x)=\int q(x-x')o(x')dx'.
	\label{eq:conv}
\end{equation}  

The PSF therefore defines how much the image of the specimen is ``blurred'' during the imaging process. The integration in \autoref{eq:conv} is over the whole space of acquired data (typically 2D or 3D). Note, that \autoref{eq:conv} applies to the situation with spatially invariant PSF. It also assumes that PSF is fully determined by the optical system. The influence of the specimen on the shape of the PSF is neglected. In a real experiment, PSF can be locally distorted by the aberrations introduced by the specimen or by the imperfections in the optical setup.

\begin{figure}[!bht]
	\centering
	\newcommand{\wf}{.48\textwidth}
	\subfloat[Linear]{
	\includegraphics[width=\wf]{figures/psfillustration/PSFAiryGauss}}
	\subfloat[Logarithmic]{
	\includegraphics[width=\wf]{figures/psfillustration/PSFAiryGauss_semilogy}}
	\caption{Intensity profile of a PSF for 1.2 NA objective using $\lambda_{em}=625$ nm emission light. The blue line represents profile of the Airy pattern, red dashed line shows the Gaussian approximation. Green vertical lines mark the first minima of the Airy pattern at $\delta=318$ nm. $\delta$ corresponds to the radius of the Airy disk. (a) Linear, (b) logarithmic plot of the intensity highlighting the secondary maxima in the Airy pattern and the differences of the Gaussian approximation at the periphery of the function.}
	\label{fig:PSF}
\end{figure}
%
Neglecting the effect of polarisation (scalar theory), the two-dimensional PSF of an optical microscope, known as the ``Airy pattern'', is described by \cite{Born1999}
%
\begin{equation}
	q(\rho)=\frac{1}{Z}\left(\frac{J_1(\alpha \rho)}{\alpha \rho}\right),
\end{equation}
%
where $Z=\int q(\rho) d\rho$ is the normalising constant, $J_1$ is the Bessel function of the first kind of order one and $\rho$ is the distance from the centre of the image. The parameter $\alpha$ depends on the emission wavelength $\lambda_{em}$ and the numerical aperture of the objective $\unit{NA}$:
%
\begin{equation}
	\alpha=2\pi\frac{\unit{NA}}{\lambda_{em}}.
\end{equation}
% 
The $\unit{NA}$ is defined as
\begin{equation}
	\unit{NA}=n\sin(\theta),
	\label{eq:NA}
\end{equation}
%
where $n$ is the refractive index of the immersion medium and $2\theta$ is the angle of the light cone entering the objective. Note that the refractive index is a fucntion of wevelength $n\sim n(\lambda)$. Dispersion in the sample and in the optics can lead to the PSF corrupted with spherical aberration when a range of wavelengths is used \cite{SCALETTAR1996}.

An intensity profile of an unaberrated PSF is shown in \autoref{fig:PSF}. The Airy pattern is also compared to the popular approximation of the PSF with a Gaussian function \cite{Zhang2007} defined by a standard deviation
%
\begin{equation}
	\sigma=\frac{\sqrt{2}}{2\pi}\frac{\lambda_{em}}{\unit{NA}}.
\end{equation}

%==========================================
%==========================================

\section{Resolution limit \label{sec:Resolution limit}}

\begin{figure}[!bht]
	\centering
	\newcommand{\wf}{.3\textwidth}	
	\newcommand{\ndir}{figures/psfillustration/}
	\begin{tabular}{ccc}
		\subfloat[$d=\delta/2$]{\includegraphics[width=\wf]{\ndir Airy05_AiryDisk}}
		& \subfloat[$d=\delta$]{\includegraphics[width=\wf]{\ndir Airy1_AiryDisk}}
		& \subfloat[$d=2\delta$]{\includegraphics[width=\wf]{\ndir Airy2_AiryDisk}}
		\tabularnewline
		\subfloat[$d=\delta/2$]{\includegraphics[width=\wf]{\ndir AiryPix05_AiryDisk}}
		& \subfloat[$d=\delta$]{\includegraphics[width=\wf]{\ndir AiryPix1_AiryDisk}}
		& \subfloat[$d=2\delta$]{\includegraphics[width=\wf]{\ndir AiryPix2_AiryDisk}}
		\tabularnewline
		\subfloat[$d=\delta/2$]{\includegraphics[width=\wf]{\ndir Airy05profile}}
		& \subfloat[$d=\delta$]{\includegraphics[width=\wf]{\ndir Airy1profile}}
		& \subfloat[$d=2\delta$]{\includegraphics[width=\wf]{\ndir Airy2profile}}
		\tabularnewline
	\end{tabular}
	\caption{Two PSFs (1.2 NA objective, $\lambda=625$ nm) separated by distance $d$. Location of the sources is indicated with red dots. Airy disk is indicated with green dashed circles. The radius of the Airy disk corresponds to $\delta=318$ nm. (a) Continuous representation of the PSFs. (b) Pixelated version with pixel-size 80 nm. (c) Blue lines show the intensity profiles along the line intersecting the sources' locations. The profiles of the individual PSFs are shown as red and magenta dashed lines.}
	\label{fig:Rayleigh}
\end{figure}
%
The ability of an optical microscope to show spatial details in the specimen structure is fundamentally limited by diffraction \cite{Born1999}. The radius $\delta$ of the Airy disk (see \autoref{fig:PSF}) is often considered as the ``classical resolution limit''. It is given by \cite{Born1999}
%
\begin{equation}
 	\delta=0.61\frac{\lambda_{em}}{\unit{NA}}, 
	\label{eq:Airy}
\end{equation}
%
where $\lambda_{em}$ is the wavelength of the emitted light and $\unit{NA}$ is the numerical aperture of the objective \autoref{eq:NA}. The resolution limit \autoref{eq:Airy} comes from the \emph{empirical} observation: two sources separated by a distance greater than $\delta$ can be ``resolved'' as two individual objects (see \autoref{fig:Rayleigh}\ccc). If the separation is smaller than $\delta$, the point spread functions overlap significantly and the sources become ``unresolved'' (see \autoref{fig:Rayleigh}\aaa). The resolution limit defined by \autoref{eq:Airy}, sometimes called  the ``Raleigh resolution limit'',  is often taken as the benchmark for different resolution techniques. Note that the ``Abbe resolution limit'' $\delta=0.5\lambda_{em}/\unit{NA}$ related to the passband of spatial frequencies is also used.

The resolution limit \autoref{eq:Airy} relates to the noise-free situation with continuous representation of the PSF (the top line of \autoref{fig:Rayleigh}). However, the pixelation and noise associated with the photon-detection process can deteriorate the resolution significantly. The influence of noise, pixelation or brightness intermittency to the resolution is discussed in \autoref{ch:Theoretical-limits-of the LM}. 

%==========================================
%==========================================

\section{Fluorescence microscopy\label{sec:Fluorescence microscopy}}

The introduction of fluorescence microscopy in the 20\ths\ century has revolutionised the use of optical microscopy in biological and medical science. Fluorescence is generated by fluorescent molecules or nanostructures (commonly called fluorophores) during the relaxation of their electronic structure to the ground state.  The fluorophores are driven into higher energetic state with an excitation light of a specific wavelength. The fluorescent light is typically shifted towards the longer wavelengths (Stokes shift),  which allows an efficient filtering of the excitation light from the fluorescence signal. Fluorescence microscopy detects the fluorescence from the sample itself (auto-fluorescence) or more commonly from the fluorophores attached to the specimen \cite{PawleyHandbook2006}.

Fluorescence labels can be attached to the structures of interest with high specificity and provide a strong intensity contrast in the microscopic image of the specimen. Fluorescent proteins (FPs) allow for direct expression of the fluorescent marker by the organism itself \cite{Fernandez-Suarez2008}. FPs have further redefined the use of fluorescence microscopy in cell biology as a nearly non-invasive and highly specific technique. 

The first fluorescent protein used in microscopy (green fluorescent protein - GFP) has been isolated from the jellyfish {\it Aequorea victoria} \cite{Tsien1998}. Nowadays, modified versions of GFP and a rich variety of fluorescent dyes cover the emission across the whole visible spectral range \cite{Fernandez-Suarez2008}. Photo-activable and photo-switchable fluorescent proteins and dyes have also been discovered, and are used in specific applications and microscopy methods \cite{Bock2007,Hirvonen2008, Rego2011a}. The 2008 Nobel Prize in Chemistry was awarded to Martin Chalfie, Osamu Shimomura, and Roger Y. Tsien for their discovery and development of the GFP.
 
%==========================================
%==========================================

\section{Quantum Dots\label{sec:Quantum-dots}}

Quantum dots (QDs), recently used in biological research, are promising fluorescent labels.  QDs are inorganic crystals composed of 100-100,000 atoms of substances such as cadmium selenide (CdSe), with diameter $\sim2-10\unit{nm}$ \cite{Alivisatos1996}. QDs are often coated with a zinc sulphide (ZnS) shell conjugated with an antibody molecule. The diameter of the coated QDs is $\sim10-30\unit{nm}$. The emission wavelength can be tuned by the diameter of the QD core. QDs tend to have a broad excitation spectrum and a narrow emission spectrum when compared to the standard fluorescent dyes/proteins. This facilitates the multicolour imaging with QDs. QDs are also exceptionally photo-stable. It has been reported that QDs illuminated with continuous $\sim 50\unit{mW}$ laser do not bleach even after $14\unit{h}$, whereas the standard fluorescent dye ``fluorescein'' completely bleaches in less than 20 mins \cite{Jaiswal2004}. Moreover, QDs are an order of magnitude brighter than the traditional fluorophores \cite{Resch-Genger2008,Walling2009}. 

Commercially available QDs have a polymer coating with a covalently attached linker, or are conjugated with an antibody molecule. This allows them to be specifically attached to the structure of interest in the specimen the same way as the standard fluorescent labels. However, coated QDs are relatively big ($\sim10-30\unit{nm}$) compared to the fluorescent dyes (fluorescenin size is $\sim1\unit{nm}$) or fluorescent proteins (GFP size is $\sim5\unit{nm}$). Therefore QDs cannot diffuse through the cell membrane, which complicates the labelling of the structures in the interior of the cell. 

An interesting property of the QDs is that they exhibit ``fluorescence blinking'' (fluorescence intermittency) under continuous excitation. QDs switch between the ON episodes of a rapid absorption-fluorescence cycling and the OFF episodes, where no light is emitted despite the continuous excitation. Both ON-time ($\tau_{ON}$) and OFF-time ($\tau_{OFF}$) probability densities follow an inverse power law $P(\tau_{ON/OFF})\propto1/\tau_{ON/OFF}^{m}$ \cite{Kuno2001, Stefani2009}. A comparison of QDs and the standard fluorescent dyes is described in the Resch-Genger et al. review article \cite{Resch-Genger2008}.

%==========================================
%==========================================

\section{Super-resolution\label{sec:super-resolution}}

The resolution limit \autoref{eq:Airy} for a far-field optical microscope has been challenged in the past two decades. Several research teams have reported sub-diffraction resolution in the fluorescent samples. In fact, the resolution limit is no longer dictated by diffraction but by the signal-to-noise ratio. 

There is a rich variety of super-resolution strategies in fluorescence microscopy. However, all these methods are based on driving the nearby fluorophores into emitting (ON) and non-emitting (OFF) states. This allows distinguishing the individual emitters separated by a sub-resolution distance.

``Selective'' activation can be achieved either by spatially structured excitation (structured illumination) or by stochastic activation of the individual fluorophores. In the stochastic activation approach, a small random subset of fluorophores is activated for each acquisition frame. For the conventional methods, the activated subset must be sufficiently small to ensure that the majority of the activated emitters are separated by distances larger than the diffraction limit. In this case each acquired frame consists of several well-separated (non-overlapping) PSFs. The individual fluorophores can be localised by, for example, fitting each PSF with a Gaussian function. Given enough detected photons, the localisation precision can be significantly higher than the resolution limit. The activation-acquisition cycle is typically repeated for several thousands acquisition frames. Super-resolution fluorescent images are produced by visualisation all the estimated fluorophores locations. 

Methods based on this simple strategy are called by a general term ``localisation microscopy'' (LM). Various names for LM have been proposed: ``Photo-Activation Localisation Microscopy'' (PALM) \cite{Betzig2006}, ``fluorescence PALM'' (fPALM) \cite{Hess2006} or ``Stochastic Optical Reconstruction Microscopy'' (STORM) \cite{Rust2006}.  

``Super-resolution Optical Fluctuation Imaging'' (SOFI) \cite{Dertinger2009} is a LM related technique. SOFI is based on higher-order statistical analysis of temporal intensity fluctuations caused by blinking behaviour of the fluorophores.

\cut{In \autoref{ch:NMF}, we discuss a new localisation microscopy method, which can deal with highly overlapping blinking sources. The individual sources can be ``separated'' computationally post acquisition and localised in the same manner as in the conventional LM techniques. The method allows (but is not limited to) using QDs as exceptionally bright and photo-stable fluorescence labels.}

The structured illumination based methods modulate the fluorescence behaviour of the molecules within the diffraction-limited area. The nearby molecules are driven to either ON or OFF states, which facilitates their discrimination.  These methods include ``Saturated Structured Illumination microscopy'' (SSIM) \cite{Gustafsson2000,Heintzmann2002} and ``STimulated Emission Depletion microscopy'' (STED) \cite{Hell1994}. 

\cut{The last chapter of this thesis (\autoref{ch:LSSIM}) introduces a combination of ``(linear) structured illumination microscopy'' with line scanning approach. This combination allows high resolution imaging in relatively thick fluorescent samples, which are challenging for most super-resolution techniques. The ``linear'' stands for the fact that the response of the fluorophores remains proportional to the excitation light. In this case the resolution remains limited, but can be increased by up to a factor of two. }


%==========================================
%==========================================
\clearpage
\section{Overview of the thesis}

In \autoref{ch:NMF} we discuss a new approach to localisation microscopy. Application of the machine learning technique non-negative matrix factorisation (NMF) enables us to computationally separate images of individual blinking fluorophores despite their high mutual overlap in the recorded data. We show that with this approach we can use quantum dots (QDs) as extremely bright and stable fluorescent labels for super-resolution microscopy. \cut{QDs are extremely bright and stable fluorophores, and the single QDs can be recorded using a standard CCD camera even without electron-multiplying (EM) enhancement. Data acquisition is therefore simple and does not require any specialised hardware. It can be performed on any wide-field epi-fluorescence microscope equipped with a camera, which makes this super-resolution method very accessible to a wide scientific community. }

\Autoref{ch:Theoretical-limits-of the LM} discusses the resolution criterion for noisy and pixelated data in terms of so-called fundamental resolution measure (FREM). We show that intermittency of the sources' brightness (blinking) can be beneficial and provide higher resolution when compared to sources with static intensity. 

In \autoref{ch:LSSIM} we present structured illumination microscopy (SIM) combined with line scanning (LS). Our goal is to introduce the SIM into more realistic and biologically relevant settings. Line scanning reduces the out-of-focus fluorescence background, which improves the quality of the illumination pattern. The method enables resolution improvement in relatively thick and densely labelled fluorescent samples. The reconstructed images reveal high details of the specimen's inner structure, and suffer less from the artefacts when compared to the conventional SIM methods.  

\Autoref{ch:Summary} gives a short overview of possible extensions of the current work and contains the final summary of the thesis. 

% This is copied from the first year report: 
%Recently, several super-resolution fluorescence microscopy methods (providing resolution beyond the classical resolution limit) have been developed. Microscopic images of fluorescently labelled samples with sub-resolution details have been obtained by using wide range of ideas and strategies. STED (Stimulated Emission Depletion) microscopy \citep{Westphal2003} reduces the individual excitation focal spots through non-linear interaction between two laser pulses of particular spatio-temporal profiles. This excitation spot is swept across the sample (scanning microscopy) and fluorescence signal from sub-resolution volume is registered at each position. Reducing the size of the PSF must be compensated with smaller displacement of the excitation spot and requires then higher acquisition time. SIM (Structured illumination microscopy) \citep{Gustafsson2000,Heintzmann2002} can be regarded as highly parallel form of scanning microscopy. The sample is illuminated with spatially periodic patterns which enables to encode high spatial Fourier frequencies into the detected image. Several `scans' of the pattern across the sample enable us to extract this information and the region of the observed Fourier frequencies can be enlarged. This in turn leads to super resolution in the real space. $\mathrm{I^{n}M}$ \citep{Gustafsson1999} and 4pi microscopy \citep{HellStelzer1992} introduce interference of excitation and (or) emission light and improve resolution mainly along the axial direction. A different approach is used in single-molecule localisation microscopy (LM) \citep{Betzig1995}. This technique takes advantage of the fact that a single isolated point source can be localised with high precision. A Gaussian approximation (mean $\mu$, variance $s^{2}$) of the point spread function (PSF - an ideal image of a point source) can be fit to the recorded image of a point source and its position ($\mathbf{\hat{\mu}}$) can then be estimated with precision $\mathbf{\hat{\sigma}}_{xy}\approx s/\sqrt{N}$ where $N$ is the number of detected photons \citep{Hess2006}. It is possible to detect more than $10^{4}$ photons from a single fluorophore and thus the achievable localisation precision can be better than $10\,\mathrm{nm}$ \citep{Churchman2005}. The most successful in terms of number of biological publications has been techniques based on (f)PALM/STORM principles: \textbf{PALM} (\emph{Photo-Activated Localisation Microscopy}) \citep{Betzig2006}, \textbf{fPALM} (\emph{Fluorescence PALM}) \citep{Hess2006} and \textbf{STORM} (\emph{Stochastic Optical Reconstruction Microscopy}) \citep{Rust2006_STORM} are three different names for almost identical technique (differences being the type of fluorophore used and fitting algorithms). (f)PAM/STORM methods use the photo-activation property of several fluorescent dyes (or pairs of dyes, STORM) or photo-activable fluorescence proteins ((f)PALM). Under normal conditions these molecules are in a non-fluorescent (`OFF') state. However, after an exposure to the light with $\lambda_{\mathrm{act}}$ wavelength they can be activated into a fluorescent (`ON') state. In this state they can be excited by light with a different wavelength $\lambda_{\mathrm{exc}}$ to emit fluorescent light with a peak at $\lambda_{\mathrm{em}}$ (typically $\lambda_{\mathrm{act}}<\lambda_{\mathrm{exc}}<\lambda_{\mathrm{em}}$). Usually, the sample is continuously exposed to the excitation light $\lambda_{\mathrm{exc}}$ and the activation light $\lambda_{\mathrm{act}}$ is applied in discrete pulses. After each activating pulse $\lambda_{\mathrm{act}}$ a small random subset of fluorophores is activated. Due to the continuous excitation $\lambda_{\mathrm{exc}}$ it emits fluorescence light. If the activated subset is small enough, the individual fluorophores randomly scattered across the sample are well separated (their PSFs do not overlap) and can be localised by fitting the individual PSFs into the pattern. After a certain time of emitting the fluorophore turns back into the dark (OFF) state again. This can be either reversible (`switching', STORM) or irreversible (`bleaching', (f)PALM). An extension of the STORM to three dimensions has been shown in \citep{Huang2008}. A cylindrical lens was inserted in the imaging system introducing astigmatism. Out-of-focus PSF becomes asymmetric (elongated along one lateral direction). The z-dimension is thus encoded in the asymmetry of the PSF when shifting along the axial (z) direction. The resolution improvement down to 20~nm (corresponding to $\lambda_{\mathrm{em}}/30$) was demonstrated \citep{Betzig2006,Rust2006_STORM}. However, a long acquisition time (2-12 hours in the original publications \citep{Betzig2006}) required to collect sufficient density of the fluorophores is the main drawback of these methods even though the required acquisition time has been shortened by several orders of magnitude recently (in 2008 PALM images with $\sim60\,\mathrm{nm}$ resolution ($\lambda_{\mathrm{em}}/8$) with a frame rate 25~s was reported in studies of slow cellular motion dynamics \citep{Shroff2008LiveFPALM}). A fast algorithm for estimation of the position of a large number of individual molecules ($10^{5}$ combined fits and Cramer-Rao lower bound on parameter precision calculation per second) has been presented \citep{Smith2010}. The algorithm is implemented on graphical processing unit and can speed up the post-acquisition analysis of the images by a factor of 10-100 when compared to a modern central processing unit. The algorithm has been shown to reach Cramer-Rao lower bounds on a parameter estimation. Another fast and robust algorithm for data processing consisting of noise reduction, detection of likely fluorophore position, high precision localisation and subsequent visualisation of found fluorophores position has been proposed in \citep{Wolter2010}. fPALM/STORM based techniques can only deal with non-overlapping sources. Separation of individual emitters with overlapping PSFs can significantly speed the acquisition procedure (scaling inversely with the number of molecules localised per bright spot \citep{Small2009}). In 2005 there has been published a method exploiting the fluorescence intermittency (`blinking') of quantum dots under continuous excitation \citep{Lidke2005}. A time series of the blinking quantum dots was recorded and analysed using \emph{Independent Component Analysis} (ICA). The fastICA algorithm \citep{Hyvarinen2000} has been used. As the blinking of the individual quantum dots is independent with respect to each other, individual emitters even with overlapping PSFs can be separated. Localisation of two quantum dots separated down to $23$~nm (corresponding to $\lambda_{\mathrm{em}}/30$) has been reported \citep{Lidke2005}. Further exploration of the technique for more than two sources and for different configuration of the experiment can be found in \citep{Lidke2007}. A Bayesian approach to the blinking of the individual fluorophores has been presented in poster in 2008 \citep{Harrington2008}. A localisation of several quantum dots within the diffracted limited volume has been shown. A method for measurement of sub-resolution distances using the quantum dots has been published in \citep{Lagerholm2006}. However, discrete ON-OFF blinking is required (only one source being ON and others OFF) as opposed to \citep{Lidke2005,Harrington2008} where only fluctuation of the individual sources is required. \textbf{SOFI} (\emph{Super-resolution Optical Fluctuation Imaging}), which was published in 2009 \citep{Dertinger2009} relies on higher-order statistical analysis of temporal intensity fluctuations (caused by blinking behaviour of the quantum dots) recorded in a sequence of images. It has been shown that the \emph{n}th order cumulant function (over time dimension) of the recorded sequence is composed of individual emitters with \emph{n}th power of their intensity images (PSFs). This yields the resolution enhancement by a factor $\sqrt{n}$ (for Gaussian approximation of PSF). Resolution improvement down to 60~nm ($\lambda_{\mathrm{em}}/10$) has been reported. In contrast to ICA based methods no prior knowledge about the number of sources is needed. Although there is no fundamental limit for resolution enhancement there are practical ones. Because the PSF is raised to the \emph{n}th power so is the molecular brightness (intensity) of each emitter. Thus the emitter that has intensity two times higher will appear $2^{n}$ times brighter in the \emph{n}th-order SOFI image. Additionally the brightness of each emitter in the SOFI image will be altered by its specific blinking behaviour. An emitter that doesn't fluctuate over the recorded time will not appear in the SOFI image at all.
%


%!TEX root =  thesis.tex
\chapter{Non-negative matrix factorisation for localisation microscopy\label{ch:NMF}}
We propose non-negative matrix factorisation (NMF) as a natural model for localisation microscopy of samples labelled with quantum dots (QDs) or other intermittent fluorophores. NMF can separate the individual highly overlapping sources with individual different shapes. The separated sources can be localised with uncertainty smaller than the diffraction limit and provide information about sub-resolution details of the sample structure. We use the NMF algorithm, which accounts for Poisson noise in recorded images. This allows us to recover the individual intermittent sources from noisy microscopic recording.

The chapter is divided into following sections: \Autoref{sec:LM} introduces localisation microscopy (LM) technique and discusses the advantages and challenges of using the quantum dots as fluorescent labels. It also contains a short review of recent methods dealing with LM data containing overlapping sources. 

\Autoref{sec:NMF} introduces non-negative matrix factorisation (NMF). Alternative algorithms applying certain constraints on the estimated results are discussed and a generative probabilistic model for NMF is mentioned.

\Autoref{sec: NMF} shows NMF as a natural model for intermittent overlapping QDs and \autoref{sec:NMF related} discusses the alternative methods used for treating QD data. A link to a standard deconvolution technique is also mentioned. 

Application of the NMF algorithm to real microscopic data is explored in \autoref{sec:NMF-for-real}. In this section we also present preliminary results to demonstrate specific problems of the topic.

We used synthetic data for analysing the performance of the algorithm in different experimental settings. The main simulations used in this chapter are described in \autoref{sec:simulations} and \autoref{sec:evaluation} explains the evaluation techniques for the comparison of the results.

Finally, the NMF reconstructed images of synthetic and real data are presented in \autoref{sec:results}. In this section we also compare NMF with two other techniques (CSSTORM, 3B analysis) dealing with similar problematic. 

%==========================================
%==========================================

\section{Localisation microscopy\label{sec:LM}}

Localisation microscopy (LM) is a conceptually simple and accessible technique for super-resolution imaging of fluorescent samples. LM takes as input a stack of images containing a number of fluorescent sources (fluorophores) with time-varying intensity and identifies the locations \cut{and point-spread functions (PSFs)} of these sources. If the sources are attached to structures of interest (e.g.\ in biological samples), then this provides useful information about the target structures. By exploiting multiple images of time-varying sources, LM can achieve a resolution beyond the classical diffraction limit of $\sim \lambda/2$ \cite{Abbe1873}, where $\lambda$ is the wavelength of the photons emitted by fluorophores. Provided enough photons are collected, the localisation of an individual source can be an order of magnitude below $\lambda/2$ \cite{Ober2004}, meaning that sources whose point spread functions overlap heavily can be resolved. 
%See \autoref{fig:Iterative restarts}, \autoref{fig:Real-data-QDrandom} and \autoref{fig:Real-data-patch-B24} for examples.

LM techniques are based on sources with transition between bright (ON) and dark (OFF) intensity states. Fluorescent proteins or organic dyes are used as fluorophores in the standard techniques (fPALM \cite{Hess2006}, STORM \cite{Rust2006}). In this case the density of the ON sources in each captured frame is controlled by photo-switching and must be optimised experimentally. High density of ON fluorophores results in overlapping sources and complicates localisation (overlapping sources are usually discarded), whereas low density leads to an excessive total acquisition time \cite{Small2009}. Several thousands frames are typically required for an image reconstruction.

%==========================================

\subsection{Quantum dots for localisation microscopy\label{sec:QD for LM}}

There has been interest in using quantum dots (QDs) as sources for localisation microscopy in recent years. QDs are an order of magnitude brighter and more photo-stable compared to the organic dyes or fluorescent proteins used in conventional LM \cite{Resch-Genger2008}. Under continuous excitation QDs exhibit a stochastic blinking between ON and OFF states \cite{Kuno2001,Stefani2009}. Excellent photo-stability, low cyto-toxicity and distinctive spectral properties make QDs very attractive for biological research. However, the stochastic blinking of QDs is impractical for standard LM techniques because the rate of switching, and hence the density of ON sources, is difficult to control. Thus QD-labelled data typically consist of highly overlapping sources, which cannot be localised with standard techniques.

%==========================================

\subsection{Overlapping sources\label{sec:Overlapping sources}}

Several techniques dealing with overlapping sources have been proposed recently. Most of these methods model the LM data using a known image of a single source, the so called point spread function (PSF). Most often a single PSF is assumed to be shared by all sources in the dataset.

There are two main groups of the algorithms addressing the overlapping sources in the LM data. The first group operates separately on each frame of the LM dataset: a method proposed by Huang et al. \cite{Huang2011} tries to fit multiple PSFs into each frame of the dataset while the CSSTORM \cite{Zhu2012} makes use of compressed sampling to recover the sparse vector representing the distribution of the fluorophores' locations. The DAOSTORM algorithm \cite{Holden2011} applies iterative fitting and subtracting procedure in each frame. CSSTORM is supposed to deal with higher densities than DAOSTORM (supplementary materials to \cite{Zhu2012}).

These methods ignore the fact that some sources can stay ON for several successive frames or can even reappear in different frames due to blinking because each frame of the dataset is treated independently. They can therefore generally deal with only moderately overlapping sources with densities $<10\unit{sources/\mu m^{2}}$ \cite{Huang2011,Holden2011,Zhu2012}.

The second group of the algorithms models LM dataset as a collection of blinking sources. They can improve the localisation for higher densities of the overlapping sources by taking the reappearance of fluorophores into account. However, these algorithms are, in general, computationally more expensive. 

Modelling the whole dataset from a known PSF with maximum posterior (MAP) fitting has been proposed by Harrington et al. \cite{Harrington2008}. Separation of several (up to five) simulated emitters contained in a disk of $100\unit{nm}$ radius has been shown. However, the technique becomes computationally very challenging for higher numbers of sources. Bayesian analysis of intermittent sources (Bayesian Blinking and Bleaching (3B) analysis) has been suggested in \cite{Cox2011}. The blinking behaviour of the fluorophores is modelled as a hidden Markov model with three distinct states: emitting, not emitting and bleached. Each source is described by its position, size of the PSF, and intensity. MAP estimates of the positions obtained from different sampling of the state sequences are used as estimated locations of the fluorophores. While the 3B analysis adjusts the width of the PSF (Gaussian approximation of the PSF \cite{Zhang2007}), it cannot deal with individually different shapes of the sources. This situation can arise, for example, in three-dimensional samples, where the overlapping sources can be located in different focal planes (see \autoref{fig:Simulted-PSF-different-focal-depths}). 
%
\begin{figure}[!htb]
	\newcommand{\widthfig}{.95\textwidth}
	\newcommand{\barspace}{-.5cm}
	\centering
	\condcomment{\boolean{includefigs}}{ 
	\subfloat[Intensity scaled to the in-focus PSF]{
	\includegraphics[width=\widthfig]{\qd S393/images/psf_outOfFocus_scaled}}\\
	\subfloat[Intensity scaled for each frame]{
	\begin{tabular}{l}
		\includegraphics[width=\widthfig]{\qd S393/images/psf_outOfFocus}\vspace{\barspace}\tabularnewline		
		\includegraphics[width=\widthfig]{\qd S393/images/psf_outOfFocus_intBars}
	\end{tabular}}
	}	
	\caption{Simulated PSF in different depths of focus.  The number in each figure indicates the distance in $\um$ from the in-focus plane. (a) Intensity scaled to the in-focus PSF. (b) Intensity scaled in each frame. The maximum intensity relative to the in-focus PSF is indicated in the bars below and corresponds to about 10\% at $1\um$ and 3\% at $1.5\um$.}
	\label{fig:Simulted-PSF-different-focal-depths}
\end{figure}
%
Moreover, 3B assumes a mono-exponential decay of the fluorescence for the individual sources. QDs have a complex blinking behaviour with power-law distribution in the histogram of ON and OFF times \cite{Shimizu2001}. This can complicate the 3B analysis of the QD data. 

Independent component analysis (ICA) have been proposed for analysis of overlapping intermittent sources in \cite{Lidke2005}. However, as we demonstrate in \autoref{sub:ICA} it is not a suitable model for noisy QD data. 

Yet another approach to the LM data with overlapping sources problem has been proposed in a method called SOFI (Super-resolution Optical Fluctuation Imaging) \cite{Dertinger2010b}. Instead of separating the individual emitters, SOFI analyses higher order statistics of the intensity fluctuation. The intensity values in the SOFI image, however, reflect the fluctuation behaviour, rather than the strength of the emitters. The non-linear relation between the sources' strength and intensity in the reconstructed image leads to structural artefacts such as apparent discontinuities, cavities and holes in otherwise continuous structures. Sources, which do not blink will not appear in the SOFI image at all. Recently, this issue has been, to a certain extent, addressed by balanced SOFI (bSOFI) in \cite{Geissbuehler2012}.

%==========================================
%==========================================

\clearpage
\section{Non-negative matrix factorisation\label{sec:NMF}}
Non-negative matrix factorisation (NMF) solves the approximative factorisation of an $N\times T$ data matrix $\bm{D}$ with non-negative entries:
%
\begin{equation}
	\bm{D}\approx\bm{WH},
	\label{eq:NMF approx}
\end{equation}
%
where $\bm{W}$ and $\bm{H}$ are $N\times K$ and $K\times T$  matrices ($K<N,T$), respectively. The factorisation is constraint to $\bm{W}$ and $\bm{H}$ with non-negative entries. 

Initial factorisation algorithms (so called positive matrix factorisation) \cite{Paatero1994} were published in 1994. However, it was in 1999 when NMF attracted attention of researchers after publication of the \emph{Nature} article by Daniel Lee and Sebastian Seung \cite{Lee1999}. NMF was presented as an efficient and powerful method for approximation of non-negative data (in their case a database of facial images) by linear combination of non-negative localised basis vectors (images of the nose, mouth, ears, eyes, etc.) An individual face from the dataset can be recovered as a non-subtractive composition of individual basis vectors. 

Lee and Seung also proposed simple multiplicative updates \cite{Lee2001} for the elements of $\bm{W}$ and $\bm{H}$
%
\begin{alignat}{1}
	w_{xk} & =\frac{w_{xk}}{\sum_{t=1}^{T}h_{kt}}\left[(\bm{D}\oslash\bm{WH})\bm{H^{\top}}\right]_{xk}\nonumber \\
	h_{kt} & =\frac{h_{kt}}{\sum_{x=1}^{N}w_{xk}}\left[\bm{W^{\top}}(\bm{D}\oslash\bm{WH})\right]_{kt}.
	\label{eq:NMF classic updates}
\end{alignat}
%
minimising KL-divergence between data matrix $\bm{D}$ and its factorised approximation $\bm{WH}$ (for details see \autoref{sec: NMF} and \autoref{app:NMF-algorithm}). The symbol ``$\oslash$'' denotes the element-wise division of matrices. \Autoref{eq:NMF classic updates} suggests that the complexity of the updates is $O\left(NKT\right)$, or more precisely $O\left(N(2KT+T+K)\right)$.

Note that updates \autoref{eq:NMF classic updates} automatically ensure that $\bm{W}$ and $\bm{H}$ remains non-negative if initialised so. Also once they become zero they remain zero for the rest of iterations. Sufficient conditions for uniqueness of solutions to the NMF problem has been studied in \cite{Donoho2004}. 

Various alternative minimisation strategies have been explored in an effort to speed up the convergence properties of the Lee \& Seung updates. A comprehensive discussion on the variety of these algorithms can be found in \cite{Berry2007}. 

%==========================================

\subsection{Additional constraints to the NMF model \label{sub:NMF constrains}}
Additional constraints can be imposed on $\bm{W}$ and $\bm{H}$ matrices. Imposing a defined ``sparsity'' on either columns of $\bm{W}$ or rows of $\bm{H}$ has been proposed in \cite{Hoyer2004} and is discussed in \autoref{sub:Hoyer}. Enforcing the temporal smoothness of $\bm{H}$ in the analysis of EEG recordings has been published in \cite{Chen2005}. Multiplicative updates for various constraints have been suggested in \cite{Chen2005,Pauca2006}  (see \autoref{app:NMF-algorithm}).

%==========================================

\subsection{Gamma - Poisson model \label{sub:GaP}}
The gamma-Poisson (GaP) model has been proposed by John Canny \cite{Canny2004} as a probabilistic model for documents. It represents a generative model for NMF \autoref{eq:NMF model}. The entries $h_{kt}$ of the intensity matrix $\bm{H}$ in \autoref{eq:NMF model element-wise} are regarded as latent variables generated from a Gamma distribution with parameters $\alpha_k, \beta_k$ and the data are modelled as a Poisson variable with mean $\bm{WH}$. Variables $\theta = \{\bm{w_k},\alpha_k, \beta_k\}; k = 1..K$ are then parameters of the GaP model.

%==========================================
%==========================================
\clearpage
\section{NMF as a natural model for QD data\label{sec: NMF}}
Non-negative matrix factorisation (NMF) \cite{Lee1999,Lee2001} is a natural model for QD data. NMF decomposes a movie of the blinking QDs into spatial and temporal parts, i.e.,\ time independent emission profiles of the individual sources and fluctuating intensities of each source, respectively. NMF imposes non-negativity constraints on both the spatial and the temporal components, which are natural constraints for the source profiles and intensities of blinking QDs.

Consider a $N\times T$ data matrix $\bm{D}$, where $N$ is the number of pixels in each frame, and $T$ is the number of time frames. The columns of $\bm{D}$ are the individual frames of the movie reshaped into $N\times 1$ vector by concatenating the columns of the image. All entries in $\bm{D}$ are non-negative, i.e.,\ $d_{xt}\geq 0$. Under the NMF model, matrix $\bm{D}$ is factorised into a $N\times K$ spatial component matrix $\bm{W}$ (images of the $K$ individual sources) and the $K\times T$ temporal component matrix $\bm{H}$ (the intensities of the sources). In fact, we relax the demand for exact factorisation by factorisation of the noisy dataset expectation value:  
%
\begin{equation}
	\mathbb{E}\left[\bm{D}\right]=\bm{WH};\;w_{xk},\, h_{kt}\geq0
	\label{eq:NMF model}
\end{equation}
%
or in element-wise form
%
\begin{equation}
	\mathbb{E}\left[d_{xk}\right]=\sum_{k=1}^{K}w_{xk}h_{kt};\;w_{xk},\, h_{kt}\geq0
	\label{eq:NMF model element-wise}
\end{equation}

The predominant noise model in microscopy imaging is Poisson noise \cite{PawleyHandbook2006}. Therefore the log-likelihood function can be expressed as
%
\begin{equation}
	\log p(\bm{D}|\bm{W},\bm{H})=\sum_{xt}\left(d_{xt}\log\sum_{k=1}^{K}w_{xk}h_{kt}-\sum_{k=1}^{K}w_{xk}h_{kt}\right)+C_{1},
	\label{eq:NMF model likelihood}
\end{equation}
%
where $C_{1}$ is independent of $\bm{W}$and $\bm{H}$. 

The Lee and Seung NMF updates \autoref{eq:NMF classic updates} minimise the divergence between the data and the NMF model
%
\begin{equation}
	\mbox{KL}(\bm{D}\parallel\bm{WH})=-\sum_{xt}\left(d_{xt}\log\sum_{k=1}^{K}w_{xk}h_{kt}-\sum_{k=1}^{K}w_{xk}h_{kt}\right)+C_{2}
	\label{eq:KL divergence}
\end{equation}
%
where $C_{2}$ is independent of $\bm{W}$and $\bm{H}$. Comparison with the log-likelihood \autoref{eq:NMF model likelihood} shows that the minimum of the divergence with positivity constrains on $\bm{W}$ and $\bm{H}$ is equivalent to the maximum of the log-likelihood. 

There is a scaling indeterminacy between $\bm{W}$ and $\bm{H}$ in the NMF model. We fix this by setting the $L_1$ norm of each column of $\bm{W}$ to 1. The background fluorescence in the images is modelled as a `flat' component $\bm{w}_K = \bfone/N$ with corresponding intensity $\bm{h}_K$. The spatial part of the background component $\bm{w}_K$ is not updated during the optimisation, while the temporal part $\bm{h}_k$ is updated to account for changes in background levels during the data acquisition (due to bleaching, for example).

The NMF model is fitted to data iteratively using multiplicative updates \autoref{eq:NMF classic updates} sequentially. Note that the divergence \autoref{eq:KL divergence} is convex wrt $\bm{H}$ and $\bm{W}$ individually, but not in both variables together \cite{Lee2001}, leading to local optima.

%==========================================
%==========================================

\clearpage
\section{Related work \label{sec:NMF related}}
This section points to published work relevant to NMF application to the QD data. An algorithm for NMF with sparsity constraints is reviewed and demonstrated on simulated data in \autoref{sub:Hoyer}. \Autoref{sub:ICA} discusses the proposed independent component analysis (ICA) as a model for QD data and \autoref{sub:RL deconvolution} shows a link between NMF and a standard Richardson-Lucy deconvolution.
%==========================================

\subsection{Hoyer's sparse NMF \label{sub:Hoyer}}
The in-focus PSF (see \autoref{fig:Simulted-PSF-different-focal-depths}, left) is a fairly compact structure with only few pixels with significant values. Constraints on sparsity of the estimated $\bm{w_k}$s would likely facilitate the estimation of the credible sources and might lead to a faster convergence to a better local minimum. 


NMF with explicit sparsity constraints has been developed by Hoyer \cite{Hoyer2004}. The ``sparsity'' of a vector $\bm{x}$ was defined as 
%
\begin{equation}
	s(\bm{x})=\frac{\sqrt{n}-L_{1}/L_{2}}{\sqrt{n}-1},
	\label{eq:Hoyers sparsity}
\end{equation}
%
where $L_1=\sum_i|x_i|$, $L_2=\sqrt{\sum_i x^2_i}$ and $n$ is the dimensionality of the vector $\bm{x}$.

Specific fixed constraints on the sparsity of the columns of $\bm{W}$ can be imposed during the optimisation. After each iteration, the columns $\bm{w}_k$s of the estimated matrix $\bm{W}$ are projected to be non-negative, have unchanged $L_{2}$ norm, but $L_{1}$ norm set to achieve the desired sparseness \autoref{eq:Hoyers sparsity}.

Note that the assumption that all columns have identical ``sparseness'' might be restrictive when out-of-focus PSFs are present. For example, the in focus PSF in \autoref{fig:Simulted-PSF-different-focal-depths} has Hoyer's sparsity $s=0.83$ while the PSF from $1 \um$ out-of-focus plane has $s=0.4$ and the PSF from $1.8 \um$ out-of-focus plane  has $s=0.1$.

Hoyer's sparse NMF algorithm minimises $\|\bm{D} - \bm{WH}\|^{2}$ rather than the KL-divergence \autoref{eq:KL divergence}. This cost function corresponds to the Gaussian rather than Poisson noise assumption, which can be significant especially for low-intensity images (fast acquisition time, for example). 

\begin{figure}[!htb] %copied from S433_report.tex
	\newcommand{\sizefig}{.9}
	\centering
	\subfloat[$10 \unit{sources/\mu m^{2}}$]{	
	\includegraphics[scale=\sizefig]{\qd S429/figures/resN_simiter1_1to14}}\\
%	\subfloat[$20 \unit{sources/\mu m^{2}}$]{	
%	\includegraphics[scale=\sizefig]{\qd S430/figures/resN_simiter1_1to14}}\\
	\subfloat[$30 \unit{sources/\mu m^{2}}$]{	
	\includegraphics[scale=\sizefig]{\qd S431/figures/resN_simiter1_1to14}}\\
%	\subfloat[$40 \unit{sources/\mu m^{2}}$]{	
%	\includegraphics[scale=\sizefig]{\qd S432/figures/resN_simiter1_1to14}}\\
	\subfloat[$50 \unit{sources/\mu m^{2}}$]{	
	\includegraphics[scale=\sizefig]{\qd S433/figures/resN_simiter1_1to14}}\\	
	\caption{$\bm{W}$ estimated with Hoyer's algorithm with no sparsity constraints. This corresponds to conventional NMF. Evaluation of the simulated data of randomly scattered sources with different densities. Shown first 14 estimated components.}
	\label{fig: Hoyer no sparsity constraint}
\end{figure}

\begin{figure}[htbp!] %copied from S433_report.tex
	\newcommand{\sizefig}{.9}
	\centering
	\subfloat[$10 \unit{sources/\mu m^{2}}$]{	
	\includegraphics[scale=\sizefig]{\qd S429/figures/res07_simiter1_1to14}}\\
%	\subfloat[$20 \unit{sources/\mu m^{2}}$]{	
%	\includegraphics[scale=\sizefig]{\qd S430/figures/res07_simiter1_1to14}}\\
	\subfloat[$30 \unit{sources/\mu m^{2}}$]{	
	\includegraphics[scale=\sizefig]{\qd S431/figures/res07_simiter1_1to14}}\\
%	\subfloat[$40 \unit{sources/\mu m^{2}}$]{	
%	\includegraphics[scale=\sizefig]{\qd S432/figures/res07_simiter1_1to14}}\\
	\subfloat[$50 \unit{sources/\mu m^{2}}$]{	
	\includegraphics[scale=\sizefig]{\qd S433/figures/res07_simiter1_1to14}}\\	
	\caption{Sparsity constraints $s=0.7$ on $\bm{W}$ estimated with Hoyer's algorithm from simulated data of randomly scattered sources with different densities. Shown first 14 estimated components.}
	\label{fig: Hoyer sparsity 0.7}
\end{figure}

We used simulated data of randomly scattered sources with densities $10-50\,\um^{-2}$ to explore the ability of the Hoyer's algorithm to recover credible sources. The blinking intensity was assumed to be uniformly distributed on the interval $[0, 5000]$ photons. The background was set to $100$ photons/pixel and data were corrupted with Poisson noise. Prior to the evaluation with the Hoyer's algorithm, the true background value was subtracted from the data, clipping negative pixels to zero. The number of components $K$ was set to the true value used for simulation $K=K_{true}$. The algorithm was run for 1000 iterations. Running the algorithm for longer (2000, 5000) iterations did not improve the estimated results. 

\Autoref{fig: Hoyer no sparsity constraint} shows estimated $\bm{W}$ with Hoyer's algorithm without sparsity constraints. This corresponds to conventional NMF. Note that most of the $\bm{w_k}$ for higher densities contain multiple sources \autoref{fig: Hoyer no sparsity constraint}\bbb,\ccc. Imposing the sparsity constraints $s=0.7$ on the columns of $\bm{W}$, estimated from the true PSF, gives better estimated sources for densities$<30\um^{-2}$  \autoref{fig: Hoyer sparsity 0.7}\aaa, however, for dense data the method fails to recover the individual sources and gives unsatisfactory results, see \autoref{fig: Hoyer sparsity 0.7}\ccc.

%==========================================

\subsection{Independent component analysis\label{sub:ICA}}
\begin{figure}[!htb] % this figure is copied from ~/DTC/paper/NMFLM.tex
	\condcomment{\boolean{includefigs}}{
	\newcommand{\sizefig}{.4}
	\centering
	\subfloat[NMF (noise)]{
	\begin{tabular}{c}
		\includegraphics[scale=\sizefig]{\qd S310/images/nmf_noise_loc_c1} \tabularnewline
		\includegraphics[scale=\sizefig]{\qd S310/images/nmf_noise_loc_c2}\tabularnewline
	\end{tabular}}
	\subfloat[ICA (noise)]{
	\begin{tabular}{c}
		\includegraphics[scale=\sizefig]{\qd S310/images/ica_noise_loc_c1} \tabularnewline
		\includegraphics[scale=\sizefig]{\qd S310/images/ica_noise_loc_c2}\tabularnewline
	\end{tabular}}	
	\subfloat[ICA (noise free)]{
	\begin{tabular}{c}
		\includegraphics[scale=\sizefig]{\qd S310/images/ica_noNoise_loc_c1} \tabularnewline
		\includegraphics[scale=\sizefig]{\qd S310/images/ica_noNoise_loc_c2}\tabularnewline
	\end{tabular}}}
	\caption{Comparison of the components separated with (a) NMF and (b) ICA for	simulated noisy data of two blinking QDs separated by $0.5 \unit{pixel}$ (which corresponds to $50\unit{nm}$ or $\lambda/12$). (c) ICA for noise-free data.  Blue pixels contain negative values. The true and the estimated positions are shown as red circles and green crosses, respectively. The airy disk is shown as a green circle (radius $333\unit{nm}$).}
	\label{fig:Comparison of NMF and ICA}
\end{figure}

The independent component analysis (ICA) algorithm \cite{Hyvarinen2000} has been used for separating the overlapping QDs \cite{Lidke2005}. ICA allows each source to have a different individual PSF.  However, the ICA model allows negative entries in the individual PSFs and does not account for noise in the measured data, which can make recovery of the individual sources difficult in realistic noise levels (see \autoref{fig:Comparison of NMF and ICA}). 

\Autoref{fig:Comparison of NMF and ICA}\bbb shows results from $10^{3}$ simulated frames containing two sources with blinking intensity uniformly distributed on the interval $[0, 1500]$ photons and with background $100$ photons/pixel. The true background level was subtracted (clipping any negative values to zero) prior to the ICA evaluation. We used {\tt `tanh'} as a nonlinearity option in the fixed-point algorithm \cite{Hyvarinen2000}, and the number of sources was set to $K=2$. 


%==========================================

\subsection{Richardson-Lucy deconvolution \label{sub:RL deconvolution}}
%This is from DecovolutionNotes.lyx file.
There is a link between NMF and classical Richadson-Lucy deconvolution. An observed ``blurred'' (diffraction limited) image $\bm{i}$ ($N\times1$ vector) can be expressed as a (discretised) convolution 
%
\begin{equation} 
	i_x=\sum_{j=1}^{N}o_{j}w_{x-j}, 
\end{equation} 
%
where $\bm{o}$ ($N\times1$) is the original (unblurred) object which represents locations and intensities of fluorescent sources. $\bm{w}$ ($N\times1$) is an image of point spread function (PSF) centred in the middle of the image. Richardson \cite{Richardson1972} and Lucy \cite{Lucy1974} published an iterative deconvolution technique for astronomical images with known PSF. They used Bayes theorem as a `hint' for an iterative update of $\bm{o}$. This update is usually referred to as Richardson-Lucy (RL) deconvolution algorithm and is identical to the Lee-Seung NMF update with generalised KL-divergence objective function \cite{Lee2001}. 

Holmes \cite{Holmes:92} derived the RL updates based on maximum likelihood estimation of the model with Poisson noise using the expectation-maximisation algorithm. He also proposed an update for $\bm{w}$ so that the method can be used as a blind deconvolution algorithm (PSF is not known). This is sometimes referred to as a `blind RL algorithm'. 

The updates for $\bm{o}$ and $\bm{w}$ are technically identical to the Lee and Seung NMF updates (KL divergence as an objective function). However, (blind) RL deconvolution solves a different problem than NMF. RL deconvolution estimates one PSF ($\bm{w}$), which is shared by all sources. The deconvolution is performed for each frame separately, independent on the rest of the dataset. NMF models the whole dataset as a collection of individual (and in general different) PSFs (columns of $\bm{W}$) each changing intensity over time (rows of $\bm{H}$). While one source which appears in $n$ different frames is treated as $n$ different individual sources by RL, NMF can identify it as a single source. 

Modified updates imposing radial symmetry constraints on the PSF were also proposed. There exist several modified updates derived using EM algorithm which impose some constraints on $\bm{o}$ or $\bm{w}$. Joshi \cite{Joshi:93} gives updates, where Good's roughness measure ($\int\frac{\left|\nabla f(x)\right|^{2}}{f(x)}dx$) on the original image $\bm{o}$ is used as a regularisation term. This biases the solution towards the `smooth' images and avoids speckle artefacts in the reconstructions, that are sometimes experienced in deconvolution methods. 

Fish et al. \cite{Fish1995} use RL `blind' algorithm (updates on both $\bm{o}$ and $\bm{w}$) but after some number of iterations they fit an approximation of the PSF to the estimated $\bm{w}$ and use this fit as a new $\bm{w}$. They claim that in noisy images this `semi-blind' deconvolution can perform better than the one with known PSF. The comparison of the regularised RL versions and some other deconvolution techniques has been shown in \cite{Kempen1997BA,Verveer1999}. RL usually performs well for noisy images.

%==========================================
%==========================================

\clearpage
\section{Simulations \label{sec:simulations}}
In this section we describe the generation of the simulated datasets used in testing the performance of the algorithm in different experimental settings.
%Main competitors 3B \cite{Cox2011}, CSSTORM \cite{Zhu2012} and SOFI \cite{Dertinger2009}.	

The parameters of the simulations were chosen to correspond to real experimental data with quantum dots (QD635, \emph{Invitrogen}).  \Autoref{tab:Simulations parameters} summarises the main simulation parameters. Radius of an Airy disk (classical resolution limit) for parameters from \autoref{tab:Simulations parameters} is $\delta=293 \unit{nm}$. FWHM of a Gaussian approximation of the in-focus PSF is $260\unit{nm}$ ($\sigma = 111\unit{nm}$) \cite{Zhang2007}.
%
\begin{table}[!h]
	\centering
	\begin{tabular}{|c|c|l|}
		\hline 
		\bf Parameter 		& \bf Note  							& \bf Value\tabularnewline
		\hline %\hline 
		$\lambda_{em}$ 	& wavelength of the emission light 	& 625 nm\tabularnewline
%		\hline 
		NA 				& numerical aperture 				& 1.3\tabularnewline
%		\hline 
		RI 				& refraction index 					& 1.5\tabularnewline
%		\hline 
		pixel-size 			& size of a pixel in image plane	& 80 nm\tabularnewline
%		\hline 
		$T$ 				& number of frames  					& $50-1000$\tabularnewline
%		\hline 
		$\unit{mean}(n_{phot})$ & mean number of photons / source / frame 	& 2500\tabularnewline
%		\hline 
		$\max(n_{phot})$ 	& max number of photons / source / frame 	& 5000\tabularnewline
%		\hline 
		$b$ 				& background \# of phot/pixel/frame 			& 100\tabularnewline
%		\hline
		noise			& 	{\centering -}				& Poisson\tabularnewline
		\hline
	\end{tabular}
	\caption{Main parameters for data simulations.}
	\label{tab:Simulations parameters}
\end{table}

The blinking behaviour of the QDs was simulated as either:
%
\begin{enumerate}
	\item
	Uniform random number between $0$ and $\max(n_{phot})$.
	\item
	Telegraph process with switching rate $\gamma$. Difference between the sampling rate of the experiment and the blinking of the fluorophores is considered by simulating the blinking behaviour on the oversampled time axes followed with averaging over several bins (see \autoref{sub:results - blinking behaviour}). 
\end{enumerate}
 
%==========================================
\subsection{Randomly scattered sources\label{sub:Simul random}}
The ability of the algorithm to separate individual overlapping sources was tested on simulated data of randomly scattered fluorophores. The density of the sources was in a range $\rho=10-50 \unit{sources/\mu m^{2}}$. This density range corresponds to $\sim 3-14$ sources in an Airy disk, respectively, for parameters from \autoref{tab:Simulations parameters}. Several frames of the simulated dataset for four different densities is shown in  \autoref{fig:simulated data random}. The mean projection of the frames, which corresponds to a wide-field image, is shown in \autoref{fig:simulated data random - mean}.

\begin{figure}[!htb]	
	\newcommand{\widthfig}{1\textwidth}
	\centering	
	\subfloat[density $10\um^{-2}$ ($14$ sources)]{
	\includegraphics[width=\widthfig]{\qd S455/images/dpixc_1to8_dens10}}
	
	\subfloat[density $30\um^{-2}$ ($43$ sources)]{
	\includegraphics[width=\widthfig]{\qd S455/images/dpixc_1to8_dens30}}
	
	\subfloat[density $50\um^{-2}$ ($72$ sources)]{
	\includegraphics[width=\widthfig]{\qd S455/images/dpixc_1to8_dens50}}		
	\caption{First eight frames of simulated randomly scattered sources with density $10-50\um^{-2}$. The area of the frame is $1.2\times1.2\unit{\mu m}$ ($15\times15$ pixels)}
	\label{fig:simulated data random}
\end{figure} 
%
\begin{figure}[!htb]	
	\newcommand{\widthfig}{.25\textwidth}
	\centering	
	\subfloat[density $10\um^{-2}$]{
	\includegraphics[width=\widthfig]{\qd S455/images/dpixc_mean_dens10}}\hspace{.3cm}	
	\subfloat[density $30\um^{-2}$]{
	\includegraphics[width=\widthfig]{\qd S455/images/dpixc_mean_dens30}}\hspace{.3cm}	
	\subfloat[density $50\um^{-2}$]{
	\includegraphics[width=\widthfig]{\qd S455/images/dpixc_mean_dens50}}		
	\caption{Mean projection of the simulated frames \autoref{fig:simulated data random}. The sources' positions are marked with red dots.}
	\label{fig:simulated data random - mean}
\end{figure} 

%==========================================
\clearpage
\subsection{Artificial structure\label{sub:Simul hash}}
% 
\cut{confusion with densities -> in simulations the density is in pixels^{-1} but here is in um^{-1}}
\begin{figure}[!htb]	
	\newcommand{\widthfig}{.9\textwidth}
	\centering	
%	\subfloat[linear density $0.3\um^{-1}$ ]{
%	\includegraphics[width=\widthfig]{\qd S569/figures/dpixc_1to8_dens3}}
	
	\subfloat[linear density $\mu=7.5\um^{-1}$, distance between lines $d=150\unit{nm}$]{
	\includegraphics[width=\widthfig]{\qd S569/figures/dpixc_1to8_dens6}}
	
	\subfloat[linear density $15\um^{-1}$, distance between lines $d=150\unit{nm}$]{
	\includegraphics[width=\widthfig]{\qd S569/figures/dpixc_1to8_dens12}}		
	
	\subfloat[linear density $12.5\um^{-1}$, distance between lines $d=100\unit{nm}$]{
	\includegraphics[width=\widthfig]{\qd S575/figures/dpixc_1to8_dens10}}		

	\caption{First eight frames of the simulated dataset. The area of the frame is $1.7\times1.7\um$ ($21\times21$ pixels).}
	\label{fig:simulated data hash}
\end{figure} 

A dataset with sources arranged in a shape of a hash symbol (\#) was used for testing the algorithm to recover structural details in the sample. The vertical parallel lines were aligned with the pixels grid, the horizontal lines were slightly tilted to investigate the possible effect caused by the geometrical configuration of the sources with respect to the pixel grid. 
%
\begin{figure}[!htb]	
	\newcommand{\widthfig}{.3\textwidth}	
	\centering	
%	\subfloat[$\mu=0.3\um^{-1}$]{
%	\includegraphics[width=\widthfig]{\qd S569/figures/dpixc_mean_dens3__bar04um}}\hspace{.3cm}	
	\subfloat[$\mu=7.5\um^{-1}$, $d=150\unit{nm}$]{
	\includegraphics[width=\widthfig]{\qd S569/figures/dpixc_mean_dens6_bar04um}}\hspace{.3cm}		
	\subfloat[$\mu=15\um^{-1}$, $d=150\unit{nm}$]{
	\includegraphics[width=\widthfig]{\qd S569/figures/dpixc_mean_dens12_bar04um}}\hspace{.3cm}	
	\subfloat[$\mu=12.5\um^{-1}$, $d=100\unit{nm}$]{
	\includegraphics[width=\widthfig]{\qd S575/figures/dpixc_mean_dens10_bar04um}}		
	\caption{Sum projection of the simulated frames \autoref{fig:simulated data hash}. The sources' positions are marked with red dots. Scale bar $400 \unit{nm}$.}
	\label{fig:simulated data hash - mean}
\end{figure} 
%
The distance $d$ between the parallel lines and the linear density of the sources $\mu$ were two main parameters of the structure. The brightness and the background values are shown in \autoref{tab:Simulations parameters}. \Autoref{fig:simulated data hash} shows several frames of the simulated dataset for different linear densities $\mu$ and distances between parallel lines $d$. The distance $d=150\unit{nm}$ corresponds to the half of the Airy disk radius. The mean projections of the frames are shown in \autoref{fig:simulated data hash - mean}.  

%==========================================
%==========================================

\clearpage
\section{Evaluation of the results\label{sec:evaluation}}

The performance of the algorithm applied on a simulated dataset can be quantitatively measured, because the true locations of the sources are known. We used several measures to compare the algorithm performance on simulated datasets consisting of randomly scattered in-focus PSFs (\autoref{fig:simulated data random}). 

\begin{figure}[!h]
	\centering
	\includegraphics[width=1.0\textwidth]{figures/TPFNFPillustration/TPFNFPs}
	\caption{True positives TP, false positives FP and false negatives FN illustration. A red dot represents the true location with a circle of radius $r$, a green cross denotes an estimated position.}
	\label{fig:TPFPFN}
\end{figure}

The individual estimated sources $\bm{w_k}$ were localised by ML fitting of a Gaussian approximation of the PSF \cite{Zhang2007}. We used a greedy algorithm to assign the estimated locations ($E$) to their nearest true positions ($T$). Only one estimated position was assigned to each true position. If the distance between the estimated and the true position was smaller than a threshold $r$, then the source was consider as a true positive (TP). A true position with no estimated source within a disk of radius $r$ was false negative (FN), whereas an estimated position further than $r$ from any true position was considered as false positive (FP), see \autoref{fig:TPFPFN}. $M$ estimated sources in the proximity of one true source are counted as $1$TP and $(M-1)$FP (\autoref{fig:TPFPFNcombi}, left). One estimated source in proximity of $M$ true sources gives $1$TP and $(M-1)$FN (\autoref{fig:TPFPFNcombi}, right).

We set the threshold $r=\sigma/2$, where $\sigma=\frac{\sqrt{2}}{2\pi}\frac{\lambda_{em}}{NA}$ is the standard deviation of the in-focus PSF Gaussian approximation \cite{Zhang2007}. For the parameters used in our simulations (see \autoref{tab:Simulations parameters}) the threshold corresponds to $r=56\unit{nm}$ ($0.7\unit{pixels}$). 

\begin{figure}[!h]
	\centering
	\includegraphics[width=1.0\textwidth]{figures/TPFNFPillustration/TPFNFPs_comb}
	\caption{There is only one estimated position assigned to each true position. Two estimated sources in the proximity of one true source are counted as $1$TP and $1$FP (left). One estimated source in proximity of two true sources gives $1$TP and $1$FN (right)}
	\label{fig:TPFPFNcombi}
\end{figure}

%Localisation error was estimated as an average distance between true locations and all estimated locations classified as TP. 
The estimated density of the sources was counted as the number of all TP divided by the area of the image.

The average precision (AP) \cite{Salton1986,Everingham2009} was used to summarise both localisation precision and ability to recover the individual sources. The estimated positions $e_k$ are ranked according to the square root of the source's mean brightness
%
\begin{equation}
	b_k=\sqrt{\bar{N_k}}
\end{equation}
%
because the Cramer-Rao lower bound on localisation precision scales as $1/\sqrt{N}$, where $N$ is a number of emitted photons (see \autoref{ch:Theoretical-limits-of the LM} for details). For results of the NMF evaluation is $N$ retrieved from the matrix $\bm{H}$ as a mean along the rows
%
\begin{equation}
	\bar{N_k}=\underset{t}{\unit{mean}}(h_{kt}).
\end{equation}

The interval $[l_{min},l_{max}]$ between the dimmest $l_{min}$ and the brightest $l_{max}$ source intensity is divided into number of intervals (confidence levels) $l_{i}$ defined by the steps in the sorted intensities of all sources. For each confidence level $l_i$ only the sources with $b_k$ above $l_i$ are considered. True positives ($\unit{TP}_{i}$), false negatives ($\unit{FN}_{i}$) and false positives ($\unit{FP}_{i}$) are computed for each confidence level $l_{i}$.

Precision $P$ and recall $R$ are computed from $\unit{TP}(l_{i})$, $\unit{FP}(l_{i})$ and $\unit{FN}(l_{i})$ for each confidence level $l_i$:
%
\begin{align} \label{eq:TP,FN} 
	P(l_i)& = \frac{\unit{TP}(l_i)}{\unit{TP}(l_i)+\unit{FP}(l_i)},\\
	R(l_i)& = \frac{\unit{TP}(l_i)}{\unit{TP}(l_i)+\unit{FN}(l_i)}.
\end{align}
%
An example of precision $P(l_{i})$ and recall $R(l_{i})$ curves for different confidence levels is shown in \autoref{fig:PRcurve}\aaa. 
%
\begin{figure}[!h]
	\newcommand{\widthfig}{.5\textwidth}
	\newcommand{\sizefig}{.4}
	\centering
	\subfloat[]{
	\includegraphics[scale=\sizefig]{\qd S407/images/PRconfidence}}
	\subfloat[]{
	\includegraphics[scale=\sizefig]{\qd S407/images/PR_Pinterpol}}
	\caption{(a) Example of the precision $P(l_{i})$ (blue) and recall $R(l_{i})$ (green) curve. (b) The precision/recall curve $P(R)$ (blue) with interpolated precision $P_{interp}(\tilde{R}$) (red).}
	\label{fig:PRcurve}
\end{figure}

Following \cite{Everingham2009}, the precision/recall (PR) curve $P(R)$ (\autoref{fig:PRcurve}\aaa) is interpolated for $11$ equally spaced recall levels $\tilde{R}_{i}\in[0:.1:1]$ by taking the maximum precision for which the corresponding recall exceeds $\tilde{R}_{i}$ (\autoref{fig:PRcurve}\bbb):
%
\begin{equation}
	P_{interp}(\tilde{R})=\max_{R;R\geq \tilde{R}}P(R).
\end{equation}
%
The precision/recall (PR) curve is interpolated in order to reduce the impact of ``wiggles'' in the PR curve (see \autoref{fig:PRcurve}\bbb). Note that to obtain a high AP, the method must have precision at all levels of recall, penalising methods that can accurately estimate only few very bright sources. 
% ``wiggles'' in the precision/recall curve, caused by small variations in the ranking of examples. 

Average precision (AP) is a mean of interpolated precision:
%
\begin{equation}
	AP=\frac{1}{11}\sum_{\tilde{R}}{P_{interp}(\tilde{R})}.
	\label{eq:AP}
\end{equation}

%==========================================
%==========================================

\clearpage
\section{NMF for realistic microscopy datasets \label{sec:NMF-for-real}}

NMF becomes challenging when applied to a dataset with large number ($\sim 10^3$) of images, each containing more than $10^4$ pixels and more than $10^2$ QDs. Beside an excessively large computational time, the local minima in NMF fitting complicate the optimisation \cite{Kim2008}. 

We address this partly by dividing the data into overlapping patches, so that NMF is applied to each patch individually (see \autoref{sec:preproc}).  In the end, the results from the patches are ``stitched'' back together. 

We also develop methods to reduce local optima problems in the fitting procedure (\inmf{} algorithm discussed \autoref{sub: Iterative restarts}). 

The results of the NMF can be used in two different ways. The separated individual sources $\bm{w}$ can be localised and the estimated positions can be used either directly or to create a sub-resolution image very much like in the conventional LM techniques (see \autoref{sub:Localisation-and-stitching}). A different approach avoids the localisation step and creates the super-resolution image directly from the estimated $\bm{w}$s by combining their ``squeezed'' versions (see \autoref{sub:visualisation}). 

The whole pipe-line for NMF evaluation of a realistic microscopy dataset is described in this section.  The individual steps of the procedure are illustrated on simulated data. 

%==========================================

\subsection{Pre-processing \label{sec:preproc}}

Raw data are calibrated such that the image intensity corresponds to the photon counts. Each image is divided into patches of $n_x\times n_y$ pixels with $o$ pixels overlap. We usually use $n_x=n_y=25$ and $o = 5$. The overlap has been chosen as the estimated extent of a single in-focus point spread function. Each patch $p$ is reshaped into a $N\times1$ vector ($N=n_x n_y$)  by concatenation of columns. All $T$ frames then create a $N \times T$ data matrix $\bm{D}_{p}$.

To detect patches with low signal, the maximum intensity pixel in the time average of each patch $m_{p}=\max_{i}\left\langle \bm{D}_{p}(i,t)\right\rangle _{t}$ is compared to the maximum intensity pixel of the average of the whole data $m=\max_{i}\left\langle \bm{D}(i,t)\right\rangle _{t}$. The patches with $m_{p}/m< t_{m}$ are considered to contain only weak signal and are not considered for further evaluation. For our evaluation we usually set $t_{m}=0.25.$

%==========================================
\subsection{Estimation of number of sources $K$\label{sub:Estimation-of-number-of-sources}}

NMF model requires prior knowledge about the number of sources $K$ to be separated. Estimation of $K$ is a difficult task for noisy datasets. In preliminary work we explored this on simulated data ($\lambda_{em}=655\unit{nm}$, $NA=1.2$) with NMF model fitted for a range of $K$ values. 
%
\begin{figure}[!ht]
	\centering
	\newcommand{\sizee}{.25}		
	\newcommand{\sizebb}{.6}
	\newcommand{\ima}{$2\delta$} 
	\newcommand{\imb}{$1.5\delta$}
	\newcommand{\imc}{$\delta$}
	\newcommand{\pca}{, PCA}
	\newcommand{\data}{, wide-field}
	\newcommand{\lbd}{, lower bound}
	\newcommand{\mxc}{, max correlation}
	
	\subfloat[\ima \data]{
	\includegraphics[scale=\sizebb]{\qd S300/images/reslocalized_nc11_sc20}}
	\subfloat[\imb \data]{
	\includegraphics[scale=\sizebb]{\qd S300/images/reslocalized_nc11_sc15}}
	\subfloat[\imc \data]{
	\includegraphics[scale=\sizebb]{\qd S300/images/reslocalized_nc11_sc10}}\\
	
	\caption{Sum of the simulated frames. Red marks indicate the locations of the sources. Green circle shows the Airy disk (with radius $\delta$).  Ten sources are randomly distributed on a disk with radius \ima \ (left column), \imb \ (middle column) and \imc \ (right column). The border of the disk in (a) and (b) is marked with red dashed circle.}	
	 \label{fig:K estimation data}
\end{figure}

\Autoref{fig:K estimation data} illustrates three different simulated datasets with 10 sources randomly scattered within an area of radius $2\delta$, $1.5\delta$ and $\delta$, where $\delta$ was equal to the diameter of an Airy disk (diffraction limit), shown as a green circle in \autoref{fig:K estimation data}\ccc\ ($\delta=0.61\frac{\lambda_{em}}{NA}$). This corresponds to the sources densities of $2.4$, $4.4$ and $10$ sources per Airy disk or $7$, $13$ and $29$ sources per $\um^{2}$, respectively. The mean of the simulated frames, which corresponds to a wide-field image, is shown as a grey-value image. Red marks indicate the true positions of the sources. Ten datasets with different geometrical configurations of the randomly scattered sources were simulated for each source density. 

\begin{figure}[!htb]
	\centering
	\newcommand{\sizee}{.24}		
	\newcommand{\sizebb}{.6}
	\newcommand{\ima}{$2\delta$} 
	\newcommand{\imb}{$1.5\delta$}
	\newcommand{\imc}{$\delta$}
	\newcommand{\pca}{, PCA}
	\newcommand{\data}{, data}
	\newcommand{\lbd}{, lower bound}
	\newcommand{\mxc}{, max correlation}
	
%	\subfloat[\ima \data]{
%	\includegraphics[scale=\sizebb]{\qd S300/images/reslocalized_nc11_sc20}}
%	\subfloat[\imb \data]{
%	\includegraphics[scale=\sizebb]{\qd S300/images/reslocalized_nc11_sc15}}
%	\subfloat[\imc \data]{
%	\includegraphics[scale=\sizebb]{\qd S300/images/reslocalized_nc11_sc10}}\\
	\begin{tabular}{ccc}
		\subfloat[\ima \pca]{
		\includegraphics[scale=\sizee]{\qd S301/images/pca_nc20}}&
		\subfloat[\imb \pca]{
		\includegraphics[scale=\sizee]{\qd S301/images/pca_nc15}}&
		\subfloat[\imc \pca]{
		\includegraphics[scale=\sizee]{\qd S301/images/pca_nc10}}\tabularnewline
		%	
		%	\subfloat[\ima \llk]{
		%	\includegraphics[scale=\sizee]{\qd S303/images/LogLikPoisson_nc20}}
		%	\subfloat[\imb \llk]{
		%	\includegraphics[scale=\sizee]{\qd S303/images/LogLikPoisson_nc15}}
		%	\subfloat[\imc \llk]{
		%	\includegraphics[scale=\sizee]{\qd S303/images/LogLikPoisson_nc10}}\\
		
		\subfloat[\ima \lbd]{
		\includegraphics[scale=\sizee]{\qd S303/images/LowerBound_nc20}}&
		\subfloat[\imb \lbd]{
		\includegraphics[scale=\sizee]{\qd S303/images/LowerBound_nc15}}&
		\subfloat[\imc \lbd]{
		\includegraphics[scale=\sizee]{\qd S303/images/LowerBound_nc10}}\tabularnewline
		
		\subfloat[\ima \mxc]{
		\includegraphics[scale=\sizee]{\qd S303/images/MaxCorrInResid_nc20_NMF}}&
		\subfloat[\imb \mxc]{
		\includegraphics[scale=\sizee]{\qd S303/images/MaxCorrInResid_nc15_NMF}}&
		\subfloat[\imc \mxc]{
		\includegraphics[scale=\sizee]{\qd S303/images/MaxCorrInResid_nc10_NMF}}
	\end{tabular}
	%
	\caption{$K$ estimation for $10$ sources contained within a disk with radius \ima \ ({\it left column}), \imb \ ({\it middle column}) and \imc \ ({\it right column}). Lines for three datasets with different configuration of the sources are shown. $K_{true}$ is marked with red vertical line.}	\label{fig:K estimation}
\end{figure}

The likelihood of the model \autoref{eq:NMF model likelihood} is increasing with higher $K$, because the noisy data can always be fitted better with a model containing higher number of components ($K$). The Bayesian Information Criterion (BIC) \cite{Bishop2006} is a simple model comparison method, adding a penalty term to the likelihood penalising for the $NK$ parameters contained in $\bm{W}$. The models with larger $K$ are therefore more heavily penalised. BIC, however, did not provide satisfactory results. 
We therefore tried to estimate the number of sources $K$ using:
%
\begin{enumerate}
	\item 
	\emph{Principal Component Analysis (PCA)}	
	A crude estimation of $K$ can be obtained from a position of a ``kink'' in the plot of sorted principal values \autoref{fig:K estimation}\aaa-\ccc. However, the ``kink'' is not obvious in the noisy data or data with high density of blinking sources, see \autoref{fig:K estimation}\ccc.
	\item
	\emph{A variational lower bound (LB)} 	
	A variational approximation of the GaP model \autoref{sub:GaP} provides lower bound $\mathcal{L}$ on the likelihood $p(\bm{D}|K, \theta)$ by approximately integrating out the latent variables $\bm{h}_k$ \cite{Buntine2006}. To obtain the marginal likelihood $p(\bm{D}|K)$ it would be necessary to also integrate out $\theta$, but this is computationally challenging. We show in \autoref{fig:K estimation}\ddd-\fff\ that in fact the lower bound already underestimates the value of $K$, so that $p(\bm{D}|K)$ would likely peak at even lower values of $K$ and thus systematically underestimate the number of sources.
	
	\item 
	\emph{Analysis of correlations in residuals (ACR)}	
	An alternative approach for estimating $K$ is to analyse the residuals ``data-model''. The entries of the $N\times T$ residual matrix $\bm{S}$:
	%
	\begin{equation}
		s_{nt}=\frac{d_{nt}-\sum_{k=1}^{K}w_{nk}h_{kt}}{\sqrt{\sum_{k=1}^{K}w_{nk}h_{kt}}}.
	\end{equation}
	%
	The factor $1/\sqrt{\sum_{k=1}^{K}w_{nk}h_{kt}}$ is applied in order to standardise the residuals (zero mean and unit variance) of Poisson distributed data. We can then compute the $N\times N$ correlation matrix 
	%
	\begin{equation}
		\bm{C}_{S}=\bm{SS}^{T},
	\end{equation}
	%
	and the $N\times N$ matrix of the correlation coefficients $\bm{R}_{S}$ with entries 
	%
	\begin{equation}
		r_{ij}=\frac{c_{ij}}{\sqrt{c_{ii}c_{jj}}}.
		\label{eq:Correlation in residuals}
	\end{equation}
	
	Underestimation of the number of sources ($K<K_{true}$) will lead to correlations between some pixels as the model will try to explain multiple sources with one component. For $K\geq K_{true}$ the correlations are expected to drop to a base level and the residuals become uncorrelated. We can pick the value of $K$ for which the maximum of the residual correlations decrease to a certain level and where further increase of $K$ does not give any further improvement \autoref{fig:K estimation}\gggg-\iii.
\end{enumerate}

A reliable estimation of $K$ is a difficult task for higher source densities. \Autoref{fig:K estimation hist} shows the histograms of the estimated $K$s for ten different geometrical configurations of the sources with a given density. From the three methods presented in this section  (\autoref{fig:K estimation hist}\aaa-\ccc), the analysis of the correlations in residuals (ACR) shows the best performance.
%
\begin{figure}[!hbt]
	\newcommand{\sizef}{.4}		
	\centering
	\subfloat[PCA]{
	\includegraphics[scale=\sizef]{\qd S300/images/KestHist_PCA_Kink}}
	\subfloat[Variational lower bound]{
	\includegraphics[scale=\sizef]{\qd S300/images/KestHist_lb}}\\
	\subfloat[Maximum correlations in residuals]{
	\includegraphics[scale=\sizef]{\qd S301/images/KestHist_maxC_Kink}}
	\subfloat[NMF with iterative restarts (\inmf{})]{
	\includegraphics[scale=\sizef]{\qd S560/figures/KestHist_NMFiter}}
	\caption{Histograms of the $K$ estimations ($K_{true}=10$) with (a) PCA, (b) variational lower bound, (c) analysis of correlations in residuals and (d) iterative NMF. Histograms are from the evaluation of simulated data of randomly scattered emitters: ten sources within a disk of $\delta$ (blue), $1.5\delta$ (green) and $2\delta$ (red). Ten different geometrical configurations were simulated for each density.}
	\label{fig:K estimation hist}
\end{figure}

Both LB and ACR require evaluation of the model for a reasonable range of possible $K$s. The range can be estimated from PCA  (\autoref{fig:K estimation hist}\ccc), because the principal coefficients can be computed directly from the data matrix $\bm{D}$.

In the following section we will be discussing an iterative procedure of the NMF algorithm (\inmf{}) which can deal with moderate overestimation of $K$  (estimated from PCA). The correct number of sources can be estimated additionally by analysing the optimised matrix $\bm{W}$ and selecting the ``credible'' sources $\bm{w_k}$. Therefore evaluation for only one overestimated value of $K$, rather than a range of $K$s, is required. The histogram of the $K$s estimated with the iterative algorithm is shown in  \autoref{fig:K estimation hist}\ddd\ for comparison. \inmf{} gives the most accurate estimates. 

%==========================================
\clearpage
\subsection{Tackling local optima in NMF fitting with iterative restarts \label{sub: Iterative restarts}}

Although the Lee and Seung algorithm is convex with respect to $\bm{W}$ and $\bm{H}$ separately, it is non-convex in both simultaneously \cite{Lee2001}. Multiple restarts can be used to address the problem of local optima, but we have not found good solutions with this approach on the QD data. Instead, we exploit some prior knowledge about the problem, namely that PSFs are likely to have a fairly compact structure (see \autoref{fig:Simulted-PSF-different-focal-depths}). As the estimated sources $\bm{w_k}$ are normalised to have the $L_1$ norm equal to one (i.e.,\ $\sum_{j}w_{jk}=1$), we use the inverse $L_2$ norm to rank the columns $\bm{w_k}$'s of the matrix $\bm{W}$. Note that Hoyer's sparsity measure \autoref{eq:Hoyers sparsity} is a $L_1/L_2$ measure normalised to the $[0..1]$ interval \cite{Kim2008}. 

This leads to an iterative NMF algorithm (we denote it as \inmf{}), where on iteration $(j+1)$ the first $j$ sorted sources $\{ \bm{w} \}_1^j$ (and corresponding $\{ \bm{h} \}_1^j$) are used as initial values for the first $j$ columns of $\bm{W}$ (and the corresponding rows of $\bm{H}$). The remaining components are re-initialised from a uniform random distribution. Initial values of $\bm{W}$ and $\bm{H}$ for the $(j+1)$th iteration are therefore composed of the $j$ ``sparsest'' components of the previous iteration and $(K-j)$ randomly initialised components. The procedure runs until $j=K$. The \inmf{} algorithm is summarised in \autoref{alg:restarts}. 

We used a crude over-estimation of $K$ with PCA because it can be computed directly from data $\bm{D}$ prior to the evaluation: 
%
\begin{enumerate}
	\item
	We compute the sorted principal coefficients $\lambda_{j}$ of $\bm{D}$ ($\lambda_{1}>\lambda_{2}>...$). 
	\item
	$K$ is (over) estimated as the number of components which satisfy $\lambda_{j}/\lambda_{1}>t_{PCA}$, where $t_{PCA}$ is a threshold. 
\end{enumerate}
%
User should be able to test the source estimation procedure on a patch where the number of sources can be guessed (e.g. an area with sparse sources) to get a notion about the threshold. The threshold $t_{PCA}$ should be set such that it slightly overestimates the true number of sources.

\begin{algorithm}[hbt]
	\caption{Iterative restarts of the NMF (\inmf{} algorithm).}	
	\label{alg:restarts}
	\begin{enumerate}
		\item Set $\bm{W}_{init}$ and $\bm{H}_{init}$ as random positive matrices.
		\item Iterate for $j=1:K$ ($K$ is the (over) estimated number of sources.)
		\begin{enumerate}
			\item Run NMF with $\bm{W}_{init}$ and $\bm{H}_{init}$ as initial values.
			\item Sort columns of $\bm{W}$ according to $L_2$ norm and permute rows of $\bm{H}$ correspondingly.
			\item Replace first $j$ columns of $\bm{W}_{init}$ with first $j$ columns of sorted $\bm{W}$.
			\item Replace last $j+1:K$ columns of $\bm{W}_{init}$ with positive random vectors.
			\item Replace first $j$ rows of $\bm{H}_{init}$ with first $j$ rows of sorted $\bm{H}$.
			\item Replace last $j+1:K$ rows of $\bm{H}_{init}$ with positive random vectors.
		\end{enumerate}
	\end{enumerate}    
\end{algorithm}

The motivation for the iterative procedure \autoref{alg:restarts} is to progressively exploit the credible (and therefore sparse) components from the data while keeping full flexibility of the NMF. It should be noted that in contrast to the Hoyer's sparse NMF (\autoref{sub:Hoyer}), where the ``sparsity'' on the $\bm{w_k}$ is imposed as a ``hard'' constraint, \inmf{} leads to a ``soft'' enhancement of the $\bm{w_k}$'s sparsity. The sparse components are preferably reused in the following iterative restarts but are still allowed to change during the iterations. 

\inmf{} is illustrated on simulated data of slanted line with eight attached PSFs in \autoref{fig:Iterative restarts}. The parameters of the simulations are discussed in  \autoref{sec:results} with illustration of typical frames of the dataset \autoref{fig:Data-true-estimations}\aaa\ and true sources \autoref{fig:Data-true-estimations}\bbb.

The algorithm was run for $K=15$, with the last component reserved for background. The results of the first run (random initialisation) are shown in \autoref{fig:Iterative restarts}\aaa. The individual PSFs (see \autoref{fig:Data-true-estimations}\bbb) are spread across all $\bm{w}$s, and many of them contain a mixture of multiple PSFs. This is a typical solution corresponding to a local minimum of the objective function \autoref{eq:KL divergence}. As the iterative procedure progresses, realistic sources are gradually recovered, see \autoref{fig:Iterative restarts}\bbb. After eight iterations, the first eight $\bm{w}$'s show credible PSFs, while the rest represent only noise, see \autoref{fig:Iterative restarts}\ccc. Further iterations do not have a significant effect on the already estimated PSFs, see \autoref{fig:Iterative restarts}\ddd. 

\begin{figure}[!htb]
	\newcommand{\widthfig}{.95\textwidth}
	\newcommand{\barspace}{-.7cm}
	\condcomment{\boolean{includefigs}}{ 
	\centering
		
		\subfloat[run 1 ]{
		\begin{tabular}{l}
			\includegraphics[width=\widthfig]{\qd S382/images/w_restart0_1toEnd}\vspace{\barspace}\tabularnewline
			\includegraphics[width=\widthfig]{\qd S382/images/w_restart0_1toEnd_intBars}
			\tabularnewline
		\end{tabular}}
		
		\subfloat[run 4 ]{
		\begin{tabular}{l}
			\includegraphics[width=\widthfig]{\qd S382/images/w_restart3_1toEnd}\vspace{\barspace}\tabularnewline
			\includegraphics[width=\widthfig]{\qd S382/images/w_restart3_1toEnd_intBars}
			\tabularnewline
		\end{tabular}}
		
		\subfloat[run 8 ]{
		\begin{tabular}{l}
			\includegraphics[width=\widthfig]{\qd S382/images/w_restart7_1toEnd} \vspace{\barspace}\tabularnewline
			\includegraphics[width=\widthfig]{\qd S382/images/w_restart7_1toEnd_intBars}
			\tabularnewline
		\end{tabular}}
		
		\subfloat[run 14 ]{
		\begin{tabular}{l}
			\includegraphics[width=\widthfig]{\qd S382/images/w_restart13_1toEnd} \vspace{\barspace}\tabularnewline
			\includegraphics[width=\widthfig]{\qd S382/images/w_restart13_1toEnd_intBars}\tabularnewline
		\end{tabular}}		
	}
	\caption{Illustration of the iterative restart procedure. Estimated sources after (a) 1, (b) 4, (c) 8 and (c) 14 runs of the algorithm. Bars below the figures show the maximum of the intensity image $\bm{w}$.}
	\label{fig:Iterative restarts}
\end{figure}

The \inmf{} procedure leads to better local minima of the NMF optimisation problem. The $L_{2}$ norm sorting of the recovered $\bm{w}$s after each iteration (step 2a in \autoref{alg:restarts}) ensures that the sparsest components will be reused in subsequent evaluation. Gradually increasing number of sparse $\bm{w}$s with small $L_{2}$ norm is reused in subsequent restarts (step 2c in \autoref{alg:restarts}), whereas the $\bm{w}$s with large $L_{2}$ norm replaced by a random vector after each run (step 2d in \autoref{alg:restarts}). This ``soft'' sparsity enhancement allows for higher flexibility of the evaluated $\bm{w}$s and allows recovery of the sources with different individual sparsities such as the sources from different focal depths shown in \autoref{fig:Iterative restarts}\ddd. This is an advantage compared to the ``hard'' sparsity enhancement used in the Hoyer's algorithm (\autoref{sub:Hoyer}).

The ``good'' sources, representing individual PSFs, can be identified after the termination of \inmf{} by analysing the resulting $\bm{W}$. In \autoref{fig:Iterative restarts}\ddd only first eight $\bm{w}$s show the ``credible'' PSFs, while rest of the $\bm{w}$s represent only noise (except for the last one, which models the homogeneous background offset).

\begin{figure}[!htb]
	\newcommand{\widthfig}{.95\textwidth}
	\newcommand{\barspace}{-.7cm}
	\condcomment{\boolean{includefigs}}{ 
	\centering
	\subfloat[$K=15$]{
	\begin{tabular}{l}
		\includegraphics[width=\widthfig]{\qd S382/images/w_restart13_1toEnd} \vspace{\barspace}\tabularnewline
		\includegraphics[width=\widthfig]{\qd S382/images/w_restart13_1toEnd_intBars}\tabularnewline
	\end{tabular}}		
		
	\subfloat[$K=30$]{
	\begin{tabular}{l}
		\includegraphics[width=\widthfig]{\qd S382/images/w_nc30_1to15}\vspace{\barspace}\tabularnewline
		\includegraphics[width=\widthfig]{\qd S382/images/w_nc30_1to15intBars}\tabularnewline
		\includegraphics[width=\widthfig]{\qd S382/images/w_nc30_16to30}\vspace{\barspace}\tabularnewline
		\includegraphics[width=\widthfig]{\qd S382/images/w_nc30_16to30intBars}\tabularnewline
	\end{tabular}}	
		
	\subfloat[$K=45$]{
	\begin{tabular}{l}
		\includegraphics[width=\widthfig]{\qd S382/images/w_nc45_1to15}\vspace{\barspace}\tabularnewline
		\includegraphics[width=\widthfig]{\qd S382/images/w_nc45_1to15intBars}\tabularnewline
		\includegraphics[width=\widthfig]{\qd S382/images/w_nc45_16to30}\vspace{\barspace}\tabularnewline
		\includegraphics[width=\widthfig]{\qd S382/images/w_nc45_16to30intBars}\tabularnewline
		\includegraphics[width=\widthfig]{\qd S382/images/w_nc45_31to45}\vspace{\barspace}\tabularnewline
		\includegraphics[width=\widthfig]{\qd S382/images/w_nc45_31to45intBars}\tabularnewline
	\end{tabular}}		
	}
	\caption{\inmf{} evaluation of the simulated dataset \autoref{fig:Data-true-estimations}\aaa\ for different numbers of overestimated sources $K$. Bars below the figures show the maximum of the intensity image $\bm{w}$s multiplied with the mean intensity estimated from the corresponding $\bm{h}$s.}
	\label{fig:iterative restarts robustness}
\end{figure}

\Autoref{fig:iterative restarts robustness} illustrates the ``robustness'' of \inmf{} with respect to the initial number of estimated sources $K$. Resulting $\bm{w}$s of the dataset \autoref{fig:Data-true-estimations}\aaa\ evaluation for initial number of sources set to $K=15$, $30$ and $45$ are shown in \autoref{fig:iterative restarts robustness}\aaa,\bbb\ and \ccc, respectively. In all cases, the eight different PSFs shown in \autoref{fig:Data-true-estimations}\bbb\ were recovered, while the remaining $K-8$ estimated $\bm{w}$s are representing noise (last component models the homogeneous background offset).

\begin{figure}[!htb]
	\newcommand{\widthfig}{.95\textwidth}
	\newcommand{\barspace}{-.7cm}
	\centering
	\begin{tabular}{l}
		\includegraphics[width=\widthfig]{\qd S584/figures/w} \vspace{\barspace}\tabularnewline
		\includegraphics[width=\widthfig]{\qd S584/figures/w_intBars}\tabularnewline
	\end{tabular}
	\caption{Multiple random restarts. Bars below the figures show the maximum of the intensity image $\bm{w}$s multiplied with the mean intensity estimated from the corresponding $\bm{h}$s.}
	\label{fig:random restarts}
\end{figure}

To make a fair comparison with standard NMF, we made $15$ conventional NMF evaluations of the dataset with matrices $\bm{W}$ and $\bm{H}$ initialised with random values every time. \Autoref{fig:random restarts} shows the result of the evaluation with highest likelihood \autoref{eq:NMF model likelihood} (lowest cost function \autoref{eq:KL divergence}). The ``credible'' PSFs are distributed across all the available $\bm{w}$s and several $\bm{w_k}$s ($\bm{w_{1}}$, $\bm{w_{3}}$ and $\bm{w_{5}}$, for example) contain combination of multiple PSFs. Comparison with \autoref{fig:iterative restarts robustness}\aaa\ demonstrates the superiority of the \inmf{} results. 

%==========================================

\subsection{Classification of the estimated sources\label{sub:Classification-of-sources}}

The estimated sources can greatly vary in quality. While some $\bm{w}$'s are credible representation of a PSF, there are often $\bm{w}$'s which contain multiple PSFs while some correspond to background noise due to overestimation of $K$ (\autoref{sub:Estimation-of-number-of-sources}). Sources located close to the patch border, and therefore partially missing, should also be identified. These sources will likely appear in the adjacent patch entirely, because the overlap of the patches is set to approximately the extent of the (in focus) PSF (\autoref{sec:preproc}).

If all the sources are expected to be in-focus and therefore have a fairly compact PSF with one global maximum (left side of \autoref{fig:Simulted-PSF-different-focal-depths}), we can use a simple procedure for identification of reasonable $\bm{w}$s:
%
\begin{enumerate}
	\item
	Each estimated source $\bm{w}$ is convolved with in-focus point spread function (PSF) (generated from the parameters of the experimental setup). This is to smooth the noise in the results and to enhance the structures at the scale of PSF. 
	\item
	The number of local maxima with intensity larger than $50\%$ of the global maximum are counted. The threshold $50\%$ is arbitrary and it reflects our empirical experience that the peaks with intensity less than half of the brightest peak's intensity are not very visible in the scaled image of $\bm{w}$.
\end{enumerate}

Only the sources with one major local maximum of the convolution of PSF and $\bm{w}$ are considered for further evaluation. The distance of the maximum from the edge can indicate a partially missing source. 

\begin{figure}[!htb]
	\newcommand{\fw}{.98\textwidth}
	\newcommand{\barspace}{-.55cm}
	\centering
	\begin{tabular}{l}			
		\includegraphics[width=\fw]{\qd S455/images/resw_1to16_col}\vspace{\barspace}\tabularnewline
		\includegraphics[width=\fw]{\qd S455/images/resw_1to16_col_barInt}\tabularnewline
		\includegraphics[width=\fw]{\qd S455/images/resw_17to32_col}\vspace{\barspace}\tabularnewline
		\includegraphics[width=\fw]{\qd S455/images/resw_17to32_col_barInt}\tabularnewline
		\includegraphics[width=\fw]{\qd S455/images/resw_33to48_col}\vspace{\barspace}\tabularnewline
		\includegraphics[width=\fw]{\qd S455/images/resw_33to48_col_barInt}\tabularnewline
		\includegraphics[width=\fw]{\qd S455/images/resw_49to64_col}\vspace{\barspace}\tabularnewline
		\includegraphics[width=\fw]{\qd S455/images/resw_49to64_col_barInt}\tabularnewline
		\includegraphics[width=\fw]{\qd S455/images/resw_65to80_col}\vspace{\barspace}\tabularnewline
		\includegraphics[width=\fw]{\qd S455/images/resw_65to80_col_barInt}\tabularnewline
	\end{tabular}	
	\caption{Selection of the credible $\bm{w}$s (here ordered by $L_1$ norm). The green and blue boxes indicate the estimated ``credible'' sources (with only one major global maximum). The sources with blue frame have the maximum closer than two pixels to the border and can be therefore considered as partly missing sources. The sources with red frame have two local maxima of comparable strength. Bars under the figures show the normalised maximum value of the estimated $\bm{w}_k$ multiplied with mean brightness of the source estimated from the intensity matrix $\bm{H}$. The true number of sources in this simulated dataset was $K_{true}=72$.}. 
	\label{fig:good w}	
\end{figure}
%
The process is illustrated in \autoref{fig:good w} on $\bm{w}$ estimated from the simulated dataset of $72$  randomly scattered sources with density $50\um^{-2}$ (\autoref{fig:simulated data random}\ccc). The $\bm{w}$s considered as ``credible'' are indicated with blue or green frame. The blue frame shows the sources with maximum closer than two pixels from the border. The red frame shows the $\bm{w}$ with two local minima of similar strength (at least $50\%$ of the strength of the stronger maximum).

This approach would, however, fail when used on data with out-of-focus PSF, because the images of the out-of-focus PSF do not have one compact global maximum (\autoref{fig:Simulted-PSF-different-focal-depths} right). To accommodate for the individually different shape of the PSF we have to use a different approach. 

One possibility is to compute a set of ``features'' on each estimated $\bm{w_k}$ (and possibly on the corresponding $\bm{h_k}$) and use a linear classifier to identify the class of each estimated source. The possible ``features'' can include, for example, the $L_2$ norm, the number of clusters in the thresholded image, the maximum of the cross-correlation with the PSF, a measure of smoothness of the estimated result, the distance of the global maximum from the edge, and many others. Each $\bm{w_k}$ then represents a point in a high-dimensional feature space. The linear classifier assumes that the individual classes can be separated with linear manifolds in the feature space.

However, the linear classifier has to be trained on a set of labelled data. The training therefore requires a manual labelling of at least several hundreds of $\bm{w}$s (assignment of the class to each $w_k$). In an ideal world, one training set would be sufficient for different datasets. The classifier, once trained, would be applicable for results from different datasets taken in a range of experimental conditions. However, our experience is that the classification performance varies significantly with change of the experimental parameters (size of the patch, pixel-size, background levels). The performance, of course, depends on the set of ``features'' computed for each $\bm{w}$ and the development of some ``universal features'' might be a topic of future work. 

 
%We use a linear classifier to automate the quality assessment of the estimated $\bm{w}_k$. We compute a set of features based on intensity values of $\bm{w}_k$, thresholded background image $\bm{b}_k=\bm{w}_k<t_b$, thresholded foreground image $\bm{f}_k=\bm{w}_k>t_f$ and an image of $\bm{w}_k$ smoothed with estimated in-focus PSF. The features are listed in \autoref{tab:Features}a and five classes for $\bm{w}_k$ are listed in \autoref{tab:Classes}b.
%
%
%\begin{table}[!h]
%	\begin{centering}		
%		\footnotesize{\subfloat[Features]{
%		\begin{tabular}{|c|c|}
%			\hline 
%			\textbf{\#} & \textbf{note}\tabularnewline
%			\hline
%			\hline 
%			\textbf{1} & $l^{2}$ norm ($\sqrt{\sum_{j}w_{jk}^{2}}$)\tabularnewline
%			\hline 
%			\textbf{2} & Smoothness of $\bm{w}_k$ (discrete version of $\int\left|\frac{\partial}{\partial x}w_k(x)\right|dx$).\tabularnewline
%			\hline 
%			\textbf{3} & Smoothness of $\bm{b}_k$ (discrete version of $\int\left|\frac{\partial}{\partial x}b_k(x)\right|dx$).\tabularnewline
%			\hline 
%			\textbf{4} & Maximum of the cross-correlation between $\bm{w}_k$ and PSF.\tabularnewline
%			\hline 
%			\textbf{5} & Difference between PSF smoothed image and original $\bm{w}_k$.\tabularnewline
%			\hline 
%			\textbf{6} & Sum of the foreground $\bm{f}_k$. $\sum_{j}a_{jk}$ $ $\tabularnewline
%			\hline 
%			\textbf{7} & Distance of the global maximum from the nearest edge.\tabularnewline
%			\hline
%		\end{tabular}}				
%		\textbf{\hspace{.2cm}}\subfloat[Classes]{
%		\begin{tabular}{|c|c|}
%			\hline 
%			\textbf{\#} & \textbf{note}\tabularnewline
%			\hline
%			\hline 
%			\textbf{1} & One credible PSF\tabularnewline
%			\hline 
%			\textbf{5} & One PSF partly missing \tabularnewline
%			\hline 
%			\textbf{2} & Two credible PSFs\tabularnewline
%			\hline 
%			\textbf{3} & Three credible PSFs\tabularnewline
%			\hline 
%			\textbf{4} & Multiple credible PSFs\tabularnewline
%			\hline 
%			\textbf{0} & Noise \tabularnewline
%			\hline
%		\end{tabular}}}
%	\end{centering}	
%	\caption{(a) Features and (b) classes for classification of $\bm{w}_k$.}\label{tab:Features}\label{tab:Classes}
%\end{table}

%==========================================

\subsection{Localisation and stitching\label{sub:Localisation-and-stitching}}
The individual estimated sources classified as a credible representation of the PSF can be localised. Conventional LM techniques often apply the maximum likelihood fitting of an in-focus PSF Gaussian approximation to the estimated images \cite{Hess2006}. The localisation precision is estimated from the number of photons emitted by the sources in the frame from which the source was localised. The \inmf{} estimated intensity matrix $\bm{H}$ gives us access to the entire intensity profile of the source. We can therefore estimate number of all photons emitted by the source during the measurement, maximum intensity of each source or a variance of the blinking over time. 

The sources close to the edge can be problematic to localise. If the source represents an in-focus PSF, then it should appear entirely in the adjacent patch and can be localised there. When dealing with images with mostly in focus PSFs we can simply discard the sources classified as ``partly missing'' (\autoref{sub:Classification-of-sources}). More problematic are the out-of-focus PSFs with extent larger than the overlap area. These sources have to be first stitched together before further processing.

%==========================================

\subsection{Visualisation of the results\label{sub:visualisation}}

\begin{figure}[!hbt]
	\newcommand{\sizef}{.8}		
	\newcommand{\textgaussdiff}{Gaussian, $\sigma^{2} \propto$ 1/intensity}
	\newcommand{\textgauss}{Gaussian, $\sigma=3$ pixels}
	\newcommand{\textpowers}{Powers, $p=30$}
	\centering
	\subfloat[\textgaussdiff]{
	\includegraphics[scale=\sizef]{\qd S575/figures/oneeval_gaussFiltered_coordsEstTrue_diffSigma_bar04um}}
	\subfloat[\textgaussdiff]{
	\includegraphics[scale=\sizef]{\qd S575/figures/gaussFiltered_coordsEstTrue_diffSigma_bar04um}\label{fig:visualisation gf}}\\	

%	\subfloat[\textgauss]{
%	\includegraphics[scale=\sizef]{\qd S575/figures/oneeval_gaussFiltered_coordsEstTrue_bar04um}}
%	\subfloat[\textgauss]{
%	\includegraphics[scale=\sizef]{\qd S575/figures/gaussFiltered_coordsEstTrue_bar04um}}\\	

	\subfloat[\textpowers]{
	\includegraphics[scale=\sizef]{\qd S575/figures/res_truepos_bar04um}}
	\subfloat[\textpowers]{
	\includegraphics[scale=\sizef]{\qd S575/figures/res_meaniter_coordTrue_bar04um}}
	\caption{Visualisation of the results. {\it Left column} shows the results of one \inmf{} evaluation. {\it Right column} shows the sum of ten \inmf{} evaluations of the same dataset. (a,b) The conventional visualisation by placing Gaussians located at the positions of the estimated sources (green dots). (c,d) Powers of $\bm{w_k}$s. The true sources' locations are indicated with red dots.  Scale bar $400 \unit{nm}$.}
	\label{fig:visualisation gaussf}
\end{figure}

The conventional way for visualisation of the LM results (STORM, PALM) is to sum Gaussian functions placed in the estimated locations. The variance $\sigma^{2}$ of each Gaussian reflects the ``uncertainty'' of the estimated position. This is usually set to be proportional to the inverse of number of photons $N$ emitted by the source. The motivation behind this is the Cramer-Rao (CR) lower bound on the localisation accuracy (see \autoref{ch:Theoretical-limits-of the LM} for details)
%
\begin{equation}
	\sigma_{CR}^{2} \approx \sigma_{Airy}^{2}/N,
\end{equation}
%
where the  $\sigma_{Airy}^{2}$ is the variance of the PSF Gaussian approximation. As the $\sigma_{CR}$ is typically considerably smaller than the resolution limit, the rendered image can provide super-resolution information about the specimen's structure. 
	
In terms of the \inmf{} procedure, this method replaces the credible estimated $\bm{w}$s with an ideal, sub-resolution PSFs centred at the estimated source's location. The intensity values for each source can be estimated from the intensity time profiles of each source (rows of $\bm{H}$). 

The conventional visualisation of the \inmf{} evaluation of the synthetic dataset (illustrated in \autoref{fig:simulated data hash}) representing an artificial structure ($\mu = 12.5\um^{-1}$, $d=100 \unit{nm}$) is shown in \autoref{fig:visualisation gaussf}\aaa,\bbb. The standard deviation of each Gaussian was set to $\sigma=20\sigma_{CR}$. The left column displays the result of one evaluation, the right column shows the sum of ten evaluations (discussed below) of the same dataset.

\begin{figure}[!hbt]
	\newcommand{\sizef}{.8}			
	\newcommand{\widthfig}{1\textwidth}	
	\centering
	\subfloat[Estimated $\bm{w_k}$s.]{
	\includegraphics[width=\widthfig]{\qd S575/figures/demowpow_rf1_pow1}}\\
	\subfloat[$\bm{w_k}$s resampled by a factor $r=4$ and taken to the power $p=5$]{
	\includegraphics[width=\widthfig]{\qd S575/figures/demowpow_rf4_pow5}}	
	\caption{Illustration of the $\bm{w_k}$s ``squeezing''. Eight (out of 60) selected $\bm{w_k}$s  shown. The number in the top left corner is the index $k$ in the $L_{2}$ norm sorted $\bm{w}$s. (a) shows the \inmf{} estimated $\bm{w_k}$, (b) is the ``squeezed'' version  $\bm{w_k^p}$ by taking the up-sampled ($r=4$) results (a) to the power $p=5$.}
	\label{fig:demo pow w}	
\end{figure}

However, we can use the estimated sources $\bm{w}$s directly without replacing them with ``ideal'' PSFs. By taking the pixelwise power $p>1$ of the estimated sources $\bm{w}^p$ we achieve a ``shrinking'' of the individual $\bm{w}$ while keeping some characteristics of each source's shape (elongation along a certain direction, for example). 

Oversampling of $\bm{w_k}$s (for example zero-padding of the Fourier transform of the $\bm{w_k}$'s image) is needed before taking the higher powers $p$. \Autoref{fig:demo pow w} shows the original estimated $\bm{w_k}$s and the corresponding up-sampled (by a factor of $r=4$) $\bm{w^p_k}$s taken to the power $p=5$. 

This approach allows taking into account even the $\bm{w_k}$s containing multiple sources ($\bm{w_{32}}$ in \autoref{fig:demo pow w}, for example).

If we normalise the $L_1$ norm of $\bm{w}^p$ to one ($\sum_x \bm{w}^p(x)=1$), we can reconstruct a ``super-resolution'' image by summing all $\bm{w_k}^p$, weighted by the corresponding mean intensity $\unit{mean}(\bm{h_k})$. 

As we show in \autoref{fig:visualisation gaussf}\ccc, the visualisation of a single \inmf{} evaluation of a linear structure with high density of sources can lead to rather discontinuous image of the underlying structure because only a subset of the sources is recovered. Sum of multiple \inmf{} runs is required to give smoother representation of the structure \autoref{fig:visualisation gaussf}\ddd. 
%
\begin{figure}[!hbt]
	\newcommand{\sizef}{.48}			
	\newcommand{\widthfig}{1\textwidth}	
	\centering
	\subfloat[Wide-field]{
	\includegraphics[scale=\sizef]{\qd S575/figures/wf_bar04um}}
	\subfloat[$p=5$]{
	\includegraphics[scale=\sizef]{\qd S575/figures/res_meaniter_rf4_pow5_coordTrue_bar04um}}	
	\subfloat[$p=10$]{
	\includegraphics[scale=\sizef]{\qd S575/figures/res_meaniter_rf4_pow10_coordTrue_bar04um}}	
	\subfloat[$p=30$]{
	\includegraphics[scale=\sizef]{\qd S575/figures/res_meaniter_rf4_pow30_coordTrue_bar04um}}	
	\caption{(a) Sum projection of the simulated dataset. (b-d) Visualisation of the sum projections of ten different \inmf{} evaluations using $\bm{w}^p$ for different values of $p$. Images of $\bm{w}$s were up-sampled by a factor of $r=4$. Scale bar $400 \unit{nm}$.}
	\label{fig:demo pow w result}	
\end{figure}

\Autoref{fig:demo pow w result} show sum projection of ten \inmf{} runs for different values of $p$. This image can be compared with the conventional visualisation in \autoref{fig:visualisation gaussf}\ccc,\ddd. The number of \inmf{} runs has to be set by user and will depend on desired ``smoothness'' of the reconstructed images. The denser labelling requires more \inmf{} run (see \autoref{sec:Discussion} for further discussion). 

%==========================================
%==========================================
\clearpage
\section{Results \label{sec:results}}
We used simulated data for exploring the behaviour of \inmf{} in different experimental regimes. In \autoref{sub:results - blinking behaviour} and \ref{sub:results - number of frames} we used an average precision and an estimated density as a quantitative quality assessments of the algorithm performance on simulated data. 

\Autoref{sub:results - comparison} shows a qualitative comparison of \inmf{} results with competitive techniques CSSTORM and 3B analysis dealing with overlapping sources. Average precision was used as a quantitative measure of the performance on simulated datasets consisting of randomly scattered sources with different densities.

The comparison of the three techniques on a simulated dataset of an artificial sub-resolution structure is shown in \autoref{sub:results - comparison - structure}. The results of Richardson-Lucy deconvolution and second order SOFI are shown for further comparison. 

The unique ability of \inmf{} to recover different individual overlapping PSFs is illustrated on simulated data and on randomly scattered out-of-focus QDs in \autoref{sub:results - out of focus PSF real data}.

\Autoref{sub:results - tubulin} shows \inmf{} reconstruction of a real biological sample labelled with QDs, revealing sub-diffraction details of tubulin structures.

%==========================================

\subsection{Effect of the blinking behaviour \label{sub:results - blinking behaviour}}
The mechanism of the QD blinking is a complex and still not fully understood process \cite{Stefani2009}. Both ON and OFF time probability densities follow an inverse power law rather than an exponential decay observed in conventional fluorophores \cite{Kuno2001}. The blinking of different QDs can therefore vary greatly with a large scale of ON and OFF time periods. NMF does not make any assumption about the intensity time profiles (rows of $\bm{H}$) and can recover a great variety of blinking patterns. As we discuss later, the actual time ordering of the acquired frames is irrelevant for \inmf{} evaluation. 

The effect of the different blinking patterns on the performance of the \inmf{} algorithm was tested on simulated data. We used randomly scattered sources with different densities in a range of $10-50\um^{-2}$. The simulated datasets are illustrated in \autoref{fig:simulated data random} and the parameters of the simulations are in \autoref{tab:Simulations parameters}. 

\begin{figure}[!h]
	\centering
	\newcommand{\sizef}{.35}
	\subfloat[uniform]{
	\includegraphics[scale=\sizef]{\qd S455/images/blinkmatS422}}
	\subfloat[telegraph ($\gamma=0.5$)]{
	\includegraphics[scale=\sizef]{\qd S455/images/blinkmatS445}}\\
	\subfloat[telegraph ($\tilde{\gamma}=0.05$) downsampled]{
	%\includegraphics[scale=\sizef]{\qd S455/images/blinkmatS450}}
	%{\includegraphics[scale=\sizef]{\qd S455/images/blinkmatS557}} % same varaince as random
	\includegraphics[scale=\sizef]{\qd S455/images/blinkmatS565}} % half variance compared to random
	\subfloat[telegraph ($\tilde{\gamma}=0.005$), a-synchronic]{
	\includegraphics[scale=\sizef]{\qd S455/images/blinkmatS455}}	
	\caption{Examples of blinking behaviour of one source (50 points out of 1000).}
	\label{fig:blinking}
\end{figure}\noindent
%
Four different blinking behaviours with fixed number of emitted photons (equal mean value) were considered: 
%
\begin{enumerate}[(a)]
	\item
	Uniform random distribution of intensities between $0$ and $\max(n_{phot})$, illustrated in  \autoref{fig:blinking}\aaa. 
	\item
	A telegraph process with transition probability $\gamma=0.5$, where the intensity is switching between $0$ and $\max(n_{phot})$, shown in \autoref{fig:blinking}\bbb.
	\item
	A downsampled telegraph process. The time axes was oversampled $q$ times and a telegraph process with a rate $\tilde{\gamma}$ was generated. Finally, the blinking was under-sampled $q$ times. \Autoref{fig:blinking}\ccc\ illustrates the result for $\tilde{\gamma}=0.05$ and $q=100$, which corresponds to $10\times$ faster blinking than the sampling frequency of the measurement. This leads to the averaging (smoothing) of the intermittent behaviour.
	\item
	Similar to (c), but for $\tilde{\gamma}=0.005$. This represents more realistic intermittent behaviour than the ``binary'' telegraph process described in (b), keeping the same blinking rate. The switching between two states is no longer synchronised with the sampling, which gives rise to intermediate intensity values, \autoref{fig:blinking}\ddd. 
\end{enumerate}
%
The variance of the four different intensity profiles is shown in \autoref{fig:blinking var}.
	
\begin{figure}[!h]
	\centering
	\newcommand{\sizefig}{.40}
	\subfloat[Variance of blinking]{
	%\includegraphics[scale=\sizefig]{\qd S455/images/blinkmat_varS422S445S450S455}
	%\includegraphics[scale=\sizefig]{\qd S455/images/blinkmat_varS422S445S557S455} %the faster telegraph the same variance as random
	\includegraphics[scale=\sizefig]{\qd S455/images/blinkmat_varS422S445S565S455} %the faster telegraph the half variance as 
	\label{fig:blinking var}}
	\subfloat[average precision]{
	%\includegraphics[scale=\sizefig]{\qd S455/images/AP_letters_S422S445S450S455}
	%\includegraphics[scale=\sizefig]{\qd S455/images/APerrorbar_lettersS422S445S557S455} %the faster telegraph the same variance as random
	\includegraphics[scale=\sizefig]{\qd S455/images/APerrorbar_lettersS422S445S565S455} %the faster telegraph the half variance as random
	\label{fig:AP blinking}}
	\caption{(a) Variance of the intensity time profiles for four different blinking behaviours shown in \autoref{fig:blinking} and (b) corresponding average precision of the estimated results.}		
	\label{fig:variance and AP}
\end{figure}

All datasets were evaluated with \inmf{}. The number of sources $K$ was set to $K=K_{true}+10$, where $K_{true}$ is the true number of emitters used for simulation. 

The average precision \autoref{eq:AP} for four blinking behaviours shown in \autoref{fig:blinking} is plotted in \autoref{fig:AP blinking}. The mean and the standard deviation of the results from five different geometric configurations of randomly scattered sources is shown. 

\Autoref{fig:variance and AP} suggests that the AP is proportional to the variance of the blinking rather than to the time series structure of the blinking. In fact, NMF updates \autoref{eq:NMF classic updates} are insensitive to permutations of time frames. The \inmf{} algorithm therefore does not take the time-series structure of the data into account. This is a drawback of the NMF model, because the correlations between the adjacent time frames provide valuable information. Note that the 3B algorithm  \cite{Cox2011} exploits this information by modelling the blinking behaviour of the fluorophores with a Markov chain.

%==========================================

\subsection{Effect of the number of frames\label{sub:results - number of frames}}
%Simulated data of randomly scattered sources (\autoref{sec:simulations}) were used to evaluate effect of the number of frames in dataset. We used a dataset with $3000$ frames and used first $[50,\,100,\,200,\,500,\, 1000]$ frames for evaluation. The achieved average precision (AP) for three different densities of the sources are shown in \autoref{fig:AP on frames}. 
%%
%\begin{figure}[!h]
%	\newcommand{\sizef}{.7}
%	\centering
%	\includegraphics[scale=\sizef]{\qd S486/figures/APS486S476S481}
%	\caption{Average Precision}
%	\label{fig:AP on frames}
%\end{figure}
%
%For low densities ($10\um^{-2}$) the results reach the plateau AP about $90\%$ for 100 frames (blue curve in \autoref{fig:AP on frames}). Adding more frames does not increase AP.

What is the optimal way of acquiring data when we have a limited total acquisition time? Is it better to use longer acquisition time per frame to acquire smaller dataset with better signal-to-noise ratio in each frame? Or rather to record large number of noisy frames with as fast acquisition as possible? 

To address these questions we tested the algorithm on simulated data of randomly scattered sources (\autoref{sub:Simul random}). The parameters of the simulation were taken from \autoref{tab:Simulations parameters} but the maximum of the sources' intensity $\max(n_{phot})$ was set to 300. This represents weak sources recorded with fast acquisition time. The telegraph process shown in \autoref{fig:blinking}\ccc\ was used for simulated intensity profiles. 


\begin{figure}[!htb]	
	\newcommand{\widthfig}{1\textwidth}
	\centering	
	\subfloat[original data]{
	\label{fig:subsampled data original}
	\includegraphics[width=\widthfig]{\qd S537/figures/dpixc_1to8_subf1}}
	
	\subfloat[downsampled $2\times$]{
	\label{fig:subsampled data 2}
	\includegraphics[width=\widthfig]{\qd S537/figures/dpixc_1to8_subf2}}
	
	\subfloat[downsampled $10\times$]{
	\label{fig:subsampled data 10}
	\includegraphics[width=\widthfig]{\qd S537/figures/dpixc_1to8_subf10}}
	
%	\subfloat[subsampled $20\times$]{
%	\includegraphics[width=\widthfig]{\qd S537/figures/dpixc_1to8_subf20}}		
	\caption{First eight frames of (a) original data and (b-c) downsampled simulated data. }
	\label{fig:subsampled data}
\end{figure} 
%
Several frames of the dataset are shown in \autoref{fig:subsampled data original}. From this dataset consisting of $1000$ frames we generated four more datasets by downsampling the data $q=2,\,5,\,10$ and $20$ times by summing the $q$ subsequent frames. If  we neglect the read-out noise of the camera, this corresponds to data taken with $q$ times longer acquisition time per frame. The downsampled data consist of $1000/q$ frames. Several frames of the downsampled data for $q=2$ and $10$ are shown in \autoref{fig:subsampled data 2} and \hyperref[fig:subsampled data 10]{c}, respectively. 
%
\begin{figure}[!htb]	
	\newcommand{\wf}{.45\textwidth}
	\newcommand{\sizef}{.38}
	\centering
	\subfloat[Estimated density]{
	\includegraphics[scale=\sizef]{\qd S537/figures/densitytrueestS537S527S532}}
	\subfloat[Average precision]{
	\includegraphics[scale=\sizef]{\qd S537/figures/APS537S527S532}}
	\caption{(a) Estimated density and (b) average precision as a function of number of downsampled frames. Mean and standard deviation of three different geometrical configurations of the randomly scattered sources is shown.}
	\label{fig:AP subsampled}
\end{figure}

Three different densities of the sources were considered. Each dataset was simulated three times \cut{make more simulations?}  with different geometrical configurations of the sources. Mean values with standard deviations of the average precision and the estimated density are plotted in \autoref{fig:AP subsampled}. 

The downsampling of the dataset decreases the performance of the \inmf{} algorithm. The signal-to-noise ratio increases in each frame, due to downsampling, however, the variance of the blinking decreases because the ``ON'' and ``OFF'' states average out. The increase of the signal-to-noise ratio does not compensate for the deterioration of the AP due to decreased blinking variance (\autoref{sub:results - blinking behaviour}). 

Note, that only the blue curve in \autoref{fig:AP subsampled}\aaa\ corresponding to $10\unit{\mu m}^{-2}$ reaches the true density. Both green and red curves underestimate the true density by a factor of two and three, respectively, even for data with fastest sampling. 

\cut{\emph{Maybe rather try this experiment with uniform random blinking. Because here the blinking rate is set to 0.5 (every other frame switch) and therefore any subsampling makes it worse.}}

%==========================================

\subsection{Comparison with other methods - randomly scattered sources\label{sub:results - comparison}} %Main competitors 3B \cite{Cox2011}, CSSTORM \cite{Zhu2012} and SOFI \cite{Dertinger2009}.	
%timing: NMF algorithm 1: 25x25x1000 frames, 50 sources ->18 min on jupiter1, 21x21x1000 frames 80 sources -> 30 minutes on jupiter1
We used simulated data of randomly scattered overlapping sources with densities $10$ to $50\ \unit{sources/\mu m^{2}}$ (\autoref{sub:Simul random}) to compare performance of \inmf{} with CSSTORM \cite{Zhu2012}, and the 3B analysis \cite{Cox2011} (discussed in \autoref{sec:Overlapping sources}). The code for both 3B and CSSTORM is freely available.  

A margin of three pixels was left empty in each simulated frame to ensure that there are no partially missing PSFs. The sum projections of the frames for densities $10 \um^{2}$ and $40 \um^{2}$ are shown in \autoref{fig:density 10 wf} and \ref{fig:density 40 wf}, respectively. Several individual data frames are shown in \autoref{fig:simulated data random}, illustrating highly overlapping sources (dataset displayed in \autoref{fig:simulated data random} does not contain the 3 pixel empty margin). We used three different geometrical configurations of the randomly scattered sources. The blinking behaviour was simulated as a telegraph process with asynchronous recording as described in \autoref{sub:results - blinking behaviour} (d) and illustrated in \autoref{fig:blinking}\ddd.

The true background value of $100$ photons per pixel per frame was subtracted (clipping any negative values to zero) before CSSTORM and 3B evaluation. The true PSF was provided to both CSSTORM and 3B algorithms. 

\begin{figure}[!h]
	\centering
	\newcommand{\sizef}{.38}
	\subfloat[density]{
	\includegraphics[scale=\sizef]{\qd S612/images/densitytrueest3simiteriNMFCSSTORMCSSTORMind3B}}
	\subfloat[AP]{
	\includegraphics[scale=\sizef]{\qd S612/images/AP3simiter_iNMFCSSTORMCSSTORMind3B}}
%	\subfloat[density]{
%	\includegraphics[scale=\sizef]{\qd S455/images/densitytrueestiNMFCSSTORMCSSTORMind3B}}
%	\subfloat[AP]{
%	\includegraphics[scale=\sizef]{\qd S455/images/AP_iNMFCSSTORMCSSTORMind3B}}
	
	\caption{Comparison of the \inmf{}, CSSTORM and 3B evaluation of the randomly scattered PSFs. The blue line corresponds to \inmf{}, green line to the projected CSSTORM image, red curve to the values estimated from the individual frames of the CSSTORM and cyan line to the 3B evaluation. (a) Estimated density, (b) average precision. The mean and the standard deviation from three different configurations of the randomly scattered sources are shown. A small random offset ($\pm0.5$) to the horizontal values was added to each dataset for the sake of clarity.}
	\label{fig:comparison AP,dens}
\end{figure}
%
The estimated density and the AP values obtained from the results of the \inmf{} algorithm are shown as blue lines in \autoref{fig:comparison AP,dens}. The mean and the standard deviation from three different configurations of the randomly scattered sources are shown. The visualisation of the results (\autoref{sub:visualisation}, $p=4,\,q=4$) for datasets with four different densities is shown as a grey-scale image in \autoref{fig:density 10 NMF}-\ref{fig:density 40 NMF}. The green crosses show the maximum likelihood fit of a Gaussian function to the \inmf{} estimated $\bm{w_k}$. Only ``credible'' $\bm{w}$s were used. The selection of the ``credible'' sources is described in \autoref{sub:Classification-of-sources}.

CSSTORM processes each input frame individually, independent of the rest of the dataset. This method tries to recover a sparse distribution of the active (``ON'') fluorophores in each frame considering a known PSF (shared with all sources). The output for each frame is an image showing the possible positions of the sources on a sub-pixel grid (8 times oversampled). Following \cite{Zhu2012}, we estimated the position of each source as a centre of mass of the small clusters formed on a sub-pixel grid. The AP (the mean of AP from individual frames) and the estimated density (the mean of estimated densities from the individual frames) are denoted as \textsf{CSSTORMind} and are shown as red curves in \autoref{fig:comparison AP,dens}. 

We also processed the sum of all CSSTORM output frames which summarises all the estimated sources. The summed image was filtered with Gaussian kernel ($\sigma=1\unit{pixel}$, which corresponds to $\sigma=10\unit{nm}$). The result for different densities of the sources is shown in \autoref{fig:density 10 CSSTORM}-\ref{fig:density 40 CSSTORM}. The local maxima stronger than $5\%$ of the global maximum were identified (green crosses in \autoref{fig:density 10 CSSTORM}-\ref{fig:density 40 CSSTORM}). We chose the threshold of $5\%$ because for this value the number of local maxima roughly corresponds to the true number of sources $K_{true}$. The positions of the local maxima were used as the estimated sources' positions for computation of the AP. The true positives were considered for density estimation (\autoref{sec:evaluation}). The results are denoted as \textsf{CSSTORM} and are shown as green curves in \autoref{fig:comparison AP,dens}.

%parameters adjusted for the 3B evaluation: \texttt{blur.mu=0.37261, blur.sigma=0.1, intensity.rel\_sigma=}$K_{true}$
As the last comparison technique, we used the 3B analysis for the simulated datasets. The prior parameters for the size of the PSF were adjusted to the true values. Also the true number of sources $K_{true}$ was used as an initial number of spots in the model. The 3B algorithm was run for at least 30 iterations. Following \cite{Cox2011}, the output coordinates of the 3B analysis were placed on a $100\times$ oversampled grid ($0.8\unit{nm}$ pixel-size) and convolved with a Gaussian ($\sigma=10\unit{pixels}$, which corresponds to $\sigma=8\unit{nm}$). The resulting image is shown as a grey-scale image in \autoref{fig:density 10 3B}-\ref{fig:density 40 3B}. Similar to analysis of the projected CSSTORM, we identified local maxima in the images (green crosses in \autoref{fig:density 10 3B}-\ref{fig:density 40 3B}). Only the maxima above a certain threshold were considered for evaluation. The threshold was set individually for each image, such that the number of local maxima roughly corresponds to the number of sources considered by the 3B analysis after the last iteration. The estimated density and the AP are denoted as \textsf{3B} and are shown in \autoref{fig:comparison AP,dens} as cyan lines. 

\Autoref{fig:comparison AP,dens} suggests that CSSTORM cannot recover enough sources in individual frames (\textsf{CSSTORMind}, red lines). The density is severely underestimated, which leads to many false negatives (FN) and therefore low recall values \autoref{eq:TP,FN} which penalises AP. However, CSSTORM recovers some subset of the sources in each frame and the sum projection show dramatically improved AP and density estimation (green lines in \autoref{fig:comparison AP,dens}). 

The AP of the CSSTORM results is comparable with the AP of the \inmf{} algorithm \autoref{fig:comparison AP,dens}\bbb. \inmf{} performs slightly better at the higher densities of the sources (56\% as opposed to 44\% at density $50\um^{-2}$). However, visual inspection of the results shown in \autoref{fig:comparison density 10}\bbb,\ccc\ reveals that \inmf{} can discriminate even very close sources, while CSSTORM approximates these sources by one intensity maximum in the middle (bottom right corner in \autoref{fig:comparison density 10}\bbb,\ccc, for example). This is even more pronounced in the regions with higher densities. For example, in the bottom part of \autoref{fig:comparison density 40}\bbb,\ccc\ the sources organised in approximately parallel lines are represented by one intensity ``crest'' in the middle in the CSSTORM image \autoref{fig:density 10 CSSTORM}, whereas \inmf{} managed to pick almost all the individual sources \autoref{fig:density 10 NMF}. 

3B performs significantly worse than both \inmf{} and CSSTORM for this simulated data. The visualisation of the 3B results, shown in \autoref{fig:density 10 3B}, underlines the poor performance of 3B for this simulated dataset.

It should be noted that the estimated density and AP from the sum projection for CSSTORM and 3B results are dependent on the threshold for considering local maxima (see above). There a number of possibilities how to choose the threshold value, but we tried to relate the number of local maxima to quantities that are possible to interpret in terms of each algorithm. For CSSTORM we chose the threshold to obtain number of local maxima approximately equivalent to $K_{true}$. For 3B we matched the number of local maxima to the numbers of sources considered by 3B algorithm after last iteration. In our opinion, these threshold settings can be used for a fair comparison with \inmf{}.
%
\begin{figure}[!p]
	\centering
	\newcommand{\sizef}{.95}
	\subfloat[wide-field]
	{
	\includegraphics[scale=\sizef]{\qd /S500/figures/wf_dens10_simiter1_bar4um}
	\label{fig:density 10 wf}
	}
	\subfloat[\inmf]
	{
	\includegraphics[scale=\sizef]{\qd /S500/figures/demo_dens10_simiter1_bar4um}
	\label{fig:density 10 NMF}
	}\\
	\subfloat[CSSTORM]
	{
	\includegraphics[scale=\sizef]{\qd S492/figures/demo_dens10_simiter1_bar4um}
	\label{fig:density 10 CSSTORM}
	}
	\subfloat[3B]
	{
	\includegraphics[scale=\sizef]{\qd S494/figures/demo_dens10_simiter1_bar4um}
	\label{fig:density 10 3B}
	}	
	\caption{Comparison of the results for simulated of randomly scattered sources with density $10\um^{-2}$ ($14$ sources in total). Sum projection of the dataset with true sources' positions marked with red dots is shown in (a). Red circles show the true locations of the sources. The radius of the circles $r=0.7\unit{pixels}$ ($56\unit{nm}$) indicates the true-positive threshold distance. For further information see \autoref{sec:evaluation}. Scale bar 400\unit{nm}.}
	\label{fig:comparison density 10}
\end{figure}

%\begin{figure}[p]
%	\centering
%	\newcommand{\sizef}{.95}
%	\subfloat[wide-field]
%	{
%	\includegraphics[scale=\sizef]{\qd /S500/figures/wf_dens20_simiter1_bar4um}
%	\label{fig:density 20 wf}
%	}
%	\subfloat[NMF]
%	{
%	\includegraphics[scale=\sizef]{\qd /S500/figures/demo_dens20_simiter1_bar4um}
%	\label{fig:density 20 NMF}
%	}\\
%	\subfloat[CSSTORM]
%	{
%	\includegraphics[scale=\sizef]{\qd S492/figures/demo_dens20_simiter1_bar4um}
%	\label{fig:density 20 CSSTORM}
%	}
%	\subfloat[3B]
%	{
%	\includegraphics[scale=\sizef]{\qd S501/figures/demo_dens20_simiter1_bar4um}
%	\label{fig:density 20 3B}
%	}	
%	\caption{Comparison of the results for simulated of randomly scattered sources with density $20\um^{-2}$ ($14$ sources in total). Sum projection of the dataset is shown in (a). Red circles show the true locations of the sources. The radius of the circles $r=0.7\unit{pixels}$ ($56\unit{nm}$) indicates the true-positive threshold distance. For further information see \autoref{sec:evaluation}. Scale bar 400\unit{nm}.}
%	\label{fig:comparison density 20}
%\end{figure}

%\begin{figure}[p]
%	\centering
%	\newcommand{\sizef}{.95}
%	\subfloat[wide-field]
%	{
%	\includegraphics[scale=\sizef]{\qd /S500/figures/wf_dens30_simiter1_bar4um}
%	\label{fig:density 30 wf}
%	}
%	\subfloat[NMF]
%	{
%	\includegraphics[scale=\sizef]{\qd /S500/figures/demo_dens30_simiter1_bar4um}
%	\label{fig:density 30 NMF}
%	}\\
%	\subfloat[CSSTORM]
%	{
%	\includegraphics[scale=\sizef]{\qd S492/figures/demo_dens30_simiter1_bar4um}
%	\label{fig:density 30 CSSTORM}
%	}
%	\subfloat[3B]
%	{
%	\includegraphics[scale=\sizef]{\qd S495/figures/demo_dens30_simiter1_bar4um}
%	\label{fig:density 30 3B}
%	}	
%	\caption{Comparison of the results for simulated of randomly scattered sources with density $30\um^{-2}$ ($14$ sources in total). Sum projection of the dataset is shown in (a). Red circles show the true locations of the sources. The radius of the circles $r=0.7\unit{pixels}$ ($56\unit{nm}$) indicates the true-positive threshold distance. For further information see \autoref{sec:evaluation}. Scale bar 400\unit{nm}.}
%	\label{fig:comparison density 30}
%\end{figure}

\begin{figure}[p]
	\centering
	\newcommand{\sizef}{.95}
	\subfloat[wide-field]
	{
	\includegraphics[scale=\sizef]{\qd /S500/figures/wf_dens40_simiter1_bar4um}
	\label{fig:density 40 wf}
	}
	\subfloat[\inmf]
	{
	\includegraphics[scale=\sizef]{\qd /S500/figures/demo_dens40_simiter1_bar4um}
	\label{fig:density 40 NMF}
	}\\
	\subfloat[CSSTORM]
	{
	\includegraphics[scale=\sizef]{\qd S492/figures/demo_dens40_simiter1_bar4um}
	\label{fig:density 40 CSSTORM}
	}
	\subfloat[3B]
	{
	\includegraphics[scale=\sizef]{\qd S516/figures/demo_dens40_simiter1_bar4um}
	\label{fig:density 40 3B}
	}	
	\caption{Comparison of the results for simulated of randomly scattered sources with density $40\um^{-2}$ ($58$ sources in total). Sum projection of the dataset with true sources' positions marked with red dots is shown in (a). Red circles show the true locations of the sources. The radius of the circles $r=0.7\unit{pixels}$ ($56\unit{nm}$) indicates the true-positive threshold distance. For further information see \autoref{sec:evaluation}. Scale bar 400\unit{nm}.}
	\label{fig:comparison density 40}
\end{figure}


%%%% original figures grouped together according to the method
%\begin{figure}[!h]
%	\centering
%	\newcommand{\sizef}{1}
%	\subfloat[density $10\um^{-2}$, $14$ sources]{
%	\includegraphics[scale=\sizef]{\qd /S500/figures/wf_dens10_simiter1_bar4um}}
%	\subfloat[density $20\um^{-2}$, $29$ sources]{
%	\includegraphics[scale=\sizef]{\qd S500/figures/wf_dens20_simiter1_bar4um}}\\
%	\subfloat[density $30\um^{-2}$, $43$ sources]{
%	\includegraphics[scale=\sizef]{\qd S500/figures/wf_dens30_simiter1_bar4um}}
%	\subfloat[density $40\um^{-2}$, $58$ sources]{
%	\includegraphics[scale=\sizef]{\qd S500/figures/wf_dens40_simiter1_bar4um}}
%	
%	\caption{Sum projection of four datasets with different densities of the randomly scattered sources. Red circles show the true sources. The radius of the circles $r=0.7\unit{pixels}$ ($56\unit{nm}$) indicates the true-positive threshold distance. For further information see \autoref{sec:evaluation}. Scale bar 400\unit{nm}.}
%	\label{fig:wf demo}
%\end{figure}
%
%\begin{figure}[!h]
%	\centering
%	\newcommand{\sizef}{1}
%	\subfloat[density $10\um^{-2}$]{
%	\includegraphics[scale=\sizef]{\qd /S500/figures/demo_dens10_simiter1_bar4um}}
%	\subfloat[density $20\um^{-2}$]{
%	\includegraphics[scale=\sizef]{\qd S500/figures/demo_dens20_simiter1_bar4um}}\\
%	\subfloat[density $30\um^{-2}$]{
%	\includegraphics[scale=\sizef]{\qd S500/figures/demo_dens30_simiter1_bar4um}}
%	\subfloat[density $40\um^{-2}$]{
%	\includegraphics[scale=\sizef]{\qd S500/figures/demo_dens40_simiter1_bar4um}}
%	
%	\caption{Demonstration of the NMF results. The results are visualised as a weighted sum of $\bm{w}^p$ (with$p=5$) as described in \autoref{sub:visualisation}. Red circles show the true sources. The radius of the circles indicates the true-positive threshold distance $r=56\unit{nm}$. For further information see \autoref{sec:evaluation}. Green crosses show the local maxima in the image. ($4\times$ oversampled original). Scale bar 400\unit{nm}.}
%	\label{fig:NMF demo}
%\end{figure}
%
%\begin{figure}[!h]
%	\centering
%	\newcommand{\sizef}{1}
%	\subfloat[density $10\um^{-2}$]{
%	\includegraphics[scale=\sizef]{\qd S492/figures/demo_dens10_simiter1_bar4um}}
%	\subfloat[density $20\um^{-2}$]{
%	\includegraphics[scale=\sizef]{\qd S492/figures/demo_dens20_simiter1_bar4um}}\\
%	\subfloat[density $30\um^{-2}$]{
%	\includegraphics[scale=\sizef]{\qd S492/figures/demo_dens30_simiter1_bar4um}}
%	\subfloat[density $40\um^{-2}$]{
%	\includegraphics[scale=\sizef]{\qd S492/figures/demo_dens40_simiter1_bar4um}}
%	
%	\caption{Demonstration of the CSSTORM results. Gaussian filtered sum projection  of all output frames is shown. Red circles show the true sources. The radius of the circles indicates the true-positive threshold distance $r=56\unit{nm}$. For further information see \autoref{sec:evaluation}. Green crosses show the local maxima in the image. ($8\times$ oversampled original). Scale bar 400\unit{nm}.}
%	\label{fig:CSSTORM demo}
%\end{figure}
%
%\begin{figure}[!h]
%	\centering
%	\newcommand{\sizef}{1}
%	\subfloat[density $10\um^{-2}$]{
%	\includegraphics[scale=\sizef]{\qd S494/figures/demo_dens10_simiter1_bar4um}} %70 iterations>20 points (initial 10 points); 50 iterations>18 points
%	\subfloat[density $20\um^{-2}$]{
%	\includegraphics[scale=\sizef]{\qd S501/figures/demo_dens20_simiter1_bar4um}}\\ %38 iterations>39 points (initial 29 points)
%	\subfloat[density $30\um^{-2}$]{
%	\includegraphics[scale=\sizef]{\qd S495/figures/demo_dens30_simiter1_bar4um}} %217 iterations>58 points (initial 40 points); 50 iterations>43 points.	
%	\subfloat[density $40\um^{-2}$]{
%	\includegraphics[scale=\sizef]{\qd S516/figures/demo_dens40_simiter1_bar4um}} %50 iterations>62 points (initial 58 points)
%	
%	\caption{Demonstration of the 3B results. Red circles show the true sources. The radius of the circles indicates the true-positive threshold distance $r=56\unit{nm}$. For further information see \autoref{sec:evaluation}. Green crosses show the local maxima in the 3B image. ($100\times$ oversampled original). Scale bar 400\unit{nm}.}
%	\label{fig:3B demo}
%\end{figure}

%==========================================

\clearpage
\subsection{Comparison with other methods - artificial structure\label{sub:results - comparison - structure}}
The experiments shown above are useful for a quantitative comparison of the different methods. Randomly scattered sources are, however, of little practical interest. The main motivation of super-resolution microscopy is to recover sub-diffraction details of a sample structure. Therefore we used simulated data with sources attached to an artificial structure to further compare the performance of the three methods. The simulated data are illustrated in \autoref{sub:Simul hash} with main parameters shown in \autoref{tab:Simulations parameters}. The distance between the parallel lines was set to $d=150\unit{nm}$ ($1.8 \unit{pixels}$), which corresponds to a half of the Airy disk's radius. The linear density was set to $\mu = 15\um^{-1}$ which corresponds to approximately $67\ \unit{nm}$ spacing between adjacent sources. We used the same blinking behaviour (\autoref{fig:blinking}\ddd) as for the randomly scattered sources describe above. The sum projection of the data frames, equivalent to the wide-field image is shown in \autoref{fig:comparison hash dens 12}\aaa.
%
\begin{figure}[!h]
	\centering
	\newcommand{\sizef}{.95}
	\newcommand{\wf}{.45\textwidth}
	\subfloat[wide-field]
	{
	\includegraphics[width=\wf]{\qd /S570/figures/wf_bar4um_arrows}
	\label{fig:hash wf  dens 12}
	}
	\subfloat[\inmf]
	{
	\includegraphics[width=\wf]{\qd /S570/figures/res_bar4um_arrows}
	\label{fig:hash NMF dens 12}
	}\\
	\subfloat[CSSTORM]
	{
	\includegraphics[width=\wf]{\qd S571/figures/res_bar4um_arrows}
	\label{fig:hash CSSTORM  dens 12}
	}
	\subfloat[3B]
	{
	\includegraphics[width=\wf]{\qd S572/figures/res_bar4um_arrows}
	\label{fig:hash 3B  dens 12}
	}	
	\caption{Evaluation of the artificial structure data with three different methods. The parallel lines are separated by $d=150 \unit{nm}$ ($1.8 \unit{pixels}$). The sources (indicated as red dots in (a)) are distributed along the lines with a linear density $15 \um^{-1}$. Arrows in a wide-field image (a) point at sub-resolution features of the specimen (further discussed in the text). Scale bar 400\unit{nm}.}
	\label{fig:comparison hash dens 12}
	% there is a Richardson lucy deconvolution of this dataset in S570/figures ....
\end{figure}
%

To achieve a smoother representation of the underlying structure, we used the sum of ten \inmf{} evaluations (see discussion in \autoref{sub:visualisation}) rather than only one \inmf{} run used for the data of randomly scattered sources in the section above. The result is visualised in \autoref{fig:comparison hash dens 12}\bbb. The same dataset was evaluated with 3B and CSSTORM, shown in  \autoref{fig:comparison hash dens 12}\ccc\ and \autoref{fig:comparison hash dens 12}\ddd, respectively. Only one run of 3B (24 iterations) and CSSTORM has been used. 

3B completely fails to recover the double parallel lines, replacing them with one intensity crest in between of the lines (blue arrow in \autoref{fig:comparison hash dens 12}\ddd). CSSTORM shows the double line structure of the hash symbol at the periphery of the specimen (green arrows in \autoref{fig:comparison hash dens 12}\ccc), however the double lines joins into a single line close to the centre of the cross  (blue arrow in \autoref{fig:comparison hash dens 12}\ccc). The hole ($150 \times 150 \unit{nm}$) in the middle of the specimen is completely unresolved and is replaced by intensity maximum (red arrow in \autoref{fig:comparison hash dens 12}\ccc). 

\inmf{} shows the double line structure all the way along the artificial specimen (green arrows in \autoref{fig:comparison hash dens 12}\bbb) and  the hole in the middle of the structure is clearly visible (red arrow in \autoref{fig:comparison hash dens 12}\bbb). \Autoref{fig:demo pow w result}\ddd\ demonstrates that the double lines and the hole in the middle can be observed even for a structure with lines as close as $d=100\unit{nm}$ ($1.25 \unit{pixels}$). 

The results of one evaluation of \inmf{} do not provide satisfactory representation of the structure. As shown in an image constructed from the powers of $w$ (\autoref{fig:visualisation gaussf}\ccc), only several individual PSFs are recovered from the highly overlapping sources (the adjacent sources are closer than $\sim \lambda_{em}/10)$. The visualised image consists of disconnected individual blobs. Several runs of \inmf{} for the same dataset are needed to gradually fill the disconnected structure (see \autoref{fig:visualisation gaussf}\ddd\ and \ref{fig:comparison hash dens 12}\bbb).

\begin{figure}[!htb]
	\centering
	\newcommand{\wf}{.3\textwidth}
	\subfloat[RL]{
	\includegraphics[width=\wf]{\qd /S613/images/doublecross_deconvlucy500itMean_bar400nm}}
	\subfloat[RL ind]{
	\includegraphics[width=\wf]{\qd /S613/images/doublecross_deconvlucy500itIndiv_bar400nm}}
	\subfloat[SOFI 2\textsuperscript{nd} order]{
	\includegraphics[width=\wf]{\qd /S613/images/doublecross_sofi2nd_bar400nm}}
	\caption{(a) Richardson-Lucy (RL) deconvolution of the artificial structure sum projection.  (b) Sum of the RL deconvolutions of the individual frames. (c) Second order SOFI image. Red dots show the locations of the sources. Scale bar 400 nm.}
	\label{fig:Cross RL SOFI}
\end{figure}
%
The dataset was also evaluated with Richardson-Lucy (RL) deconvolution ({\tt deconvlucy} function in MATLAB) with provided known (true) PSF and run for 100 iterations. The true background offset of 100 photons was subtracted (clipping the negative values to zeros) before the evaluation.

RL deconvolution of the dataset sum projection (wide-field image) is shown in \autoref{fig:Cross RL SOFI}\aaa. We also applied RL deconvolution to each frame of the dataset. The sum projection of deconvolved frames is shown in \autoref{fig:Cross RL SOFI}\bbb. 

\Autoref{fig:Cross RL SOFI}\ccc\ shows the application of the second order SOFI image (see \cite{Dertinger2010b} and discussion in \autoref{sec:Overlapping sources}). The second order SOFI corresponds to the variance of the pixels intensity along the frames. 

Neither RL deconvolved images nor the second order SOFI was capable of discriminating the sub-resolution features of the artificial structure (compare with \autoref{fig:comparison hash dens 12}).



%\begin{figure}[!h]
%	\centering
%	\newcommand{\sizef}{.95}
%	\subfloat[wide-field]
%	{
%	\includegraphics[scale=\sizef]{\qd /S568/figures/wf_bar4um}
%	\label{fig:hash wf}
%	}
%	\subfloat[NMF]
%	{
%	\includegraphics[scale=\sizef]{\qd /S568/figures/res_bar4um}
%	\label{fig:hash NMF}
%	}\\
%	\subfloat[CSSTORM]
%	{
%	\includegraphics[scale=\sizef]{\qd S566/figures/res_bar4um}
%	\label{fig:hash CSSTORM}
%	}
%	\subfloat[3B]
%	{
%	\includegraphics[scale=\sizef]{\qd S567/figures/res_bar4um}
%	\label{fig:hash 3B}
%	}	
%	\caption{Scale bar 400\unit{nm}.}
%	\label{fig:comparison hash}
%\end{figure}

%==========================================

\subsection{Comparison with other methods - computational time}

For our computer (Intel(R) Core(TM)2 Duo @ 2GHz processor with 3GB of RAM), the computational time for the simulated dataset ($21\times21\times1000$ frames) was:

\begin{tabular}{ll}
	{\bf\inmf} & $\sim 20 \unit{mins}$ for one complete run with $K=50$ sources and $K$ restarts.\\
	
	{\bf CSSTORM} & $\sim 260 \unit{mins}$.\\
	
	{\bf 3B analysis} & $>12$ hours for $30$ iterations.\\
\end{tabular}

Note that the \inmf{} images shown in \autoref{fig:hash NMF dens 12} are results of 10 \inmf{} evaluation. The computation time is therefore comparable ($\sim 200 \unit{mins}$) to the CSSTORM method. 

%\begin{description}
%
%	\item[\inmf:]
%	$\sim 20 \unit{mins}$ for one complete run with $K=50$ sources and $K$ restarts.
%	
%	\item[CSSTORM:]
%	$\sim 260 \unit{mins}$.
%	
%	\item[3B analysis:]
%	$>12$ hours for $\sim30$ iterations.
%\end{description}
% utah: Intel(R) Core(TM)2 Duo CPU     E8400 @2GHz with 3GB of RAM
% jupiter2:Intel(R) Xeon(R) CPU           X5450  @3GHz with 31GB of RAM
% 21x21x10^3 frames 
% NMF fir 80 sources takes 30 min
% CSSTORM 260 min
% 3B more than 5 hours for 50 iterations

%==========================================
\subsection{Comparison with other methods - parameters setting}

Each of the method requires a number of parameters to be explicitly set by user prior to the evaluation. The explicit parameters are summarised in the following table (n.a. stands for not applicable):

\begin{table}[!h]
\centering
	\begin{tabular}{|l||c|c|c|}
		\hline
		{\it parameter}				& {\bf\inmf} & {\bf CSSTORM} & {\bf 3B analysis}\\ \hline
		
		PSF description		& NO  	& YES	& YES \\ \hline
		\# of iterations in one run	& YES	& NO 	& YES \\ \hline
		\# of runs				& YES	& n.a. 	& n.a.   \\ \hline	
		\# of sources estimation	& YES	& NO	& YES \\ \hline
		patch size and overlap	& YES 	& YES	& YES \\ \hline
	\end{tabular}
\end{table}

For visualisation purposes all methods require setting of the oversampling rate of the resulting images of the results in addition. There is also a parameter for a slight ``blurring'' of the results: the variance of the Gaussian kernel for 3B and CSSTORM  (see \autoref{sub:results - comparison}) and the ``power'' parameter $p$ for \inmf{} (see \autoref{fig:demo pow w result}).

Note that these are only the parameters explicitly set by user. There are more parameters within each algorithm that are pre-set to their ``optimal'' values. 


%==========================================
\clearpage
\subsection{Out of focus PSFs\label{sub:results - out of focus PSF}\label{sub:results - out of focus PSF real data}}
%
\begin{figure}[!htb]	
	\newcommand{\widthfig}{0.95\textwidth}
	\newcommand{\barspace}{-.5cm}
	\condcomment{\boolean{includefigs}}{ 
	\centering	
	\subfloat[Data]{	
	\begin{tabular}{l}
		\includegraphics[width=\widthfig]{\qd S382/images/dpixc_1to8}		
	\end{tabular}}\\	
	\subfloat[True sources]{
	\begin{tabular}{l}
		\noindent		
		\includegraphics[width=\widthfig]{\qd S382/images/wtrue_l2sort}\vspace{\barspace}\tabularnewline
		\includegraphics[width=\widthfig]{\qd S382/images/wtrue_l2sort_intBars}\tabularnewline
	\end{tabular}}\\
	\subfloat[True sources corrupted with noise]{
	\begin{tabular}{l}
		\includegraphics[width=\widthfig]{\qd S382/images/wtrue_l2sort_noise}\vspace{\barspace}\tabularnewline
		\includegraphics[width=\widthfig]{\qd S382/images/wtrue_l2sort_noise_intBars}\tabularnewline
	\end{tabular}}\\	
	\subfloat[Estimated sources]{
	\begin{tabular}{l}
		\includegraphics[width=\widthfig]{\qd S382/images/resw_1to8}\vspace{\barspace}\tabularnewline
		\includegraphics[width=\widthfig]{\qd S382/images/resw_1to8_intBars}\tabularnewline
	\end{tabular}}}
	\caption{Simulated data of eight sources. (a) Eight frames (out of 500) of the simulated data set. (b) The true sources. (c) Noisy version of the true sources with their maximum intensity. (d) The first 8 estimated sources (see \autoref{fig:Iterative restarts}\ddd\ for all $\bm{w}$'s.) Bars under the figures show the maximum of the intensity image of $\bm{w}_k$. }
	\label{fig:Data-true-estimations}
\end{figure} 
%
\noindent
\inmf{} has a unique capability of recovering sources with different individual PSFs, because there is no assumption about the shape of the estimated components $\bm{w_k}$ in the NMF updates \autoref{eq:NMF classic updates}. We demonstrate this interesting feature on simulated data of eight blinking QDs attached to a bar slanting in depth, see \autoref{fig:Data-true-estimations}\aaa. The individual simulated sources were separated by $370 \unit{nm}$ ($1.15\times$ radius of the Airy disk $\delta$, $2.6 \um$ total length) in the projected plane and the axial difference between the tips of the bar was $1.6 \um$. Other parameters of the simulation were: emission wavelength $625\unit{nm}$, numerical aperture $1.3$, refractive index $1.5$, edge size of a pixel in the image plane $100 \unit{nm}$, $T=500$, mean number of photons per source $1500$, background photons/pixel $70$, uniform distribution of blinking (\autoref{fig:blinking}\aaa). 

The true sources (individual PSFs) are shown in \autoref{fig:Data-true-estimations}\bbb\ and their noisy versions (obtained from the frame with the maximum intensity of each source) are shown in \autoref{fig:Data-true-estimations}\ccc. The \inmf{} result is shown in \autoref{fig:Data-true-estimations}\ddd\ (Several steps of the procedure are illustrated in \autoref{fig:Iterative restarts}). The correspondence of the estimated sources $\bm{w}$ (first eight out of 16 sources form \autoref{fig:Iterative restarts}) to the true sources shown in \autoref{fig:Data-true-estimations} demonstrates the ability of \inmf{} to recover sources with individually different shapes from noisy data with highly overlapping emitters (see \autoref{fig:Data-true-estimations}). 

\begin{figure}[htb]
	\newcommand{\sizeresw}{.85}
	\newcommand{\wf}{.95\textwidth}
	\newcommand{\barspace}{-.6cm}
	\condcomment{\boolean{includefigs}}{ 
	\centering
	\begin{tabular}{l}
		\subfloat[Data]{
		\includegraphics[width=\wf]{\qd S392/images/dpixc_randind}}
	\end{tabular}		
	\subfloat[Estimated sources]{
	\begin{tabular}{l}			
		\includegraphics[width=\wf]{\qd S392/images/resw_1to11}\vspace{\barspace}\tabularnewline
		\includegraphics[width=\wf]{\qd S392/images/resw_1to11_intBars}\tabularnewline
		\includegraphics[width=\wf]{\qd S392/images/resw_12to22}\vspace{\barspace}\tabularnewline
		\includegraphics[width=\wf]{\qd S392/images/resw_12to22_intBars}\tabularnewline
		\includegraphics[width=\wf]{\qd S392/images/resw_23to33}\vspace{\barspace}\tabularnewline
		\includegraphics[width=\wf]{\qd S392/images/resw_23to33_intBars}\tabularnewline			
	\end{tabular}}	
	}
	\caption{Real data of randomly scattered QDs. (a) Eleven randomly selected frames (out of $1,000$) of the overlapping PSFs produced by blinking QDs. (b) Estimated sources $\bm{w}_k$ sorted according to their estimated mean brightness. Bars below each figure show the maximum of the $\bm{w}_k$ multiplied by the mean brightness of the source estimated from $\bm{H}$.}
	\label{fig:Real-data-QDrandom}	
\end{figure}
%
To demonstrate the recovery of individual different PSF in realistic experimental settings, we applied \inmf{} on a movie of real out-of-focus blinking QDs. We analysed $1000$ frames acquired with $50\unit{ms/frame}$ acquisition time (the total acquisition time was $\sim$ one minute). Several frames of the dataset are shown in \autoref{fig:Real-data-QDrandom}\aaa. The over-estimated number of sources $K=33$ was estimated from the principal values of the data as described in \autoref{sub:Estimation-of-number-of-sources}. 

The images of evaluated $\bm{w}$s are shown in \autoref{fig:Real-data-QDrandom}\aaa. Credible out-of-focus PSFs from different focal depths (cf. \autoref{fig:Simulted-PSF-different-focal-depths}) have been recovered (the first two rows in \autoref{fig:Real-data-QDrandom}\bbb). The $\bm{w}_k$s in the last row of \autoref{fig:Real-data-QDrandom}\bbb\ are mostly noise contribution. The mean brightness of these sources (estimated form $\bm{H}$) is less than 10\% of the brightest $\bm{w}_k$ (see bars under individual images in \autoref{fig:Real-data-QDrandom}\bbb).

It should be noted that the recovery of different individual PSFs is beyond ability of either 3B or CSSTORM. Both methods require known PSF, which is shared by all emitters. 3B can adjust for the size of the PSF, however, the shape (Gaussian)  remains identical for all sources. 

In theory, independent component analysis (ICA), discussed in \autoref{sub:ICA}, allows recovery of different individual PSFs. However, as we demonstrated in \autoref{sub:ICA}, ICA's performance is poor when applied to noisy data. The results of the ICA evaluation (FastICA algorithm \cite{Hyvarinen2000}) of data from \autoref{fig:Data-true-estimations} and \autoref{fig:Real-data-QDrandom} are shown in \autoref{fig:ICA slanted bar} and \autoref{fig:ICA QD random}, respectively. The background was subtracted (clipping any negative values to zero) prior to the ICA evaluation. The number of sources was set to $K=K_{true}$ in \autoref{fig:ICA slanted bar} and $K=20$ in \autoref{fig:ICA QD random} (we set $K=20$ because this corresponds to the number of ``credible'' PSF recovered with \inmf{} in \autoref{fig:Real-data-QDrandom}\bbb). We used {\tt 'tanh'} as the nonlinearity option in the fixed-point algorithm.

\begin{figure}[htb]
	\newcommand{\wf}{.22}
%	\newcommand{\wf}{.95\textwidth}
	\newcommand{\barspace}{-.6cm}	
	\centering
	\includegraphics[scale=\wf]{\qd S382/images/ic_c1}
	\includegraphics[scale=\wf]{\qd S382/images/ic_c2}
	\includegraphics[scale=\wf]{\qd S382/images/ic_c3}
	\includegraphics[scale=\wf]{\qd S382/images/ic_c4}
	\includegraphics[scale=\wf]{\qd S382/images/ic_c5}
	\includegraphics[scale=\wf]{\qd S382/images/ic_c6}
	\includegraphics[scale=\wf]{\qd S382/images/ic_c7}
	\includegraphics[scale=\wf]{\qd S382/images/ic_c8}
	\caption{ICA evaluation of the simulated slanted bar (\autoref{fig:Data-true-estimations}). Blue pixels indicate negative values.}
	\label{fig:ICA slanted bar}
\end{figure}
%
For the simulated data of slanted bar \autoref{fig:Data-true-estimations}\aaa, ICA completely fails to discriminate the out-of-focus overlapping sources \autoref{fig:ICA slanted bar}. Only the in-focus PSFs are more or less recovered (components 3 and 7 in \autoref{fig:ICA slanted bar}). Other components represent  a combination of several PSFs together. All the components contain negative values (blue pixels in \autoref{fig:ICA slanted bar}).

\begin{figure}[htb]
	\newcommand{\wf}{.25}
%	\newcommand{\wf}{.95\textwidth}
	\newcommand{\barspace}{-.6cm}	
	\centering
	\includegraphics[scale=\wf]{\qd S392/images/ic_c1}
	\includegraphics[scale=\wf]{\qd S392/images/ic_c2}
	\includegraphics[scale=\wf]{\qd S392/images/ic_c3}
	\includegraphics[scale=\wf]{\qd S392/images/ic_c4}
	\includegraphics[scale=\wf]{\qd S392/images/ic_c5}
	\includegraphics[scale=\wf]{\qd S392/images/ic_c6}
	\includegraphics[scale=\wf]{\qd S392/images/ic_c7}
	\includegraphics[scale=\wf]{\qd S392/images/ic_c8}
	\includegraphics[scale=\wf]{\qd S392/images/ic_c9}
	\includegraphics[scale=\wf]{\qd S392/images/ic_c10}\\
	\includegraphics[scale=\wf]{\qd S392/images/ic_c11}
	\includegraphics[scale=\wf]{\qd S392/images/ic_c12}
	\includegraphics[scale=\wf]{\qd S392/images/ic_c13}
	\includegraphics[scale=\wf]{\qd S392/images/ic_c14}
	\includegraphics[scale=\wf]{\qd S392/images/ic_c15}
	\includegraphics[scale=\wf]{\qd S392/images/ic_c16}
	\includegraphics[scale=\wf]{\qd S392/images/ic_c17}
	\includegraphics[scale=\wf]{\qd S392/images/ic_c18}
	\includegraphics[scale=\wf]{\qd S392/images/ic_c19}
	\includegraphics[scale=\wf]{\qd S392/images/ic_c20}
	\caption{ICA evaluation of the randomly scattered out of focus QDs (\autoref{fig:Real-data-QDrandom}). Blue pixels indicate the negative values.}
	\label{fig:ICA QD random}
\end{figure}
%
For the real data of randomly scattered QDs \autoref{fig:Real-data-QDrandom}\aaa, some of the ICA estimated sources resemble the out-of-focus PSFs (for example, components 4 and 7 in \autoref{fig:ICA QD random}), however most of the sources contain large regions of negative values (blue pixels in \autoref{fig:ICA QD random}) and the overall quality is inferior to the \inmf{} results \autoref{fig:Real-data-QDrandom}\bbb. Most of the estimated components clearly combine several overlapping PSFs together (components 1, 9 and 12, for example). 
%==========================================
\clearpage
\subsection{Real data: QD stained tubulin fibres\label{sub:results - tubulin}}
We applied the pipeline described in \autoref{sec:NMF-for-real} to a stack of $T=10^{3}$ frames ($128\times128$ pixels) of $\alpha$-tubulin fibres of a 3T3 fibroblast cell imuno-labelled with QDs (QD625, \emph{Invitrogen}). The experimental parameters are shown in \autoref{tab:parameters experiment}.

The time average of the dataset, which corresponds to the wide-field image, is shown as a grey-valued image in \autoref{fig:tubulin WF and NMF}\aaa. The quantum dots are attached to the tubulin creating fine linear structures with sub-diffraction details. 

\begin{figure}[!htb]
	\centering
	\condcomment{\boolean{includefigs}}{ 
	\includegraphics[scale=.9]{\qd S364/results/dataChunks}}
%	\includegraphics[scale=.9]{\qd S580/figures/chunks}	
	\caption{Division of the dataset into smaller patches. Time average of all frames is shown as a grey-valued image. Boxes with thick lines were used for \inmf{} evaluation (boxes with thin lines were considered to be empty). The index of the patches and the (over) estimated numbers of sources ($K$) are shown in each box.}
	\label{fig:Patches}
\end{figure}
%
The dataset was divided into $25 \times 25$ patches (\autoref{fig:Patches}), and only patches with sufficiently strong signal (thick boxes in \autoref{fig:Patches}) were considered for further evaluation. The number of sources within each patch is over-estimated via principal components analysis (see \autoref{sub:Estimation-of-number-of-sources}). Each patch was evaluated with the \inmf{} algorithm.

\begin{figure}[!htb]
	\newcommand{\widthf}{.95\textwidth}
	\newcommand{\barspace}{-.6cm}
	\condcomment{\boolean{includefigs}}{ 
	\centering
	\subfloat[Data]{			
	\begin{tabular}{l}
		\includegraphics[width=\widthf]{\qd S364/results/dpixc_randind}\tabularnewline
	\end{tabular}
	\label{fig:Real-data-patch-B24}}						
	
	\subfloat[Estimated sources]{				
	\begin{tabular}{l}
		\includegraphics[width=\widthf]{\qd S364/results/resw_B24_1to14}\vspace{\barspace}\tabularnewline
		\includegraphics[width=\widthf]{\qd S364/results/resw_B24_1to14_intBars}\tabularnewline
		\includegraphics[width=\widthf]{\qd S364/results/resw_B24_15to28}\vspace{\barspace}\tabularnewline
		\includegraphics[width=\widthf]{\qd S364/results/resw_B24_15to28_intBars}\tabularnewline
		\includegraphics[width=\widthf]{\qd S364/results/resw_B24_29to42}\vspace{\barspace}\tabularnewline
		\includegraphics[width=\widthf]{\qd S364/results/resw_B24_29to42_intBars}\tabularnewline
		\includegraphics[width=\widthf]{\qd S364/results/resw_B24_43to56}\vspace{\barspace}\tabularnewline
		\includegraphics[width=\widthf]{\qd S364/results/resw_B24_43to56_intBars}\tabularnewline			
	\end{tabular}
	\label{fig:Real-data-patch-B24 estimated W}}
	}
	\caption{Real data - patch \texttt{B24} from \autoref{fig:Patches}. (a) 14 randomly selected frames (out of $10^{3}$) of the tubulin structure stained with QDs. (b) Estimated sources $\bm{w}_k$ sorted according to their $L_{2}$ norm (shown all $K=56$ sources). Bars below each figure show the maximum of the $\bm{w}_k$ intensity image multiplied with the mean intensity of the source estimated from the matrix $\bm{H}$.}
	
\end{figure}
%
Several time frames of the patch \texttt{B24} from \autoref{fig:Patches} are shown in \autoref{fig:Real-data-patch-B24}. \Autoref{fig:Real-data-patch-B24 estimated W} displays the \inmf{} estimated $\bm{w}$s (for $K=56$).


%\begin{figure}[!htb]
%	\newcommand{\wf}{.9\textwidth}
%	\centering
%	\includegraphics[width=\wf]{\qd S364/results/all_mean_zoh}
%	\caption{Wide - filed data.}
%\end{figure}
%
%\begin{figure}[!htb]
%	\newcommand{\wf}{.9\textwidth}
%	\centering
%	\includegraphics[width=\wf]{\qd S364/results/all_gauss3_int}
%	\caption{QD - labelled tubulin. Gaussian filtered.}
%\end{figure}
%
%\begin{figure}[!htb]
%	\newcommand{\wf}{.9\textwidth}
%	\centering
%	\includegraphics[width=\wf]{\qd S364/results/all_powFromLabel}
%	\caption{QD - labelled tubulin. PSF squeezed.}
%\end{figure}
%
%\begin{figure}[!htb]
%	\newcommand{\wf}{.9\textwidth}
%	\centering
%	\includegraphics[width=\wf]{\qd S364/results/all_powFromLabel}
%	\caption{QD - labelled tubulin. PSF squeezed.}
%\end{figure}
%
%\begin{figure}[!htb]
%	\newcommand{\wf}{.9\textwidth}
%	\centering
%	\includegraphics[width=\wf]{\qd S364/results/meanResIm_power05_scalebar500nm}
%	\caption{QD - labelled tubulin. PSF squeezed. Mean of 10 evaluations. Scale bar $500 \unit{nm}$.}
%\end{figure}

%\begin{figure}[!htb]
%	\newcommand{\wf}{.45\textwidth}
%	\centering
%	\subfloat[wide-field]{
%	\includegraphics[width=\wf]{\qd S580/figures/wf_bar05um}}
%	\subfloat[NMF]{
%	\includegraphics[width=\wf]{\qd S580/figures/resIm_hot_bar05um}}
%%	\includegraphics[width=\wf]{\qd S580/figures/comparisonROI2}
%
%	\caption{QD - labelled tubulin. PSF squeezed. Mean of 5 evaluations. Scale bar $500 \unit{nm}$.}
%\end{figure}

\begin{figure}[!htb]
	\newcommand{\wf}{.47\textwidth}
	\centering
	\subfloat[wide-field]{
	\includegraphics[width=\wf]{\qd S580/figures/wf_roi123_bar05um}}
	\subfloat[NMF]{
	\includegraphics[width=\wf]{\qd S580/figures/resIm_bar05um}}\\
	\caption{Tubulin labelled with QDs. Comparison of (a) wide-field and (b) \inmf{} evaluation. Details in the highlighted regions are shown in \autoref{fig:tubulin details}. Scale bar $500 \unit{nm}$.}
	\label{fig:tubulin WF and NMF}
\end{figure}

\begin{figure}[!hb]
	\centering
	\includegraphics[width=.95\textwidth]{\qd S580/figures/compareROI123hot}
	\caption{Wide field (WF) and \inmf{} results for regions marked in \autoref{fig:tubulin WF and NMF} with coloured boxes. \inmf{} results are shown in false colours to enhance the contrast of dim features. Scale bars $200 \unit{nm}$.} 
	\label{fig:tubulin details}
\end{figure}
% 
Each patch was evaluated five times with \inmf{}. The results were visualised by ``squeezing'' $\bm{w}$s, as described in \autoref{sub:visualisation}, using the over-sampling by a factor $r=4$ and power $p=30$. The resulting images for five different \inmf{} evaluations were summed together to create a sub-resolution image. The final image of the whole dataset was created by tiling results for the individual patches. The border pixels of neighbouring patches were removed to avoid overlaps of the results.

\Autoref{fig:tubulin WF and NMF} compares the wide-field (WF) image with \inmf{} evaluated results. The close-up of the highlighted regions in \autoref{fig:tubulin WF and NMF}\aaa\ for WF and \inmf{} is displayed in \autoref{fig:tubulin details} (using false colours to enhance contrast of the dim features in the \inmf{} results). Sub-resolution details of the tubulin structure such as fibre crossing (left part of \autoref{fig:tubulin details}) or twisting of fibres (right part of \autoref{fig:tubulin details}) are revealed in visualised \inmf{} results.



%We trained the classifier on $10^{3}$ labelled $\bm{w}_k$, computed by NMF from a real dataset (\autoref{fig:Patches}). Confusion matrix of the ten-fold cross validation is shown in \autoref{tab:Confusion-matrix}. From all $\bm{w}_k$s classified as good sources (class 1) 89\% were correct, while the rest 11\% being spread into classes for two sources (6\%), half missing source (3\%), noise (2\%) and multiple sources (1\%).
%
%\begin{table}[!h]
%	\subfloat[Counts]{
%	\begin{tabular}{|c||c|c|c|c|c|c|c|}
%		\hline 
%		\textbf{Class} & \textbf{0} & \textbf{1} & \textbf{2} & \textbf{3} & \textbf{4} & \textbf{5} & \tabularnewline
%		\hline
%		\hline 
%		\textbf{0} & \textcolor{red}{130} & 4 & 6 & 1 & 18 & 26 & \tabularnewline
%		\textbf{1} & 6 & \textcolor{red}{335} & 21 & 0 & 3 & 13 & \tabularnewline
%		\textbf{2} & 15 & 43 & \textcolor{red}{89} & 0 & 9 & 16 & \tabularnewline
%		\textbf{3} & 3 & 0 & 6 & \textcolor{red}{3} & 9 & 2 & \tabularnewline
%		\textbf{4} & 29 & 7 & 23 & 1 & \textcolor{red}{32} & 1 & \tabularnewline
%		\textbf{5} & 12 & 12 & 8 & 0 & 0 & \textcolor{red}{187} & \tabularnewline
%		\hline
%	\end{tabular}}
%	\hspace{.2cm}
%	\subfloat[Percentage (sum over rows gives 100\%).]{
%	\begin{tabular}{|c||c|c|c|c|c|c|c|}
%		\hline 
%		\textbf{Class} & \textbf{0} & \textbf{1} & \textbf{2} & \textbf{3} & \textbf{4} & \textbf{5} & \tabularnewline
%		\hline
%		\hline 
%		\textbf{0} & \textcolor{red}{70} & 2 & 3 & 1 & 10 & 14 & \tabularnewline
%		\textbf{1} & 2 & \textcolor{red}{89} & 6 & 0 & 1 & 3 & \tabularnewline
%		\textbf{2} & 9 & 25 & \textcolor{red}{52} & 0 & 5 & 9 & \tabularnewline
%		\textbf{3} & 13 & 0 & 26 & \textcolor{red}{13} & 39 & 9 & \tabularnewline
%		\textbf{4} & 31 & 8 & 25 & 1 & \textcolor{red}{34} & 1 & \tabularnewline
%		\textbf{5} & 5 & 5 & 4 & 0 & 0 & \textcolor{red}{85} & \tabularnewline
%		\hline 
%	\end{tabular}}
%	\caption{Confusion matrix for 10-fold cross validation. Correctly classified $\bm{w}_k$ are on diagonal highlighted in red.} \label{tab:Confusion-matrix}
%\end{table}

\begin{figure}[!h]
	\centering
	\newcommand{\sizef}{.95}
	\newcommand{\wf}{.45\textwidth}
	\subfloat[wide-field]
	{
	\includegraphics[width=\wf]{\qd /S580/figures/wfB11_bar4um}
	\label{fig:B11 wf}
	}
	\subfloat[\inmf]
	{
	\includegraphics[width=\wf]{\qd /S580/figures/resB11_bar4um_arrows}
	\label{fig:B11 NMF}
	}\\
	\subfloat[CSSTORM]
	{
	\includegraphics[width=\wf]{\qd S582/figures/res_bar4um_arrows}
	\label{fig:B11 CSSTORM}
	}
	\subfloat[3B]
	{
	\includegraphics[width=\wf]{\qd S583/figures/res_bar4um_arrows2}
	\label{fig:B11 3B}
	}	
	\caption{Comparison of the evaluation of a patch shown in green box in \autoref{fig:tubulin WF and NMF}. Arrows in (b) and (c) point at differences in structure recovered with CSSTORM and \inmf{}, respectively. Arrows in (d) point at local maxima of the 3B results, corresponding to the bright pixels in the wide-field image show in (a). Scale bar 400\unit{nm}.}
	\label{fig:B11 comparison}
\end{figure}

We compared performance of \inmf{}, CSSTORM and the 3B analysis on one patch of real data. Visualisation of results from $20\times20$ patch covering the green box in \autoref{fig:tubulin WF and NMF} are shown in \autoref{fig:B11 comparison}. The structure revealed with \inmf{} and CSSTORM (completely unresolved in wide-field image) differs in several places. The arrows in \autoref{fig:B11 NMF} and \autoref{fig:B11 CSSTORM}  point at some differences in  \inmf{} and CSTORM recovered sample structure, respectively. However, unlike in the simulated artificial structure (\autoref{fig:comparison hash dens 12}), we do not have ground truth for this dataset. Quantitative comparison of the results is therefore difficult. The 3B analysis (20 iterations of the algorithm) delivered very poor results \autoref{fig:B11 3B} and did not provide any further information about the sub-diffraction structure. Coloured arrows point at local maxima, which correspond to the bright pixels in wide-field image \autoref{fig:B11 wf}. These are the only significant features of the 3B-recovered image.

\begin{table}[!h]	
	\centering
%	\begin{tabular}{|c|c|c|}
	\begin{tabular}{|c|c|l|}
		\hline 
		\bf Parameter & \bf Note  & \bf Value\tabularnewline
		\hline
%		\hline 
		$\lambda_{ex}$ & excitation light & 405 nm\tabularnewline
%		\hline 
		$\lambda_{em}$ & emission light & 625 nm\tabularnewline
%		\hline 
		$t_{exp}$ & exposure time  & 50 ms\tabularnewline
%		\hline 
		NA & numerical aperture & 1.4\tabularnewline
%		\hline 
		RI & refraction index & 1.52\tabularnewline
%		\hline 
		pixel-size & size of a pixel in image plane & 79 nm\tabularnewline
%		\hline 
		QD & quantum dots  & QD625\tabularnewline
%		\hline 
		$T$ & number of frames  & $10^{3}$\tabularnewline
		\hline
	\end{tabular}
	\caption{Parameters of the experiment.}\label{tab:Parameters of the (a) simulations (b) real data}
	\label{tab:parameters experiment}
\end{table}

%==========================================
%==========================================

\clearpage
\section{Discussion\label{sec:Discussion}}

\inmf{} can recover individual highly overlapping sources. In the preliminary work \cite{Mandula2010b}, we demonstrated on simulated data with realistic levels of emitted photons and background, that \inmf{} can separate \emph{two} sources as close as $\sim \delta/7$ ($\sim \lambda_{em}/15 = 40 \unit{nm}$ for our simulations with $\lambda_{em}=625 \unit{nm}$) even for emitters with uniformly distributed blinking behaviour (see \autoref{fig:blinking}\aaa), where both sources are present in almost all frames. However, as we discuss below, the situation becomes increasingly difficult with larger number of sources in the sub-resolution area.

\Autoref{fig:density 40 NMF} suggests that it is possible to identify most of the sources for densities up to $\rho=40\um^{-2}$ (46 identified sources out of 58). Further increase of $\rho$ leads to the significant underestimation of the density (blue lines in \autoref{fig:comparison AP,dens}\aaa). The average distance $\bar{r}$ between two nearest neighbours for randomly scattered sources with density $\rho$ is given by \cite{Frieden1991}
%
\begin{equation}
	\bar{r}=\frac{1}{2\sqrt{\rho}}.
\end{equation}
%
The mean nearest neighbour distance corresponding to the density $\rho$ of randomly scattered sources used for our simulations (see \autoref{sub:results - comparison}) is summarised in the following table:
%
\begin{table}[!h]
	\centering
	\begin{tabular}{l|ccccc}
		$\rho$ [$\unit{\um^{-2}}$]		& 10		& 20		& 30		& 40		& 50\\ \hline
		$\bar{r}$ [$\unit{nm}$]		& 158	& 112	& 91		& 79		& 71
	\end{tabular}
	\caption{2D density $\rho$ of the randomly scattered sources and the corresponding mean nearest neighbour distance $\bar{r}$.\label{tab:density2dist}}
\end{table}

The density $40\um^{-2}$ corresponds to $\sim 10$ sources within an Airy disk or mean distance between the nearest neighbours of $79\unit{nm}$ (see \autoref{tab:density2dist}). However, the probability of the source to appear (``ON'') in each frame is 0.5 (see \autoref{fig:blinking}\ddd), which on average reduces the density of the sources in each recorded frame by a factor of two. For data with simulated density of $40\um^{-2}$ the mean distance between the nearest neighbouring ``ON'' sources in each frame is therefore $112 \unit{nm}$, (see \autoref{tab:density2dist}).  

\Autoref{fig:visualisation gaussf}\ccc\ shows that \inmf{} cannot separate all the individual sources uniformly distributed on a line with $\sim \delta/3$ spacing ($80 \unit{nm}$, linear density $12.5 \um^{-1}$ or average spacing of ``ON'' sources $160 \unit{nm}$). One \inmf{} evaluation recovered only a subset of all the sources. In some cases one \inmf{} estimated source represents, in fact, several close emitters. That is why some of the estimated locations (green dots in \autoref{fig:visualisation gaussf}\bbb) fall in between two neighbouring emitters (red dots in \autoref{fig:visualisation gaussf}\bbb). Multiple runs of \inmf{} can recover varying subsets of sources. 
Therefore the sum of these evaluations can provide more complete information about the structure, see \autoref{fig:visualisation gf}, \ref{fig:hash NMF dens 12} and \ref{fig:tubulin WF and NMF}. 


QDs are characterised by broad absorption profiles and a narrow and spectrally tuneable emission spectrum. A range of colours (determined by the size of QD's core) is readily available on market. It is therefore possible to label the specimen with a mixture of QDs with a variety of colours and record the intermittent sources in several different spectral channels. This would lead to a reduction in density of the QDs in each colour channel and facilitate the separation of the individual sources. 

\inmf{} does have any constraints on the shape of the estimated sources. We presented this as an advantage in \autoref{sub:results - out of focus PSF real data}, because such flexibility makes the recovery of different PSFs possible (\autoref{fig:Data-true-estimations} and \ref{fig:Real-data-QDrandom}\bbb). However, there is a lot of information about PSF (compact and sparse object, circular symmetry) which can constrain the space of ``credible'' $\bm{w}$s, and therefore make the recovery of the sources easier. The sparse NMF algorithm, discussed in \autoref{sub:Hoyer}, is an example of reducing the excessive degrees of freedom. As we demonstrated in \autoref{fig: Hoyer sparsity 0.7}, it does not provide satisfactory results, though. Note that \inmf{} uses only a ``soft'' enhancement of the $\bm{w}$s sparsity  (see discussion in \autoref{sub: Iterative restarts}). 

Additional information about $\bm{W}$ or $\bm{H}$ (for example, upper bound on number of emitted photons) can be used in in two different approaches. In the first approach, we specify the constraints on the $\bm{W}$ and (or) $\bm{H}$ and formulate NMF as an optimisation problem (minimising \autoref{eq:KL divergence}) subject to these constraints. Hoyer's sparse NMF \autoref{sub:Hoyer} is in this category. 

In the other, rather heuristic, approach we can use the standard unconstrained NMF and employ the additional information as a quality criterion for the estimated $\bm{W}$ and $\bm{H}$. Results which does not satisfy this criterion can be recomputed by, for example, randomisation or splitting very bright sources or $\bm{w}$s with multiple local maxima into two or more individual components. \inmf{} belongs to the second category.

The disadvantage of the second approach is that it does not reduce the parameter space of the optimisation. Instead of reducing the number of local minima, it tries to search for the ``better'' ones. On the other hand, all the flexibility of NMF is maintained, which makes the method robust. It also reduces the number of parameters, which need to be set by the user.

It should be noted that NMF is applicable to any intermittent fluorescent dyes, such as blinking fluorescent dyes used in dSTORM technique \cite{VandeLinde2011}. Data with high densities of activated (and therefore overlapping) sources can be processed with \inmf{}. Such data require shorter total acquisition time compared to the conventional LM methods, where the individual emitters are separated physically by keeping the number of activated sources sufficiently small. However, \inmf{} is impractical for data with high bleaching rates. For example, in the standard fPALM techniques \cite{Hess2006}, each emitter is activated for one data frame (or few adjacent frames) and then irreversibly destroyed by photo-bleaching.  Even though \inmf{} is applicable to such data, it does not make use of the method's major strength, which is the identification of the sources reappearing throughout the dataset.  
 
Poor performance of the 3B analysis in \autoref{sub:results - comparison} was surprising. Many of well separated sources in \autoref{fig:density 40 3B} were completely missed in the evaluation of simulated data. The method showed weak results even when applied to an artificial structure \autoref{fig:hash 3B  dens 12} or a real dataset of QD labelled tubulin \autoref{fig:B11 3B}. Despite the ability to resolve structures on $50 \unit{nm}$ scale claimed in \cite{Cox2011}, the double lines in  separated by $150 \unit{nm}$ were completely unresolved. The data proved to be too difficult for the method even though the simulated blinking of the sources was generated with a Markov process (without bleaching), one of the assumptions of model, and the prior parameters (PSF, number of sources, blinking rates) of the model were set close to the true values. We spent considerable amount of time to test different parameters' prior values but could not improve the method performance. 

On the other hand, the performance of the CSSTORM was surprisingly high. Average precision and estimated density curves \autoref{fig:comparison AP,dens} were only marginally inferior to the \inmf{} results, even though we tested the method on sources densities four times higher than in the original publication \cite{Zhu2012} ($50 \um^{-2}$ as opposed to $12 \um^{-2}$). However, \inmf{} provided higher details in the recovered artificial structure \autoref{fig:comparison hash dens 12}. Comparison of the CSSTORM and \inmf{} on real data of QD labelled tubulin in \autoref{fig:B11 comparison} is difficult due the fact that the true underlying structure is unknown. 

The computational time of one run of \inmf{} is approximately $10\times$ faster than for CSSTORM (\autoref{sub:results - comparison}). However, for visualisation shown in \autoref{fig:B11 comparison} we used mean of five evaluations of \autoref{alg:restarts} (see \autoref{sub:results - comparison - structure}) and therefore the computational time (about 2 hours for  $10^{3}$ frames with $21 \times 21$ pixels) is comparable for both methods. 

Ability to recover the individually different overlapping sources makes \inmf{} unique when compared with other methods. The only alternative method, ICA, proved to be unsuitable model for noisy data \autoref{fig:Comparison of NMF and ICA} and \ref{fig:ICA slanted bar}. The recovery of different shapes of PSF can be used for determination of the axial position of the emitters \autoref{fig:Simulted-PSF-different-focal-depths} by, for example, determination of the diamterer of the outermost ring \cite{Speidel2003}. However, separation of the overlapping out-of-focus and in-focus PSFs might be problematic due to large difference in the intensity brightness. Photons in the out-of-focus PSF are distributed over much larger area making the PSF considerably dimmer (the maximum brightness of the PSF $1\ \um$ out of focus is only 10\% of the in focus one in \autoref{fig:Simulted-PSF-different-focal-depths}). \inmf{} separated components six to eight in \autoref{fig:iterative restarts robustness} are partially missing on the overlap with the in-focus source. This bright source took over a part of the weaker one. This effect is apparent in all evaluations shown in \autoref{fig:iterative restarts robustness}.

The more promising strategy for determining the axial position of the emitters might be using a specially designed PSF, such as double helix PSF \cite{Quirin2011} or PSF with introduced astigmatism \cite{Huang2008}. Such PSF specifically changes its shape with the axial location. Moreover, the PSF, however distorted, remains fairly compact over an interval of several hundreds nanometres. The difference in brightness of in-focus and out-of-focus PSF is less pronounced (the in-focus PSF is less bright than the one in the system without aberrations), facilitating the separation of individual overlapping emitters. Testing of the \inmf{} for data with special designed PSFs \cite{Huang2008,Quirin2011} might be a subject for future work. 

%==========================================
%==========================================
\clearpage
\section{Conclusion\label{sec:Conclusion}}

In this chapter we demonstrated non-negative matrix factorisation (NMF) as a natural model for microscopic samples labelled with quantum dots. We described a practical pipeline for evaluation and visualisation of realistic datasets and the individual steps of the pipeline were illustrated on simulated data. 

We introduced a procedure of NMF with iterative restarts (\inmf{}), which leads to better local minima in the optimisation procedure and shows robustness in terms of estimated number of sources contained in the data. 

We introduced average precision (AP) as a quantitative measure  of the algorithm performance and used it for exploring behaviour of \inmf{} in different experimental settings. We also used AP for quantitative comparison of \inmf{} with CSSTORM and the 3B analysis demonstrating superior performance of \inmf{} on simulated data of highly overlapping sources. 

The unique ability of \inmf{} to recover individually different sources from data with highly overlapping emitters was demonstrated on simulated three-dimensional object and on real data consisting of randomly scattered out-of-focus QDs. 

Finally, we used \inmf{} for evaluation of larger area of a biological sample with QD labelled tubulin structures. We demonstrated ability of \inmf{} to show sub-resolution features in the specimen.

In conclusion, non-negative matrix factorisation enlarges the family of localisation microscopy techniques and enables using quantum dots as fluorescent labels. It is a promising technique with potential to deliver super-resolution images of three dimensional samples. 
%future work? 
%!TEX root = thesis.tex
\chapter{Theoretical Limits for LM \label{ch:Theoretical-limits-of the LM}}

In this chapter, we discuss the resolution limit and its application to localisation microscopy (LM) from the theoretical point of view. In \autoref{sec:FREM} we compare the classical resolution limit with fundamental resolution measure (FREM), introduced by Ram et al. \cite{Ram2006}. FREM accommodates the resolution criterion for situation of pixelated data corrupted with noise.

\Autoref{sec:CR} introduces the \CR lower bound as a theoretical framework for description of the estimator covariance matrix. In \autoref{sec:FREM orig} we use the \CR lower bound to show the derivation of the Ram's original FREM formula. We also discuss the limitations and problems of the original FREM. In \autoref{sub:An-alternative-derivation-FREM} we derive an alternative version of FREM and demonstrate that this version fixes strange and inconsistent behaviour of the original FREM. In \autoref{sec:FREM for blinking} we derive FREM for two emitters with intermittent intensity. This expression is relevant to localisation microscopy with blinking fluorophores, such as QDs (discussed in \autoref{ch:NMF}). The parameters used for data simulation are discussed in \autoref{sec:FREM simulations}. \Autoref{sec:FREM results} compares FREM for blinking and static sources in different experimental conditions and identifies the regions and experimental parameters setting, where the intermittent behaviour of the intensity allows considerable higher resolution. The discussion of the results is in \autoref{sec:FREM discussion}. 

Details of the derivations are shown in \autoref{app:Appendix2}.

%==========================================
%==========================================

\section{Fundamental resolution measure (FREM)\label{sec:FREM}}

The classical resolution limit \autoref{eq:Airy} discussed in \autoref{sec:Resolution limit} relates to an empirical observation and does not take into account the statistical nature of the photon detection process. It applies to noise-free situation. It also assumes that data were collected with sufficiently small pixels and the effect of pixelation is neglected. 

Ram et al. \cite{Ram2006,Ram2006b} revised the resolution limit and defined a new measure, which considers the statistical process of the photon detection on a pixelated grid of a camera. The so-called \emph{Fundamental resolution measure} (FREM) refers to the achievable precision of the estimator on distance between two sources. FREM reflects the fact, that the ``resolution limit'' is different for sources with different noise levels. If we want to ``resolve'' two sources, the necessary separation must be larger for weak emitters with high background than for bright sources with low background values.

Ram et al. defined FREM as a \CR lower bound on the standard deviation of the source separation estimation. FREM therefore does not provide ``resolution criterion'' such as \autoref{eq:Airy}, but gives as a notion about variability we can expect if we try to measure the distance between two emitters. We can set the ``resolution'' limit arbitrarily according to the measurement precision we are willing to accept. A natural choice for the ``acceptable precision'' is the distance between the sources. I.e. the standard deviation of the source separation measurement is equal to the separation itself. We use this ``natural resolution criterion'' throughout this chapter. 

It is important to note that FREM defined as a \CR lower bound does not consider any specific algorithm for the separation estimation. FREM is derived from the generative model of the dataset.  The standard deviation lower bound can be achieved only with the ``most optimal'' algorithm. 

%For the standard LM techniques such as PALM and STORM \cut{(\autoref{sub:PALM,-STORM})} the spatial resolution limit is determined by the localisation precision for an individual source, because only individual, well separated sources are considered for localisation. 

%The \CR lower bound for the position estimation of a single source detected by a CCD camera is derived in \cite{Ram2006,Ram2006b}. The variance is shown to be proportional to $1/\Lambda$, where $\Lambda$ is the number of photons emitted by the source. 

%==========================================
%==========================================

\section{\CR lower bound\label{sec:CR}}

\CR lower bound is a theoretical framework for description of the estimator covariance matrix. If $\mathcal{L}(\theta)=\log p(x|\theta)$ is a log-likelihood function for data $X$, then a covariance matrix $\bm{Q}$ of an unbiased estimator of $\hat{\theta}$ is bounded by \cite{Rao1945,Cover1991} 
%
\begin{equation}
	\bm{Q}\geq\bm{I}^{-1}(\theta),
	\label{eq:Covariance vs Fisher information}
\end{equation}
%
where the Fisher information matrix $\bm{I}(\theta)$ can be expressed in two equivalent formulas
%
\begin{equation}
	I_{ij}(\theta)=-\E\left[\frac{\partial^2\mathcal{L}}{\partial\theta_i\partial\theta_j}\right]=\E\left[\frac{\partial\mathcal{L}}{\partial\theta_i}\frac{\partial\mathcal{L}}{\partial\theta_j}\right].
	\label{eq:Fisher information general}
\end{equation}

The inequality \autoref{eq:Covariance vs Fisher information} is in the sense that $\bm{Q}-\bm{I}^{-1}(\theta)$ is a non-negative definite matrix.

%==========================================
%==========================================
\section{Original FREM formula\label{sec:FREM orig}}

Ram et al. \cite{Ram2006} considered two sources separated by a distance $d$ and derived the Fisher information
%
\begin{equation}
	I(d)=\frac{1}{4}\sum_{k=1}^N\frac{\left[\Lambda_1q_k'(-\frac{d}{2})-\Lambda_2q_k'(\frac{d}{2})\right]^2}{\Lambda_1q_k(-\frac{d}{2})+\Lambda_2q_k(\frac{d}{2})+b},
	\label{eq:Ram FREM}
\end{equation}
%
where $\Lambda_i$ is the intensity of the $i$th source, $b$ is the background level in each pixel, $q_k(z)=\int_{\Gamma_k}q(x-z)dx$ is the pixelated version of a point spread function translated by $z$ with $\Gamma_k$ being an area of the $k$th pixel. The corresponding pixelated derivative is $q'_k(z)=\int_{\Gamma_k}\frac{\partial q(x-z)}{\partial x}dx$. 

The inverse of the Fisher information bounds the variance of the estimator on $d$ 
%
\begin{equation}
	\var(d)\geq I^{-1}(d).
\end{equation}
%
FREM as a lower bound on the standard deviation is therefore
%
\begin{equation}
	\unit{FREM}=\sqrt{I^{-1}(d)}.
	\label{eq:FREM}	
\end{equation}

A short summary of the derivation is shown in \autoref{app:Appendix2}. 

Closer inspection of FREM derived from the Fisher information given by \autoref{eq:Ram FREM} reveals problematic behaviour of FREM in the limits $d\rightarrow 0$ (see discussion in \autoref{app:Appendix2}). The limit of very close emitters $d\rightarrow0$ gives, as we would expect, zero Fisher information $I(d)\rightarrow0$, and therefore FREM$\rightarrow\infty$. However, this is only for situation, when the sources have equal intensities $\Lambda_1=\Lambda_2$. For emitters of unequal strength $\Lambda_1\neq\Lambda_2$ the variance remains finite even for sources infinitely close. 

Another serious problem with this expression is that the sources are assumed to be located at $\pm d/2$, which implicitly assumes the knowledge of the origin. It is therefore not surprising that the formula \autoref{eq:Ram FREM} gives $I(d)\neq0$ (i.e. finite FREM) even when one source is missing ($\Lambda_i=0$), because, in fact, only one source is needed to determine the distance $d/2$. 

In the following section we present an alternative derivation of FREM, which these problems. Our version gives diverging FREM for $d\rightarrow0$ for sources with different intensities. It also diverges in the situation when one of the sources is missing. For sources with equal intensities $\Lambda_1=\Lambda_2$ our version and the original version of FREM give identical results. 

%==========================================
%==========================================

\section{An alternative derivation of FREM\label{sub:An-alternative-derivation-FREM}} 
 
We assume two sources located along a line at positions $c_1$ and $c_2$ with intensities $\Lambda_1$ and $\Lambda_2$, respectively. If both sources have identical PSFs (here denoted as $q(x)$) we can express the intensity as:
%
\begin{equation}
	\lambda(\bm{c})=\Lambda_1q(x-c_1)+\Lambda_2q(x-c_2).
	\label{eq:lambda}
\end{equation}
%
The distance between the two sources is $d=c_1-c_2$, which is a linear combination $\bm{a}^{T}\cdot\bm{c}$ of the variable $\bm{c}=(c_1,c_2)^{T}$, where $\bm{a}=(1,-1)^{T}$. The variance of $d$ is therefore given by 
%
\begin{alignat}{2}
	\var(d)
	&=\var(\bm{a}^{T}\cdot\bm{c})\nonumber\\
	&=\bm{a}^{T}\cdot\bm{Q}\cdot\bm{a},
	\label{eq:var d from Q}
\end{alignat}
%
where $\bm{Q}$ is the covariance matrix with lower bound given by the inverse of the Fisher information matrix (see \autoref{eq:Covariance vs Fisher information} and \autoref{eq:Fisher information general}):
%
\begin{equation}
	\bm{Q}\geq\bm{I}^{-1}(\bm{c})=\frac{1}{I_{11}I_{22}-I_{12}^2}\left(
	\begin{array}{cc}
		I_{22} & -I_{12}\\
		-I_{12} & I_{11}
	\end{array}\right).
	\label{eq:inverse I}
\end{equation}
%
Expressing the elements of the covariance matrix $\bm{Q}$ from \autoref{eq:inverse I} and substitution to \autoref{eq:var d from Q} gives the expression for $\var(d)$ from the elements of the Fisher information matrix
%
\begin{alignat}{2}
	\var(d)
	&=Q_{11}+Q_{22}-2Q_{12}\nonumber\\
	&\geq\frac{I_{11}+I_{22}+2I_{12}}{I_{11}I_{22}-I_{12}^2}.
	\label{eq:variance d alternative}
\end{alignat}

We assume that the recorded images are corrupted with Poisson noise only (denoted here as $\Po(n;\lambda)$, or sometimes in a shorter version $\Po(\lambda)$, leaving only the expectation value $\lambda$ as an argument). The probability distribution of $n_k$ photons detection in the $k$th pixel is therefore
%
\begin{equation}
	p(n_k|\bm{c})=\Po\left(n_k;\lambda_k(\bm{c})\right),
\end{equation}
%
where $\lambda_k$ is the expected intensity in pixel $k$. It is obtained by integration of the intensity distribution $\lambda(x)$ from \autoref{eq:lambda} over the area of a pixel $\Gamma_k$:
%
\begin{equation}
	\lambda_k(\bm{c})=\int_{\Gamma_k}\Lambda_1q(x-c_1)+\Lambda_2q(x-c_2)dx+b.	
	\label{eq:intensity pixel}
\end{equation}
%
The constant $b$ is a homogeneous background in each pixel.

If we suppose uncorrelated noise between pixels, we get the log-likelihood function for $N$ pixels: 
%
\begin{equation}
	\mathcal{L}=\sum_{k=1}^N\log p(n_k|\bm{c})=\sum_{k=1}^N\log\left[\Po\left(n_k;\lambda_k(\bm{c})\right)\right].
	\label{eq:FREM likelihood Poisson}
\end{equation}
%
Inserting $\mathcal{L}$ into \autoref{eq:Fisher information general}, the elements of the Fisher information matrix become (see \autoref{eq:app-Fisher Information alternative - Individual} in \autoref{app:Appendix2} for details)
%
\begin{equation}
	I_{ij}(\bm{c})=\sum_{k=1}^N\frac{1}{\lambda_k}\frac{\partial\lambda_k}{\partial c_i}\frac{\partial\lambda_k}{\partial c_j};\; \ i,j=\{1,2\}.
	\label{eq:FI - entries}
\end{equation}
%
By substitution from \autoref{eq:intensity pixel} we get for the individual elements of the Fisher information matrix (see \autoref{eq:app-Fisher Information alternative - Individual} in \autoref{app:Appendix2} for details): 
%
\begin{equation}
	I_{ij} =\Lambda_i\Lambda_j\sum_{k=1}^{K}\frac{q'_k(c_i)q'_k(c_j)}{\Lambda_1q_k(c_1)+\Lambda_2q_k(c_2)+b};\; \ i,j=\{1,2\},
	\label{eq:FI - individual}
\end{equation}
%
where $q_k(c_i)$ and $q'_k(c_i)$ are the pixelated versions (pixel area $\Gamma_k$) of the PSF and the derivative, respectively:
%
\begin{alignat*}{2}
	q_k(c_i) & =\int_{\Gamma_k}q(x-c_i)dx\\
	q'_k(c_i) & =\int_{\Gamma_k}\frac{\partial q(x-c_i)}{\partial x}dx.
\end{alignat*}

For equally strong sources ($\Lambda_1=\Lambda_2=\Lambda$) we get a compact expression for the entries of the Fisher information: 
%
\begin{equation}
	I_{ij} =\Lambda\sum_{k=1}^{K}\frac{q'_k(c_i)q'_k(c_j)}{q_k(c_1)+q_k(c_2)+b/\Lambda};\; \ i,j=\{1,2\},
	\label{eq:FI - individual - equal strength}
\end{equation}
%
and due to the symmetry of the entries ($I_{11}=I_{22}$ and $I_{12}=I_{21}$) the variance can be expressed as
%
\begin{equation}
	\var(d)\geq\frac{2}{I_{11}-I_{12}}.
	\label{eq:var symmetric}
\end{equation}
%
Inserting the matrix elements \autoref{eq:FI - individual - equal strength} into \autoref{eq:var symmetric} shows that for situations where the background level is considerably smaller than the intensity $b/\Lambda\ll1$, the lower bound on variance scales as
%
\begin{equation}
	\var(d)\propto\frac{1}{\Lambda}. 
\end{equation}
%
However, the exact value depends on the shape of the PSF $q(x)$.

In \autoref{app:Appendix2} we show the equivalence of the original FREM \autoref{eq:Ram FREM} and our version \autoref{eq:FI - individual} for sources with equal strength ($\Lambda_1=\Lambda_2$). However, as we demonstrate in \autoref{sec:comparison orig and new FREM}, the expression gives very different results for sources of unequal intensity. 

FREM computed from \autoref{eq:FI - individual} have reasonable behaviour in the limits $d\rightarrow0$ and $d\rightarrow\infty$ (see \autoref{sec:Appendix FI alternative} for details). The limit of very close sources ($d\rightarrow0$) gives FREM$\rightarrow\infty$ for any value of $\Lambda_i$ and $\Lambda_j$. Also, in contrast to the original FREM expression, FREM diverges if one of the sources is missing $\Lambda_i=0$, because we do not make any assumption about the symmetry with respect to the origin. 

For well separated sources ($d\rightarrow\infty$) the off-diagonal elements of the Fisher information matrix vanish ($I_{ij}=0$ for $i\neq j$) and from \autoref{eq:variance d alternative} we get
%
\begin{equation}
 	\var(d)\geq\frac{1}{I_{11}}+\frac{1}{I_{22}}.
\end{equation}
%
Fraction $1/I_{ii}$ is the lower bound for the variance of an individual source $s_i$ localisation. The bound on the total variance is therefore composed from the sum of bounds on variances for localisation of individual sources, as we expect.


%==========================================
%==========================================

\section{FREM for blinking sources\label{sec:FREM for blinking}}

Fundamental resolution measure discussed in the previous section considers only the total number of photons $\Lambda_i$ emitted by each source $s_i$. In this section we derive FREM for sources with intermittent intensity and compare it to the ``static'' FREM derived above. 

To address this question we assume a simple model of Poisson distributed data with expected pixel values $\lambda_k$ (\autoref{eq:intensity pixel}). To account for the intermittent behaviour of the intensity, we turn the intensity vector $\bm{\Lambda}=(\Lambda_1,\Lambda_2)$ into a random variable distributed over four distinctive states (indexed with a superscript $\alpha$):
%
\begin{equation}
	\left\{ \bm{\Lambda}^{\alpha=1}=(\Lambda_1,0),\,\bm{\Lambda}^{\alpha=2}=(0,\Lambda_2),\,\bm{\Lambda}^{\alpha=3}=(\Lambda_1,\Lambda_2),\,\bm{\Lambda}^{\alpha=4}=(0,0)\right\},
	\label{eq:intensity states}
\end{equation}
%
which is a simple model of, for example, two blinking quantum dots. The expected intensity in the $k$th pixel when $\bm{\Lambda}$ is in the state $\bm{\Lambda}^\alpha$ is then $\lambda_k^\alpha=\lambda_k(\bm{\Lambda}^\alpha)$:
%
\begin{alignat}{4}
	\lambda_k^{\alpha=1}&=\Lambda_1q_k(x-c_1) & &+b,\nonumber\\ 
	\lambda_k^{\alpha=2}&=&\Lambda_2q_k(x-c_2) &+b,\nonumber\\ 
	\lambda_k^{\alpha=3}&=\Lambda_1q_k(x-c_1)&+\Lambda_2q_k(x-c_2)&+b,\nonumber\\ 
	\lambda_k^{\alpha=4}&=& &+b,
	\label{eq:lambda states}
\end{alignat}
%
where homogeneous background $b$ was added to each pixel.

%==========================================
\subsection{Averaging the Fisher information\label{sub:avg FI}}

The ``averaging'' of the Fisher Information matrix presented in this section assumes knowledge of the intensity state (ON/OFF) of each source in every acquired frame. This information is not accessible in the real situation. However, we show the derivation to emphasise the difference between this approach and the more realistic situation, where the intensity states are described by probability distribution (\autoref{sub:FI int out}). 

If the intensity states $\bm{\Lambda}$ \emph{were known}, we would write the log-likelihood function as 
%
\begin{equation}
	\mathcal{L}(\theta,\Lambda)=\sum_{k=1}^K\log\left(l_k(\theta,\bm{\Lambda})\right).
\end{equation}
%
and the expected Fisher information matrix would become (see \autoref{app:Appendix2} for details)
%
\begin{equation*}
	I(\theta) = \int_{\bm{\Lambda}}p(\bm{\Lambda})I_{\bm{\Lambda}}(\theta)d\bm{\Lambda},
\end{equation*}
%
where $I_{\bm{\Lambda}}(\theta)$ is the Fisher information computed for a specific value of $\bm{\Lambda}$ (see \autoref{eq:FI - entries}).
%
For discrete states of $\bm{\Lambda}^\alpha$ shown in \autoref{eq:lambda states} we get
%
\begin{equation}
	I(\theta)=\sum_{\alpha}p(\bm{\Lambda}^\alpha)I_{\bm{\Lambda}^\alpha}(\theta),
	\label{eq:FI avg}
\end{equation}
%
where the Fisher Information for every configuration of $\bm{\Lambda}^\alpha$ is averaged with weights $p(\bm{\Lambda}^\alpha)$. 


%%
%\begin{equation}
%	I_{ij}(\bm{c})=\sum_{t=1}^T\sum_{\alpha=1}^4p(\bm{\Lambda}^\alpha_t)\sum_{k=1}^N\frac{1}{\lambda_k(\bm{c},\bm{\Lambda}^\alpha_t)}\frac{\partial\lambda_k(\bm{c},\bm{\Lambda}^\alpha_t)}{\partial c_i}\frac{\partial\lambda_k(\bm{c},\bm{\Lambda}^\alpha_t)}{\partial c_j},
%	\label{eq:Fisher Information Blinking Cheating}
%\end{equation}
%%
%which is the expectation value (with respect to the states $\bm{\Lambda}$) of the Fisher information matrix \autoref{eq:FI - entries} for each time frame, followed by the summation over all frames.

%==========================================

\subsection{Integrating over the intensity states\label{sub:FI int out}}
%
However, we assume that the variable $\bm{\Lambda}$ is fully described by the probability $p(\bm{\Lambda})$ over the states. The exact state in time frame is unknown. Therefore we have to integrate over $\bm{\Lambda}$. The likelihood function is then
%
\begin{alignat}{2}
	l(\theta)
	&=\prod_{k=1}^Np(n_k|\theta)\nonumber\\
	&=\prod_{k=1}^N\int_{\bm{\Lambda}}p(n_k,\bm{\Lambda}|\theta)\nonumber\\
	&=\prod_{k=1}^N\sum_{\alpha=1}^4p(n_k|\theta,\bm{\Lambda}^\alpha)p(\bm{\Lambda}^\alpha).
	\label{eq:FREM likelihood Lambda integrated out}
\end{alignat}
%
This complicates the evaluation of the Fisher information matrix \autoref{eq:Fisher information general} because of the summation within the logarithm in the log-likelihood
%
\begin{equation}
	\mathcal{L}(\theta)=\log l(\theta)=\sum_k\log\left(\sum_{\alpha=1}^4p(n_k|\theta,\bm{\Lambda}^\alpha)p(\bm{\Lambda}^\alpha)\right).
	\label{eq:log likelihood integrated out}
\end{equation}
%
In \autoref{app:Appendix2} we show that the Fisher information matrix for uniform distribution $p(\bm{\Lambda}^\alpha)=\frac{1}{4}$ over the four intensity states \autoref{eq:intensity states} is given by
%
\begin{equation}
	I_{rs}(\theta) =\sum_{k=1}^N\E_k\left[\frac{\left(\sum_{\alpha=1}^4\frac{\partial\Po(\lambda_k^\alpha)}{\partial c_r}\right)\left(\sum_{\alpha=1}^4\frac{\partial\Po(\lambda_k^\alpha)}{\partial c_s}\right)}{\left(\sum_{\alpha=1}^4\Po(\lambda_k^\alpha)\right)^2}\right],
	\label{eq:Fisher Information Blinking Integrating Out}
\end{equation}
%
where $\E_k\left[.\right]$ represents the expectation value with respect to $p(n_k,\bm{\Lambda}|\theta)$ (see \autoref{eq:log likelihood integrated out}). 

Expressing the derivatives and the expectation value gives
%
\begin{alignat}{2}
	I_{rs}(\theta)
	&=\frac{1}{4}\sum_{k=1}^N\left(\frac{\partial\lambda_k^{\alpha=r}}{\partial c_r}\right)\left(\frac{\partial\lambda_k^{\alpha=s}}{\partial c_s}\right)\times \nonumber\\
	&\times \sum_{n_k\geq0}\left[\frac{\left(\sum_{\alpha=\{r,3\}}\Po(n_k;\lambda_k^\alpha)\frac{(n_k-\lambda_k^\alpha)}{\lambda_k^\alpha}\right)\left(\sum_{\alpha=\{s,3\}}\Po(n_k;\lambda_k^\alpha)\frac{(n_k-\lambda_k^\alpha)}{\lambda_k^\alpha}\right)}{\sum_{\alpha=1}^4\Po(n_k;\lambda_k^\alpha)}\right].
	\label{eq:FI-blinking}
\end{alignat}
%
In \autoref{app:Appendix2} we show that the limit $d\rightarrow0$ gives $\var(d)\rightarrow\infty$ and the limit $d\rightarrow\infty$ gives $\var(d)\geq\frac{1}{I_{11}}+\frac{1}{I_{22}}$. We also show, that for well separated sources ($d\rightarrow\infty$) and negligible background ($b\ll\Lambda$) the variance $\var(d)$ is identical for both blinking and static situation, if the total number of emitted photons is kept constant. 

%==========================================
%==========================================
\clearpage
\section{Experimental parameters and numerical evaluations\label{sec:FREM simulations}} 
We made a comparison of the original FREM formula computed from the Fisher information \autoref{eq:Ram FREM} with our proposed fixed FREM formula computed from \autoref{eq:FI - individual}, for sources with static intensity. We also compared the static situation with FREM for sources with intermittent intensity computed from \autoref{eq:FI-blinking}.

We considered $625\unit{nm}$ emission light wavelength and $1.2\unit{NA}$ objective. Images were pixelated with $80\times80\unit{nm}$ pixels. Various intensity of the emitters $\Lambda_i$ and pixel background levels $b$ were considered.

The pixelated version $q_k(c_i)$ of the continuous PSF $q(x-c_i)$ and the corresponding derivatives $q'_k(c_i)$ from \autoref{eq:Ram FREM} and \autoref{eq:FI - individual} were computed by summing $10\times10$ pixels of $10\times$ oversampled images (approximation of the continuous PSF $q(x)$ on the $8\times8 \unit{nm}$ grid). The pixelated $\lambda^\alpha_k$ in \autoref{eq:Fisher Information Blinking Integrating Out} was computed in a similar manner. 

The expectation values in \autoref{eq:Fisher Information Blinking Integrating Out} were evaluated using the expression \autoref{eq:FI-blinking}. The set of images for a range $n_k=[0..n_{max}]$ was computed to perform the summation $\sum_{n_k\geq0}$. The value of $n_{max}$ was set such that the Poisson cumulative distribution function $F$ for the maximum intensity $\max_{k,\alpha}(\lambda_k^\alpha)$ satisfies $F(n>n_{max})>1-t$, with $t=10^{-6}$.

%==========================================
%==========================================

\clearpage
\section{Results\label{sec:FREM results}}

We computed the FREM for simulated datasets corresponding to different experimental settings, such as the separation of the sources $d$, the total number of emitted photons by each source $\Lambda$ and the background offset $b$ in the recorded frames. FREM gives us the lower bound on the standard deviation ($\sqrt{\var(d)}$) for the measurement of the source separation $d$. The source separation equivalent to FREM ($d$=FREM) can be considered as a ``natural resolution limit'', which takes the statistical nature of the photon detection into account.

%==========================================
\subsection{Comparison of the original and proposed FREM formula\label{sec:comparison orig and new FREM}}

We compared FREM computed from the original \autoref{eq:Ram FREM} and our proposed \autoref{eq:FI - individual} formula of the Fisher information for two static sources. It can be shown (\autoref{app:Appendix2}), that if the sources have equal strength ($\Lambda_1=\Lambda_2$), both formulas give the identical results. However, for unequal sources $\Lambda_1\neq\Lambda_2$ the 
FREM values differ significantly. 
%
\begin{figure}[!hbt]
	\centering
	\newcommand{\wf}{.49\textwidth}
	\includegraphics[width=\wf]{\qd gFREM/images/FREM_statVsRAM_longrange_bg100fix}
%	\begin{tabular}{cc}
%		\subfloat[FREM (fixed background 100 photons)]{\includegraphics[width=\wf]{\qd gFREM/images/FREM_statVsRAM_longrange_bg100fix}}
%		&\subfloat[ratio]{\includegraphics[width=\wf]{\qd gFREM/images/FREM_staticVsRAM_ratio_longrange_bg100fix}}
%	\end{tabular}	
	\caption{Comparison of the original FREM formula computed from \autoref{eq:Ram FREM} (dashed line) and our proposed FREM formula \autoref{eq:FI - individual} (solid line) for two sources with unequal intensities $\Lambda_2=2\Lambda_1$ and background $b=100$ phot/pixel. The black dotted curve corresponds to FREM=$d$. This would be a straight line with unit gradient in a linear plot. \cut{(b) Ratio of the curves showing how many times is FREM higher for our proposed formula compared to the original formula.}} 
	\label{fig:Comparison FREM Ram and fix}
\end{figure}

The sources $s_i$ were represented with an in-focus PSF centred at $c_i$. The intensity of $s_2$ was set to double of the intensity of $s_1$: $\Lambda_2=2\Lambda_1$. Three different intensity levels $\Lambda_1=500,\,3000$ and $10^4$ photons with homogeneous background $b=100\unit{photons/pixel}$ were considered. \Autoref{fig:Comparison FREM Ram and fix} shows FREM (lower bound on $\sqrt{\var(d)}$) for a range of sources separations $d$ evaluated with the original FREM, computed from \autoref{eq:Ram FREM} (dashed line) and our proposed FREM \autoref{eq:FI - individual} (solid line).

The original FREM formula gives consistently lower FREM (dashed curves are under the solid lines for the whole range of $d$ in \Autoref{fig:Comparison FREM Ram and fix}). The original FREM (dashed curves) also tends to finite values even for $d\rightarrow 0$. We discuss this behaviour further in \autoref{sub:LL surface}. 

%	\Autoref{fig:Comparison FREM Ram and fix}\bbb\ shows the ratio of the curves
%	%
%	\begin{equation}
%		r=\frac{\unit{FREM}^{static}(d)}{\unit{FREM}^{orig}(d)},
%	\end{equation} 
%	%
%	and shows how many times the original FREM is lower compared to our proposed expression. The difference is less significant for larger separation of the sources, however the formulas never converge to the same value (see \autoref{app:Appendix2} for details).

%==========================================

\subsection{FREM for static and blinking sources\label{sub:FREM static vs blinking}}
%
In order to compare the blinking situation \autoref{eq:Fisher Information Blinking Integrating Out} with the static case \autoref{eq:FI - individual} we evaluated FREM as a function of the source separation $d$. For the blinking situation we considered equal strength of the sources
%
\begin{equation}
	\Lambda_1^{blink}=\Lambda_2^{blink}=2\Lambda
\end{equation}
%
and the homogeneous background $b^{blink}$ in each pixel of each frame. Because the sources are ``ON'' only in 50\% of the cases (see \autoref{eq:lambda states}), the total number of emitted photons per source per frame is $\Lambda$ on average. 

For the static case we considered the situation of two sources emitting with equal intensities. To keep the total number of emitted photons per frame equal to the blinking case, we set the intensity 
%
\begin{equation}
	\Lambda_1^{static}=\Lambda_2^{static}=\Lambda.
\end{equation}
%
The background values are equal for the blinking and the static case $b^{blink}=b^{static}$.
%
\begin{figure}[!hbt]
	\centering
	\newcommand{\wf}{.49\textwidth}
	\begin{tabular}{cc}
		\subfloat[FREM (fixed $b=$100 phot/pixel)]{\includegraphics[width=\wf]{\qd gFREM/images/FREM_longrange_bg100fix}
		\label{fig:FREM fixed bg}}
		%For bg=0 the ratio should converge to 1/sqrt(2)=0.7
		&\subfloat[Ratio of the curves form (a)]{\includegraphics[width=\wf]{\qd gFREM/images/FREM_ratio_longrange_bg100fix}
		\label{fig:FREM ratio fixed bg}}\tabularnewline
		\subfloat[FREM (fixed $\Lambda=1500$ photons)]{\includegraphics[width=\wf]{\qd gFREM/images/FREM_longrange_int3000fix}
		\label{fig:FREM fixed int}}		
		%For bg=0 the ratio should converge to 1/sqrt(2)=0.7
		&\subfloat[Ratio of the curves form (c)]{\includegraphics[width=\wf]{\qd gFREM/images/FREM_ratio_longrange_int3000fix}
		\label{fig:FREM ratio fixed int}}
	\end{tabular}	
	\caption{{\it Left:} FREM (a) for fixed background $b=100$ photons and three different intensities $\Lambda$ of the sources and (c) for fixed total number of emitted photons $\Lambda=1500$ and three different values of background $b$. Dashed lines correspond to the blinking situation \autoref{eq:Fisher Information Blinking Integrating Out}, solid lines correspond the static situation \autoref{eq:FI - individual}. {\it Right:} Ratio $r$ of the static (solid) to blinking (dashed) curves showing how many times is FREM for the blinking situation lower compared to the static situation. The classical resolution limit $\delta$ (radius of an Airy disk) corresponds to $\delta=320$ nm.}	
	\label{fig:FREM static blinking}
\end{figure}
%
Comparison of FREM as a function of the separation $d$ for the blinking and the static case is shown in \autoref{fig:FREM static blinking}. Three different values of the total number of photons $\Lambda$ were considered in the semi-logarithmic plot \autoref{fig:FREM fixed bg}. All curves are computed for a fixed background level $b=100$ photons/pixel. 

The ratio of the FREM curves
%
\begin{equation}
	r=\frac{\unit{FREM}^{static}(d)}{\unit{FREM}^{blink}(d)}
	\label{eq:ratio}
\end{equation} 
%
for the blinking and the static case are shown in \autoref{fig:FREM ratio fixed bg}. The plot shows how many times is the FREM for the blinking situation lower when compared to the static situation. \Autoref{fig:FREM fixed int} shows the FREM curves for three different background values $b$. The total number of emitted photons per source was set to $\Lambda =1.5\cdot 10^3$ photons.

$\unit{FREM}^{blink}$ (dashed curves) is in general lower than $\unit{FREM}^{static}$. The exception is a small region centred at around $300\unit{nm}$ (see \autoref{sub:Int out vs avg} for more discussion). The difference between the curves is most pronounced for closely spaced sources ($d<100\unit{nm}$) and data with high signal-to-noise ratio (red curves - bright sources with low background). 

\begin{figure}[!htb]
	\centering
	\newcommand{\wf}{.48\textwidth}
	\subfloat[Static]{
	\includegraphics[width=\wf]{\qd gFREM/images/FREM_Static_sep40fix}
	\label{fig:FREM int bg static}
	}
	\subfloat[Blinking]{
	\includegraphics[width=\wf]{\qd gFREM/images/FREM_IntOut_sep40fix}
	\label{fig:FREM int bg blinking}
	}\\
	\subfloat[Static - top view]{
	\includegraphics[width=\wf]{\qd gFREM/images/FREM_Static_flat_sep40fix}
	\label{fig:FREM int bg static top}
	}
	\subfloat[Blinking - top view]{
	\includegraphics[width=\wf]{\qd gFREM/images/FREM_IntOut_flat_sep40fix}
	\label{fig:FREM int bg blinking top}
	}
	\caption{{\it Top:} FREM for (a) static and (b) blinking situation for two sources separated by $d=40$ nm. {\it Bottom:} Top view on the surfaces. Red plane corresponds to situation when FREM is equal to the separation of the sources $d=40$ nm. The region, where the surface is above the red plane (in black) does not allow precise estimation of $d$ ($\unit{FREM}>d$). Examples of the data frames corresponding to the points in the $\Lambda \times b$ plane are shown in  \autoref{fig:two sources int bg}.}
	\label{fig:FREM int bg}
\end{figure}

%
For further comparison of the static and the blinking FREM, we fixed the separation between the two sources to $d=40\unit{nm}$ and computed FREM for a range of background $b$ and intensity  $\Lambda$ values. \Autoref{fig:FREM int bg} compares the static (left) with the blinking situation (right). The red plane corresponds to the ``natural'' resolution limit $\unit{FREM}=d$ (see \autoref{sec:FREM}), where the lower bound on standard deviation of the distance estimation $\sqrt{\var(d)}$ is equal to the separation $d$. In the region, where the black surface is above the red plane, the distance estimation is very imprecise. This region corresponds to the sources closer than the ``natural resolution limit''.  These (black) regions can be easily observed from the top view shown in the bottom plots of \autoref{fig:FREM int bg} demonstrating the increase of the ``resolution region'' for the blinking case.

\begin{figure}[!hbt]
	\centering
	\newcommand{\wf}{.14\textwidth}
	\newcommand{\dirim}{\qd gFREM/images/psf2/text_twosources_}
	\newcommand{\vs}{.4}
	\begin{tabular}{c|ccccc}
		\begin{sideways}\hspace{\vs cm}$b=300$\end{sideways}
		&\includegraphics[width=\wf]{\dirim int2500_bg300}
		&\includegraphics[width=\wf]{\dirim int2000_bg300}
		&\includegraphics[width=\wf]{\dirim int1500_bg300}
		&\includegraphics[width=\wf]{\dirim int1000_bg300}
		&\includegraphics[width=\wf]{\dirim int500_bg300}\\
		\begin{sideways}\hspace{\vs cm}$b=200$\end{sideways}
		&\includegraphics[width=\wf]{\dirim int2500_bg200}
		&\includegraphics[width=\wf]{\dirim int2000_bg200}
		&\includegraphics[width=\wf]{\dirim int1500_bg200}
		&\includegraphics[width=\wf]{\dirim int1000_bg200}
		&\includegraphics[width=\wf]{\dirim int500_bg200}\\
		\begin{sideways}\hspace{\vs cm}$b=100$\end{sideways}
		&\includegraphics[width=\wf]{\dirim int2500_bg100}
		&\includegraphics[width=\wf]{\dirim int2000_bg100}
		&\includegraphics[width=\wf]{\dirim int1500_bg100}
		&\includegraphics[width=\wf]{\dirim int1000_bg100}
		&\includegraphics[width=\wf]{\dirim int500_bg100}\\
		\begin{sideways}\hspace{\vs cm}$b=10$\end{sideways}
		&\includegraphics[width=\wf]{\dirim int2500_bg10}
		&\includegraphics[width=\wf]{\dirim int2000_bg10}
		&\includegraphics[width=\wf]{\dirim int1500_bg10}
		&\includegraphics[width=\wf]{\dirim int1000_bg10}
		&\includegraphics[width=\wf]{\dirim int500_bg10}\\
		\hline	
		&$\Lambda=2500$ & $\Lambda=2000$ & $\Lambda=1500$ & $\Lambda=1000$ & $\Lambda=500$\\
	\end{tabular}
	\caption{Illustration of two simulated sources separated by $d=40$ nm with intensity $\Lambda$ (total number of emitted photons per source per frame) and the background $b$. Data were corrupted with Poisson noise. The red dots indicate the positions of the sources. The layout corresponds to \autoref{fig:FREM int bg}. Numbers at the top and the bottom of each figure state the ratio $r_B=100\unit{FREM}^{blink}/d$ and $r_S=100\unit{FREM}^{stat}/d$ for the blinking and the static situation, respectively, indicating the how many percent of the separation $d$ represents the FREM value. The smaller the values, the higher the precision of the distance estimator.}
	\label{fig:two sources int bg}
\end{figure}
%
\Autoref{fig:two sources int bg} shows the noisy images of two sources with parameters $b$ and $\Lambda$ with layout similar to the graphs in \autoref{fig:FREM int bg}.  The black regions from \autoref{fig:FREM int bg static top} and \autoref{fig:FREM int bg blinking top} correspond to extremely noisy data (top right corner) and the high FREM values are not surprising. 

For high signal-to-noise ratio data (high $\Lambda$ and low $b$,  see bottom-left corner of \autoref{fig:FREM int bg} and \autoref{fig:two sources int bg}) the lower bound on the standard deviation of the separation $d=40\unit{nm}$ estimation can be as low as $7\unit{nm}$ for the blinking case (\autoref{fig:FREM int bg blinking}). For the static case the values of FREM are approximately three times higher ($\sim 20\unit{nm}$,  \autoref{fig:FREM int bg static}). Note that the FREM surface in \autoref{fig:FREM int bg blinking} for the blinking situation has much steeper increase from the sub $40\unit{nm}$ region than the surface for the static case \autoref{fig:FREM int bg static}.

%\clearpage
%\begin{figure}[!bt]
%	\centering
%	\newcommand{\wf}{.15\textwidth}
%	\newcommand{\dirim}{\qd gFREM/images/psf2/text_}
%	\newcommand{\vs}{.4}
%	\begin{tabular}{c|ccccc}
%		\begin{sideways}\hspace{\vs cm}$b=300$\end{sideways}
%		&\includegraphics[width=\wf]{\dirim int2500_bg300}
%		&\includegraphics[width=\wf]{\dirim int2000_bg300}
%		&\includegraphics[width=\wf]{\dirim int1500_bg300}
%		&\includegraphics[width=\wf]{\dirim int1000_bg300}
%		&\includegraphics[width=\wf]{\dirim int500_bg300}\\
%		\begin{sideways}\hspace{\vs cm}$b=200$\end{sideways}
%		&\includegraphics[width=\wf]{\dirim int2500_bg200}
%		&\includegraphics[width=\wf]{\dirim int2000_bg200}
%		&\includegraphics[width=\wf]{\dirim int1500_bg200}
%		&\includegraphics[width=\wf]{\dirim int1000_bg200}
%		&\includegraphics[width=\wf]{\dirim int500_bg200}\\
%		\begin{sideways}\hspace{\vs cm}$b=100$\end{sideways}
%		&\includegraphics[width=\wf]{\dirim int2500_bg100}
%		&\includegraphics[width=\wf]{\dirim int2000_bg100}
%		&\includegraphics[width=\wf]{\dirim int1500_bg100}
%		&\includegraphics[width=\wf]{\dirim int1000_bg100}
%		&\includegraphics[width=\wf]{\dirim int500_bg100}\\
%		\begin{sideways}\hspace{\vs cm}$b=10$\end{sideways}
%		&\includegraphics[width=\wf]{\dirim int2500_bg10}
%		&\includegraphics[width=\wf]{\dirim int2000_bg10}
%		&\includegraphics[width=\wf]{\dirim int1500_bg10}
%		&\includegraphics[width=\wf]{\dirim int1000_bg10}
%		&\includegraphics[width=\wf]{\dirim int500_bg10}\\
%		\hline
%		&$\Lambda=2500$ & $\Lambda=2000$ & $\Lambda=1500$ & $\Lambda=1000$ & $\Lambda=500$\\
%	\end{tabular}
%	\caption{Illustration of a simulated source with intensity $\Lambda$ (total number of emitted photons) and background level $b$ corrupted with Poisson noise. Red dot indicates the position of the source. The numbers $p$ in each frame shows the lower bound on localisation precision $p=\sqrt{var(c_1)}$ along one direction.}
%	\label{fig:PSF int bg}
%\end{figure}
%
\clearpage
\begin{figure}[!bh]
	\centering
	\newcommand{\wf}{.45\textwidth}
	\subfloat[b=100]{
	\includegraphics[width=\wf]{\qd gFREM/images/FREMdip_conjgrad_Lambda500_b100}
	\label{fig:FREM dip}}
	\subfloat[b=0]{
	\includegraphics[width=\wf]{\qd gFREM/images/FREMdip_conjgrad_Lambda500_b0}
	\label{fig:FREM dip b=0}}
	\caption{FREM and measured standard deviation ($\sqrt{\var(d)}$) for the estimation of the separation between two simulated noisy static sources ($\Lambda_1=\Lambda_2=500$ photons). A homogeneous background of (a) $b=100$ photons/pixel and (b) $b=0$ photons/pixel was added to each simulated image before realisation of Poisson noise.  FREM is shown as a red dashed line. Estimated standard deviation from 1000 different realisation of the Poisson noise is plotted with blue circles. (a) FREM curve shows a distinct ``dip'' at $300 \unit{nm}$ when background noise is present in the recorded images. (b) For background-free data FREM curves are monotonically decreasing.}
	\label{fig:FREM dip both}
\end{figure}
%
The FREM formula for static sources derived from \autoref{eq:FI - individual} shows an interesting behaviour for weak sources with large background values. The red dashed curve in \autoref{fig:FREM dip} shows FREM for two sources of equal intensity $\Lambda_1=\Lambda_2=500$ photons with background $b=100$ photons/pixel. This parameter settings corresponds to the top line in \autoref{fig:two sources int d}. Contrary to our intuition, FREM is not monotonic with separation $d$, and the FREM curve shows a ``dip'' at $d\approx300$ nm.

\begin{figure}[!bt]
	\centering
	\newcommand{\wf}{.12\textwidth}
	\newcommand{\dirim}{\qd gFREM/images/psf3/text_twosources_}
	\newcommand{\vs}{.05}
	\begin{tabular}{c|cccccc}
%		\begin{sideways}\hspace{\vs cm}$\Lambda=3000$\end{sideways}
%		&\includegraphics[width=\wf]{\dirim int3000_d50}
%		&\includegraphics[width=\wf]{\dirim int3000_d100}
%		&\includegraphics[width=\wf]{\dirim int3000_d150}
%		&\includegraphics[width=\wf]{\dirim int3000_d200}
%		&\includegraphics[width=\wf]{\dirim int3000_d250}
%		&\includegraphics[width=\wf]{\dirim int3000_d300}\\
		\begin{sideways}\hspace{\vs cm}$\Lambda=500$\end{sideways}
		&\includegraphics[width=\wf]{\dirim int500_d10}
		&\includegraphics[width=\wf]{\dirim int500_d50}
		&\includegraphics[width=\wf]{\dirim int500_d100}
		&\includegraphics[width=\wf]{\dirim int500_d200}
		&\includegraphics[width=\wf]{\dirim int500_d300}
		&\includegraphics[width=\wf]{\dirim int500_d400}\\		
		\begin{sideways}\hspace{\vs cm}$\Lambda=1000$\end{sideways}
		&\includegraphics[width=\wf]{\dirim int1000_d10}
		&\includegraphics[width=\wf]{\dirim int1000_d50}
		&\includegraphics[width=\wf]{\dirim int1000_d100}
		&\includegraphics[width=\wf]{\dirim int1000_d200}
		&\includegraphics[width=\wf]{\dirim int1000_d300}
		&\includegraphics[width=\wf]{\dirim int1000_d400}\\
		\begin{sideways}\hspace{\vs cm}$\Lambda=1500$\end{sideways}
		&\includegraphics[width=\wf]{\dirim int1500_d10}
		&\includegraphics[width=\wf]{\dirim int1500_d50}
		&\includegraphics[width=\wf]{\dirim int1500_d100}
		&\includegraphics[width=\wf]{\dirim int1500_d200}
		&\includegraphics[width=\wf]{\dirim int1500_d300}
		&\includegraphics[width=\wf]{\dirim int1500_d400}\\		
		\begin{sideways}\hspace{\vs cm}$\Lambda=2000$\end{sideways}
		&\includegraphics[width=\wf]{\dirim int2000_d10}
		&\includegraphics[width=\wf]{\dirim int2000_d50}
		&\includegraphics[width=\wf]{\dirim int2000_d100}
		&\includegraphics[width=\wf]{\dirim int2000_d200}
		&\includegraphics[width=\wf]{\dirim int2000_d300}
		&\includegraphics[width=\wf]{\dirim int2000_d400}\\
		\begin{sideways}\hspace{\vs cm}$\Lambda=2500$\end{sideways}
		&\includegraphics[width=\wf]{\dirim int2500_d10}
		&\includegraphics[width=\wf]{\dirim int2500_d50}
		&\includegraphics[width=\wf]{\dirim int2500_d100}
		&\includegraphics[width=\wf]{\dirim int2500_d200}
		&\includegraphics[width=\wf]{\dirim int2500_d300}
		&\includegraphics[width=\wf]{\dirim int2500_d400}\\		
		\hline	
		&$d=10$ & $d=50$ & $d=100$ & $d=200$ & $d=300$ & $d=400$\\
	\end{tabular}
	\caption{Two simulated sources separated by distance $d$ [nm] with intensity $\Lambda$ [total number of emitted photons per source]. The background was set to $b=100$ photons/pixel and the images were corrupted with Poisson noise. The red dots indicate the positions of the sources. Numbers at the top and the bottom of each figure state the ratio $r_B=100\unit{FREM}^{blink}/d$ and $r_S=100\unit{FREM}^{stat}/d$ for the blinking and the static situation, respectively, indicating the how many percent of the separation $d$ represents the FREM value. The smaller the values, the higher the precision of the distance estimator. The classical resolution limit $\delta$ (radius of an Airy disk) corresponds to $\delta=320$ nm.}
	\label{fig:two sources int d}
\end{figure}
%
After careful checking of the derivation and the numerical calculations of the curves (see \autoref{sec:Appendix dip comments} in \autoref{app:Appendix2} for details), we interpret the dip to be a result of two competing factors. The first factor is the decrease of FREM with increasing separation $d$. This is in accordance with our intuition, that the separation between the sources becomes progressively easier to estimate with increasing distance $d$ between the sources (reducing their mutual overlap). While this is the case for the background-free data \autoref{fig:FREM dip b=0}, the situation is more complicated for data with noisy background. Weak sources can ``disappear'' in the strong background noise if they are well separated (cf. right of the top line in \autoref{fig:two sources int d}). The same sources are easier to detect if they are close, because their overlapping PSFs create a bright object in the noisy background (left of the top line in \autoref{fig:two sources int d}). The ``visibility'' of the sources is the second competing factor, which decreases with $d$. The dip in the FREM curves therefore represents an ``optimal separation'' $d$, where the sources are already sufficiently separated to be localised with good precision but still ``visible'' due to their overlapping PSFs. Further increase of the localisation precision with $d$ is not sufficient to compensate for the fact that sources ``disappear'' in noise.

A qualitative confirmation of this counter-intuitive behaviour is shown on simulated data in \autoref{fig:FREM dip both}. We simulated two sources separated with a distance $d$. The intensity and the background was set to the same values as for the theoretical FREM curves ($\Lambda_1=\Lambda_2=500$ photons, $b=100$ photons/pixel, see top row in \autoref{fig:two sources int d}). For each separation $d$ we created $M=1000$ images with different realisation of Poisson noise. Using conjugate gradient optimisation (NETLAB function {\tt conjgrad} \cite{Nabney}), we found the maximum-likelihood estimator of the positions $\bm{c}_{ML}=(c^{ML}_1,c^{ML}_2)$ of two PSFs (see \autoref{eq:intensity pixel}). The initial position for the fitting procedure was set to the true values $\bm{c}_{true}$. The standard deviation of the $M$ estimators $d^{ML}=|c^{ML}_1-c^{ML}_2|$ is plotted with blue circles in \autoref{fig:FREM dip} and shows a ``dip'' similar to the one in the theoretical FREM curve. For zero background ($b=0$) FREM is monotonically decreasing (the dashed red line in \autoref{fig:FREM dip b=0}). Standard deviation of the $M$ estimators is plotted with blue circles in \autoref{fig:FREM dip b=0}. No dip is observed in this case. 

Note that the measured values of the standard deviation (blue circles) are lower than the theoretical lower bound (red dashed curve). This is likely due to the initialisation of the maximum-likelihood fitting with $\bm{c}_{true}$. Random initialisation would be more appropriate, but the measured standard deviation does not provide curves smooth enough to show the ``dip'' clearly.

\cut{Note that FREM for blinking sources is strictly monotonically decreasing with $d$. }


%==========================================
%==========================================

\clearpage
\section{Discussion\label{sec:FREM discussion}}

This section contains a general discussion about FREM and the results presented above. We also discuss and explain the strange behaviour of the original FREM curves. 

In \autoref{sub:LL surface} we give some insight into the behaviour of FREM by visualisation of the expected log-likelihood surface and considering the Fisher information as a measure of the surface's curvature. \Autoref{sub:Int out vs avg} compares the difference between the averaging and integrating over the intensity states $\bm{\Lambda^\alpha}$ in the Fisher information matrix for blinking sources. In \autoref{sub:scaling} we comment on the scaling of FREM with background and the sources' intensity and in \autoref{sub:noise} we shortly discuss noise in the recorded images.

%==========================================
\subsection{Visualisation of the expected log-likelihood surface\label{sub:LL surface}}
\begin{figure}[hbt!]
	\centering
	\newcommand{\sizeff}{.18}
	\newcommand{\sizegg}{.16}
	\newcommand{\ndir}{\qd gFREM/images/LLsurface/}
	\begin{tabular}{cccc}
		\subfloat[$d=300$ nm]{\includegraphics[scale=\sizegg]{\ndir surf_d300_int11000_int21000_bg100}} 		
		& \subfloat[$d=200$ nm]{\includegraphics[scale=\sizegg]{\ndir surf_d200_int11000_int21000_bg100}} 		
		& \subfloat[$d=50$ nm]{\includegraphics[scale=\sizegg]{\ndir surf_d50_int11000_int21000_bg100}} 		
		& \subfloat[$d=0$ nm]{\includegraphics[scale=\sizegg]{\ndir surf_d0_int11000_int21000_bg100}} 		
		\tabularnewline
		\subfloat[$d=300$ nm]{\includegraphics[scale=\sizegg]{\ndir cont_d300_int11000_int21000_bg100}} 
		& \subfloat[$d=200$ nm]{\includegraphics[scale=\sizegg]{\ndir cont_d200_int11000_int21000_bg100}} 		
		& \subfloat[$d=50$ nm]{\includegraphics[scale=\sizegg]{\ndir cont_d50_int11000_int21000_bg100}} 		
		& \subfloat[$d=0$ nm]{\includegraphics[scale=\sizegg]{\ndir cont_d0_int11000_int21000_bg100}} 		
		\tabularnewline
	\end{tabular}
	\caption{Surface of the expected log-likelihood \autoref{eq:Expected log-likelihood} as a function of $\bm{c}=(c_1,\, c_2)$ for different separation $d$ between the two sources, located at $\bm{c}^{true}$ (marked with red dot). The point where the sources exchange their locations is marked with blue cross. The classical resolution limit corresponds to $\delta=320$ nm. Movement along the ``top-left to bottom-right'' diagonal represents moving the points apart (see \autoref{fig:FI space demo} for details).}	
	\label{fig:Expected-log-likelihood-Surface}
\end{figure}

In order to understand the behaviour of the Fisher information matrix, we visualised the surface of the expected log-likelihood \autoref{eq:FREM likelihood Poisson} as a function of the parameter $\bm{c}=(c_1,\, c_2)$ in \autoref{fig:Expected-log-likelihood-Surface}:
%
\begin{alignat}{2}
	\E_{p(n|\lambda^{true})}\left[\mathcal{L}(\bm{c})\right]
	&=\E_{p(n|\lambda^{true})}\left[\sum_{k=1}^N\log p\left(n_k|\lambda_k(\bm{c})\right)\right]\nonumber\\
	&=\sum_{k=1}^N\left(\lambda_k^{true}\log\lambda_k(\bm{c})-\lambda_k(\bm{c})\right)+A.
	\label{eq:Expected log-likelihood}
\end{alignat}
%
Note that the expectation is taken with respect to the ``true'' distribution $\lambda^{true}=\lambda(\bm{c}^{true})$, while the log-likelihood $\mathcal{L}$ is a function of $\bm{c}$. $A$ is independent of $\bm{c}$. 

The surface in \autoref{fig:Expected-log-likelihood-Surface} shows the average log-likelihood for a model with two sources $s_1$ and $s_2$ located at $c_1$ and $c_2$, respectively, for data generated from a model consisting of two sources $s_1^{true}$ and $s_2^{true}$ located at $c_1^{true}$ and $c_2^{true}$, respectively, corrupted with Poisson noise. Parameters of the simulation were $\Lambda=10^3$, $b=100$ and wavelength $625 \unit{nm}$.

\begin{figure}[hb]
	\newcommand{\wf}{.48\textwidth}
	\centering
	\subfloat[]{\includegraphics[width=\wf]{./figures/FisherInfoSurfaceMovementDemo/FIsurfaceDemo1}\label{fig:FI space demo1}} 
	\subfloat[]{\includegraphics[width=\wf]{./figures/FisherInfoSurfaceMovementDemo/FIsurfaceDemo2}\label{fig:FI space demo2}} \\
	\subfloat[]{\includegraphics[width=\wf]{./figures/FisherInfoSurfaceMovementDemo/FIsurfaceDemo3}\label{fig:FI space demo3}} 
	\subfloat[]{\includegraphics[width=\wf]{./figures/FisherInfoSurfaceMovementDemo/FIsurfaceDemo4}\label{fig:FI space demo4}} 
	\caption{Illustration of the translation of sources $s_1$ and $s_2$ along line $l$ and the corresponding movement in the parameter space from \autoref{fig:Expected-log-likelihood-Surface}.}
	\label{fig:FI space demo}
\end{figure}
%
The correspondence between the $(c_1,c_2)$ space of \autoref{fig:Expected-log-likelihood-Surface} and the movement of the sources is illustrated in \autoref{fig:FI space demo}. The coordinates $(c_1,c_2)$ represent the positions of two sources on a line $l$ intersecting both sources $s_1$ and  $s_2$. The origin $o=(0,0)$ corresponds to the geometric centre between $c_1^{true}$ and $c_2^{true}$. Moving along the top-left to bottom-right diagonal (\autoref{fig:FI space demo1}) represents a symmetrical movement of $s_1$ and $s_2$  in opposite directions with respect to $o$, while moving parallel to the top-right to bottom-left diagonal (\autoref{fig:FI space demo2}) represents the translation of $s_1$ and $s_2$ together along $l$, while keeping their distance from each other constant. Moving along a vertical line corresponds to the situation, where the position of $s_1$ fixed while $s_2$ is moving (\autoref{fig:FI space demo3}) and vice versa for horizontal lines (\autoref{fig:FI space demo4}).

For well-separated sources the surface (\autoref{fig:Expected-log-likelihood-Surface}\aaa) has a sharp maximum at $\bm{c}^{true}=(c_1^{true},c_2^{true})$ (red dot in \autoref{fig:Expected-log-likelihood-Surface}\aaa). In fact, there is another equivalent maximum (blue cross in \autoref{fig:Expected-log-likelihood-Surface}\aaa) as the points are interchangeable and the surface is symmetrical along top-right to bottom-left diagonal. It is important to note that the surface falls sharply in all directions around the maximum. In other words, the likelihood of a model $s_1$ and $s_2$ for data generated from $s_1^{true}$ and $s_2^{true}$ drops quickly once the $s_1$ and $s_2$ move anywhere from the ``true'' locations $\bm{c}^{true}$.

Once the true sources $s_1^{true}$ and $s_2^{true}$ come closer together (\autoref{fig:Expected-log-likelihood-Surface}\bbb,\ccc), the maximum of the surface becomes less pronounced, especially along the top-left to bottom-right diagonal. The likelihood of a model $s_1$ and $s_2$ is not very sensitive to small symmetrical movement of $s_1$ and $s_2$ with respect to $o=(0,0)$. 

Once the sources $s_1^{true}$ and $s_2^{true}$ get very close, the saddle point in $o$ disappears and turns into a flat crest (\autoref{fig:Expected-log-likelihood-Surface}\ddd). The likelihood becomes insensitive to small variations of $s_1$ and $s_2$. 

The Fisher information matrix \autoref{eq:Fisher information general} describes the curvature (Hessian) at $\bm{c}^{true}$ (red dot in \autoref{fig:Expected-log-likelihood-Surface}\aaa). For well separated sources \autoref{fig:Expected-log-likelihood-Surface}\aaa\, the curvature is very high in all directions, resulting in a large determinant of the Hessian matrix, which in turn results in a small variance $\var(d)$ of the distance $d=\left|c_1^{true}-c_2^{true}\right|$ estimation  (see \autoref{eq:inverse I}). Once the ``true'' sources get closer, the curvature at the surface maximum decreases leading to larger $\var(d)$. For very close ``true'' sources, the determinant of Hessian becomes zero, and the lower bound on $\var(d)$ diverges. 

The situation of infinitely close sources $c_1^{true}=c_2^{true}$, shown in \autoref{fig:Expected-log-likelihood-Surface}\ddd\  is equivalent to the situation with one source of double intensity and the second source missing, resutling in a divergence of $\var(d)$ in accordance with our discussion of \autoref{eq:FI - individual} of the limit $d\rightarrow 0$ (see \autoref{sec:Appendix FI alternative} in \autoref{app:Appendix2}).

\begin{figure}[!thb]
	\centering
	\newcommand{\sizeff}{.18}
	\newcommand{\sizegg}{.16}
	\newcommand{\ndir}{\qd gFREM/images/LLsurface/}
	\begin{tabular}{cccc}
		\subfloat[$d=300$ nm]{\includegraphics[scale=\sizegg]{\ndir surf_d300_int11000_int22000_bg100}} 		
		& \subfloat[$d=200$ nm]{\includegraphics[scale=\sizegg]{\ndir surf_d200_int11000_int22000_bg100}} 		
		& \subfloat[$d=50$ nm]{\includegraphics[scale=\sizegg]{\ndir surf_d50_int11000_int22000_bg100}} 		
		& \subfloat[$d=0$ nm]{\includegraphics[scale=\sizegg]{\ndir surf_d0_int11000_int22000_bg100}} 		
		\tabularnewline
		\subfloat[$d=300$ nm]{\includegraphics[scale=\sizegg]{\ndir cont_d300_int11000_int22000_bg100}} 
		& \subfloat[$d=200$ nm]{\includegraphics[scale=\sizegg]{\ndir cont_d200_int11000_int22000_bg100}} 		
		& \subfloat[$d=50$ nm]{\includegraphics[scale=\sizegg]{\ndir cont_d50_int11000_int22000_bg100}} 		
		& \subfloat[$d=0$ nm]{\includegraphics[scale=\sizegg]{\ndir cont_d0_int11000_int22000_bg100}} 		
		\tabularnewline
	\end{tabular}
	\caption{Surface of the expected log-likelihood \autoref{eq:Expected log-likelihood} as a function of $\bm{c}=(c_1,\, c_2)$ for different separations $d$ between two sources of unequal intensity $\Lambda_2=2\Lambda_1$, located at $\bm{c}^{true}$ (marked with a red dot). The position where the sources exchange their true locations is marked with a blue cross.}	
	\label{fig:LL surf different intensity}
\end{figure}
%
The symmetry of the surface breaks when we consider two sources with unequal intensity ($\Lambda_1\neq\Lambda_2$), because such sources are no longer interchangeable. The situation for $\Lambda_2=2\Lambda_1$ is shown in \autoref{fig:LL surf different intensity}. The displacement of the stronger source ($s_2$), which corresponds to the movement along the vertical lines in \autoref{fig:LL surf different intensity}\eee-\hhh\ (see \autoref{fig:FI space demo} for explanation), has a dramatic effect on the likelihood of the model. The surface drops steeply in the horizontal direction, while it decreases rather slowly along the horizontal line (displacement of the weaker source $s_1$). For the limit $d\rightarrow 0$, shown in \autoref{fig:LL surf different intensity}\ddd,\hhh, the flat crest in the origin still exists (which results in a divergence of $\var(d)$), however it is not aligned with the top-left to bottom-right diagonal as for the equal sources (see \autoref{fig:Expected-log-likelihood-Surface}\ddd,\hhh). There is a non-zero curvature along this diagonal.

The Fisher information for the original FREM formula \autoref{eq:Ram FREM} is derived from the curvature of the surface along the top-left to bottom-right diagonal (symmetrical displacement of the sources with respect to the origin cf. \autoref{fig:FI space demo}). For the symmetrical situation $\Lambda_1=\Lambda_2$ the original FREM gives the correct results (see \autoref{app:Appendix2} for a mathematical explanation), however, for the asymmetrical case $\Lambda_1\neq\Lambda_2$ the non-zero curvature along the diagonal results in finite FREM even for the limit $d\rightarrow 0$. Our proposed derivation of FREM from the Fisher information matrix (see \autoref{eq:var d from Q}) accommodates for the unequal sources correctly and gives diverging FREM for this limit. 

%==========================================

\subsection{Blinking vs static sources\label{sub:Blink vs static}}

\Autoref{fig:FREM static blinking} and \autoref{fig:FREM int bg} suggest that the intensity blinking can facilitate localisation of closely separated sources when compared to the static situation. The difference is more pronounced for data with high signal-to-noise ratio (low $b/\Lambda$). Quantum dots with intermittent intensity and an order of magnitude higher brightness than the organic fluorophores are therefore interesting for localisation microscopy even from a theoretical point of view.

%==========================================

\subsection{Integrating out $\Lambda$ vs averaging\label{sub:Int out vs avg}}
\begin{figure}[!hbt]
	\centering
	\newcommand{\wf}{.49\textwidth}
		\includegraphics[width=\wf]{\qd gFREM/images/FREM_intoutVSavg_longrange_bg100fix}
%		\subfloat[FREM (fixed $b=$100 phot/pixel)]{\includegraphics[width=\wf]{\qd gFREM/images/FREM_intoutVSavg_bg100fix}}
%		\subfloat[Ratio of the curves form (a)]{\includegraphics[width=\wf]{\qd gFREM/images/FREM_intoutVSavg_ratio_bg100fix}}
		
%	\caption{(a) Comparison of FREM computed from the Fisher information with integration over the states within the log-likelihood function \autoref{eq:FREM likelihood Lambda integrated out} (dashed lines) and the averaging of the Fisher information over different intensity states $\Lambda^{\alpha}$ \autoref{eq:FI avg} (solid lines). Dotted black lines corresponds to FREM$=d$ border. (b) Ratio of the ``sum'' (solid) to ``averaging'' (dashed) FREM curves. It shown how many times is the ``averaging'' FREM smaller than FREM derived from integrating over the states in the Fisher information matrix.}
	\caption{Comparison of FREM computed from the Fisher information with integration over the states within the log-likelihood function \autoref{eq:FREM likelihood Lambda integrated out} (dashed lines) and the averaging of the Fisher information over different intensity states $\Lambda^{\alpha}$ \autoref{eq:FI avg} (solid lines). Background was set to $b=100$ phot/pixel. Dotted black line corresponds to FREM$=d$.}
	\label{fig:FREM int out vs avg}
\end{figure}
% 
As we pointed out in \autoref{sec:FREM for blinking}, in the real situation we do not know the intensity states $\bm{\Lambda^{\alpha}}$ of the individual emitters in each frame (see \autoref{eq:intensity states}). We have to therefore integrate (sum) over these states within the likelihood function \autoref{eq:FREM likelihood Lambda integrated out}, rather than average the Fisher information over different configurations of $\bm{\Lambda^{\alpha}}$ as in \autoref{eq:FI avg}. 

To further emphasise the difference between the ``integrating over states in'' and ``averaging of'' the Fisher information we plot FREM as a function of the separation $d$ for both concepts in \autoref{fig:FREM int out vs avg}.

FREM computed from the averaged Fisher information is consistently lower for the whole range of $d$. It is also lower than the FREM curves for static sources (solid lines in \autoref{fig:FREM ratio fixed bg}), which in certain region cross the FREM corresponding to the blinking sources (dashed lines in \autoref{fig:FREM ratio fixed bg}). In other words, if we {\it knew} the intensity configurations $\bm{\Lambda}^{\alpha}$ of the blinking sources in each recorded frame, we would be able to reach the highest estimation precision. In the blinking situation with the probabilistic description of the intensity states, the integration (summation) over all possible states is required and the estimation precision is lower. 

Note, that the averaging approach does not results in divergence of the FREM for the $d\rightarrow 0$ limit (see \autoref{fig:FREM int out vs avg}\aaa). This is due to the fact that we assume the configuration of the intensity state in each frame to be {\it known}. We can therefore determine the position of each source individually from the frames, where only one source is emitting ($\bm{\Lambda^{\alpha=1}}$ and $\bm{\Lambda^{\alpha=2}}$ in \autoref{eq:intensity states}). The averaging \autoref{eq:FI avg} fills in the (otherwise zero) diagonal entries of the Fisher information matrix with non-zero values and we get a finite precision for the separation estimation even when $d=0$.

%==========================================

\subsection{Scaling of FREM for different levels of intensity and background\label{sub:scaling}}

The Fisher information matrix \autoref{eq:FI - individual - equal strength} for two sources with equal (static) intensity $\Lambda_1=\Lambda_2=\Lambda$ suggests that increasing the intensity of the sources by a factor of $M$, leads to approximately $M$ times higher Fisher information matrix entries (and therefore a $\sqrt{M}$ times lower FREM). More precisely, the dependency of the entries on the intensity and the background is as follows
%
\begin{equation}
	\left\{I_{\Lambda,b}^{static}\right\}_{ij}=\Lambda\frac{A}{B+b/\Lambda},
	\label{eq:FI static on lambda b}
\end{equation}
%
where $A$ and $B$ are independent of $\Lambda$ and $b$. The subscripts $\Lambda,\,b$ in $I_{\Lambda,b}$ express the parametric dependency of the Fisher information on the intensity and background values.  Note that the background and the intensity appear as a ratio $b/\Lambda$ in \autoref{eq:FI static on lambda b}, but the whole expression is multiplied by $\Lambda$. An $M$ fold increase of the acquisition time leads to an $M$ fold increase of both the total number of emitted photons and the background $b$. In this case we get a $\sqrt{M}$ fold decrease of FREM. For zero background $b=0$ 
%
\begin{equation}
	\unit{FREM}^{static}\propto\frac{1}{\sqrt{\Lambda}}.
\end{equation}

The scaling with $\Lambda$ and $b$ of the Fisher information entries for the blinking case is more complicated. In \autoref{app:Appendix2} we show, that the blinking situation gives results equal to the static situation (up to a factor of 1/2, accounting for half of the total number of emitted photons) for the limit of well-separated sources ($d\rightarrow\infty$) and zero background (see \autoref{eq:app-FREM blink lim infty b=0} in \autoref{app:Appendix2}). In the non-zero background situation, $b$ appears within the arguments of Poisson terms organised in a complicated fraction (see \autoref{eq:app-FREM blink limit infty bg} in \autoref{app:Appendix2}) and the dependence of the Fisher information matrix entries on the background is therefore highly non-linear.  

%==========================================

\subsection{A note on noise\label{sub:noise}}

The Poisson distribution models noise associated with the photon detection \cite{PawleyHandbook2006}. This noise, derived from the nature of the signal itself, is often called ``shot noise'' and is present even under ideal imaging conditions free of any noise introduced by the sensor. Noise introduced by the sensor is usually divided into two components: the dark noise (or dark current), which represents the electrons thermally generated in the detector, and the read-out noise (or read noise) associated with analogue-to-digital conversion of the signal. The dark noise (Poisson distributed) is efficiently eliminated by the cooling of the sensor and is negligible in high-performance cameras.  The read-out noise is assumed to follow a Gaussian distribution and is characterised by a standard deviation (often called r.m.s.).  The read-out noise depends on the read-out frequency and can be reduced by optimising the design of the detector's electrical circuits. The standard deviation of the read-out noise for a typical scientific CCD camera is 5-10 electrons. 

We assume Poisson distributed data throughout this chapter. We therefore neglect the read-out noise of the sensor and assume only the shot noise to be present in the recorded images. Ram et al. \cite{Ram2006} shows a modification of the Fisher information for a model combining both Poisson (shot) and Gaussian (read-out) noise.

%==========================================
%==========================================

\clearpage
\section{Conclusions} 

The alternative derivation of the fundamental resolution measure (FREM) provides a correction to the original formula published by Ram et al. \cite{Ram2006}. The results presented in \autoref{fig:FREM static blinking} suggest that the blinking sources can significantly increase the localisation precision compared to the static situation if the total number of emitted photons is kept equal. The increase of the localisation precision for blinking sources is stronger for close ($d<50\unit{nm}$) and bright ($\Lambda>1000$ photons/source/frame) sources with lower background levels ($b<100$ photons/pixel/frame). For well-separated sources ($d\rightarrow\infty$ limit) the static and the blinking situation give identical results. 

Background has a large impact on the localisation precision and it is desirable to keep it as low as possible. In practice a large proportion of the background intensity comes from the out of focus light. Typically TIRF (Total Internal Reflection Fluorescent microscopy) illumination of the sources is used to reduce the out-of-focus blur. However, different techniques such as sparse illumination of the sample can be used. 

Note that the \CR lower bound approach gives a lower bound for the variance of the distance $d$ estimator. The actual ``localisation precision'' depends on the specific algorithm we use for the localisation of the individual sources. For example, our proposed LM algorithm discussed in \autoref{ch:NMF} computationally separates the overlapping sources before localisation. The ability to separate the individual emitters is the limiting factor in this case. It is also important to note that the ``localisation'' becomes much harder with increasing number of emitters in the sub-diffraction area.

In summary bright sources with low background levels (high signal-to-noise ratio) are desirable for the LM techniques. In this setting the blinking provides lower FREM values for closely spaced sources. This makes bright sources with intermittent intensity, such as quantum dots, an interesting candidate for localisation microscopy.


%\begin{figure}[thb]
%	\centering
%	\newcommand{\sizeff}{.18}
%	\newcommand{\sizegg}{.16}
%	\newcommand{\ndir}{\qd gFREM/images/LLsurface/}
%	\begin{tabular}{cccc}
%		\subfloat[$d=400$ nm]{\includegraphics[scale=\sizegg]{\ndir cont_d400_int11000_int21000_bg100}} 
%		& \subfloat[$d=300$ nm]{\includegraphics[scale=\sizegg]{\ndir cont_d300_int11000_int21000_bg100}} 		
%		& \subfloat[$d=200$ nm]{\includegraphics[scale=\sizegg]{\ndir cont_d200_int11000_int21000_bg100}} 		
%		& \subfloat[$d=100$ nm]{\includegraphics[scale=\sizegg]{\ndir cont_d100_int11000_int21000_bg100}} 		
%		\tabularnewline
%		\subfloat[$d=400$ nm]{\includegraphics[scale=\sizegg]{\ndir cont_d400_int1500_int2500_bg100}} 
%		& \subfloat[$d=300$ nm]{\includegraphics[scale=\sizegg]{\ndir cont_d300_int1500_int2500_bg100}} 		
%		& \subfloat[$d=200$ nm]{\includegraphics[scale=\sizegg]{\ndir cont_d200_int1500_int2500_bg100}} 		
%		& \subfloat[$d=100$ nm]{\includegraphics[scale=\sizegg]{\ndir cont_d100_int1500_int2500_bg100}} 		
%		\tabularnewline
%		\subfloat[$d=400$ nm]{\includegraphics[scale=\sizegg]{\ndir cont_d400_int1500_int2500_bg500}} 
%		& \subfloat[$d=300$ nm]{\includegraphics[scale=\sizegg]{\ndir cont_d300_int1500_int2500_bg500}} 		
%		& \subfloat[$d=200$ nm]{\includegraphics[scale=\sizegg]{\ndir cont_d200_int1500_int2500_bg500}} 		
%		& \subfloat[$d=100$ nm]{\includegraphics[scale=\sizegg]{\ndir cont_d100_int1500_int2500_bg500}} 		
%
%	\end{tabular}
%	\caption{Top row: int=1000, bg=100; bottom int=1000, bg=100}	
%\end{figure}
%



%% This is for discussion
%In the blinking case \autoref{eq:Fisher Information Blinking Integrating Out} the dependency on $\Lambda$ is complicated as the expectation in \autoref{eq:Fisher Information Blinking Integrating Out} cannot be simplified and $\var(d)$ depends on $\Lambda$ through the parameter $\lambda_k^i(\Lambda)$ of the Poisson distribution in \autoref{eq:Fisher Information Blinking Integrating Out}. This gives rise to a non-linear relationship between $\var(d)$ and $\Lambda$.
%%
%The comparison of the blinking and the static case for three different values of the mean source intensity $\Lambda$ is shown in \autoref{fig:FREM static blinking}\aaa. In this figure the background intensity was fixed to $bg=100$~photons. The intensity of the blinking sources was set to $2\Lambda$ to keep the mean number of detected photons constant for the static and the blinking case (the blinking sources are on average `ON' only half of the time and so the average intensity is $\Lambda$). 
%
%Localisation precision $\sqrt{\var(d)}$ for two sources separated by 40~nm with equal intensities $\Lambda=10^3$ photons for different background levels is shown in \autoref{fig:effect of the background and illustration of the noisy sources with background}\aaa. This graph correspond to two sources separated by $d=40$~nm with $\Lambda=10^3$~photons ($2\Lambda$ for blinking case). For lower background ($bg<200$) the blinking sources allow higher localisation precision (lower $\var(d)$). Illustration of this data with different background levels is shown in \autoref{fig:effect of the background and illustration of the noisy sources with background}b-e. 
%
%%==========================================
%%==========================================
%

%!TEX root = thesis.tex
\chapter{Line scan structured illumination microscopy}
%LS-SIM

\section{Structured illumination microscopy applied to thick samples}

%==========================================
%==========================================

\section{Line scan - structured illumination microscopy}

%==========================================
%==========================================

\section{Results}

%==========================================
%==========================================

\section{Discussion}

%==========================================
%==========================================

\section{Conclusion}
%!TEX root = thesis.tex
\chapter{Conclusions and Future Work\label{ch:Summary}}

\section{Conclusion}

In this thesis we have made three contributions to super-resolution methods for fluorescence microscopy:

We have shown that non-negative matrix factorisation with iterative restarts (\inmf{}) can separate highly overlapping intermittent sources with arbitrary shape. \inmf{} is comparable in performance to other recently published methods (CSSTORM and 3B analysis). We introduced average precision (AP) as a quantitative measure for comparing the performance of the \inmf{} algorithm. AP can be used for data, with known true locatins of the sources (e.g. simulated data). We compared \inmf{} with CSSTORM and the 3B analysis and demonstrated superior performance of \inmf{} on simulated data of highly overlapping sources. We described a pipeline for evaluating and visualising realistic datasets, and used \inmf{} to show super-resolution images of experimental data consisting of tubulin structures labelled with quantum dots. \inmf{} is a promising and very accessible technique with the potential to deliver super-resolution images of three-dimensional samples.

The combination of structured illumination with line scanning (LS-SIM) presented in this thesis provides images of thick fluorescent samples with resolution improvement in the lateral plane. Line scanning reduces the out-of-focus background and the LS-SIM images suffer less from reconstruction artefacts when compared to conventional structured illumination. LS-SIM reveals the fine details of biological specimens' inner structures with higher resolution than line-confocal microscopy and with the image quality superior to conventional structure illumination.

In addition we discuss the theoretical resolution limit for noisy and pixelated datasets. We present an alternative derivation of fundamental resolution measure (FREM), correcting the original formula published by Ram et al. \cite{Ram2006}. We show that fluorescence intermittency (such as quantum dots blinking) can be beneficial for resolution when compared to the sources with static intensity.

%==========================================
%==========================================

\section{Future Work}

The unique ability of the \inmf{} algorithm to recover sources with different shapes discussed in \autoref{sub:results - out of focus PSF real data} allows extension of the super-resolution imaging to three dimensional samples. The axial position of a fluorophore can be determined from the shape of the recovered out-of-focus PSF by, for example, determination of the diameter of the outmost ring \cite{Speidel2003}. However, the conventional out-of-focus PSF decreases quickly in brightness when compared to the in-focus PSF (see \autoref{fig:Simulted-PSF-different-focal-depths}). This makes it difficult to separate overlapping sources located in different focal planes. The brightness of a tailored PSF, such as the double helix PSF \cite{Quirin2011} or the PSF with introduced astigmatism \cite{Huang2008} is less sensitive to defocus. The out-of-focus PSF remains compact over defocus of several micrometres. On the other hand, the in-focus PSF is less bright than the one in the system without aberrations. The axial position is determined from the specific changes of the PSF shape. Testing the \inmf{} algorithm on data with a tailored PSF is a logical extension of the current work. 

We also want to apply \inmf{} to specimens labelled with standard organic fluorophores dyes. For example, dSTORM \cite{VandeLinde2011} exploits the repetitive transfer of conventional fluorescent probes between bright ON states  and stable and reversible dark OFF states. This results in blinking of the fluorescent sources. The determination of the overlapping sources with \inmf{} can significantly speed up the data acquisition.

Separate publications from \autoref{ch:NMF} and \autoref{ch:Theoretical-limits-of the LM} are in preparation.
\appendix
%!TEX root = thesis.tex
\chapter{NMF Algorithm\label{app:NMF-algorithm}}

Classic NMF updates \cite{Lee2001} minimises the generalised KL divergence between the data matrix $\bm{D}$ and its factorised version $\bm{WH}$ (see \autoref{eq:KL divergence})
%
\begin{equation}
	\mbox{KL}(\bm{D}\parallel\bm{WH})=-\sum_{xt}\left(d_{xt}\log\sum_{k=1}^Kw_{xk}h_{kt}-\sum_{k=1}^Kw_{xk}h_{kt}\right)+C,
	\label{app-eq:KL}
\end{equation}
where $C$ is a constant independent on $\bm{W}$ or $\bm{H}$. The optimisation can be solved by a scaled gradient descent method:
%
\begin{alignat}{1}
	\bm{W} & =\bm{W}-\alpha^{W}\frac{\partial f(\bm{W,H})}{\partial \bm{W}}\nonumber \\
	\bm{H} & =\bm{H}-\alpha^{H}\frac{\partial f(\bm{W,H})}{\partial \bm{H}},
	\label{eq:gradient descend}
\end{alignat}
%
with respect to an objective function $f(\bm{W,H})=\mbox{KL}(\bm{D}\parallel\bm{WH})$. The explicit derivation of the objective function $f$ gives
%
\begin{alignat}{1}
	\frac{\partial f(\bm{W,H})}{\partial w_{xk}} & =\sum_{t=1}^T h_{kt}-\left[(\bm{D}\oslash\bm{WH})\bm{H}^{\top}\right]_{xk}\nonumber \\
	\frac{\partial f(\bm{W,H})}{\partial h_{kt}} & =\sum_{x=1}^N w_{xk}-\left[\bm{W^{\top}}(\bm{D}\oslash\bm{WH})\right]_{xt},
	\label{eq:gradients}
\end{alignat}
%
where the symbol ``$\oslash$'' denotes the element-wise division of matrices. 

Lee and Seung \cite{Lee2001} proposed 
%
\begin{alignat}{1}
	\alpha_{xk}^{W} & =\frac{w_{xk}}{\sum_{t=1}^Th_{kt}}\nonumber \\
	\alpha_{kt}^{H} & =\frac{h_{kt}}{\sum_{x=1}^Nw_{xk}},
	\label{eq:alphas}
\end{alignat}
%
which leads to compact multiplicative updates
%
\begin{alignat}{1}
	w_{xk} & =\frac{w_{xk}}{\sum_{t=1}^Th_{kt}}\left[(\bm{D}\oslash\bm{WH})\bm{H^{\top}}\right]_{xk}\nonumber \\
	h_{kt} & =\frac{h_{kt}}{\sum_{x=1}^Nw_{xk}}\left[\bm{W^{\top}}(\bm{D}\oslash\bm{WH})\right]_{kt}.
	\label{eq:classic updates}
\end{alignat}
%

Auxiliary constraints $J^{W}(\bm{W})$, $J^{H}(\bm{H})$ can be added to the objective function $f(\bm{W,H})$ to enhance certain characteristics of $\bm{W}$, $\bm{H}$:
%
\begin{equation}
	f(\bm{W,H})=f(\bm{W,H})+\beta^{W}J^{W}(\bm{W})+\beta^{H}J^{H}(\bm{H}).
	\label{eq:objective + auxiliary}
\end{equation}
%
With a choice of
%
\begin{alignat}{1}
	\alpha_{xk}^{W} & =\frac{w_{xk}}{\sum_{t=1}^Th_{kt}+\beta^{W}\frac{\partial J^{W}}{\partial w_{xk}}}\nonumber \\
	\alpha_{kt}^{H} & =\frac{h_{kt}}{\sum_{x=1}^Nw_{xk}+\beta^{H}\frac{\partial J^{H}}{\partial h_{kt}}},
	\label{alphas + auxiliary}
\end{alignat}
%
the multiplicative updates change to
%
\begin{alignat}{1}
	w_{xk} & =\frac{w_{xk}}{\sum_{t=1}^Th_{kt}+\beta^{W}\frac{\partial J^{W}}{\partial w_{xk}}}\left[(\bm{D}\oslash\bm{WH})\bm{H^{\top}}\right]_{xk}\nonumber \\
	h_{kt} & =\frac{h_{kt}}{\sum_{x=1}^Nw_{xk}+\beta^{H}\frac{\partial J^{H}}{\partial h_{kt}}}\left[\bm{W^{\top}}(\bm{D}\oslash\bm{WH})\right]_{kt}.
	\label{eq:classic updates + auxiliary}
\end{alignat}
%!TEX root = thesis.tex
\chapter{Resolution limit for blinking QDs\label{app:Appendix2}}

%==========================================
%==========================================

This is derivation of the fisher information for Poisson distributed
variable $X$ with mean $\lambda(\theta)$

\begin{equation}
X\sim\Po(n;\lambda)=\frac{\lambda^{n}e^{-\lambda}}{n!}.
\end{equation}



\section{Likelihood}

Likelihood of the Poisson distributed variable with detection $n_k$ in K pixels: 
%
\begin{equation}
	l(\theta)=\prod_{k=1}^Kl_k=\prod_{k=1}^K\frac{\lambda_k^{n_k}e^{-\lambda_k}}{n_k!},
	\label{eq:app-Likelihood of Poisson}
\end{equation}
%
where $l_k(\theta)=p(n_k|\theta)$ to emphasise the dependency on the parameter $\theta$.

Log-Likelihood:
\begin{equation}
	\LL=\sum_{k=1}^K\LL_k, 
\end{equation}
%
where
%
\begin{equation}
	\LL_k=n_k\log\lambda_k-\lambda_k-\log n_k!. 
	\label{eq:app-loglik}
\end{equation}


%==========================================
%==========================================

\section{Fisher Information}

Fisher information can be expressin in these eqvivalent forms:
\begin{equation}
	I(\theta)=-\E\left[\frac{\partial^2\LL}{\partial\theta^2}\right]=\E\left[\left(\frac{\partial\LL}{\partial\theta}\right)^2\right]=\E\left[\left(\sum_k\frac{\partial\log(l_k)}{\partial\theta}\right)^2\right]=\E\left[\left(\sum_k\frac{1}{l_k}\frac{\partial l_k}{\partial\theta}\right)^2\right].
	\label{eq:app-Fisher Info Definition}
\end{equation}

Therefore
%\begin{alignat}{2}
%	I(\theta) 
%	& =\E\left[\left(\sum_k\frac{1}{l_k}\frac{\partial l_k}{\partial\theta}\right)\left(\sum_m\frac{1}{l_m}\frac{\partial l_m}{\partial\theta}\right)\right]\\
%	& =\E\left[\sum_k\frac{1}{l_k^2}\left(\frac{\partial l_k}{\partial\theta}\right)^2\right]+\E\left[\sum_k\sum_{m\neq k}\frac{1}{l_k}\frac{\partial l_k}{\partial\theta}\frac{1}{l_m}\frac{\partial l_m}{\partial\theta}\right]
%\end{alignat}
\begin{alignat}{2}
	I(\theta) 
	& =\E\left[\left(\sum_k\frac{\partial \LL_k}{\partial\theta}\right)\left(\sum_m\frac{\partial \LL_m}{\partial\theta}\right)\right]\\
	& =\E\left[\sum_k\left(\frac{\partial \LL_k}{\partial\theta}\right)^2\right]+\E\left[\sum_k\sum_{m\neq k}\frac{\partial \LL_k}{\partial\theta}\frac{\partial \LL_m}{\partial\theta}\right].
\end{alignat}
%
Because $n_k$ are iid, the second term can be expressed as 
%
\begin{equation}
	\E\left[\sum_k\sum_{m\neq k}\frac{\partial \LL_k}{\partial\theta}\frac{\partial \LL_m}{\partial\theta}\right] 
	=\sum_k\sum_{m\neq k}\E_k\left[\frac{\partial \LL_k}{\partial\theta}\right]\E_m\left[\frac{\partial \LL_m}{\partial\theta}\right],
\end{equation}
%
where
%
\begin{equation}
	\E_k\left[f(n_k)\right]=\sum_{n_k\geq0}p(n_k|\theta)f(n_k).
\end{equation}
%
But 
\begin{equation}
	\E_k\left[\frac{\partial \LL_k}{\partial\theta}\right]=\E_k\left[\frac{1}{l_k}\frac{\partial l_k}{\partial\theta}\right]=\sum_{n_k}l_k\frac{1}{l_k}\frac{\partial l_k}{\partial\theta}=\sum_{n_k}\frac{\partial l_k}{\partial\theta}=\frac{\partial}{\partial\theta}\sum_{n_k}p(n_k|\theta)=0,
\end{equation}
%
as $\sum_{n_k}p(n_k|\theta)=1$. 

The Fisher Information can then be written as 
%
\begin{equation}
	I(\theta) =\E\left[\sum_k\left(\frac{\partial \LL_k}{\partial\theta}\right)^2\right]
\end{equation}
%
and expressing the derivatives from \autoref{eq:app-loglik}
%
\begin{equation}
	\frac{\partial \LL_k}{\partial\theta}=\left(\frac{n_k-\lambda_k}{\lambda_k} \right)\frac{\partial\lambda_k}{\partial\theta},
\end{equation}
%
we get for Fisher information
%
\begin{alignat}{2}
	I(\theta)
	&=\E\left[\sum_k\frac{(n_k-\lambda_k)^2}{\lambda_k^2}\left(\frac{\partial\lambda_k}{\partial\theta}\right)^2\right]\\
	&=\sum_k\frac{1}{\lambda_k^2}\left(\frac{\partial\lambda_k}{\partial\theta}\right)^2\E\left[(n_k-\lambda_k)^2\right].
\end{alignat}
%
We recognize the variance $\var(n_k)=\E\left[(n_k-\lambda_k)^2\right]$. For Poisson variable 
%
\begin{equation}
	\var(n_k)=\mathrm{mean}(n_k)=\lambda_k,
\end{equation}
%
and therefore Fisher Information becomes
%
\begin{equation}
	I(\theta)=\sum_{k=1}^K\frac{1}{\lambda_k}\left(\frac{\partial \lambda_k}{\partial \theta}\right)^2.
	\label{eq:app-Fisher Info for Poisson}
\end{equation}

%==========================================
%==========================================
\clearpage
\section{Two sources separated by a distance $d$\label{sub:Two-sources-separated}}
This section comments on Fisher Information estimation as described by Ram et al. in \cite{Ram2006}.

For two sources separated by a distance $d$ we have a mean value of the intensity:
%
\begin{equation}
	\lambda(x)=\Lambda_1f_1(x)+\Lambda_2f_2(x),
\end{equation}
%
where $f_i$ and $\Lambda_i$ is the response function and intensity, respectively, of the source $i$. For translationally invariant PSF and in-focus sources: $f_1=q(x-\frac{d}{2})$ and $f_2=q(x+\frac{d}{2})$
%
\begin{equation}
	\lambda(d)=\Lambda_1q(x-\frac{d}{2})+\Lambda_2q(x+\frac{d}{2}),
	\label{eq:app-expected intensity d/2}
\end{equation}
%
where $q$ is the PSF of the sources. For pixelised version (integral over pixel area $\Gamma_k$) with homogeneous background $b$ in each pixel the intensity can be expressed as:
%
\begin{equation}
	\lambda_k(d)=\Lambda_1\int_{\Gamma_k}q(x-\frac{d}{2})dx+\Lambda_2\int_{\Gamma_k}q(x+\frac{d}{2})dx+b,
\end{equation}
%
so we get by plugging into \autoref{eq:app-Fisher Info for Poisson} expression for the Fisher Information:
%
\begin{alignat}{2}
	I(d)
	&=\frac{1}{4}\sum_{k=1}^K\frac{\left(\Lambda_1\int_{\Gamma_k}\partial_{x}q(x-\frac{d}{2})dx-\Lambda_2\int_{\Gamma_k}\partial_{x}q(x+\frac{d}{2})dx\right)^2}{\Lambda_1\int_{\Gamma_k}q(x-\frac{d}{2})dx+\Lambda_2\int_{\Gamma_k}q(x+\frac{d}{2})dx+b}\\
	&=\frac{1}{4}\sum_{k=1}^N\frac{\left[\Lambda_1q_k'(-\frac{d}{2})-\Lambda_2q_k'(\frac{d}{2})\right]^2}{\Lambda_1q_k(-\frac{d}{2})+\Lambda_2q_k(\frac{d}{2})+b},
	\label{eq:app-Fisher Info Pixelised - Ram}	
\end{alignat}
%
where we have set $q_k(z)=\int_{\Gamma_k}q(x-z)dx$ as the pixelised version of a point spread function translated by $z$ with $\Gamma_k$ being an area of the $k$th pixel, and $q'_k(z)=\int_{\Gamma_k}\frac{\partial q(x-z)}{\partial x}dx$ as the corresponding pixelised derivative.

FREM is defined as a lower bound standard deviation ($\sqrt{\var(d)}$) of the source separation $d$ estimation
\begin{equation}
 	\sqrt{\var(d)}\geq\textrm{FREM}=\sqrt{I^{-1}(\theta)}.
\end{equation}�

\subparagraph*{Limit $d=0$:}\ \\
If $\Lambda_1=\Lambda_2$ then $I(d=0)=0$ which means $\var(d=0)\rightarrow\infty$. However, for unequal sources $\Lambda_1\neq\Lambda_2$ this does not hold and the variance remains finite.

\subparagraph*{Limit $d\rightarrow\infty$:}\ \\
When the sources are far apart the mixing term in nominator in \autoref{eq:app-Fisher Info Pixelised - Ram} $\Lambda_1\Lambda_2\partial_{x}q(x-\frac{d}{2})\partial q(d+\frac{d}{2})=0$ as the $\partial_{x}q(x-\frac{d}{2})$ and $\partial_{x}q(x+\frac{d}{2})$ do not have common overlap. \Autoref{eq:app-Fisher Info Pixelised - Ram} then decomposes into two individual terms
%
\begin{equation}
	I(d) 
	=\frac{1}{4}\sum_{k=1}^K\left[\frac{\left(\Lambda_1q'_k(x-\frac{d}{2})\right)^2}{\Lambda_1q_k(x-\frac{d}{2})+b}+\frac{\left(\Lambda_2 q'_k(x+\frac{d}{2})\right)^2}{\Lambda_2q_k(x+\frac{d}{2})+b}\right]
\end{equation}
%
This corresponds to a sum of Fisher Information for localisation of individual sources. 

\subparagraph*{Situation with missing source $\Lambda_i=0,\ \Lambda_j\neq0$:}\ \\
Even if one of the source is missing the Fisher information is strangely non-zero $I(d)\neq0$. The variance remains finite even if one of the sources is not present! This is a consequence of the assumption that the sources are located symmetrically at $\pm d/2$ with respect to the origin. Therefore only one source is needed to determine the distance $d/2$.

%==========================================
%==========================================

\section{An alternative way to derive Fisher information for two sources separated by $d$:}
\label{sec:Appendix FI alternative}
Below we show how to fix the problems with limits for Fisher Information derived above. This gives infinite variance when one of the sources is no present. Also fix strange behaviour of unequal sources for the limit $d\rightarrow0$. 

We consider two sources located at $c_1$ and $c_2$, respectively. The expectation of the intensity is therefore expressed as (cf. \autoref{eq:app-expected intensity d/2}):
%
\begin{equation}
	\lambda_k(\bm{c})=\Lambda_1q_k(x-c_1)+\Lambda_2q_k(x-c_2)+b.
	\label{eq:app-Expected intensity c_1 c_2}
\end{equation}
%
The distance between the sources is $d=c_1-c_2$. This is a linear combination $\bm{a}^T\cdot\bm{c}$ of the variable $\bm{c}=(c_1,c_2)^T$ where $\bm{a}=(1,-1)^T$. The variance of $d$ is given by 
%
\begin{equation}
	\var(d)=\var(\bm{a}^T\cdot\bm{c})=\bm{a}^T\cdot\bm{Q}\cdot\bm{a}=Q_{11}+Q_{22}-2Q_{12},
\end{equation}
%
where $\bm{Q}$ is a covariance matrix with lower bound expressed as an inverse of the the Fisher information matrix $\bm{Q}\geq\bm{I}^{-1}(\theta)$
%
\begin{equation}
	\bm{I}(\theta)=\left(
	\begin{array}{cc}
		I_{11} & I_{12}\\
		I_{21} & I_{22},
	\end{array}\right)
\end{equation}
%
given by generalisation of \autoref{eq:app-Fisher Info for Poisson}
%
\begin{equation}
	I_{ij}(\theta)=\sum_{k=1}^K\frac{1}{\lambda_k}\frac{\partial\lambda_k}{\partial c_i}\frac{\partial\lambda_k}{\partial c_j}.
	\label{eq:app-Fisher Info general lambda}
\end{equation}
%
Not that the Fisher information matrix is symmetrical $I_{12}=I_{21}$ due to exchangebilily of the derivatives.
 
The covariance matrix $\bm{Q}$ is therefore
%
\begin{equation}
	\bm{Q}\geq\bm{I}^{-1}(\theta)=\frac{1}{I_{11}I_{22}-I_{12}^2}\left(
	\begin{array}{cc}
		I_{22} & -I_{12}\\
		-I_{12} & I_{11}
	\end{array}\right)
\end{equation}
%
and the variance of $d=c_1-c_2$ estimator
%
\begin{alignat}{2}
	\var(d)
	&=(1,-1)^T\cdot\bm{Q}\cdot(1,-1)\\
	&\geq\frac{I_{11}+I_{22}+2I_{12}}{I_{11}I_{22}-I_{12}^2}=\frac{p}{r}.
	\label{eq:app-variance alternative}
\end{alignat}

The individual terms of the Fisher Information matrix by using \autoref{eq:app-Expected intensity c_1 c_2} in \autoref{eq:app-Fisher Info general lambda}
%
\begin{equation}
	I_{ij} =\Lambda_i\Lambda_j\sum_{k=1}^K\frac{q'_k(c_i)q'_k(c_j)}{f_k(c_1,c_2)},
	\label{eq:app-Fisher Information alternative - Individual}
\end{equation}
%
where $q_k(c_i)$ is the pixelised version (pixel area $\Gamma_k$) of the PSF
%
\begin{alignat}{2}
	q_k(c_i) & =\int_{\Gamma_k}q(x-c_i)dx\\
	q'_k(c_i) & =\int_{\Gamma_k}\frac{\partial q(x-c_i)}{\partial x}dx
\end{alignat}
%
and $f_k(c_1,c_2)=\Lambda_1q_k(c_1)+\Lambda_2q_k(c_2)+b$.
 
Then the numerator $p=I_{11}+I_{22}+2I_{12}$ in \autoref{eq:app-variance alternative} is given by
\begin{equation}
	p=\sum_{k=1}^K\frac{1}{f_k(c_1,c_2)}\left[\Lambda_1^2q'^2{}_k(c_1)+\Lambda_2^2q'{}_k^2(c_2)+2\Lambda_1\Lambda_2q'_k(c_1)q'_k(c_2)\right].
\end{equation}

The terms in the denominator $r=I_{11}I_{22}-I_{12}^2$ in \autoref{eq:app-variance alternative} are given by
%
\begin{alignat}{2}
	I_{11}I_{22} & =\Lambda_1^2\Lambda_2^2\sum_{k,l}^K\frac{\left(q'_k(c_1)q'_l(c_2)\right)^2}{f_k(c_1,c_2)f_l(c_1,c_2)}\\
	I_{12}^2 & =\Lambda_1^2\Lambda_2^2\sum_{k,l}^K\frac{q'_k(c_1)q'_k(c_2)q'_l(c_1)q'_l(c_2)}{f_k(c_1,c_2)f_l(c_1,c_2)}
\end{alignat}

We now consider the limits of the very close sources ($d\rightarrow 0$) and well separated emitters ($d\rightarrow\infty$) for FREM.

\subparagraph*{Close sources limit: $d\rightarrow0 \Rightarrow c_1\rightarrow c_2 \Rightarrow q_k(c_1)\rightarrow q_k(c_2)$:}\ \\
%
\begin{equation}
	p=(\Lambda_1^2+\Lambda_2^2+2\Lambda_1\Lambda_2)\sum_{k=1}^K\frac{q'{}_k^2(c)}{f_k(c,c)},
\end{equation}
%
which can be further simplified by explicitly substituing $f_k(c_1,c_2)$
%
\begin{equation}
	p=(\Lambda_1+\Lambda_2)\sum_{k=1}^K\frac{q'{}_k^2(c)}{q_k(c)+b/(\Lambda_1+\Lambda_2)}.
\end{equation}
%
$q_k$ and $(q'_k)^2$ are strictly positive functions, therefore the sum is not zero and $p$ is non-zero for any $\Lambda_1,\,\Lambda_2$. 

The two terms in the denominator in \autoref{eq:app-variance alternative} are identical for $c_1=c_2$
\begin{equation}
	I_{11}I_{22}=I_{12}^2
	%\frac{\Lambda_1^2\Lambda_2^2}{\left(\Lambda_1+\Lambda_2\right)^2}\sum_{k,l=1}^K\frac{\left(q'_k(c)\right)^2}{\left(q_k(c)+d/(\Lambda_1+\Lambda_2)\right)}\frac{\left(q'_l(c)\right)^2}{\left(q_l(c)+d/(\Lambda_1+\Lambda_2)\right)}
\end{equation}
%
and therefore 
%
\begin{equation}
	r=I_{11}I_{22}-I_{12}^2=\det\left[\bm{I}(\theta)\right] \equiv 0
\end{equation}
%
for any $\Lambda_i$. $\bm{I}(\theta)$ is therefore a singular matrix for $d=0$ and inverse $\bm{I}^{-1}(\theta)$ does not exist for $c_1=c_2$. However,  the limit $c_1\rightarrow c_2,\,(d\rightarrow0)$ gives $p\neq0,\, r\rightarrow0$ and FREM$=\frac{p}{r}\rightarrow\infty$. FREM therefore diverges for sources with any combination of source intenisties $\Lambda_1$ and $\Lambda_2$. This is in contratst to the original FREM formula \autoref{eq:app-Fisher Info Pixelised - Ram} wich gives diveriging FREM only for equally strong sources $\Lambda_1=\Lambda_2$. 

\subparagraph*{Well separated sources limit $d\rightarrow\infty$:}\ \\
The cross term $I_{ij}$ in \autoref{eq:app-Fisher Information alternative - Individual} vanishes ($I_{ij}=0,\: i\neq j$) because of the multiplication $q'_k(c_1)q'_k(c_2)$, which is very close to zero for well separated PSFs. This assumes that the PSF (and its first derivative) decreases to negligible values for distance far from the center of the PSF. Then from \autoref{eq:app-variance alternative} 
%
\begin{equation}
	\var(d)\geq\frac{1}{I_{11}}+\frac{1}{I_{22}},
\end{equation}
%
which is the sum of bounds on variances for localisatoin of two individual sources:
%
\begin{alignat}{2}
	I_{ii}
	&=\Lambda_i^2\sum_{k=1}^K\frac{q_k'^2(c_i)}{\Lambda_1q_k(c_1)+\Lambda_2q_k(c_2)+b}\\
	&=\Lambda_i\sum_{k=1}^K\frac{q_k'^2(c_i)}{q_k(c_i)+b/\Lambda_i}.
	\label{eq:app-I d to infty}
\end{alignat}
%
We used the fact that the PSFs $q(c_1)$ and $q(c_2)$ are well separated and decrase (with its first derivatives) towards zero in the regions far from their center. Therefore the $q(c_j)$ is negligible in the region where $q'(c_i)$ ($j\neq i$) have any significant values. The term $\Lambda_jq(c_j)$ in the denominator can be therefore neglected. 


If we use Gaussian approximation of the PSF (see Zhang et al. \cite{Zhang2007})
%
\begin{equation}
	q(x-c_i)=\frac{1}{Z}\exp\left(-\frac{(x-c_i)^2}{2\sigma^2}\right)
\end{equation}
% 
with derivatives with respect to $c_i$
%
\begin{alignat}{2}
 	q'(x-c_i) 
	&=\frac{x-c_i}{\sigma^2}\frac{1}{Z}\exp\left(-\frac{(x-c_i)^2}{2\sigma^2}\right)\\
	&=\frac{x-c_i}{\sigma^2}q(x-c_i),
\end{alignat}
%
then from \autoref{eq:app-I d to infty} for the situation with negligible background $b/\Lambda\ll1$
%
\begin{equation}
	I_{ii}=\frac{\Lambda_i}{\sigma^4}\sum_k\int_{\Gamma_k}(x-c_i)^2q(x-c_i)dx.
\end{equation}
%
Now using 
%
\begin{alignat}{2}
	\sum_k\int_{\Gamma_k}(x-c_i)^2q(x-c_i)dx
	&=\int_{\mathbb{R}^2}(x-c_i)^2q(x-c_i)dx\\
	&=\sigma^2,
\end{alignat}�
% 
we obtain the terms of the Fisher information matrix
%
\begin{equation}
	I_{ii}=\frac{\Lambda_i}{\sigma^2}.
\end{equation}
%
These terms corresponds to the localisation of the individual sources. The ``localisatoin precision'' for one souces $s_i$ as used in conventional localisatoin techniques is then bounded by $\sigma/\sqrt{\Lambda_i}$, which corresponds to ``squeezing'' of the intitial `` localisation uncertainity'' $\sigma$ by the square root of the souce's intensity. 

The lower bound on the source separation estimator variance is given by
%
\begin{equation}
	\var(d\rightarrow\infty) \geq\sigma^2\left(\frac{1}{\Lambda_1}+\frac{1}{\Lambda_2}\right).
\end{equation}
%

\subparagraph*{One sources missing: $\Lambda_i=0,\ \Lambda_j\neq0$:}\ \\
$I_{ii}\equiv0$ and $I_{ij}\equiv0$ and so $\det(\bm{I}(\theta))\equiv0$, and matrix is singular. In the limit $\Lambda_i\rightarrow0$ the
variance \autoref{eq:app-variance alternative} $\var(d)\rightarrow\infty$. This is agian in contrast to the original FREM computed from \autoref{eq:app-Fisher Info Pixelised - Ram}, which gives finite FREM even for one source missing (see discussion of limits in \autoref{sub:Two-sources-separated}). 

%==========================================

\section{Comparison of the original FREM with our version}
For equally strong sources ($\Lambda_1= \Lambda_2=\Lambda$) the original FREM formula gives identical results as ours. In this situation \autoref{eq:app-Fisher Information alternative - Individual} gives equality of the diagonal terms $I_{11}=I_{22}$ for any $c_1$ and $c_2$. From \autoref{eq:app-variance alternative}
\begin{alignat}{2}
	\var(d)
	&\geq\frac{2(I_{11}+I_{12})}{I_{11}^2-I_{12}^2}\\
	&\geq\frac{2}{I_{11}-I_{12}}
	\label{eq:app-FREM equal brightness}
\end{alignat}
%
Using $q_k(c_1)=q_k(-d/2)$ and $q_k(c_2)=q_k(+d/2)$ we can rewrite the original FREM expression \autoref{eq:app-Fisher Info Pixelised - Ram} using \autoref{eq:app-Fisher Information alternative - Individual}
\begin{equation}
	\var^{ORIG}(d)\geq\frac{4}{I_{11}-2I_{12}+I_{22}},	
\end{equation}
%
which for $\Lambda_1=\Lambda_2$ ($I_{11}=I_{22}$) reduces to
%
\begin{equation}
	\var^{ORIG}(d)\geq\frac{2}{I_{11}-I_{12}}. 		
\end{equation}

Comparison with \autoref{eq:app-FREM equal brightness} shows the equality of the both formulas for the situation of equally strong sources $\Lambda_1=\Lambda_2$.

For sources with different intensity the expressions gives different resutls. If $\Lambda_2=\alpha\Lambda_1$, then can be shown, that for negligible values of $b/\Lambda_2$ the ratio between the original FREM and our proposed formula for $d\rightarrow\infty$ is $2\sqrt{2}/3$.

%==========================================
%==========================================

\section{Time distribution of the intensities - averaging\label{sec:Appendix - blinking not integrated}}
We assume a likelihood dependent on parameter $\bm{\Lambda}$ (for example, $\bm{\Lambda}=(\Lambda_1,\Lambda_2)$ - intensity of two sources in the recorede frame). If we \emph{knew the configuration} of $\bm{\Lambda}$ we would write the log-likelihood
%
\begin{equation}
	\mathcal{L}(\theta,\Lambda)=\sum_{k=1}^K\log\left(l_k(\theta,\bm{\Lambda})\right).
\end{equation}
%
Derivatives with respect to the parameter $\theta$:
%
\begin{alignat}{2}
	\frac{\partial\mathcal{L}(\theta,\bm{\Lambda})}{\partial \theta}
	&=\sum_{k=1}^K\frac{\partial\log\left(l_k(\theta,\bm{\Lambda})\right)}{\partial \theta}\\
	&=\frac{\partial\mathcal{L}(\theta,\bm{\Lambda})}{\partial \theta}. 
\end{alignat}

If we assume a probability distribution $p(\bm{\Lambda})$  of the intensity states $\bm{\Lambda}$ we can express the Fisher information (see \autoref{eq:app-Fisher Info Definition}):
%
\begin{equation}
	I(\theta) = \int_{\bm{\Lambda}}p(\bm{\Lambda})I_{\bm{\Lambda}}(\theta)d\bm{\Lambda},
\end{equation}
%
where $I_{\bm{\Lambda}}(\theta)$ is the Fisher information computed for a specific value of $\bm{\Lambda}$ (see \autoref{eq:app-Fisher Info Definition}).

For discrete states of $\bm{\Lambda}$, for example  
%
\begin{alignat}{4}
	\bm{\Lambda}
	&=\left\{\bm{\Lambda^{\alpha=1}},\,\bm{\Lambda^{\alpha=2}},\,\bm{\Lambda^{\alpha=3}},\,\bm{\Lambda^{\alpha=4}}\right\}\\
	&=\left\{[\Lambda_1,0],\,[\Lambda_2,0],\,[\Lambda_1,\Lambda_2],\,[0,0]\right\}
\end{alignat}	
% 
we get
\begin{equation}
	I(\theta)=\sum_{\alpha}p(\bm{\Lambda^\alpha})I_{\bm{\Lambda^\alpha}}(\theta),
\end{equation}
%
where the Fisher Information for every configuration of $\bm{\Lambda^\alpha}$ is averaged with weights $p(\bm{\Lambda^\alpha})$. 

%==========================================
%==========================================

\section{Time distribution of the intensities - integrating out $\Lambda$}
\label{sub:Appendix Time-distribution-Integrating out}
If we do not know the configuration of the $\bm{\Lambda^\alpha}$ in each frame, then we have to rely only on the distribution $p(\Lambda)$ and integrate over it within the likelihood function:
%
\begin{equation}
	l_k(\theta)=\int_{\Lambda}l_k(\theta,\Lambda)d\Lambda=\int_{\Lambda}p(n_k|\theta,\Lambda)p(\Lambda)d\Lambda
	\label{eq:app-log lik - int out}
\end{equation}
%
We assume four state model of two sources: $\left\{ (\Lambda_1,0),(0,\Lambda_2),(\Lambda_1,\Lambda_2),(0,0)\right\}$: 
%
\begin{alignat}{4}
	\lambda_k^{\alpha=1}&=\Lambda_1q_k(x-c_1) & &+b,\\ 
	\lambda_k^{\alpha=2}&=&\Lambda_2q_k(x-c_2) &+b,\\ 
	\lambda_k^{\alpha=3}&=\Lambda_1q_k(x-c_1)&+\Lambda_2q_k(x-c_2)&+b,\\ 
	\lambda_k^{\alpha=4}&=& &+b,
\end{alignat}
%
with uniform distribution over these states. We used uniform background intensity $b$ in each pixel of each frame. 

Assuming $p(\bm{\Lambda^\alpha})=1/4$ for $\alpha=1,..4$, then from \autoref{eq:app-log lik - int out}:
%
\begin{equation}
	l_k(\theta)=\frac{1}{4}\sum_{\alpha=1}^4\Po(\lambda_k^\alpha),
\end{equation}
%
with derivatives 
%
\begin{equation}
	\frac{\partial l_k}{\partial c_p}=\frac{1}{4}\sum_\alpha\frac{\partial\Po(\lambda_k^\alpha)}{\partial c_p}=\frac{1}{4}\sum_\alpha\left(\Po(\lambda_k^\alpha)\frac{(n_k-\lambda_k^\alpha)}{\lambda_k^\alpha}\frac{\partial\lambda_k^\alpha}{\partial c_p}\right).
\end{equation}
%
The diagonal entries of the Fisher information matrix are give by:
%
\begin{alignat}{2}
	I_{pp}(\bm{c}) & =\E\left[\left(\sum_{k=1}^N\frac{1}{l_k}\frac{\partial l_k}{\partial c_p}\right)^2\right] \\
 	& =\E\left[\left\{ \sum_{k=1}^N\left(\frac{1}{\sum_{\alpha=1}^4\Po(\lambda_k^\alpha)}\frac{\partial\sum_{\alpha=1}^4\Po(\lambda_k^\alpha)}{\partial c_p}\right)\right\} \left\{ \sum_{l=1}^N\left(\frac{1}{\sum_{\alpha=1}^4\Po(\lambda_l^\alpha)}\frac{\partial\sum_{\alpha=1}^4\Po(\lambda_l^\alpha)}{\partial c_p}\right)\right\} \right] \\
	& =\sum_{k=1}^N\E_k\left[\frac{\left(\sum_{\alpha=1}^4\frac{\partial\Po(\lambda_k^\alpha)}{\partial c_p}\right)^2}{\left(\sum_{\alpha=1}^4\Po(\lambda_k^\alpha)\right)^2}\right],
	\label{eq:app-Fisher Info Integrated Out - diagonal entries}
\end{alignat}
%
because the cross terms ($k,\, l$) in the sum (2nd row) are zeros: 
%
\begin{alignat}{2}
	\E\left[\left(\frac{\sum_{\alpha=1}^4\frac{\partial\Po(\lambda_k^\alpha)}{\partial c_p}}{\sum_{\alpha=1}^4\Po(\lambda_k^\alpha)}\right)\left(\frac{\sum_{\alpha=1}^4\frac{\partial\Po(\lambda_l^\alpha)}{\partial c_p}}{\sum_{\alpha=1}^4\Po(\lambda_l^\alpha)}\right)\right] 
	& =\E_k\left[\frac{\sum_{\alpha=1}^4\frac{\partial\Po(\lambda_k^\alpha)}{\partial c_p}}{\sum_{\alpha=1}^4\Po(\lambda_k^\alpha)}\right]\E_l\left[\frac{\sum_{\alpha=1}^4\frac{\partial\Po(\lambda_l^\alpha)}{\partial c_p}}{\sum_{\alpha=1}^4\Po(\lambda_l^\alpha)}\right]\\
 	& =\sum_{\alpha=1}^4\frac{\partial}{\partial c_p}\left(\sum_{n_k\geq0}\Po(\lambda_k^\alpha)\right)\sum_{\alpha=1}^4\frac{\partial}{\partial c_p}\left(\sum_{n_k\geq0}\Po(\lambda_l^\alpha)\right)\\
 	& =0
\end{alignat}

Expressing the derivatives and the expectation from \autoref{eq:app-Fisher Info Integrated Out - diagonal entries} we can write for the diagonal entries of the Fisher information matrix:
%
\begin{alignat}{2}
	I_{pp}(\bm{c}) 
	& =\sum_{k=1}^N\E_k\left[\left\{ \frac{\sum_{\alpha=1}^4\left(\Po(n_k;\lambda_k^\alpha)\frac{(n_k-\lambda_k^\alpha)}{\lambda_k^\alpha}\frac{\partial\lambda_k^\alpha}{\partial c_p}\right)}{\sum_{\alpha=1}^4\Po(n_k;\lambda_k^\alpha)}\right\} ^2\right]\\
 	& =\frac{1}{4}\sum_{k=1}^N\sum_{n_k\geq0}\frac{\left\{ \sum_{\alpha=1}^4\left(\Po(n_k;\lambda_k^\alpha)\frac{(n_k-\lambda_k^\alpha)}{\lambda_k^\alpha}\frac{\partial\lambda_k^\alpha}{\partial c_p}\right)\right\} ^2}{\sum_{\alpha=1}^4\Po(n_k;\lambda_k^\alpha)}
\end{alignat}

For the four states model we have $\lambda^{\alpha=3}(c_1,c_2)=\lambda^{\alpha=1}(c_1)+\lambda^{\alpha=2}(c_2)-b$ and so $\frac{\partial\lambda^{\alpha=3}}{\partial c_p}=\frac{\partial\lambda^{\alpha=p}}{\partial c_p}$ and $\frac{\partial\lambda^{\alpha=j}}{\partial c_p}=0,\, j\neq p$ for $p=\{1,2\},\: j=\{1,2,4\};$ so 
%
\begin{equation}
	I_{pp}(\bm{c}) 
	=\frac{1}{4}\sum_{k=1}^N\left(\frac{\partial\lambda_k^{\alpha=p}}{\partial c_p}\right)^2\sum_{n_k\geq0}\frac{\left\{\sum_{\alpha=\{p,3\}}\left(\Po(n_k;\lambda_k^\alpha)\frac{(n_k-\lambda_k^\alpha)}{\lambda_k^\alpha}\right)\right\}}{\sum_{\alpha=1}^4\Po(n_k;\lambda_k^\alpha)} ^2
\end{equation}


The off-diagonal entries of the  Fisher information matrix are given by:
%
\begin{alignat}{2}
	I_{pq}(\bm{c}) 
	& =\sum_{k=1}^N\E_k\left[\frac{\left(\sum_{\alpha=1}^4\frac{\partial\Po(\lambda_k^\alpha)}{\partial c_p}\right)\left(\sum_{\alpha=1}^4\frac{\partial\Po(\lambda_k^\alpha)}{\partial c_q}\right)}{\left(\sum_{\alpha=1}^4\Po(\lambda_k^\alpha)\right)^2}\right] \\
 	& =\frac{1}{4}\sum_{k=1}^N\left(\frac{\partial\lambda_k^{\alpha=p}}{\partial c_p}\right)\left(\frac{\partial\lambda_k^{\alpha=q}}{\partial c_q}\right)\\
	&\times\sum_{n_k\geq0}\frac{\left(\sum_{\alpha=\{p,3\}}\Po(n_k;\lambda_k^\alpha)\frac{(n_k-\lambda_k^\alpha)}{\lambda_k^\alpha}\right)\left(\sum_{\alpha=\{q,3\}}\Po(n_k;\lambda_k^\alpha)\frac{(n_k-\lambda_k^\alpha)}{\lambda_k^\alpha}\right)}{\sum_{\alpha=1}^4\Po(n_k;\lambda_k^\alpha)}
	\label{eq:app-Fisher Info Integrated Out-off diagonal entries}
\end{alignat}

\subparagraph*{Limit $d\rightarrow0$:}\ \\
%
When $c^1=c^2$ then $\lambda^{\alpha=1}=\lambda^{\alpha=2}$ and $\frac{\partial\Po(\lambda^{\alpha=1})}{\partial c^1}=\frac{\partial\Po(\lambda^{\alpha=2})}{\partial c^2}$.
Then all entries in $I_{pq}$ are equal and the matrix is singular. For the limit $d\rightarrow0$ the determinant $\det(\bm{I})\rightarrow0$ and the variance $\var(d)\rightarrow\infty$.

\subparagraph*{Limit $d\rightarrow\infty$:}\ \\
%
Sources are far apart and $\lambda^{\alpha=1}$ and $\lambda^{\alpha=2}$ do not have a common overlap. For $k'$ where $\frac{\partial\lambda_{k'}^{\alpha=p}}{\partial c_p}\neq0,\,\frac{\partial\lambda_{k'}^{\alpha=q}}{\partial c_p}\equiv0$ and $\Po(n_{k'},\lambda_{k'}^{\alpha=3})=\Po(n_{k'},\lambda_{k'}^{\alpha=1})$. Also $\sum_\alpha\Po(\lambda_k^\alpha)=2\Po(\lambda_k^{\alpha=p})+2\Po(b)$ in the region where $\frac{\partial\lambda^{\alpha=p}}{\partial c_p}\neq 0$.

From \autoref{eq:app-Fisher Info Integrated Out - diagonal entries} the cross terms vanish ($I_{pq}=0$ because $\frac{\partial\Po(\lambda^{\alpha=p})}{\partial c_p}\frac{\partial\Po(\lambda^{\alpha=q})}{\partial c_{q}}=0$). The diagonal elements 
%
\begin{alignat}{3}
	I_{pp} 
	& =\sum_{k=1}^N\E_k\left[\frac{\left(2\frac{\partial\Po(\lambda_k^{\alpha=p})}{\partial c_p}\right)^2}{\left(2\Po(\lambda_k^{\alpha=p})+2\Po(b)\right)^2}\right] \\
 	& =\sum_{k=1}^N\E_k\left[\frac{\left(\Po(\lambda_k^{\alpha=p})\frac{\left(n_k-\lambda_k^{\alpha=p}\right)}{\lambda_k^{\alpha=p}}\frac{\partial\lambda_k^{\alpha=p}}{\partial c_p}\right)^2}{\left(\Po(\lambda_k^{\alpha=p})+\Po(b)\right)^2}\right]\label{eq:app-FREM blink limit infty bg}
\end{alignat}
%
for $b=0$ $\left(\Po(b)=0\right)$:
%
\begin{alignat}{2}
	I_{pp} 
	 &=\sum_{k=1}^N\left(\frac{1}{\lambda_k^{\alpha=p}}\frac{\partial\lambda_k^{\alpha=p}}{\partial c_p}\right)^2\E_k\left[\left(n_k-\lambda_k^{\alpha=p}\right)^2\right]\\
 	& =\sum_{k=1}^N\left(\frac{1}{\lambda_k^{\alpha=p}}\frac{\partial\lambda_k^{\alpha=p}}{\partial c_p}\right)^2\frac{1}{4}\sum_{n_k\geq0}\left(\sum_{i=1}^4\Po(\lambda_k^\alpha)\left(n_k-\lambda_k^{\alpha=p}\right)^2\right)\\
 	& =\sum_{k=1}^N\left(\frac{1}{\lambda_k^{\alpha=p}}\frac{\partial\lambda_k^{\alpha=p}}{\partial c_p}\right)^2\frac{1}{4}\sum_{n_k\geq0}\left(2\Po(\lambda_k^{\alpha=p})\left(n_k-\lambda_k^{\alpha=p}\right)^2\right)\\
 	& =\sum_{k=1}^N\left(\frac{1}{\lambda_k^{\alpha=p}}\frac{\partial\lambda_k^{\alpha=p}}{\partial c_p}\right)^2\frac{1}{2}\lambda_k^{\alpha=p}\\
 	& =\frac{1}{2}\sum_{k=1}^N\frac{1}{\lambda_k^{\alpha=p}}\left(\frac{\partial\lambda_k^{\alpha=p}}{\partial c_p}\right)^2	\label{eq:app-FREM blink lim infty b=0}
\end{alignat}
%
which is the expression for the static sources \autoref{eq:app-Fisher Info for Poisson}, up to a factor $1/2$. The factor 1/2 comes from the fact that the source appears only in 50\% of the observations. If we keep the number of photons constant in both blinking and the static case (by reducing the intensity of the static sources by factor of two) we get identical value of $\var(d)$ for the $d\rightarrow\infty$. 

For non zero background $b>0$ we cannot simplify \autoref{eq:app-FREM blink limit infty bg} due to the background term $\Po(b)$ in the denominator. However, as the term is positive, the element $I_{pp}$ will be decreasing with increasing background. The background makes therefore the variance $\var(d)$ bigger as we would expect.

%==========================================
%==========================================

\section{Comments on ``dip'' in FREM curves for static sources}
\label{sec:Appendix dip comments}
These are several comments about the strange ``dip'' in the FREM curves for static sources (see \autoref{fig:FREM dip}). The expression for computing FREM for two sources with equal intensity is from \autoref{eq:var symmetric}
\begin{equation}
	\var(d)=\left[\frac{I_{11}-I_{22}}{2}\right]^{-1},
	\label{eq:app-vard}
\end{equation}
%
and the individual entries of the Fisher information matrix is from \autoref{eq:app-Fisher Information alternative - Individual}:
\begin{equation}
	I_{ij} =\Lambda\sum_{k=1}^{K}\frac{q'_k(c_i)q'_k(c_j)}{q_k(c_1)+q_k(c_2)+b/\Lambda};\; \ i,j=\{1,2\},
\end{equation}
%
where $q_k$ and $q'_k$ are pixelised PSF and derivative, respectively. 

For large background values ($b/\Lambda \gg \max[q_k(c_i)]$) the bottom term is nearly constant and therefore
\begin{equation}
	I_{ij} \approx C \sum_{k=1}^{K}q'_k(c_i)q'_k(c_j),
\end{equation}
where $C$ is a constant.

From \autoref{eq:app-vard}
\begin{equation}
	\var(d)\approx \left[C/2 \sum_k \left(q'_k(c_1)^2-q'_k(c_1)q'_k(c_2)\right)\right]^{-1}
\end{equation}

For Gaussian approximation of the PSF, the expression can be integrated analytically. The resulting curve sows the identical ``dip'' in the curves.

%When the background is present % <---- I think this is incorrect.... because the background in in \lambda^p_k
%%
%\begin{alignat*}{1}
%	I_{pp} & =\sum_{k=1}^N\E_k\left[\frac{\left(2\frac{\partial\Po(\lambda_k^{\alpha=p})}{\partial c_p}\right)^2}{\left(2\Po(\lambda_k^{\alpha=p})+2\Po(b)\right)^2}\right]\\
% 	& =\sum_{k=1}^N\E_k\left[\frac{\left(\Po(\lambda_k^{\alpha=p})\frac{\left(n_k-\lambda_k^{\alpha=p}\right)}{\lambda_k^{\alpha=p}}\frac{\partial\lambda_k^{\alpha=p}}{\partial c_p}\right)^2}{\left(\Po(\lambda_k^{\alpha=p})+\Po(b)\right)^2}\right]\\
% 	& =\sum_{k=1}^N\left(\frac{1}{\lambda_k^{\alpha=p}}\frac{\partial\lambda_k^{\alpha=p}}{\partial c_p}\right)^2\E_k\left[\left(\frac{\Po(\lambda_k^{\alpha=p})}{\Po(\lambda_k^{\alpha=p})+\Po(b)}\right)^2\left(n_k-\lambda_k^{\alpha=p}\right)^2\right]\\
% 	& =\sum_{k=1}^N\left(\frac{1}{\lambda_k^{\alpha=p}}\frac{\partial\lambda_k^{\alpha=p}}{\partial c_p}\right)^2\E_k\left[\left(1-\frac{\Po(b)}{\Po(\lambda_k^{\alpha=p})+\Po(b)}\right)^2\left(n_k-\lambda_k^{\alpha=p}\right)^2\right]\\
% 	& =\frac{1}{2}\sum_{k=1}^N\frac{1}{\lambda_k^{\alpha=p}}\left(\frac{\partial\lambda_k^{\alpha=p}}{\partial c_p}\right)^2-S
%\end{alignat*}
%%
%where
%%
%\begin{equation}
%	S=\sum_{k=1}^N\left(\frac{1}{\lambda_k^{\alpha=p}}\frac{\partial\lambda_k^{\alpha=p}}{\partial c_p}\right)^2\E_k\left[\left\{ \frac{2\Po(b)}{\Po(\lambda_k^{\alpha=p})+\Po(b)}+\left(\frac{2\Po(b)}{\Po(\lambda_k^{\alpha=p})+\Po(b)}\right)^2\right\} \left(n_k-\lambda_k^{\alpha=p}\right)^2\right].
%\end{equation}
%%
%This term is positive and therefore reduces $I_{pp}$.
%!TEX root = thesis.tex
\chapter{LS-SIM electronics\label{app:LSSIM electronics}}
%
\begin{figure}[!hbt]
	\centering
	\newcommand{\wf}{.8\textwidth}
%	\subfloat[Schematic of the trigger signals]{
%	\includegraphics[width=\wf]{\home Documents/Publications/LineScanPaper/figures/setupElectoricsCorrected/Media1a}}\\
%	\subfloat[Program listing]{
%	\includegraphics[width=\wf]{\home Documents/Publications/LineScanPaper/figures/setupFigure/listing}}		
	\includegraphics[width=\wf]{\home Documents/Publications/LineScanPaper/figures/setupElectoricsCorrected/Media1a}
	\caption{Schematic of the trigger signals in the microscope. The scan-mirror is the master and triggers the camera when it starts moving in one edge of the field. Upon receiving {\sf EXTERNAL\_TRIGGER} the camera starts integrating in less than a microsecond and sets {\sf FIRE} to high. This triggers an interrupt in the micro-controller and it generates {\sf LIGHT\_ENABLE} to modulate the laser. The initial $p_i$ delays determine the position of the scanning lines.
The values in the timing diagram (bottom right) were measured with a logic analyser (Logic16, Salea). These values correspond to an optimised timing of the LS-SIM system, such that the sum of the 32 line positions imprinted on the sample produced a homogeneous wide-field illumination.} 
\end{figure}



%\bibliographystyle{apalike}
\bibliographystyle{unsrt}
%\bibliographystyle{alpha}

%% If you want the bibliography single-spaced (which is allowed), uncomment the next line.
\singlespace
\bibliography{\home Documents/References/Microscopy,\home Documents/References/QuantumDots,\home Documents/References/MachineLearning,\home Documents/References/Deconvolution,\home Documents/References/SpatialPatterns}
\end{document}
