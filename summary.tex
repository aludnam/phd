%!TEX root = thesis.tex
\chapter{Summary}

We showed that non-negative matrix factorisation with iterative restarts (\inmf) can separate highly overlapping intermittent sources and is comparable to recently published methods (CSSTORM and 3B analysis) dealing with the similar problematic. We described a pipeline for evaluation and visualisation of realistic datasets and used \inmf\ to show super-resolution images of tubulin structures labelled with quantum dots. \inmf\ is a promising and very accessible technique with potential to deliver super-resolution images of three-dimensional samples.

Combination of structured illumination with line scanning (LS-SIM) presented in this thesis provides images of thick fluorescent samples with a significant resolution improvement in the lateral plane. Line scanning reduces the out-of-focus background and the LS-SIM images suffer less from the reconstruction artefacts when compared to conventional structured illumination. LS-SIM reveals fine details of a biological specimen's inner structure with higher resolution than line-confocal microscopy and with image quality superior to conventional structure illumination.

We also discuss a theoretical resolution limit for noisy and pixelated datasets. We show that fluorescence intermittency (such as quantum dots blinking) can be beneficial for resolution when compared to the static sources emitting the equal number of photons.