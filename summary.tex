%!TEX root = thesis.tex
\chapter{Summary\label{ch:Summary}}

In this thesis we have made three contributions to super-resolution methods for fluorescence microscopy:

We have shown that non-negative matrix factorisation with iterative restarts (\inmf{}) can separate highly overlapping intermittent sources,
and is comparable in performance to recently published methods (CSSTORM and 3B analysis) dealing with similar problems. We introduced average precision (AP) as a quantitative measure comparison of the algorithm performance on simulated data. We used AP for quantitative comparison of \inmf{} with CSSTORM and the 3B analysis demonstrating superior performance of \inmf{} on simulated data of highly overlapping sources. We described a pipeline for evaluation and visualisation of realistic datasets, and used \inmf{} to show super-resolution images of tubulin structures labelled with quantum dots. \inmf{} is a promising and very accessible technique with potential to deliver super-resolution images of three-dimensional samples.

The combination of structured illumination with line scanning (LS-SIM) presented in this thesis provides images of thick fluorescent samples with a significant resolution improvement in the lateral plane. Line scanning reduces the out-of-focus background and the LS-SIM images suffer less from the reconstruction artefacts when compared to conventional structured illumination. LS-SIM reveals fine details of a biological specimen's inner structure with higher resolution than line-confocal microscopy and with image quality superior to conventional structure illumination.

We also discuss a theoretical resolution limit for noisy and pixelated datasets. We presented an alternative derivation of the fundamental resolution measure (FREM), which provides correction to the original formula published by Ram et al. \cite{Ram2006}. We show that fluorescence intermittency (such as quantum dots blinking) can be beneficial for resolution when compared to the static sources emitting the equal number of photons.