%!TEX root =  thesis.tex
\chapter{Future work\label{ch:Future work}}

The unique ability of the \inmf{} algorithm to recover sources with different shapes discussed in \autoref{sub:results - out of focus PSF real data} allows extension of the super-resolution imaging to three dimensional samples. The axial position of the fluorophore can be determined from the shape of the recovered out-of-focus PSF by, for example, determination of the diameter of the outmost ring \cite{Speidel2003}. However, the conventional out-of-focus PSF decreases quickly in brightness when compared to the in-focus PSF (see \autoref{fig:Simulted-PSF-different-focal-depths}), which makes the separation of overlapping sources located in different focal planes difficult. The brightness of specially designed PSF, such as double helix PSF \cite{Quirin2011} or PSF with introduced astigmatism \cite{Huang2008} is less sensitive to the defocus because the out-of-focus PSF remains compact over defocus of several hundreds nanometres. On the other hand the in-focus PSF is less bright than the one in the system without aberrations. The axial position is determined from the specific changes of the PSF shape. Testing of the \inmf{} algorithm on the data with specially designed PSF is a logical extension of the current work. 

We also want to apply the \inmf{} to the specimens labelled with standard organic fluorophores dyes. For example, dSTORM \cite{VandeLinde2011} makes advantage of the repetitive transfer of the conventional fluorescent probes between the bright ``ON state''  and the stable and reversible dark ``OFF states''. This results in the intermittent intensity of the fluorescent sources. The determination of the overlapping sources with \inmf{} can significantly speed up the data acquisition \cite{Small2009}.

We prepare separate publications from the \autoref{ch:NMF} and \autoref{ch:Theoretical-limits-of the LM}. 